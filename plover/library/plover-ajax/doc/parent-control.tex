\documentclass[hyperref, bookmark]{article}

\usepackage[top=1in, bottom=1in, left=1in, right=1in]{geometry}
\usepackage[cm-default]{fontspec}
\usepackage{xunicode,xltxtra}
\usepackage[slantfont,boldfont]{xeCJK}
\usepackage[pdfborder={0 0 0}]{hyperref}
\usepackage{multicol}
% \usepackage{float}
\usepackage{framed}
\usepackage{fancybox}
\usepackage[vario]{fancyref}
\usepackage{url}
\usepackage{xy}

\setCJKmainfont{WenQuanYi Micro Hei}

\XeTeXlinebreaklocale "zh"
\XeTeXlinebreakskip = 0pt plus 1pt
\linespread{1.2}
\setlength{\columnsep}{.5in}

\newcommand*\xpc{ \textsc{X-Parent-Control} }
\newcommand*\ajax{AJAX}
\newcommand{\syntax}[1]{
    \vskip .2cm
    \doublebox{
        \begin{minipage}{.9\linewidth}
        #1
        \end{minipage}
    }
    \vskip .5cm
}

\newcommand\ol[1]{\begin{enumerate}{#1}\end{enumerate}}

% \newfontfamily\urlf{"Nimbus Sans L"}
% \makeatletter
% \def\url@leostyle{
%   \@ifundefined{selectfont}{\def\UrlFont{\sf}}{\def\UrlFont{\small\urlf}}}
% \makeatother
% \urlstyle{leo}

\title {\xpc \\ 父页面控制协议 }
\author{谢继雷 \\ $\infty$ \small{海宁小蜜蜂软件}}

\begin{document}

\maketitle

\abstract {
    \xpc 协议用于在 \ajax 请求中,实现由内部页面向外部传递异常的机制。
    目前 \xpc 实现了两种控制: 父页面重定向与父页面脚本调用。
    }

\vskip .5in
\begin{multicols}{2}
    \tableofcontents
\end{multicols}
\vskip .5in

\section {协议头}

    \xpc 协议向 HTTP 返回报文中添加了以下字段: (见 \Fref{tab:headers})

    \begin{table*}[ht]
        \label{tab:headers}
        \caption{\xpc 头列表}
        \begin{center}
        \begin{tabular}{l|c|l}
            \hline
              协议头  & 描述 & 格式 \\
            \hline
            X-Parent-Control    & \xpc 版本       & 1.0 \\
            X-Parent-Location   & 父页面重定向     & 相对或绝对URL \\
            X-Parent-Invoke     & 父页面脚本调用   & JavaScript 脚本 \\
            X-Parent-Auth       & 控制验证授权码   & URL 或 Javascript 的 HMAC 字符串 \\
            \hline
        \end{tabular}
        \end{center}
    \end{table*}

\section {父页面重定向}

    \syntax {
        X-Parent-Control: 1.0

        X-Parent-Location: \emph{URL-Spec}

        X-Parent-Auth: HMAC( \emph{URL-Spec} )
    }

    \emph{URL-Spec} 可以是绝对URL 或相对于当前父页面地址的相对 URL。
    当父页面收到 \xpc 控制时,应该使用 HTML4 DOM 模型的 location.href = \emph{URL-Spec} 来解释。

    安全措施: \texttt{X-Parent-Location} 必须验证。
        MITM 攻击中,入侵者篡改内部页面,插入 HTTP \texttt{Location} $(301)$ 重定向头,可以使内部页面重定向到非法内容。但父页面决定如何解释内部页面的数据。如果入侵者插入非法 \texttt{X-Parent-Location} 值,则可能导致非法的跨站请求。


\section {父页面脚本调用}

    \syntax {
        X-Parent-Control: 1.0

        X-Parent-Invoke: \emph{JavaScript}

        X-Parent-Auth: HMAC( \emph{JavaScript})
    }

    其中 \emph{JavaScript} 指定要执行的脚本。该脚本通过 JavaScript \texttt{eval} 的方法被求值。

    安全措施: \texttt{X-Parent-Invoke} 必须验证。理由是显而易见的。

\section {验证算法}

    为了防止恶意代码注入,在返回的控制信息中必须附加验证码。
    目前为了简化起见, 只使用了随机数,并用下列方式实现密钥分享,在真正应用中这是不安全的。
    将来改变这一算法应不影响程序结构。

    \ol{
        \item Parent生成隨機數 $ secret $,计算 $ E = 9*secret^2 + 102*secret + 57 $,发送 $E$ 给远程.
        \item 远程计算 $\displaystyle secret = \frac{\sqrt{E - 232} - 17}{3} $
        \item 远程计算 $H= hmac(secret, message)$,作为 HMAC 验证码
        \item 远程将 HMAC 附加到 \xpc HTTP 报文上
        \item Parent 收到 $message$,重新计算 $H$ 并核对,如果 $H$ 错误则丢弃控制信息。
    }

    验证使用如下 HMAC 算法 \cite{ebcdev.hmac}:

   \begin{leftbar}
   \begin{center}
    $ K = sha1(string(k)) $ \\
    $ hmac(K, m) = sha1((K \oplus opad) \parallel sha1((K \oplus ipad) \parallel m)) $ \\
    $ opad = 50_{16} $ \\
    $ ipad = 36_{16} $
   \end{center}
   \end{leftbar}


\section {使用}

    下图描述了 \xpc 的典型用法:
    (其中 \textsl{ 斜体 } 部分为验证相关,目前可暂时缺省)

    \subsection {父页面}

        \ol{
            \item include jquery-min.js;
            \item include jquery.sha1.js;
            \item include jquery.ajaxex.js;
            \item \textsl{ sessionSecret = secret = RANDOM; }
            \item \textsl{ E = 9*secret*secret + 102*secret + 57; }
            \item Xxx.load(AJAX-URL... \textsl{\&e=E} );
        }

    \subsection {AJAX请求}

        \ol{
            \item import ParentControl;
            \item \textsl{ E = request.getParameter("e"); }
            \item \textsl{ secret = (Math.sqrt(E - 232) - 17) / 3; }
            \item ParentControl.sendParentRedirect(response, "new-location..."
                    \textsl{, secret} );
        }

\appendix
\section {实现参考}

     引用 \cite{jq.ajaxex} 以启用父页面控制扩展。

    Servlet 支持类参考 \cite{ebcdev.xpc}。

    \bibliographystyle{plain}
    \bibliography{secca}

\section {索引}

    \listoftables

\end{document}
