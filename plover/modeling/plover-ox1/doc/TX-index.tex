\maketxtable{用户组}{group}{TX.Group.tab}{可被用户共享访问资源的用户组。}
\maketxtable{角色}{role}{TX.Role.tab}{共用角色的用户对资源具有相同的访问权限。}
\maketxtable{用户}{user}{TX.User.tab}{可登录到系统的用户帐户。}
\maketxtable{用户邮箱}{user\_email}{TX.UserEmail.tab}{管理用户的邮箱地址。}
\maketxtable{用户选项}{user\_option}{TX.UserOption.tab}{用户设置的选项。}
\maketxtable{用户配置项}{user\_preference}{TX.UserPreference.tab}{每个用户独立设置的配置项。}
\maketxtable{(无标题)}{formula}{TX.Formula.tab}{(无描述)}
\maketxtable{字段定义}{entity\_column}{TX.EntityColumn.tab}{自定义实体的字段定义,或对原有实体字段信息的重写。}
\maketxtable{实体信息}{entity\_info}{TX.EntityInfo.tab}{运行时自定义的实体信息,或对原有实体信息的重写。}
\maketxtable{自定义分类}{user\_category}{TX.UserCategory.tab}{对于文本内容,自定义分类可以提供一个下拉型选择,以方便用户输入。}
\maketxtable{自定义分类项目}{user\_category\_item}{TX.UserCategoryItem.tab}{自定义类别中的具体项目。}
\maketxtable{属性池-X40}{xpool40}{TX.XPool40.tab}{X40 属性池比 X30 属性池多出 10 个文本字段。}
\maketxtable{属性池模型}{xpool\_model}{TX.XPoolModel.tab}{定义属性池的结构和用途。}
