
一、 制定检验标准

    检验标准应以文件的形式固定下来,用以规定及指明检验作业的执行方法,以便于在繁杂的检验作业中,处理各种疏漏与不足。

    1. 明确制定检验标准的目的

    使检验人员有所依据,了解如何进行检验工作,以确保产品质量。

    2. 巳列明检验标准内容

    (1)适用范围 。列明适用于何种进料(含加工品) 或成品的检验。

    (2)检验项目。将实行检验时应检验的项目一一列出。

    (3)质量基准。明确规定各检验项目的质量基准,以此作为检验时判定的依据。如无法以文字述明,则用限度样本来表示。

    (4)检验方法。说明在检验各种项目时,使用何种检验仪器、量规、或是以感官检查(例如目视)的方式来检验; 如某些检验项目须委托其他机构代为检验,也应注明。

    (5)抽样计划。采用何种抽样计划表(例如,计数值用MIL-SID-105D,计量值用MIL-STD-414)。

    (6)取样方法。抽取样本,必须由群体批中无偏倚地随机抽取,可利用乱数表来取样,但群体批各制品无法编号时,则取样时必须从群体批的任何部位平均抽取样本。

    (7)群体批经过检验后的处置。

    ®属进料(含加工品)者,依进料检验规定的有关要点办理(如是台格批,则通知仓储人员办理人库手续,如是不合格批,则将检验情况通知采购单位,由其根据实际情况决定是否需要特采)。

    ®属成品者,依照成品质量管理作业办法有关要点办理(合格批人库或出旨,不合格批则退回生产单位检修)。

    (8)其他应注意的事项。

    ®如按特定的顺序来检验各检验项目时,必须将检验顺序列明。

    ®必要时,可将制品的蓝图或略图置于检验标准中。

    ®详细记录检验情况。

    ®检验时在样本中发现的不良品,以及在群体批次中偶然发现的不良品,应与良品交换。

    ®其他。

    3. 有关检验标准的制定与修正

    由工程单位、质量管理单位制定。

二、 制定检验作业指导书

    1.检验作业生旨导书的适用范围

    以下情况应编制检验作业指导书:

    (1)对工序质量控制计划中设置丁工序质量控制点的检验。

    (2)对关键和重要零件的检验。

    (3)新产品特有的检验活动。

    2.检验作业指寻书的内容

    (1)检验对象。主要为受检物的名称、图号,必要时还须说明其在检验流呈图上的位置(编号)。

    (2)质量特性。规定的检验项目,须鉴别的质量特性、规范要求,质量特性的重要性级别,所涉及的质量缺陷严重性级别。

    (3)检验方法。检验基准,检测程序与方法,检测中的有关计算方法,检测频次,抽样检验的有关规定及数数据。

    (4)检测手段。检测使用的工具,设备(装备)及计量器具,这些器物应处的状态,使用中必须指明的注意事项。

    (5)检验判断。正确指明对判断标准的理解,判断比较的方法、判定的原则注意事项,不合格品的处理程序和权限

    (6)记录和报告。指明需要记录的事项与方法和表格,规定要求报告的内容、方式、程序与时间。

    (7)其他。对于复杂的检验项目,检验作业指导书应给出必要的示意图表并提供有关的说明资料。

    3. 检验作业指导书的格式

    作业指导书没有固定的格式,作业指导书常采用表格或流程图形式,也可采用图文并茂的形式。
