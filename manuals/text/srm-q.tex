\section {供应商调查}

    供应商调查,在不同的阶段有不同的要求。供应商调查可以分成三种:第一种是资源市场调查;第二种是初步供应商调查;第三部是深入供应商调查。

\subsection {资源市场调查}

    资源市场调查的内容:
    \begin{enumerate}

        \item  资源市场的规模、容量、性质。例如资源市场究竟有多大范围?有多少资源量?是卖方市场还是买方市场?是完全竞争市场、垄断竞争市场还是垄断市场?是一个新兴成长的市场,还是一个陈旧的没落的市场?

        \item  资源市场的环境如何?例如市场的管理制度、法制建设、市场的规范化程度、市场的经济环境、政治环境等外部条件如何?市场的发展前景如何?

        \item  资源市场中各个供应商的情况如何?也就是指前面进行的初步供应商调查所得到的情况如何。把众多的供应商调查资料进行分析,就可以得出资源市场自身的基本情况。例如资源市场的生产能力、技术水平、管理水平、可供资源量、质量水平、价格水平、需求状况以及竞争性质等。

        \item 资源市场分析。资源市场调查的目的,就是要进行资源市场分析。资源市场分析,对于企业制定采购策略以及产品策略、生产策略等都有很重要的指导意义。

        \item  要确定资源市场是紧缺型的市场还是富余型市场?是垄断性市场还是竟争性市场?对于垄断性市场,企业应当采用断性采购策略,对盖竞争性市场,企业应当采用竞争性采购策略。例如采用招标投标制、一商多角制等。

        \item  要确定资源市场是成长型的市场还是没落型市场,如果是没落型市场则要趁旱准备替换产品,不要等到产品被淘汰了再去开发新产品。

        \item  要确定资源市场总的水平,并根据整个市场水平来选择台适的供应商。通常要选择在资源市场中处于先进水平的供应商、选择产品质优而价格低的供应商。

    \end{enumerate}

\subsection {初步供应商调查}

    \begin{enumerate}
        \item 初步调查目的与方法。所谓初步供应商调查,是对供应商基本情况的调查。主要是了解供应商的名称、地址、生产能力、能提供什么产品、价格如何、质量如何、市场份额有多大、运输进货条件如何。

        初步供应商调查的目的,是了解供应商的一般情况。其目的一是为选择最佳供应商做准备;二是为了解掌握个资源市场的情况,因为资源市场是由每一个供应商共同形成的,那么许多供应商基本情况的汇总就是整个资源市场的基本情况。

        初步供应商调查的基本方法,一般可以采用访间调查法,通过访问有关人员而获得信息。例如,可以访问供应商的市场部有关人土或者访间有关用户、有关市场主管人员,或者其他的知情人士。通过访问建立起供应商卡片。

        在开展计算机信息管理的企业中,供应商管理应当纳入计算机管理之中。把供应商卡片的内容输入到计算机中去,利用数据库进行操作、补充和利用。计算机管理有很多优越性,它不但可以很方便地储存、添加、修改、查询和删除; 而且可以很方便地统计汇总和分析,可以实现不同子系统之间的数据共享。计算机有处埋速度快岭、计算量大、储存量大、数据传递快等优点,利用计算机行供应商管理具有很多的优越性。

        \item 初步供应商分析。在初步供应商调查的基础上,要利用供应商初步调查的资料进行供应商初步析。初步供应商分析的主要目的,是比较各个供应商的优势和劣势,初步选择可能适合与企业需要的供应商。

        初步供应商分析的主要内容包括以下几个方面。

        \begin{itemize}
            \item  产品的品种、规格和质量水平是否符台企业需要?价格水平如何?只有产品的品种、规格、质适合于本企业,才能算得上企业的可能供应商,才有必要进行下面的分析。

            \item  企业的实力、规模如何?产品的生产能力如何?技术水平如何?管理水平如何?企业的信用度如何?

            对信用度的调查,在初步调查阶段可以采用访问制,从中得出一个大概的、定性的结论。分析供应商的信用程度,这是可以得到定量的结果的。

            \item  产品是竟争性商品还是垄断性商品?如果是竞争性商品,则供应商的竞争态势如何? 产品的销售情况如何? 市场份额如何? 产品的价格水平是否合适?

            \item  供应商相对于本企业的地理交通情况如何?要进行运输方式、运输时间、运输费用分析,看运输成本是否适。
        \end{itemize}
        在进行以上分析的基础上,为选定供应商提供决策支持。
    \end{enumerate}

\subsection {深入供应商调查}

    深入供应商调查是指对经过初步调查后、准备发展为自己的供应商的企业进行的更加深入仔细的考察活动。这种考察,是深入到供应商企业的生产线、各个生产工艺、质检验环节甚至管理部门,对现有的工艺设备生产技术、管理技术等进行的考察,看看能不能满足本企业所采购的产品应当具备的生产工艺条件、质保证体系和管理规范要求。有的甚至要根据生产所采购产品的生产要求; 迸行资源重组并进行样品试制,试制成功才算考察合格,只有通过深入的供应商调查才能发现可靠的供应商,建立起比较稳走的物资采购供需关系。

    进行深入的供应商调查,需要胡鄂妃较多的时间和精力,调查成本高。并不是所有供应商都是需要的,它只在一下情况下才需要。

    \begin{enumerate}
        \item 准备发展成紧密关系的供应商。例如在进行准时化(JIT)采购时,供应商的产品准时、免检、直接送上生产线进行装配。这时,供应商已经与企业结成了如同企业的一个生产车间一样的紧密关系。如果要选择这样紧密关系的供应商,就必须进行深入的供应商调查。
        \item 寻找关键零部件产品的供应商。如果企业所采购的是一种关零部件,特别是精密度高加工难度大、质量要求高、在企业的产品中起核心功能作用的零部件产品,在选择供应商时,就需要特别小心,要进行反复认真的深入考察审核。只有通过深入调查证明确实能够达到要求时,才确定发展它为企业的供应商。
    \end{enumerate}

    对于最高级的深入调查,在具体实施深入调查时,也可以分成3个阶段。

    \begin{enumerate}
        \item 通知供应商生产样品,最好生产一批样品,从其中随机抽样进行检验。如果检验不合格,允许其改进一下再生产一批,再检一次。如果还是不合格,则供应商就落选,不再进入下面的第二阶段。只有抽检合格的才能进入第二阶段。

        \item 对于生产样品合格的供应商,还要进入供应商生产过程、管理过程进行全面详细考察,检查其生产能力、技术水平、质量保障体系、装卸搬运体系、管理制度等,看看有没有达不到要求的地方,如果基本上符合要求,则深入调查可以到此结束。应商符合要求,可以中选; 如果检查结果不符合要求,则进入下面第三个阶段。

        \item 对于生产工艺、 质量保障体系、规章制度等不符台要求的供应商,要协商提出改进措施,限期改迸。供应商愿意改进并改进台格者,可以中选企业的供应商。如果应商不愿意改进,或者愿意改进但改进不合格者则落选,深入调查也至此结束。
    \end{enumerate}
