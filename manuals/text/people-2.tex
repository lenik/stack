\section {做好招聘准备}

\subsection {招聘需求分析}

    \begin{enumerate}
        \item 分析招聘环境

        确定招聘需求,首先要对招聘环境进行分析。企业外部环境包括:经济条件、劳动力市场、地理位置等; 企业内部环境包括:战略发展方向、招聘成本与财务预算,企业形象和自身条件,企业声望、企业管理水平、岗位性质、企业管理风格、用人策略等。

        \item 分析人事配置。
        人事配置指的是人与事的配置关系,目的是通过人与事的配合以及人与人的协调,充分开发利用员工的才能,实现企业目标。
    \end{enumerate}

\subsection {编制招聘计划}

    科学地设计、制定一个合理的、高效的、可行的招聘计划,对招聘工作具有十分重要的意义。虽然形式各异,但一般来说,一个完整的招聘计划主要包括以下项目:

    \begin{enumerate}
        \item 列出人员需求清单。

        人力资源需求清单主要包括招聘的职位名称、人数、任职资格要求等内容。

        \item 定信息发布的时间和渠道。

        招聘时,要根据招聘对象的特点和资金的约束,合理选择发布信息的渠道和时间。

        \item  确定招聘小组人选。

        招聘不仅仅是人力资源管理部门的工作,用人部门、相关部门都应积极参与。为了保证招聘工作的顺利开展,企业应按照精于,高效的原则建立招聘工作小组,明确分工,确定各自的职责,以及考核的标准。

        \item 提出考核方案

        考核方案是招聘计划中最核心的要素,因为考核的手段直接关系着人力资源的质量,包括考核的场所布置,题目的设计以及考核的时间和人员的安排

        \item 明确费用预算。

        招聘工作需要投人一定的费用,如资料费,广告费,招聘测试费,体检费、人才交流会费。人力资源经理在招聘计划中,应明确列出招聘工作的主要项目以及与之相应的招聘费用,在保证质量的前提下,力争以较小的代价获得同样质量的人力资源。

        \item 确定工作时间表。

        工作时间表是对整个招聘工作如何有效进行的一种安排,要尽可能详细以便于相关人员的配合。
    \end{enumerate}

\subsection {编制招聘简章}

    招聘简童是企业招聘员工的重要工具之一,招聘简章传递给潜在应聘者的信息将影响到应聘者的数量和未来的留用比率。招聘简章必须简明,吸引人,主要内容应包括以下方面

    \begin{enumerate}
        \item 企业的基本情况。

        简要介绍单位的基本情况,让应聘者有一个太致的了解。

        \item 是否经过有关方面批准。

        一般情况下,发布招聘简章必须首先经过人事部门或相关机构的审批。

        \item 应聘的基本条件。

        这是招聘简章中最重要的要素,条件界定应非常清楚,要能让应聘者很容易确认自己是否符合基本要求。另外,应聘条件应非常合理,条件太高会导致脱离企业的实际,条件太低会增加筛选的工作量和招聘的成本。

        \item 录用后的各种待遇。

        主要包括录用以后的薪酬水平、福利、教育培训、健康保障、休假等内容。待遇的介绍必须真实,客观,否则即使招聘到了合格的人力资源也无法保证人力资源队伍的稳定。

        \item 不能遗漏的事项。
        主要包括报名的方式,需携带的证件和相关材科,报名的时间、地点以及联系人。
    \end{enumerate}

\section {人员选拔面试}

    人员招聘过程中的一个重要环节,就是采用多种方法对所需要的人力资源进行选拔。选拔也称为筛选 甄选、选择、挑选。选拔是企业获取高质量人力资源的一个关键,是企业引进人力资源的过滤器。一般来说,人员选拔常用的方法有筛选简历、笔试、面试和情景模拟几种形式。

    \begin{enumerate}
        \item 筛选简历

        应聘简历是应聘者自带的个人介绍材料,筛选应聘简历并没有统一的标准。

        \item  笔试

        笔试是一种最直接而又最基本的选择方法。它是让应聘者在试卷上回答事先拟好的试题,然后根据应聘者的成绩来选拔人员的一种选择方法。

        \item  面试

        面试是通过让应聘者当面回答问题的方式,来了解应聘者的知识,业务水平、心理素质和多方面能力的一种方法。面试是企业最常用的 也是必不可少的测试手段,但也是一种操作难度较高的测评形式,随意性较大,一般的人难以掌握。
    \end{enumerate}

\section {员工录用}

    员工录用时应制定相关的政策保证企业的整体素质和竞争力。录用员工应包括以下事项:

    \begin{enumerate}
        \item 对参加面试的员工的工作能力进行系统化的评估和比较,形成定性、定量的分析表。

        \item 对受聘员工作出初步录用的决策。

        \item 对应聘者的背景进行调查、核实。

        \item 双方就薪酬福利等问题达成共识,形成录用关系。

        \item 为员工办王里相关的人职手续。

        \item 签订员工试用期合同和正式录用合同。

        \item 转移、建立人事档案关系。
    \end{enumerate}

\section {员工离职}

\subsection {对待离职员工的原则}

    员工主动离职是每个企业人力资源经王里都会遇到且必须处理酌问题,在处理时可以遵循以下原则,以取得较好的效果。

    \begin{enumerate}
        \item 适度挽留人才。

        对员工提出的离职申谙不能亳不犹豫地”有求必应“,而要根据离职人员在企业中的重要性,努力挽留有用的人才,因为离职人员当中肯定不乏对企业十分重要的优秀员工,他们的流失会给企业带来很犬的损失。

        \item 正视人才流动。

        现在的员工更多的是用知识为公司赚钱,只有真心诚意地为企业工作,才会有好的绩效。当他们有更太的梦想要去实现时,企业不妨抱着祝福之心给予鼓励,因为将来这些员工的成功或许对企业未来的业务拓展客户关系等都会有很大帮助。

        \item 消除敌视态度。

        有的企业视离职员工为”叛徒“,千方百计强留不成,就采用各种手段进行惩罚。最后,不仅没有把人留下来,反而招致了离职员工的仇视,可谓两败俱伤。因此,消除敌视态度,是做好离职员工管理的基本原则之一。

        \item 以新的眼光看待离职员工。

        企业在离职员工的身上也许曾经花费了很多的心血,但离职并不是员工与企业关系的结束,而可以看成是新的开始,若把员工离职当成学生毕业,让昔日的”叛徒“成为”校友“,就可以实现离职员工管理根本性的转变。
    \end{enumerate}

\subsection {建立离职营王里程序}

    做好离职员工管理,需有规范的工作流程。一般来说,离职处王里流程为:
    \begin{enumerate}
        \item 离职申请。
        明文规定员工应提前提出离职的时间,离职申请接收人和离职申请的基本内容等。

        \item 挽留程序。

        接到离职申请的直接主管应当与离职员工迸行沟通,对于工作称职、业绩良好的员工尽量进行挽留,并了解其离职的原因,寻找解决的办法,减少企业因员工流失而造成的损失,如果直接主管挽留无效,则可由再上一级主管人员审核是否需要挽留,并根据情况再进行挽留谈话。企业还可在批准离职前为员工提供收回辞呈的机会,以便最太限度舶挽留人才。

        \item 离职审批。
            \begin{itemize}
                \item 经挽留无效或没有挽留必要的员工,可以进人离职审批流程,按照企业组织程序进行审批。

                \item 完成审批流程后,应将有关书面文件交人力资源管理部门确认。
            \end{itemize}

        \item 工作交接。

            \begin{itemize}
                \item 人力资源管理部门收到书面审批文件后,通知有关部门主管安排离职员工的工作交接。

                \item 交接工作完成后,应由有关交接八员和负责八书面确认,方可视为交接完成。
            \end{itemize}

        \item 离职面谈。

        工作交接完成后,应由人力资源部门或公司指定的负责人与离职员工进行离职面谈,听取离职员工的建议 意见和看法。

        \item 办王里离职手续。

        离职面谈结束后,由人力资源部门为离职员工办理办公用品清点、出具工作证明、解除劳动关系和结算工资福利等手续,手续完成后,离职员工正式离职。

    \end{enumerate}

\subsection {为离职员工办理手续}

    在办理员工离职手续时,可以按照以下程序进行:

    \begin{enumerate}
        \item 工作交接。

        工作交接包括客户资料,技术资料、源代码、项目文件、工作情况等方面的交接。

        \item 证件和办公用品收回。

        具体工作有: 收回工作证、名片,检查办公桌椅是否完好,收回钥匙和非低值易耗办公用品,收回借阅图书、资料,收回公司提供的其他物品。 如宿舍、车辆等。

        \item 财务结算。

        具体有借款、贷款等应收款项,工资、奖金、福利结算,违约金、承诺合同期未满的补偿费用结算。

        \item 劳动关系解除。

        即出具工作证明,办理离职手续,转调人事关系、档案、党团关系、保险关系等。
    \end{enumerate}
