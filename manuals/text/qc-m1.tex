一、 什么是质量

    品管应该完成的所有工作,可以说就是质量管理的中心业务,而其业务又以防止不良的发生为重点,从这个意义上看,关于质量的机能,它有如图6-1所示的业务。<image />

质量不是光靠检查就能确保的,要确保质量必须能正确地执行合理的设计、正当的工程管理等品管(预防机能)业务。也就是说,质量保证工作就是调查品管业务是否实行得当,调查设计、制造、销售等各部门是否确保了目标的质量,并将此结果向经营者报告。

二、 质量保证活动范围

  质量保证活动不只是各部门在各阶段的业务执行(部门内的活动)与管理动,它还包含部门与部门之间的活动及其管理(机能管理)。更重要的是,不只是明确活动的管理方法,并要求确实执行,同时还必须对质量的管理,即质计划、质量传达、质量确认等采取相同措施。以下以项目形式列举了质量保证活动的范围:

    (1)质量的设计,新品种、新制品质量的设定,规格的设定、修正与废除。

    (2)材料的购人与保管: 材料管理、库存管理。

    (3)标准化。

    (4)工程的解析与管理。

    (5)检查及不合格产品的处理。

    (6)客诉处理、质量稽查。

    (7)例设备管理:设备的建设、预防保养、计测管理。

    (8)人事劳务管埋: 适当正确的职务分配、教育训练。

    (9) 外包、转包管理。

    (10)技术开发: 新制品的开发、研究管理、技术管理。

    (11)诊断与稽查: 质量管理实施状况的诊断、品管关系业务稽查。

    三、 质量保证体系设立

    在设立质量保证体系时,应注意以下几点

    (1)回馈的方法必须明确。

    (2)体系图的纵轴表示开发的阶段,横轴表示职别,此职别的负责人必须明确有关内容。

    (3)对体系运作的手段,用具(表单类)及运作规则必须予以确定。

    (4)决定是否可以向下一阶段运作的评价项目与评价方法必须明确。

    (5)必须由体系运作所带来的经验来修正体系。

    四、 质量组织计划

    在质量管王里的推进计划中,最重要的是“组织计划”。

    1. 组织计划应害虑的事顶:

    在组织计划中要考虑以下事项

    (1)质量管理的组织,并不是指成立质量管理部或品管科,而必须是:

    ®让所有相关人员都知道有关产品质量情报的信息系统。

    ®在质量管理活动中,动员企业内所有的部门,所有阶层人员的方法。

    (2)是否有因人而设立组织,因组织而设立工作的倾向?

    (3)是否充分进行授权? 其授权工作不能只是口头说说,必须以文件加以明确。

    (4)组织之间的结合是否完全,品管活动中动员企业所有部门,所有阶层的方法是否确立? 从品管业务角度来看 横向的联系是否充分?

    (5)变更组织时的步骤是否明确?

    2. 组织计划的原则

    如前所述,组织化并不仅仅是将业务细分,成立各种职能部门,为了达到质量目标,还必须使其分担质量管理的责任与权限。 因此,组织必须满足以下原则:

    (1)组织的上级与下级之间必须有明确的权限关系。这里所说的权限是指要求其他人活动的正常权利。

    (2)组织中的每一个人应固定地向生产线的某一主管报告,并且要明白自己必须向谁或谁向自己报告。

    (3)经营者、管理者的责任与权限必须以文件形式明确限定。

    (4)权责要相符。

    (5)必须尽量缩减管理层次。

    (6)管理者必须致力于标准化,防止例外或异常情况的发生。

    (7)每一位管理者能够使其协力帮忙的职位数(也即管理幅度)有一定限制。一般情况下,高层的管理幅度为3\~6人,中层为5\~9人,低层为7\~15人。

    (8)组织应具有弹性,必须能顺应形势的变化而有所改变。

    (9)制度必须简明,即阶层的数目不要太多。

    在这种情形下,下层职位的人只要向自己生产线的主管报告,接受其指示就行了。职位间指挥命令的混乱,只要能明确职位所应管理的项目,即能化解。
