\section {供应商选择}

\subsection {供应商选择概述}

    实际上,供应商选择融台在供应商开发的全过程中。供应商开发的过程包括了几次供应商的选择过程,在众多的供应商中,每个瓶中都要选择5\~10个供应商进入初步调查;初步调查以后,要选择1\~3个供应商,进入深入调查;深入调查之后又要做一次选择,初步确定1\~2个供应商; 初步确定的供应商进入试运行,又要考核和选择,确定最后的供应商结果。

    一个好的供应商,一是产品好,二是服务好。所谓产品好,就是要求产品质好、产品价格合适、产品先进、技术含量高、发展前 景好、产品货源稳定、供应有保障; 所谓服务好,就是要求供应商在供送货方面能够及时、有很好的技木支持和售后服务、守信用、愿意协调配合客户企业。因此一个好的供应商需要具备以下一些条件。

    \begin{enumerate.zh}
    \item 企业生产能力强。表现在产量高,规模大,生产历史长,经验丰富,生产设备好。

    \item 企业技术水平高。表现在生产技术先进,设计能力和开发能力强,生产设备先选,产品的技术含高,达到国内先进水平。

    \item 企业管理水平高。表现在有一个坚强有力的领导班子,尤其是要有一个有魄力、有能力、有管理水平的一把手、要有一个高水平的生产管理系统、还要有一个有力的,具体落实的质管保障体系,要在全企业中形成一种严肃认证、一丝不苟的工作作风。

    \item 企业服务水平高。表现在能对顾客高度负责、主动热诚认真服务,并且售后服务制度完备、且服务能力强,愿意协调配合客户企业。

    \end{enumerate.zh}

\subsection {企业供应商分类}

    一个企业的供应商数可能很多,如果不加区分,就很难实施科学的管理。企业需要对不同的供应商实施不同的关系策略;就必须对供应商进行细分。

    \begin{enumerate.zh}
    \item 按供应商的重要程度分类————模块法。

        \begin{itemize}
            \item  分类。按供应商的重要程度分类,供应商可以分为:伙伴型供应商、重点型供应商、优先型供应商、商业型供应商。
        \end{itemize}

    \item 俺采购物品的价值大小分类————80/20规则。供应商80/20规则分类法的基础是物品采购的80/20规则,其基本思想是针对不同的才有物品应采取不同的策略,同时采购工作精力也应各有侧重,相应的,对于不同物品的供应商也应采取不同的策略。

    通常80\%的采购物品(普通采购物品)占采购物品20\%的价值,而其余数量20\%的物品(重点采购物品),则占采购物品80\%的价值。相应的,可以将供应商根据80/20规则进行分类,划分为重点供应商和普通供应商,即占80\%价值的20\%的供应商为重点供应商,而其余只占20\%采购金额的80\%供应商为普通供应商。对于重点供应商应投入80\%的时间和精力进行管理与改进。

    这些供应商提供的物品为企业的战略物品或需集中采购的物品,如汽车厂需要采购的发动机和变速器,电视机厂需要采购的彩色现象管以及一些价值高、但供应不力的物品。而对于普通供应商则只需要投入20\%的时间和精力跟其交货。因为这类商品所提供的物品的运作对企业的成本和生产质量和生产的影响小,例如办公用品、维修设备、标准件等。

    \item 按供应商的规模和经营品种分类。按供应商的规模和经营品种进行供应商分类,常以供应商的规模为纵坐标,经营品种数量为横坐标进行矩阵分析。

    \item 按与供应商的关系目标分类。

        \begin{itemize}
            \item  短期目标型。

            这种类型的最主要特征是双方之间的关系为交易关系。他们希望彼此能保持较长时期的买卖关系,获得稳定的供应。但是双方所做的努力只停留在短期的交易合同上,各自关注的是如何谈判,如何提高自己的谈判技巧,不使自己吃亏,而不是如何改善自己的工作,使双方都获利。

            供应一方能够提供标准化的产品或服务,保证每一笔交易的信誉。当买卖完成时,双方关系也终止了。对于双方而言,只与业务人员和采购人员有关系,其他部门人员一般不参与双方之间的业务活动。

            \item  长期目标型。

            与供应商保持长期的关系是有好处的,双方有可能为了共同利益而对改进各自的工作感兴趣,并在此基础上建立起超越买卖关系的合作。长期目标型的特征是从长远利益出发,相互配合,不断改进产品质量与服务水平,共同降低成本,提高供应链的竞争力。同时,合作的范围遍及公司内的多个部门。

            例如由于长期合作,可以对供应商提出新的技术要求,而如果供应商目前还没有这种能力,采购方可以对供应商提供技术资金等方面的支持。

            供应商的技术创新和发展也会促进本企业产品改进,所以这样做有利于企业长远利益。比如飞机制造厂商可以对发动机生产厂商提供技术和资金以生产出技术含量更高的发动机,而发送机厂商的技术革新也会促进飞机厂商生产出新型的飞机。

            \item  渗透型。

            这种关系是在长期目标型基础上发展起来的。其管理思想是把对方公司看成自己公司的延伸,是自己的一部分。因此,对对方的关系程度又大大提高了。为了能够参与对方的业务活动,有时会在产权上也采取相应的措施,保证双方派员参与对方的有关业务活动。

            这样做的优点是可以更好地了解对方的情况,供应商可以了解自己的产品在对方是怎样起作用的,所以容易发现改进的方向。而采购方也可以知道供应商是如何制造的,对此可以提出相应的改进要求。

            \item  联盟型。

            联盟型是从供应链角度提出的。它的特点是从更长的纵向链条上管理成员之间的关系。在难度提高的前提下,高秋叶相应提高。另外,由于成员增加,往往需要一个处于供应链上的核心地位的企业出面协调成员之间的关系,它常常被成为“盟主企业”。

            \item  纵向集成型。

            这种形式被认为是最复杂的关系类型,即把供应链上的成员整合起来,像一个企业一样,但各成员是完全独立的企业,决策权属于自己。在这种关系中,要求每个企业在充分了解供应链的目标、要求,充分掌握信息的条件下,自觉做出利于供应链整体利益的决策。

        \end{itemize}

    \end{enumerate.zh}

\subsection {供应商选择方法}

    \begin{enumerate.zh}
    \item 考核选择。所谓考核选择,就是在对供应商充分调查了解的基础上,再进行认真考核、分析和比较而选择供应商的方法。

    首先应该调查了解供应商。供应商调查可以分为初步供应商调查和深入供应商调查。

    初步供应商对象的选择非常简单,选择的基本一句就是其产品的品种规格质量价格水平、生产能力、地理位置、运输条件等。在这些条件合适的供应商当中选择几个,就是初步供应商调查对象。深入供应和送你个调查分为3个阶段。

        \begin{itemize}
            \item  送样检查:通知供应商生产一批样品,随即抽样检查。检查合格进入第二阶段。检查不合格,允许再改进生产一批送检,筹建合格也可进入第二阶段。抽检不合格,供应商螺旋。

            \item  考察生产工艺、质量保障体系和管理体系等生产条件是否合格。合格者初步确定为供应商,到此结束。不合格者进入第三阶段。

            \item  生产条件改进考察。愿意改进并限期达到了改进效果者中选,不愿意改进、或愿意改进但在限期内没有达到改进效果者落选。深入调查结束。可以采用评分等方法进行评定,考察考核合格,就初步确定为企业的供应商。

        \end{itemize}

    初步确定的供应商还要进入试运行阶段进行考察考核,试运行阶段的考察考核更实际、更全面、更严格,因为这是直接面对实际的生产运作。在运作过程中,要进行所有各个评价指标的考核评估,包括产品质量合格率、按时交货率、按时交货量率、交货差错率、交货破损率、价格水平、进货费用水平、信用度、配合度等的考核和评估。在单项考核评估的基础上,还要进行综合评估。综合评估就是把各个指标就行加权平均计算而得的一个综合成绩。

    通过试运阶段,得出各个供应商的综合评估成绩,就可以基本上确定那些供应商可以入选,那些供应商被淘汰了。一般试运阶段达到有些的应该入选,达到一般或较差的供应商,应予以淘汰。

    现在一些企业为了制造供应商之间的竞争机制,创造了一个做法,就是服役选2个或3个供应商,称作A、B角或A、B、C角。A作为主供应商,分配较大的供应量。B角(或再加上C角)作为副供应商,分配较小的供应量。综合成绩为优的供应商担任A角,候补供应商担任B角。在运行一段时间后,如果A角的表现有所退步而B角的表现有所进步的话,则可以把B角提升为A角,而把原来的A角降为B角。这样无形之中造成了A角和B角之间的竞争关系,促使他们竞相改进产品和服务,似的采购企业获得更大的好处(这种现在在生活中比较常见,比如两个处于试用期的员工相互竞争一个岗位,一方面可以提高员工的水平,另一方面是企业获得了更大的利益)。

    从以上可以看出,考核选择供应商是一个较长时间的深入细致的工作。这个工作需要采购管理部门牵头负责、全厂各个部门的人员共同协调才能完成。当供应商选定之后,应当终止试用期,签订正式的供应商关系合同,进入正式运作期,开始了比较稳定的正常的物资供需关系运作。

    \item 招标选择。选择供应商也可以通过招标的方式。招标选择是采购企业采用招标的方式,吸引多个有实力的供应商来投标竞争,然后经过评标小组分析评比而选择最优供应商的方法。

        \begin{itemize}
            \item  招标选择的主要工作。
                \begin{enumerate}
                    \item 要准备一份合适的招标书。包括目标任务,完成任务的要求。
                    \item 要建立一个合适的评标小组和评标规则。
                    \item 要组织好整个招投标活动。
                \end{enumerate}

            \item  在招标活动中,广大供应商的主要工作。
                \begin{enumerate}
                    \item 起草自己的投标书参与投标竞争。
                    \item 参加招标会,进行自己的投标说明和辩论。
                \end{enumerate}

        \end{itemize}

    最后评标小组根据各个供应商的招标书及投标陈述,进行质询、分析和评比,最后得出中标的供应商。这样就最后选定了供应商。

    \end{enumerate.zh}
