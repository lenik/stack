\section {采购询价在实际运作中可能存在的问题}

    \begin{enumerate.zh}
        \item 询价信息公开面较窄,局限在有限少数的供应商,满足与三家的最低要求,排外现象较严重。很多询价项目信息不公开,不但外地供应商无从知晓相关的采购信息,而当地的供应商也会遭遇“信息失灵”,不少询价项目的金额还挺大,但是信息却处于“保密”状态,为采购人或代理机构实施“暗箱操作”提供了极大便利,一些实力雄厚的供应商只能靠边站,“望询兴叹”。

        \item 询价采购出现超范围适用,询价采购一般适用通用、价格变化小、市场货源充足的采购项目,实际工作中则是以采购项目的概算大小来决定是否采用询价方式。询价并不是通用的“灵丹妙药”,而是有确切的适用条件。实际工作中一些代理机构和采购人将寻句i啊作为主要采购方式,错误的认为只要招标搞不了的,就采用询价方式,普遍存在滥用、错用、乱用询价方式问题。代理机构隔三差五搞询价,忙得“不亦乐乎”,被琐碎的事物缠身,采购效率和规模效应低下,还有些人借询价规避招标。

        \item 询价过于倾向报价,忽视对供应商资格性审查和服务质量的考察。询价采购一般规定采购人员根据符合采购需求、质量和服务相等且报价最低的原则确定成交供应商,这是询价采购成交。供应商确定的基本原则。但是不少人片面地认为既然是询价嘛,那么谁价格低谁“中标”,供应商在恶性的“价格战”中获利无几,忽视产品的质量和售后服务。

        \item 确定被询价的供应商主观性和随意性大。被询价对象应由询价小组确定,但是往往被采购人员或代理机构“代劳”,在确定询价对象时会凭个人好恶取舍,主观性较大。询价采购一般规定从符合相应资格条件的供应商名单中确定不少于3家的供应商,一些采购人员和代理机构怕麻烦不愿意邀请过多的供应商,只执行规定的“下限”。某代理机构的询价资料中被询价的供应商一律为3家,还有些询价项目,参与的供应商只有2家,甚至只有1家。询价一般不设询价保证金。

        \item 询价采购的文件过于单薄,往往就是一张报价表,基本的合同条款也会被省略。法律规定询价采购应制作询价通知书,在一些询价采购活动中,询价方一般不会制作询价通知书,多采取电话通知方式,即使制作询价通知书,内容也不够完整,且规范性较差,价格构成、评标成交标准、保证金、合同条款等关键性的内容表述不全,影响了询价的公正性,不少询价采购结束后采购双方不签合同,权利义务不明确,引发不必要的纠纷。

        \item 询价小组组成的专业化水准很低,更多的是专家人数根本无法达到2/3。试想让“外行”来从事询价,确实让人不放心。

        \item 采购活动的后续工作比较薄弱。不搞询价采购活动记录,不现场公布询价结果,询价方式随意性大。一些企业尝试采用电话询价、传真报价、网上竞价等方式搞询价采购,尽管这些有便利之处,但不宜过多地使用。询价采购一般规定在询价过程中供应商一次报出不得更改的价格,采用非现场方式搞询价寻在舞弊漏洞,采购方有机会随意更改任何一家供应商的报价,或者给有关供应商“通风报信”。

    \end{enumerate.zh}
