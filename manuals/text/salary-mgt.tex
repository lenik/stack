第四节 员工蕲管埋

    通过薪酬调查,企业可以了解劳动力市场的需求状况,掌握各种类型人才的价格行情,从而制定正确的薪酬策略。有效地控制企业的人力成本。

一.员工薪酬调查

    1.薪酬调查的原则

    (1)协商原则。

    由于薪酬管理政策及薪酬数据在许多企业属于商业秘密,不愿意让其他企,所以在进行薪酬凋查时,要由企业人力资源部与对方人力资源部,或经理与对方总经理直接进行联系,本着双方相互交流的精神,协商调查事宜。

    (2)资料准确性原则。
    由于很多企业都对本企业的薪酬情况守口如瓶,所以,有些薪酬信息很可能是道听途说得来的。这些信息往住不全面,有些甚至是错误的,准确性较差。另外,在取得某职位的薪酬水平的同时,要比较一下该职位的职责是否与本企业的职责完全相同,不要因为职位名称相同就误以为工作内容和工作能力要求也一定相同。

    (3)资料随时更新原则

    随着市场经济的发展和人力醐市场的完善,人力资源的市场变动会越来越频繁,企业的薪酬水平也会随企业的效益和市场中人力资源的供需状况而变化,所以薪酬调查的资料要随时注意更新,如果一直沿用以前的调查数掘很可能会作出错误的判断。

    2. 薪酬调查的渠道

    进行薪酬凋查,通常有以下几种渠道:

    (1) 企业之间的调查。

    由于我国的薪酬调查系统和服务还没有完善,所以最可靠和最经济的薪酬调查渠道还是企业之间的相互调查。相关企业的人力资源部可以采取联合调查的形式,共享相互之间的薪酬信息。

    这种相互调查是一种正是的调查,也是双方受益的调查。调查可以采取座谈会、问卷调查等多种形式。

    (2)委托专业机构惊醒调查。

    现在,在北京、上海和沿海一些城市均设有提供薪酬调查的管理顾问公司和人才服务公司。通过这些专业机构,会较少人力资源部门的工作量,省去了企业之间的协调费用,但需要想委托的专业机构支付一定的费用。

    (3)从公开的信息中了解。

    有些企业在发布招聘广告时会写上薪酬待遇,调查人员稍加留意就可以了解到这些信息;另外,某些城市的人才交流部门也会定期发布一些职位的薪酬参考信息。同一职位的薪酬,一般分为高、中、低三挡,,但由于它覆盖面广,涵盖的内容多,所以对有些企业并没有意义。

    通过与其他企业调入本企业的应聘人员进行交谈,也可以了解一些该企业的薪酬情况。

    3.薪酬调查的实施。

    (1)确定调查目的。

    人力资源经理首先应该弄清楚调查的目的和调查的结果的用途,再开始制定调查计划。一般而言,调查的结果可以为以下工作提供依据:整体薪酬水平的调整,薪酬结果的调整,薪酬晋升政策的调整,具体职位薪酬水平的调整等。

    (2)选定相关市场。

    选定相关市场对于薪酬调查非常重要,因为不同行业可能有很多不同的薪酬结构,故很难互相比较,有时即使职位名称相同,工作的内容及植物也可能有很大的分别。相关劳动力市场通常是企业获取人才的场所。

    (3)确定调查范围。

    调查范围包括企业的类型以及调查对象的数目等,这些通常要根据调查目的而定。一般情况下,要选择那些与企业处于同一行业,在同一劳动力市场上有竞争行为的、实力超过自己或大致相同的企业。

    (4)选择调查方式。

    确定了调查的目的和调查范围,就可以选择调查方式。具体的调查可采用基准职位比较法、基准职位转换法、工作分类法、点数比较法、全球定位法、两端定位法。

    (5)收集和分析资料
    初步审阅调查所得的资料,若无重大的错误及矛盾,可将资料输入电脑,进行分析,计算出每一职位的最高和最低薪酬,加权平均或算数平均额、中位数,然后再将从工作评价中获得的职位等级与薪酬调查中所得的对应薪酬平均数或中位数,回程市场薪酬分布图。将分析后的资料,归类或编制成图表,作为确立薪酬水平的依据。

    二、员工薪酬设计

    在进行过薪酬调查后,就要进行员工的薪酬设计。

    1.薪酬设计原则

    一个科学合理的薪酬系统在设计时应遵循一下原则:

    (1)一致原则。

    一致原则也称公平原则,是之薪酬结构与组织层次、职位设计之间形成的对等、协调关系。具体而言,在职位薪酬结构的设计中,需要贯彻与职位价值相一致的原则;在技能薪酬机构的设计中,需要贯彻与员工能力价值相一致的原则。因为公平是薪酬系统的基础,只有在员工认为薪酬系统是公平的前提下,才可能产生认同感和满意度,才可能产生薪酬的激励作用。

    (2)激励原则。

    企业想要获得真正具有竞争力的优秀人才,必须制定出一套对人才具有吸引力并在行业中具有竞争力的薪酬系统。如果企业制定的薪资水平太低,那么必然在与其他企业的人才竞争中处于劣势地位,甚至连本企业的优秀人才也会流失。那么,什么样的薪酬系统才具有竞争力呢?除较高的薪资水平和正确的薪酬价值取向外,灵活多元化的薪酬结构也越来越引起人们的兴趣。

    (3)激励原则。

    一个科学合理的薪酬系统对员工的激励是最持久也是最根本的,因为科学合理的薪酬系统解决了人力资源所有问题中最基本的分配问题。

    (4)经济原则。

    经济原则从表面上看与竞争原则和激励原则是相互对立的————竟争原则和激励原则提供较高的薪酬水平,而经济原则提倡较低的薪酬水平,但实际上三者并不矛盾,而是统一的。

    (5)合法原则。

    薪酬系统的合法性是必不可少的,合法是建立在遵守围家相关政策,法律法规和企业一系列管理制度基础之上的合法。如果企业的薪酬系统与现行的国家政策和法律法规,企业管理制度不相符合,企业应该迅速地进行改进,以具有台法性。

    2.选择薪酬设计方法

    (1)基准职位定价法

    即利用基准职位的市场薪酬水平和基准职位的工作评价结果建立薪酬政策,进而确定薪酬结构。基准职位定价法能够较好地兼顾薪酬的外部竟争性和内部一致性原则,在比较规范的,市场相关性较强的企业薪酬结构中应用比较广泛。

    (2)直接定价法。

    即企业内所有职位的薪酬完全由外部市场决定,根据外部市场各职位的薪酬水平,直接建立企业内部的薪酬结构。这是一座哦好难过完全市场导向型薪酬结构的设计方法,体现了外部竞争性,但忽略了内部一致性,比较适合于市场驱动型企业,其员工的获取及薪酬水平的确定直接与市场挂钩。

    (3)设定工资调整法。

    即企业根据经营状况自行设定基准职位的薪酬标准,然后再跟据工作评价结果设计薪酬结构,这种薪酬结构的设计比较重视内部一致性原则,但忽略了外部竞争性,比较适合与劳动力市场接轨程度较低的企业。

    (4)当前工资调整法。

    即在当前工资的基础上对原企业薪酬结构进行调整或再设计。薪酬结构调整的本质是对员工利益的再分配,这种凋整将服从于企业内部管理的需要。

    3. 薪酬设计步骤

    企业薪酬方案设计的目的,就是在尽可能降低成本的情况下,获得人才、羸得利润,充分发挥薪酬促进公平、激励和吸引人才、留住人才的作用。

    (1)制定薪酬政策统。

    薪酬政策线是指薪酬中值点所形成的趋势线,它的主要作用是确定企业薪酬的总体趋势。

    (2)确定薪酬等级。

    即一个薪酬结构内部划分多少等级,最高等级与最低等级之间的薪酬差,相邻薪酬等级的级差等。

    (3)确定薪酬等级范围。

    即依照每个薪酬中值点确定最高值,最低值及不同等级的薪酬标准交叉或重叠度。

    (4)调整薪酬结构。

    即根据企业管理的其他特殊要求,对薪酬结构进行局部和定期的调整。

