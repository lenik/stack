\section {签订采购合同的要点}

    企业在与供应商签订采购合同时,应注意以下几点。

    \begin{enumerate}

        \item 项目相物料名称、规格、数量、单价、总价、交货日期及地点、需与请购单及决策单所列相符。

        \item 付款办法按照买卖双方约定的条件付款,一般付款的方式可以分为以下两种:
            \begin{itemize}
                \item  一次性付款。约定供应商将物料运抵企业,经企业人员验收合格后一次付清。
                \item  分期付款。依金额大小、及供应期间的长短分为几期,例如:
                    \begin{itemize}
                        \item  第一期为预订期(订金),于签订合约并办理保证经认可后给付,其数额以不超过采购总价 30\%为限。
                        \item 第二期款,依供应进度至一半或物料运抵企业时付40\%。
                        \item 第三期款(即尾款),物料运抵企业经验收合格后给付,但末期款应不少于工程部价的 10\%为宜。
                \end{itemize}
            \end{itemize}
        \item 廷期罚款于合同书中约定,供应商须配合企业生产进度物料最迟在几月几日以前,全部送达交验,除因天灾及不抗力的事故外。倘逾期,每天供应商应赔偿企业采购金额干分之几的违约金。

        \item 解约办法于合同书中约定,是供应商不能保持进度或不能符合规格要求时的解约办法,以保障企业权益。

        \item 验收与保修于合同书中约定,供应商物料送交企业之后,须另立保修,自验收日起保修一年(或几年)。在保修期内如有因劣质无聊而导致损坏者,供应商与15日内无偿修复;否则企业得另请修理,其所有费用由供应商负责偿付。

        \item 保证责任于合同书中的约定,供应商应找实力雄厚的企业担保供应商履行合同所订明的一切规定,保证期间包含无聊运抵企业经验收至保修期满为止。保证人应负责赔偿企业因供应商违约所蒙受的损失。

        \item 其他附加条款视物料的性质与需要而增列。例如:定制马克、继电器及变压器等类的物料,须于合同中约定。出厂前供应商应会同企业技术人员实施各项性能试验,合格后方可交货。
    \end{enumerate}
