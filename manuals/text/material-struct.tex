\subsection {产品是由众多 “单层结构” 组成的}

    物料和由物料组成的 "单层结构” 是产品结构的基本单元。

    任何一个产品都是由若干个 "单层结构” 组成的,单层结构中的上层物料,我们称为 “母件 (parent)”,下层物料称为 “子件(component)"。有些软件采用 “父件” 的叫法,但就一般公司的组成来讲,通常称母公司和子公司,而不用 "父公司”。我们认为母子的血缘关系较之父子更为明确,符合信息数据准确定义的要求,所以采用 "母件”的叫法。

    单层结构是由一个母件和从属于母件的一个或多个子件组成的。如果对应设计图纸,母件指的是组装图上的装配件,子件是零件明细表中众多的零部件。不少企业在自行开发管理软件时,由于没有 "单层结构” 的概念,对所有产品的描述都从最顶层一直扎到最底层,相同的“单层结构”在各个产品文件中多次重复出现,数据的冗余量很大,极不合理。

    以图 \ref{fig:bomBySimples} 的产品结构为例,X 作为最上层的母件是一个要销售出厂的产品,它由A、B、C、D 4个子件组成。件X同件A、B、C、D组成一个"单层结构”,在MRP系统中用 "单层物料单 (Sing1e level B0M)” 报表格式表示。图 13.1 中,件 B 对应于 X来讲是子件,但它对应于件E、F来讲又是母件,并一起组成一个第2个层次的单层结构。同理,件E同件G、H、I; 件 D同件 I又组成位于不同层次的单层结构。 任何一个产品都是由这样众多 "单层结构” 组成的。

    \begin{figure}[!htbp]
        \centering
        \begin{tikzpicture}[
                    part/.style={draw, thick, fill=white, drop shadow, inner sep=2mm,
                        font=\bfseries, },
                    group/.style={draw, ellipse, inner sep=0 },
                    ]
            \node[part] (X) {X}
                    [edge from parent fork down]
                child { node[part] (A) {A} }
                child { node[part, anchor=north] (B) {B}
                    child { node[part, anchor=north, xshift=-1em] (E) {E}
                        child { node[part] (G) {G} }
                        child { node[part] (H) {H} }
                        child { node[part] (I1) {I} } }
                    child { node[part, xshift=1em] (F) {F} } }
                child { node[part] (C) {C} }
                child { node[part, anchor=north] (D) {D}
                    child { node[part] (I2) {I} } }
                ;
            \node[group, fit=(X)(A)(B)(C)(D)] {};
            \node[group, fit=(B)(E)(F)] {};
            \node[group, fit=(D)(I2)] {};
            \node[group, fit=(E)(G)(H)(I1)] {};
        \end{tikzpicture}
        \caption{产品是由众多“单层结构”组成的} \label{fig:bomBySimples}
    \end{figure}

    母件同子件的关系可以是一对一,也可以是一对多,但必须是惟一的。如果众多子件中有品种或数量的差异,就是另一个不同的单层结构,是不同的母件,应有不同的母件代码。如果子件有多个可选品种,则按模块化产品结构处理,母件是虚拟件,子件是可选件。

    母件同子件之间的连线对加工件来讲是加工流程(工艺路线)和加工周期,对采购件则是入库前的采购流程和采购周期(供应商的制造周期、 运输、 通关、 检验等),单层结构上每一项方框代表的是物料的完工状态,可以入库存储,而不是正在工序之间流动、处于尚未成形的状态。

    只有明白产品是由多个单层结构组成的道埋,才能处理好“借用”、"通用”、"选用”之间的区别,正确编制物料清单。在 MRP 系统中,对每项 "单层结构” 的文件只须建立一次,所有借用这个单层结构的产品就可以共享。

    建立物料清单是从建立一个个反映 “单层结构” 的 "单层物料单” 开始的,系统会根据各单层结构母件同子件的相互关系,自动逐层地把所有相关的单层结构挂接起来,最后形成整个产品的产品结构。
