\section {公司采购合同的管理}

    采购合同的管理应当做好一下几方面的工作。

    \begin{enumerate}
        \item 加强对公司采购合同签订的管理

        加强对采购合同签订的管理,一是要对签订合同的准备工作加强管理,在签订合同之前,应当认真研究市场要和货源情况掌握企业的经营情况、库存情况和合同对方单位的情况,一句企业的销售任务收集各方面的信息,为签订合同、确定合同条款提供信息依据。另一方面,是要对签订合同过程加强管理,在签订合同时,按照有关的合同法规规定的要求严格审查,使签订的合同合理合法。

        \item 建立台同管理机构和管理制度

        保证合同的履行企业应当设置关门机构或人员,建立合同登记、回报检查制度,以同意保管合同、统一监督和检查合同的执行情况,及时发现问题,采取措施,处理违约,提出索赔,解决纠纷,保证合同的履行。同时,可以加强与合同对方的联系,密切双方的协作,以利于合同的实现。

        \item 处理好合同纠纷

        当企业的经济合同发生纠纷时,双方当事人可协商解决。协商不成时,企业可以向国家工商行政管理部门申请调解或仲裁,也可以直接向法院起诉。

        \item 信守合同,树立企业良好形象

        合同履行情况的好怀,不仅关系到企业经营活动的顺利进行,而且也关系到企业的声誉和形象。因此,加强合同管理,有利于树立良好的企业形象。

        \item 来购合同的归档管理

        按照商晶种类分类归档;采购合同复印件与供应商档案;采购台同原件单独归档、统一管理。

        \item 采购台同的跟进

        以采购台同的条款检查供应商的表现;采购台同期限的预警;在采购台同期限前一个月对供应商的表现做综合评估,以确定是否续约;可在电脑系统中设定合同期限预警程序。

    \end{enumerate}

