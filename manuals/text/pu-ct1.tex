\section {签订采购合同时的注意事项}

    在签订采购台同时特别要指出的有如下3个方面。

    \begin{enumerate}
        \item 《合同法》第 145 条规定,“当事人没有约定交付地点或者约定不明确,标的物需要运输的,出卖人将标的物交付给第一承运人后,标的物的毁损、灭失的风险由买受人承担。” 由此看来,在签订采购合同时,一定要注明具体的交付地点避免不必要的风险承担。

        \item 关于在涂运输标的物的风险转移。《台同法》第 144 条规定,“出卖人出卖交由承运人运输的在途标的物,除当事人另有约定的以外,毁损、灭失的凤险自合同成立时由买受人承担。本着对采购方(即买受人)有利的角度出发,在签订合同时,应明确约定标的物在运输途中出现毁灭、灭失由出卖人负责,把可能出现的风险降到最低。

        \item 违约金的问题。《合同法》第 114 条规定,违约金和违约行为造成的损失有密切联系,若违约金低于损失,可以请求法院予以增加; 反之,可以要求法院予你减少。在实际签订合同中,应本着诚实信用、公平的原则,最好不约定违约金。如需约定,双方要制定合理的违约金范围,当发生违约事件时,不至于严重损害一方权益。
   \end{enumerate}
