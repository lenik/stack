2.采购管理

    采购管理是指为保障企业物料供应而对企业进行的采购活动的计划、组织、指挥、协调和控制的管理活动。采购管理活动具体包括制定采购计划,对采购活动、采购人员、采购资金、运储进行管理,并进行采购评价和采购监控。也包括建立采购管理组织、采购管理机构、采购基础建设等。

    采购与采购管理的区别主要在于:前者是一种作业活动,是为完成指定的采购任务而进行具体操作的活动;一般是由采购员承担;而采购管理是管理活动,是面向整个企业的,不但面向全体采购员,而且也面向企业组织其他人员(进行有关采购的协调配合工作),一般由企业的采购科(部、处)长或供应科(部、处)长或企业副总来承担。


    (1)采购管理的职能。采购管理工作的主要职能体现在3个方面。
    ® 保障供应是其首要职能。采购要实现对整个企业的物资供应,保障企业生产和生活的正常进饨企业生产需要原材料、零配件、机器设备和工具,生产线一开动,这些东西必、须样样到位,缺少哪一样,生产线就开动不起来。

    ® 供应链管理职能,传统的采购管理观念,一般把保障供应看成是采购管理唯一的职能。但是随着社会的发展,特别是20世纪90年代供应链的思想出现以后,人们对采购管理的只能有了进一步的认识,即认为采购管理应当还有第二个重要职能,那就是供应链的管理,特别是上游供应链的管理。

    ® 资源市场信息管理。在企业中,只有采购管理部门天天和资源市场打交道,采购管理部门除了是企业和资源市场的物资输入窗口外,同时也是企业和资源市场的信息接口。所以采购管理除了保障物资供应、建立友好的供应商关系之外,还要随时掌握市场信息,并反馈到企业管理层,为企业的经营决策提供即使、有力的支持。

    (2) 采购管理的重要性。企业的基本职能是为社会提供产品和服务。这个基本职能可以分解成物资销售、物资生产和物资采购3个子职能。其中采购对于企业来讲至关重要,只要表现在以下几个方面。

    ® 物资采购为企业保障供应、维持正常生产、降低缺货风险创造条件。很显然,物资供应是物资生产的前提条件。生产所需要的原材料、设备和工具都要由物资采购来提供。没有采购就没有生产条件,没有物资供应就不可能进行生产。

    ® 物资采购供应的物资的质量好坏直接决定了本企业生产的产品质量的好坏。能不能生产出合格的产品,取决于物资采购所提供的原材料及设备工具的质量的好坏。

    ® 物资采购的成本构成了物资生产成本的主体部份,其中包括采购费用、进货费用、仓储费用、流动资金占用费用以及管理费用等。物资采购的成本太高,将会大大降低企业生产的经济效益,甚至亏损,致使物资生产成为没有意义的事情。

    ® 物资采购是企业和资源市场的关系接口,是企业外部供应链的操作点。只有通过物资采购部门人员与供应商的接触和业务交流,才能把企业与供应商联结起来,形成一种相互支持、相互配合的关系。在条件成熟以后,可以组织成一种供应链关系,从而使企业在管理方面、效益方面都登上一个崭新的台阶。

    ® 物资采购是企业与市场信息的接口,物资采购人员直接和资源市场打交道。资源市场和销售市场是交融混杂在一起的,都处在大市场之中。所以,物资采购人员比较容易获得市场信息,是企业的市场信息接口,可以为企业及时提供各种各样的市场信息,供企业进行管理决策。

    ® 物资采购是企业科学管理的开端。企业物资供应是直接和生产相联系的。物资供应模式往往会在很大程度上影响生产模式。例如,实行准时采购制度,则企业的生产方式就会改成看板方式,企业的生产流程,搬运方式也都要做很大的变革;如果要实行供应链采购则需要实行供应商掌握库存、多频次、少批量补充货物的方式,这也将大大改变企业的生产方式和搬运方式。所以 ,物资采购部门每提供中种科学的物
资采购供应模式必然会要求生产方式、物料搬运方式都做相应的变动,以共同构成一种科学管理模式,可见,这种科学管理摸式是以物资采购供应作为开端而运作起来的。

3.采购管理作业流程与工作内容

    采购管理的工作过程如图 1-1所示。
    <image />

    (1)采购管理组织。采购管埋组织是采购管理最基本的组成部分。为了搞好企业复杂多的采购管理工作,需要有一个合理的管理机制和一个精悍的管理组织机构,要有一些能千的管理人员和操作人员。

    (2)需求分析。需求分析就是要弄清楚企业需要采购什么品种、需要采购多少、什么时候需要什么品种、需要多少等问题。作为全企业的物资采购供应部门,应当掌握全企业的物资需求情况并制定物料需求计划,从而为制定出科学合理的采购订货计划做准备。

    (3)资源市场分析。资源市场分析就是根据企业所需求的物资品种,分析资源市场的情况,包括资源分布情况、供应商情况、品种质量、价格情况、交通运输情况等。资源市场分析的重点是供应商分析和品种分析。分析的目的,是为制定采购订货计划做准备。

    (4)制定采购定制计划。制定采购定制计划是根据需求品种情况和供应商的情况,制定出切实可行的采购订货计划,包括选定供应商、供应品种、具体订货策略、运输进货策略以及具体的实施进度计划等。具体地解决什么时候订货、订购什么、订多少、向谁订、怎样订、怎样进货、怎样支付等具体的计划问题,为整个采购订货规划一个蓝图。

    (5)采购计划实施。采购计划实施就是把上面制订的采购订货计划分配落实到人,根据既定的进度实施,具体包括联系指定的供应商,进行贸易谈判、签订订货合同、运输进货、到货验收入库、支付货款以及善后处理等。通过这样的具体活动,最后完成了一次完整的采购活动。

    (6)采购评估与分析。采购评估与分析就是在一次次采购完成以后对这次采购的评估,或月末、季末、年末对一定时期内的采购活动的总结评估。主要在评估采购活动的效果、总结经验教训、找出问题、提出改进方法等。通过总结评估,可以肯定成绩、发现问题、制定措施、 改进工作,不断提高采购管理水平。

    (7)采购监控。采购监控是指对采购活动进行的监控活动动,包括对采购有关人员、采购资金、采购事物活动的监控。

    (8)采购基础工作。采购基础工作是指为建立科学、有效的采购系统,需要进行的一些基础建设工作,包括管理基础工作、软件基础工作及硬件基础工作。

    4.采购作业曼本原则

    经过不断的探索与总结发现,采购必须要围绕价格、质量、需求时间、地点等5要素进行实施,才会取得不错的效果。这也就是我们平时所说的“5R”原则。适时(Right time)、适质(right qua1ity)、适量(right quantity)、适价(right price)、适地(right place)。

    下面就商品采购的5R原则进行介绍。

    (1) 适价(Right Price)。价格是企业领导在采购环节最关心的要点之一,也是采购人员最敏感的焦点。因此,采购人员不得不把相当多的时间与精力放在与供应商的“砍价”上。物品的价格与该物品的种类、是否长期购买、时候大量购买及市场供求关系有关。同时物品的价格还与采购人员对该物品市场状况熟悉的情况也有关系,如果采购人员未能把我市场脉搏,供应商在报价时就可能“蒙骗”采购人员。一个合适的价格往往要经过以下几个环节的努力才能获得。

    ®多渠道获得报价。 这不仅要求有渠道供应商报价,还应该要求一些新供应商报价。企业与某些现有供应商的合作可能已达数年之久,但他们的报价未必优惠。获得多渠道的报价后,企业就会对该物品的市场价有一个大概的了解,并进行比较。

    ®比价。俗话说 “货比三家”,因为专业采购所买的东西可能是一台价值百万或干万元的设备,或采购金额达千万元的零部件,这就要求采购人员必须谨慎行事。由于供应商的报价单中所包含的条件往往不同,故采购人员必须将不同供应商报价中的条件转化一致后,只有这样才能获得真是可信的比较结果。

    ®议价 经过比价环节后,筛选出价格最适的2\~3个报价环节。随着进一步的深入沟通,不仅可以将详细的采购要求传达绐供应商,而且可进一步"杀价",供应商的第一次报价往往含有“水分”。但是,如果采购物品为卖方市场,即使面对面地与供应商一家,最后所获得的实际效果可能要比预期的要低。

    ® 定价: 经过上述 3个环节后,买卖双方均可接受的价格便作为日后的正式采购价,一般需保持2\~3个供应商的报价。这两3个供应商的价格可能相同,也可能不同。

    (2)适质(Right Quality) 一个不重视品质的企业在今天激的的市场竞争环境中跟本无法立足。一个优秀的采购人员不仅要做一个精明的商人,同时也要在一定程度上扮演管理人员的角色,在日常的采购工作中要安排部分时间去推动供应商改善、稳定物品品质。

    采购物品品质达不到使用要求的严重后果是显而易见的。来料品质不良,往往导致企业内部相关人员花费大的时间与精力去处理,会增加大量的管理费用;来料品质不良,往住在重检、挑选上花费额外的时间和精力,造成检验费用增加;来料品质不良,导致生产线返工增多,降低生产质量,降低生产效率。因来料品质不良而导致生产计划推迟进行,有可能引起不能按承诺的时间想客户交货,会降低客户对企业的信任程度;若因来料品质不良引起客户退货,有可能令企业蒙受损失,严重的还会丢失客户。

    (3)适时(Right Time)。企业已安排好生产计划,若原材料未能如期达到,往往会引起企业内混乱,即产生停工待料。当产品不能按计划出货时,会引起客户强烈不满。若原材料提前太多时间买回来放在仓库里等着生产,又会造成库存过多,大量积压采购资金,这是企业很忌讳的事情。故采购人员要扮演协调者与监督者的角色,去促使供应丹商按预定时间交货。对某些企业来讲,交货时机很重要。

    (4)适量(Right Quantity)。批量采购虽有可能获得数量折扣,但会积压采购资金,太少又不能满足生产需要,故合理确定采购数量相当关键。一般按经济订购量采购,采购人员不仅要监督供应商准时交货,还要强调按订单数量交货。

    (5)适地 (Right Place)。天时不如地利,企业往往容易在与距离较近的供应商的合作中取得主动,企业咋选择试点供应商时最好选择近距离供应商来实施。近距离供货不仅似的买卖双方沟通更加方便,处理事务更加快捷,亦可降低采购物流成本。

    总之,只有综合考虑才能实现最佳采购,这需要采购实施人员在长期的实际操作中积累经验。
