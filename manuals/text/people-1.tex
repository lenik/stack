\documentclass[CJK]{z-article}
\begin{document}
    一、组织结构设计

    1.什么是组织结构设计

    组织结构是反映组织内部构成部分和各个部分间所确定的关系的组织形式。组织结构设计,实质上是一个组织变革的过程,它是把企业的任务、流程、权利和责任重新进行有效组合和协调的一种活动。其目的就是通过组织结构设计, 大幅度地提高企业的运行效率和经济效益。

    2.组织结构设计原则

    (1)组织的目标性。使组织内各部门的职能得到充分发挥,达到组织整体经营目标和各部门的分目标。

    (2)组织的成长性。考虑组织的经营业绩与持续成长。

    (3)组织的稳定性。随着组织成长而逐步调整组织结构,但应考虑频繁的权责、流程变更将使员工的信心动摇。

    (4)组织的简单性。简单的组织结构将有助于内部协调与人力分配。

    (5)组织的弹性。既保持基本形态,又能配合各种环境和条件的变化。

    (6)组织的均衡性。各部门业务量的均衡,将有助于内部的平衡与分工。

    (7)指挥的统一性。如一人同时接受两人以上的管理,将使其产生无所适从的感觉。

    (8)权责明确化。权责或职责不清将使工作发生重复或遗漏,推诿现象,易便员工产生挫折感。

    (9)作业制度化。明确的制度与标准作业可减少摸索时间, 提高作业效率。

    3.组织结构设计应注意的问题

    (1)企业组织结构的动态管理

    企业的组织结构不是一成不变的,它应根据市场和客户的需要,随时实施动态的变革,使企业永远充满活力。企业的组织结构必须实施动态管理,才能使企业在激烈的市场竞争中永远立于不败之地。

  (2)组织结构设计没有最好,只有最合适。

    很多企业都在追求最好的组织结构设计,实际上组织结构设计没有现成的“菜单”。组织结构设计没有最妩 只有最合适。

    组织结构设计的四个最合适的标准

    1.市场是检验组织结构设计好坏的第一标准。

    2.客户的需要是检验组织结构设计好坏的第二标准。

    3.操作是否顺畅是检验组织结构设计好坏的第三标准。

    4.运行效率是检验组织结构设计好坏的第四标准。

  (3)恰当地处理“集权”和“分权”的关系。

    在组织结构设计中,要恰当地处理好“集权”和“分权”的关系。权力过于集中虽然有助于防范各种经济风险,决策效率高,但会影响下级的积极性,有时决策人物变更或因出差等原因暂时离开时,会影响基层的工作效率。当权力分散时, 虽然下级的积极性较高,基层工作效率也比较高,但是企业容易发生各种经济风险, 企业高层的决策效率因此会降低。

    比较好的办法是适度分权,即企业的决策权力相对集中后,对下级单位或个人采取适度分权(或称为有限度分权)的方法。例如, 在对外招待费的使用上, 部门经理有500元以下的权力, 副总或总监有500\~800元的权力,超过权限的则需要请示总经现,既不要人人都可以随便进行对外招侍,也不要使下级一点权力都没有, 每次对外招待都要事先请示总经理。

二、 岗位设计
    岗位也称职位。在组织中,在一定的时间内, 当由一名员工承担若干项任务, 并具有一定的职务、责任和权限时,就构成一个岗位。
    1.岗位分类
    一般情况下, 企业的岗位分为表1-1中所示的几种。
    \begin{table}
		\centering
        \caption{岗位分类表}
        \begin{tabular}{c|c}
            \hline  岗位名称&岗位内容 \\
            \hline  生产岗位&从事制造、安装、维护及为制造做辅助工作的岗位 \\
            \hline  执行岗位&从事行政工作或服务性工作的岗位 \\
            \hline  专业岗位&从事各类专业技术工作的岗位 \\
            \hline  监督岗位&从事监督、监察企业各项工作的岗位 \\
            \hline  管理岗位&从事部门、科室管理工作的岗位 \\
            \hline  决策岗位&主要指企业的高级管理层 \\
            \hline
        \end{tabular}
    \end{table}

    2.岗位设置的原则

    (1)岗位设置应符合最低数量原则。

    设置的岗位数量要尽可能少,目的是使所有的工作尽可能集中,不要过分分散。从经济角度来说,不必投人很多人工成本。每一个人,每一个岗位的工作人员都应承担更多的责任。

    (2)要求所有岗位实现最有效的配合。

    设置岗位的时侯,应对承担的责任进行划分。一般区分为主要,部分和支持三类责任,并以此确定配合关系。支持责任是指责任很轻,只协助他人工作而应负的那部分责任。每个人的主要、部分和支持责任一一定要划分清楚。

    (3)每个岗位能否在组织中发挥最积极的作用。

    岗位设置的第三个原则是每个岗位能否在整个组织中发挥最积极的作用。每一个岗位都要有相应的主责,有些则是支持性工作。例如基层员工要有2\~5项主责, 如果工作分工里没有主责,只是负部分责任或支持性的责任,那么这个员工的积极性就会受到影响, 会认为自己是跑龙套肌,只给别人摇旗呐喊。

    (4)每个岗位与其他岗位的关系是否协调。

    协调是指岗位之间的责任不能交又,也不存在空白。例如,应避免某一责任张先生负主责,李先生也负主责,两个人分不清到底谁应负主责,出了事谁负主要责任,在工作中谁负责管理。另外,一项职能如果没有人负主贵,就是岗位职责出现了空白。

    如果某一项工作,既有负主责的人,又有配合的人,还有做支持性工作的人. 就表示冈位之间配合得很好。

    (5)岗位设置是否符台经济、科学和系统化的原则。

    岗位设置如果体现了经济、科学、合理和系统化的原则,那么岗位设置对企业的经济效益将起到积极的作用。企业都在追求自己的经济效益对于人工成本的控制也是企业控制成本的重要组成部分。如果岗位设置得特别多,参与这项工作的人就多,企业支付的费用相应增加,这不符合经济原则。如果岗位设置过少, 某些事情无人管理, 或者某一个岗位的员工负担特别重而产生怨气, 这项工作就做不好,所以岗位设置要体现经济化原则,要符合科学原理。另外, 企业规范化管理体系是一个大的完整的系统,岗位设置要和组织结构设计、职能的分配相吻合,要符合系统化原则。同时, 岗位设置也为岗位描述岗位评价和薪酬福利体系设计提供支持,成为统刑整体。

    3.岗位评价内容

    岗位评价工作包括以下内容

    (1)按工作性质,将企业的全部岗位分为若干太类。

    (2)收集汇总有关岗位的信息和资料。

    (3)建立专门组织, 配备专门人员, 系统掌握岗位评价的基本理论和实施办法。

    (4)找出与岗位有直接联系的,密切相关的各种因素。

    (5)规定统一的评价指标和衡量标准,设计各种问卷和表格,打分评价。

    (6)总结经验, 及时调整; 评价小组对评价标准的掌握可能会有偏差,应分析、寻找其形成的原因,然后移交绐负责薪酬设计的部门作为基础资料。

    三 、 职位说明书编写
    1.职位说明书内容
    (1) 职位墓本信息。

    职位基本信息也称工作标志,包括职位名称、所在部门、直接上级、定员、部门编码、职位编码等。

    (2)工作目标与职责。

    重点描述从事该职位的工作所要达到的工作目标, 以及该职位的主要职责和权限等。

    (3)工作内容。

    这是最主要的内容,详细描述该职位所从事的具体工作,并全面、详尽地写出完成工作目标所要做的每一项工作,包括每项工作的阐述 流程,工作联系和工作权限。还可以同时描述每项工作的环境和条件, 以及在不同阶段所使用的不同的工具和设备。

    (4)工作的日寸间特征。

    反映该职位通常表现的工作肘间特征,例如, 在流水线上可能需要三班倒, 在高科技企业需要经常加班,建筑施工人员要经常外出作业,一般管理人员则正常上下班等。

    (5)工作完成结果及建议考核标准。

    反映该职位完成的标准,以及如何根据完成工作情况进行考核, 具体内容通常与该鱼且织的考核制度相结合。

    (6)教育背景。

    从事该职位目前应具有的最低学历要求。在确定教肓背景时应考虑如果让一位新员工任职,他最低应是什么学历,而不一定是当前在职员工的学历。

    (7)工作经历。

    反映从事该职位之前,应具有的最起码的工作经验要求。一般包括两方面, 一是专业经历要求,即相关的知识经验背景; 另一个可能需要的是本组织内部的工作经历要求,尤其针对组织中的中、高层管理职位,在担任这些管理职位之前,通常要求在组织其他职位上工作过或对其他职位的工作有一定了解, 才可能胜任该职位。

    (8)专业技能、证书与其他能力。

    主要反映从事该职位应具有的基本技术和能力。某些职位对专业技能要求较高, 没有此项专业技能就无法开展工作,比如“投资部主管”, 如果没有证券、会计等相关基础知识以及国家金融政策,法规知识, 就根本无法开展工作。而相比之下, 另一些职位则对某些能力要求更为明确, 比如“市场部主管”职位,要求具有较强的公关能力,语言表达能力。

    (9) 专门培训。

    反映从事该职位前,应进行的基本的专业培训。具体是指员工在具备了教育水平、工作经历、技能要求之后, 还必须经过哪些方面的培训。

    (10)体能要求。

    对于体力劳动型的工作,这项要求非常重要。

    2.职位说明书编制应汪意事顶

    职位说明书一般由人力资源部统一制作,归档并管理。然而, 职位说明书的内容并不是一成不变的。实际工作中, 组织内经常出现职位增加或撒销的情况,更普遍的情形是某项工作的职责和内容出现变动。每一次工作信息的变化都应及时记录在案,并迅速调整职位说明书。在这种情况下, 一般由职位所在部门负责人向人力资源部提出申请,并填写标准的职位说明书修改表,由人力资源部据此对职位说明书作出相应的修改。

\end{document}
