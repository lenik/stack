\section {供应商者核}

    这里讲的供应商考核,主要是指同供应商签订正式合同以后正式运作期间,对供应商整个运作活动的全面考核,这种考核应当比试运作期间更全面。主要从以节几方面进行考核。

    \begin{enumerate.zh}

        \item 产品质量。产品质量是最重要的因素,在开始运作的一段时间内,都要加强对产品质的检查。检查可以分为两种,一种是全检,另一种是抽检。全检工作量大,一般可以用抽检的方法。质的好坏可以用质量合格率来描述 。

        \item 交货期。交货期也是一个很重要的考核指标参数。考察交货期主要是考察供应商的准时交货率。准时交货率可以用供应商的准时交货的次数与总交货次数之比来衡量。

        \item 交货量。考察交货量主要是考核按时交货量,按时交货量可以用按时交货量率来评价。按时交货量率是指给指定交货期内的实际交货量与与期内应当完成交货量的比率。

        \item 工作质量。考核工作质量,可以用交货差率和交货破损率来描述。

        \item 价格。考核供应商的价格水平,可以和市场同档次产品的平均价和最低价进行比较,分别用市场平均价格比率和市场最低价格比率来表示。

        \item 进货费用水平。考核供应商的进货费用水平,可以用进货费用节约率来来考核。

        \item 信用度。信用度主要考核供应商履行自己的承诺、以诚待人、不故意欠账的程度。

        \item 配合度。主要考核供应商的协调精神。在和供应相处过程中,常常因为环境的变化或具体情况的变化,需要把工作任务进行调变更。这种变更可能要导致供应商的工作方式的改变,甚至导致供应商要做出一点牺牲。这时可以考察供应商在这些方面积极配合程度。另外,如工作出现了困难,或者发生了问题,可能有时也需要供应商配台才能解决。在这样的时侯,都可以看出供应商的配台程度。

    \end{enumerate.zh}

    考核供应商的配合度,靠人们的主观评分来考核。主要找与供应商相处的有关人员,让他们跟据这个方面的体验为供应商评分。特别典型的,可能会有上报或投诉的情况。这时可以把上报或投诉的情况也作为评分依据。

    可以看出,前7项都是客观评价,第8项是主观评价。客观评价都是客观存在的,而且可以精确计量的,而主观评价主要靠人的主观感觉来评价。
