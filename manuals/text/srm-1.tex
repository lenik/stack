一,供应商管理概述

    1. 供应商理的含义

    供应商,是指可以为企业生产提供原材料、设备、工具及其他资源的企业。供应商可以是生产企业,也可以是流通企业。

    所谓供应商管理,就是对供应商的了解、选择、开发、使用和控制等综合性的管理工作的总称。其中,了解是基础,选择、开发、控制是手段,使用是目的。

    采购管理和供应商管理的关系: 企业要维持正常生产,就必须要有一批可靠的供应商为其提供各种各样的物资。因此,供应商对企业的物资供应起着非常重要的作用,采购管管就是直接和供应商打交道而从供应商采购获得各种物资的。因此,采购管理的一个重要工作,就是要搞好供应商管理。

    2. 供应商理的目的

    供应商管理的目的,就是要建立起一个稳定可靠的供应商队伍,为企业生产提供可靠的物资供应。

    供应商是一个与购买者相独立的利益主体,而且是个追求利益最大化为目的的利益主体。按照统的观念,供应商和购命买者是利益互相对立的矛盾对立体,供应商希望从购买者手中多得一点,购买者希望向供应商少付一点为此常常斤斤计较。某些供应商往往在物资商品的质量、数量上做文章,以劣充优、降低质量标准、减少数量、甚至制造假冒伪劣产品坑害购买害。购买者为了防止伪劣质次产品入库需要花费很多人力物力加强物资检验,大大增加了物资采购检验的成本。因此供应商和购买者之间,既互相依赖、又互相对立,彼此相处总是一种提心吊胆,精密设防的紧张关系。这种紧张关系,对双方都不利,对购买者来说,物资供应没有可靠的保证、产品质量
没有保障、采购成本太高这些都直接影响企业生产和成本效益。

    相反,如果找到一个好的供应商,它的产品质保证供应; 按时交货,这样,采购时就可以非常放心,不但物资供应稳定可靠、质优
价廉、准时供货,而且双方关系融洽、互相支持、共同协调。这样对企业采购管理、对企业的生产和成本效益都会有很多好处。

    最重要的是,好的供应商可以提升企业的竟争力。

    为了创造出这样一种供应商关系局面,克服传统的供应商关系观念,有必要注重供应商的管理工作。通过多方面的持续努力,去了解、选择、开发供应商,合理使用和控制供应商,建立起一支可靠的供应商队,为企业生产提供稳定可靠的物资供应保障。

    3. 供应商管理的几个基本环节

    (1)供应商调查: 供应商调查的目的,就是了解企业有哪些可能的供应商,各个供应商的基本情况如何,为企业了解资源市场以及选择正式供应商准备。

    (2)资源市场调查:资源市场调查的目的,就是在供应商调查的基础上,进一步了解掌握整个资源市场的基本情况和基本性质。是买方市场还是卖方市场?是竞争市场还是垄断市场?是成长的市场还是没落的市场?此外,还需了资源生产能力、技术水平、管理水平以及价格水平等,为制定采购决策和选择供应商做准备。

    (3)供应商开发:在供应商调查和资源市场调查的基础上,可能会发现比较好的供应商,但是还不一定马上得到一个完全合乎乎企业要求的供应商,还需要在现有的基础上进一步加以开发,才能得到一个基本合乎企业需要的供应商。将一个现有的原型供应商转化成为一个基本符合企业需要的供应商的过程,就是一个开发过程。具体包括供应商深入调查、供应商辅导,供应商改进、供应商考核等活动。

    (4)供应商考核: 供应商考核是一个很重要的工作,它分布在各个阶段。在供应商开发过程中需要考核,在供应商选择阶段需要考核、在供应商选择阶段也需要考核。不过每个阶段考核的内容和形式并不完全相同。

    (5)供应商选择: 在供应商考核的基础上,选定台适的供应商。

    (6)供应使用: 与选定的供应商开展正常的业务活动。

    (7)供应商励与控制: 这是指在使用供应商过程中的激励和控制。
