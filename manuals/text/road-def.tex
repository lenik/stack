1. 工艺路线的定义

    工艺路线是说明零部件加工或装配过程的文件。在MRP系统中,编制工艺路线是在工艺过程卡的基础上进行的,但又有所不同。前者是指导加工制造的技术文件,后者是计划文件或管理文件。为了使报表比较简练,通常并不详细说明与编制计划没有直接关系的技术要求和方法 (必要时可在注释中说明)。
    工艺路线的典型报表格式如表16.2所示。
    <image />

2. 工艺路线的作用

    工艺路线主要有以下作甩

    ®计算加工件的提前期,提供运行 MRP 的计算数据;

    ®计算占用工作中心的负荷小时,提供运行能力计划的数据;

    ®计算派工单中每道工序的开始时间和完工时间;

    ®提供计算加工成本的标准工时数据;

    ®按工序踉踪在制品,是工序跟踪报告的依据。

3. 工艺路线孑艮表

    工艺路线报表有以下特点

    (1)除说明工序顺序、工序名称、工作中心代码及名称外,工艺路线报表还把加工过程和时间定额汇总到一起显示。制定工时定额同编制工艺在同一部门进行。工艺人员掌握时间定额,有助于分析所制订的工艺的经济合理性。

    (2)除列出准备和加工时间外,还列出排队时间和等待/传送时间,也就是加工前后的缓冲时间。

    (3)表上列出的准备和加工的标准时间,是用来计算标准成本用的,考虑了操作工人的人数。在计算计划跨度和能力计划负荷时,要被相应的工人数除,才是真正占用工作中心的时间,再连同传送、排队时间作为编制计划进度的依据。例如,工序 10 的加工标准时间为 0.5,这是计算成本用的数据,即加工一件消耗 0.5 工时,再乘以工时费率。如果计算计划跨度和能力,要被这道工序的加工工人数 2 除,即 0.25 小时,也就是占用工作中心的时间。

    (4)每道工序对应一个工作中心,说明物料的形成同工作中心的关系,也用来说明工作中心负荷是由于加工哪些物料形成的。

    (5)工艺路线表包括了外协工序、外协单位代码和外协费用。外协工序的时间可记入工序的传送时间字段中,表示工件从送出到收回的时间,是固定提前期。

    (6) 除了说明基本的工艺路线外,还要说明各种可能替代的工艺路线,便于在调整计划或主要工艺路线上的设备出现故障时替代。当一个零件可以通过多种的工艺路线完成,或有若干替代工艺路线,可对不同工艺路线赋予不同代码。若工艺路线比较定型,使用同一类型工艺路线的零件较多时(如成组加工),但每件的加工时间又不同时,也可对典型的工艺路线编碼。利用软件的复制功能,建立新的工艺路线时只需修改少量数据,简化数据录入。此外,和建立物料清单一样,也要说明工艺路线的生效期和失效期。

    (7)系统在模拟计划时,如需要调整工艺可先对工艺路线的工序变化、时间定额、加工成本进行模拟,然后再做决策。

4. 工艺路线与物料清单结合应用

    工艺路线有时要结合物料清单来应用。例如,对由企业提供毛坯或原材料委托外协加工的物料。由于原材料作为企业的采购件必须列入产品结构,而外协件作为原材料的母件,不在产品结构的最低层,不能定为采购件,但却要作为采购作业对待。 这时,可用一种变通的办法,把外协件定义为加工件,但其工艺路线只有一道外协工序(这种处理方法还要依软件功能而定)。图16.3是电阻件外协弯脚的例子。
    <image />

    从逻辑讲,工艺路线中可以把"设计”、"运输” 等非工艺性作业,作为独立需求件性质的一道工序来处理。也可以把供应商或分包商的计划,作为一种特定的工序对待,只计算提前期和总费用。只要符合运算逻辑,根据软件的条件,可以有各种灵活的变通应用。

    总之,可以按照逻辑关系,灵活运用工艺路线的功能。它可以包括任何占用工作中心需要计算时间的工序,也可以是既无工作中心又无费用但要增加时间的 "工序”。

    在机械产品中,往往有两件相关的物料分别加工后,再合在一起加工,然后又分开加工的情况。如箱体和箱盖、连杆体和连杆盖、涡轮副或油泵油偶件等(又称配作件。在采用数控机床加工的情况下,某些配作可以减少)。 这些加工件的工艺路线需特殊处理。不同工序加工批量不同时,如车削与热处理,有时需要把热处理作为单独物料处理,在物料清单上要增加一个层次。

    换句话说,编制物料清单要同工艺路线结合起来考虑,在流程工业中这种特点更为普遍。
