\section {选择供应商}
    采购管理的永恒主题是:提高质量、降低成本、提高效率、保证供应。质量和成本是选择供应商的两项重要指标,通称 “比质比价”。但是,在成本方面,仅仅 "比价”是不够的,除此之外,还有一些其他重要的要素,同样需要考虑。也就是要遵照总体拥有成本 (TCO) 的原则精神来选择供应商。

\subsection {选择供应商的标准}

    选择供应商要注意以下各个方面:

    \begin{itemize}
        \item 信誉: 包括供应商所处的行业地位、管理水平等。选用信誉和履约率高的供应商,可以免去企业采购人员大量的催货等无效作业,而把时间和精力用在提高采购管理业务上。如果企业产品中的配套件和零部件都是由一流的供应商提供的,企业产品的品牌度也会同样提升。

        \item 技术: 要考察供应商的可持续发展能力,关注供应商的产品研发能力、制造工艺、现有及可能扩充的能力以及其利用情况,以及技术发展战略。

        \item 质量: 供应商全面质量管理(TQM)执行情况、不是表面上是否通过 ISO 9000 论证,而是真正保证质量持续稳定并不断提高。要注意供应商的退货记录

        \item 成本: 我们讲关注企业成本而不是价格,是通过观察分析,研究供应商在控制成本上的种种措施。同时分析批量折扣、运输条件,不断优化。

        \item 服务: 有些零部件的使用需要供应商提供培训,关注供应商的服务网点的分布状况,对投诉和保修能否提供保证。

        \item 位置: 供应商所处的地理位置,直接影响定货批量,运输方式和运输成本。在条件相同的情况下,尽量本地化。汽车行业实施 JIT 模式,很注重地理位置这个因素,主要汽车总装厂与零部件生产厂的平均距离如表\ref{tab:avgdist}
            \footnote {《中国计算机报》2005第1期B17页} 所示:

        \begin{table}[h]
            \centering
            \caption{总装厂与零部件生产厂的平均距离} \label{tab:avgdist}
            \begin{tabular}{|c|c|c|c|c|c|}
                \hline
                公司名 & 丰田 & 日产 & Chrysler & Ford & GM \\ \hline
                距离(km) & 95.3 & 183.3 & 875.3 & 818.8 & 687.2 \\ \hline
            \end{tabular}
        \end{table}

        \item 沟通 指信息沟通的有效手段。如信息门户,EDI,Extranet,能否实现信息集成和协同商务,快速响应需求市场的变化等。

    \end{itemize}

\subsection {减少同一物料供应商的好处}

    很多企业由几家供应商供应同一种物料(原材料或配套件),认为这样可以有一个较好的谈判优势,不至于被一家供应商所“挟持”。 同时,万一一家供应商发生不可抗拒的意外,如非法破坏、各种自然牢害、交通事放、国际贸易争执、战争等,还可以从其他供应商获取货源。

    但是,我们也要注意到,减少供应商带来的好处。在风险较小的情况下,应当本着合作伙伴和供需链竞争的原则,遵照总体拥有成本(TCO)的精神来精选供应商。

    这些好处是:
    \begin{itemize}
        \item 简化计划及调配 供应商数量少,肯定会大太简化计划分配的工作量,减少几家之间的争执和矛盾,巩固合作伙伴关系。

        \item 经济批量、优惠条件 向少数供应商订货,每家的订货数量肯定要大,形成对供应商有利的定货批量,需方可以争取折扣优惠,实现双赢。

        \item 减少专用工艺装备费用 对一些需要特殊工艺装备加工的零部件,由多家供应,每家都至少要准备一套,必然增加制造总成本。

        \item 有利质量控制 集中供应,有利于控制质量。

        \item 简化运输 少数供货来源,必然减少运输调度,提高运输效率。

        \item 减少库存 受定货批量限制,多家供应必然会增加库存。

        \item 降低采购费 综合上述各种原因,减少供应商,遵照 TC0 的原则,反而有可能降低采购成本,这方面的成本降低弥补了谈判地位不利的缺陷。

        \item 加强台作伙伴关系、共同提高竟争力 供应单位精简集中了,使企业有可能集中精力发展与最具优势的供应伙伴加强合作,共同创新,持续地推出有竞争力的产品和服务。

    \end{itemize}

    在 MRP II 系统的企业实施评价的 ABCD 定级标准中,有许多考核指标,其中 "供应厂商是否按时交货“,”到货物料的合格率”是两项主要的内容。看起来这似乎不是本企业的问题,完全是供应厂商的事,但是从采购管理角度来看,认为这是企业没有选对供应厂商,没有同供应厂商配合好的结果,责任仍在本企业。

    采购管理采用集中还是分散的问题,主要出现在一个企业或集团有多个工厂的情况。

    集中采购的优势是: 选择较好的供应商,建立长期密切合作关系。便于协调同一个供应商的定单,获取较大的折扣。可以精简企业的采购组织机构,有利于培养专业采购人员等。

    主张分散采购的理由是能够实时响应各个工厂独特需求的变化,各工厂有其特有的地理位置和运输距离,需要区别对待。有些工厂同现有供应商已建立良好的合作和信任关系等。

    因此,不能笼统地采取集中还是分散的采购模式。多个工厂也有生产性质相同与否的区别,如家电行业集团内部有多个生产制冷设备的工厂,许多通用的原材料和配件就可以集中采购,分别发货到使用点。汽车行业可以集中采购钢板,棉纺企业可以集中采购原棉,食品行业可以集中采购某些农产品等。所谓集中,主要是采购事务处理的集中,信息的集中,而不是集中到仅仅一个供应商。有些案例反映集中采购是"侵犯分公司的采购权利",其实,采购只是执行计划,本来没有什么 "权利” 可言,这种陈旧观念和带来的隐含陋习必须改变。

    多数企业是采取两者结台的方案,原则上是各个工厂使用品种相同的物料,采取集中采购。而各个工厂使用的特殊物料,由各个工厂自己采购。ERP 系统对多工厂管理的情况主要提供集成的信息,便于企业自己选择各种适用的采购方案,而不应强行规定或限制使用某种模式的业务流程。

\section {集中与分散采购}

    采购管理采用集中还是分散的问题,主要出现在一个企业或集团有多个工厂的情况。

    集中采购的优势是:选择较好的供应商,建立长期密切合作关系。便于协调同一个供应商的订单,获取较大折扣。
    可以精简企业的采购组织机构,有利于培养专业采购人员等。

    主张分散采购的理由是能够实施相应各个工厂独特需求的变化,各工厂有其特有的地理位置和运输距离,需要区别对待。有些工厂同现有供应商一建立良好的合作和信任关系。

    因此,不能笼统地采取集中还是分散的采购模式。多个工厂也有生产性质相同与否的区别,如家电行业集团内部有
    多个生产制冷设备的工厂,许多通用的原材料和配件就可以集中采购,分别发货到使用点。汽车行业可以集中采购钢板,棉纺企业可以集中采购原棉,食品行业可以集中采购某些农产品等。所谓集中,主要是采购事务处理的集中,信息的集中,而不是集中到仅有一个供应商。有些案例反应集中采购是“侵犯分公司的采购权利”,其实,采购只是执行计划,本来没有什么“权利”可言,这种陈旧观念和带来的隐含陋习必须改变。

    多数企业是采取两者结合的方案,原则上是各个工厂使用品种相同的物料,采取集中采购。而各个工厂使用的特殊物料,由各个工厂自己采购。ERP系统对多工厂管理的情况主要提供集成的信息,便于企业自己选择各种使用的采购方案,而不应强行规定或限制使用某种模式的业务流程。
