\subsection {物料清单的作用}

    物料清单是企业所有核心业务都要用到的共享管理文件,关键词是“共享”,就是说,它对任何业务都是重要的,不是某一种业务所独占的文件。但要求物来斗清单全面细致、使用最频的是计划与控制部门。物料清单的作用可以分以下几方面来谈。

    \begin{enumerate}
    \item 使系统识别产品结构

        用计算机辅助管理,首先要使系统能够 "读出” 企业制造的产品结构和所有涉及到的物料。为了便于计算机识别,必须把用图表达的产品结构转换成数据报表格式,也就是物料清单。物料清单同产品结构图所说明的内容是一致的。本书将以家庭和办公室最常见的电子挂钟为例来讲解物料清单

    \item 联系与沟通企业各项业务的纽带

        物料清单是运行 ERP 内部集成系统的主导文件,企业各个业务部门都要依据统一的物料清单进行工作。

        \begin{itemize}
        \item 它确定配置产品需用的可选伟计算累计提前期、是销售部门洽谈客户定单的依据;

        \item 物料清单是按照实际的生产装配顺序编制的,是计划部门编制生产计划和采购计划的依据;

        \item 是仓库部门向生产工位配套发料的依据;

        \item 是跟踪物流,工序及生产过程、追溯任务来源的依据;

        \item 是供应部门采购和外协的依据;

        \item 是改进产品设计的3化工作(标准化、系列化、通用化)需要参照的重要文件;

        \item 是成本部门计算成本的依据;

        \item 是销售部门投标报价的依据。当遇到一个新产品要报价时,如果新产品同老产品是结构相似、工艺相近,负责报价的部门可以从各个老产品中,摘出相似的相关单层结构,重新“拼凑”成一个结构近似的假想新产品。运用系统的成本累加功能计算出它的成本,作为迅速而比较准确的报价依据。这里,又一次用到了单层结构的概念;

        \item 如果把工艺装备也包括在物料清单中,就可以同步地生成生产准备计划,并按所加工的零件数量合理分摊消耗工艺装备的“间接”成本。
        \end{itemize}
    \end{enumerate}

    不难看出,上述各项业务涉及销售、计划、生产、供应、物料、成本、设计、工艺等部门。物料清单体现了信息集成和共享。对一个制造业企业来讲,实现信息化管理离了物罪斗清单是不能运行的。

    我国有一些企业,如大连起重机厂的销售计划科就曾用拼凑单层产品结构的方法形成模拟产品,迅速做出新产品的报价;也有一些企业,如哈尔滨汽车电器厂的财务科在产品结构的基础上,用自底向上累计的方法计算成本。这些例子都发生在20世纪80年代,是国内企业一些部门在没有听说过 MRP 的情况下的管理创新,说明 MRP 这种“产品结构”的思维方法反映了生产客观规律,对任何地域和体制都是适用的,不能简单地把它看成是“舶来品”。只是我国企业的这些管理创新,没有人对它再进行深层次的研究,提升为管理理论。这也反映我围对应用科学的忽视,值得深刻反思。

    正因为物料清单是系统识别产品的依据,说明产品是如何“做出来”的,要为企业众多部门所应用。它的准确度必须要求达到100%。物料清单是“时间坐标上的产品结构”数据模型的报表形式,说明动态的期量标准,如果模型不准确,运行 MRP 的结果会完全失去意义。

    建立物料清单,从表面上看似乎会给企业的某些部门增加了工作量,但从企业管理整体来看,将减少各个部门在查询、统计、传递和复制报表等方面的大量的工作,要用全局观点来对待物料清单。在 ERP 内部集成系统同 CAD 和 PDM(产品数据管理)系统集成的情况下,可以在设计图纸的基础上,经过适当补充调整,转换成制造用的物料清单,从而减少大量的重复工作。
