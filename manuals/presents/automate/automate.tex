\documentclass[CJK]{z-beamer}
\usepackage{ifthen}
\usepackage{multicol}
\usepackage{booktabs}           % toprule, bottomrule
\usepackage{tabularx}
\usepackage{tikz}
    \usetikzlibrary{arrows, chains, patterns}

\usetheme{Warsaw}
\usecolortheme{spruce}

\title{软件构建自动化}
\subtitle{从单元测试到连续集成}
\author{谢继雷}
\authorEmail{xjl@bee32.com}

\begin{document}

    \begin{frame}
        \begin{center}
            \titlepage
        \end{center}
    \end{frame}

\part {软件构建技术}

    \section {构建工具}
        \itemssframe<+->{构建工具: 集成开发环境}{
            \item 构建目标 (Build Target)
            \item 构建配置 (Build Configuration)
            \item Auto Build
            \item Custom Build
        }
        \itemssframe<+->{构建工具: Makefile}{
            \item ...
        }
        \itemssframe<+->{构建工具: Ant}{
            \item ...
        }
        \itemssframe<+->{构建工具: Maven}{
            \item ...
        }

    \section {大型项目的构建}
        \itemssframe<+->{管理依赖项}{
            \item ...
        }
        \itemssframe<+->{制品}{
            \item ...
        }

\part {软件测试技术}
    \section {单元测试}
    \section {集成测试}
    \section {性能测试}
    \section {测试分析}
    \section {测试自动化}

\part {软件集成技术}

    \section {软件产品支持}
        \itemssframe<+->{软件的版本管理}{
            \item ...
        }
        \itemssframe<+->{软件的开发文档}{
            \item 软件的编程接口
        }
        \subsection{软件的规格参数}
            \begin{frame}
                \frametitle{软件的规格参数: 特性矩阵}
            \end{frame}

            \begin{frame}
                \frametitle{软件的规格参数: 依赖图谱}
            \end{frame}

        \itemssframe<+->{软件的用户手册}{
            \item ...
        }

    \section {连续集成系统}
        \itemssframe<+->{热部署}{
            \item Tomcat
        }

\end{document}
