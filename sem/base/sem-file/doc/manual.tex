\section{用户文件}
首先介绍一下用户文件管理采用的技术及设计思路。
传统的文件系统是给予路径的树状结构,基于标签的文件系统实际上是一个网状结构,但是对于海量增长的用户数据,树形的文件管理已经显得不足。而标签式文件管理具有层次简单,检索方便等优点。微软已经开始在window
7中尝试做这个事情了。Windows 7的库对音乐,图片,以及office文档的支持已经十分完善,但是其他应用目前还跟不上。
参考\hyperlink{???}{http://storizabautpipol.com/2012/04/17/organizing-{}our-{}growing-{}personal-{}data-{}files/}
用户文件管理模块中,在浏览器中实现用标签形式来对用户上传的文件进行管理,抛弃以往的树形目录结构,是我们的一个积极的尝试。

\subsection{用户文件管理}
上传用户文件分为两个步骤
\button{ I}.上传文件
\button{ II}.对上传的文件设置标签等信息
\opset{用户文件管理}{
    \item \ops{上传文件}{
        \item 点击\button{上传}按钮,弹出上传文件对话框
            \screenshot{8.png}
        \item 点击\button{选择文件}按钮,选择需要上传的文件
        \item 点击右侧的小箭头图标,单个上传文件
        \item (如果要取消上传)最右边的取消按钮,取消上传对应的文件
        \item 点击\button{上传}按钮,批量上传文件
        \item 点击\button{取消}按钮,取消全部选中文件的上传
    }
    \item \ops{对上传的文件设置标签等信息}{
        \item 选择刚刚上传的文件,点击工具栏上的\button{修改}按钮,弹出修改文件信息对话框
            \screenshot{9.png}
        \item 选择档案日期,客户
        \item 点击按钮,弹出选择客户对话框(如果该文件对应某个客户)
            \screenshot{10.png}
        \item 选择客户,点击\button{确定}按钮,完成选择
        \item 编辑\textbox{文件名称}
        \item 编辑\textbox{文件标题}
        \item 编辑\textbox{描述信息}
        \item 点击按钮,弹出选择经办人对话框(如果需要)
            \screenshot{11.png}
        \item 点击\button{确定}按钮,选择\textbf{经办人}
        \item 点击添加标签按钮,弹出添加标签对话框(标签需在标签设置中预置)
        \item 点击\button{确定}按钮,给文件添加选中的标签
        item 点击\button{保存}按钮,保存更改
    }
    \item \ops{删除文件}{
        \item \screenshot{7.png}
    }
    \item \opset {检索用户文件} {
        \item \screenshot{6.png}
    }
}

\subsection{文件标签设置}
\opset{文件标签设置}{
    \item \ops{新建文件标签}{
        \item  点击工具栏上的\button{新建}按钮,弹出新建文件标签对话框
            \screenshot{1.png}
        \item  输入\textbox{标签名称}
        \item  输入\textbox{描述}
        \item  点击\button{保存}按钮,完成新增标签
    }
    \item \ops{查看文件标签}{
        \item  选择文件标签
        \item  点击工具栏上的\button{查看}按钮,查看文件标签详细
    }
    \item \ops{修改文件标签}{
        \item  点击工具栏上的\button{修改}按钮,弹出修改标签对话框
        \item  修改相关信息,点击\button{保存}按钮,完成修改
            \screenshot{3.png}
    }
    \item \ops{删除文件标签}{
        \item  点击工具栏上的\button{删除}按钮,弹出确定删除对话框
            \screenshot{2.png}
        \item  点击\button{确定}按钮,删除文件标签
    }
}
