电子文档及其特点

1.电子文档的概念

    (1)对电子文档的理解。人们对电子文件与文档的认识也是处在一个不断深化的过程中,从 Machine-readable Document(机读文档) Machine-readable Archives 机读档案) Electronic、(到Records(电子文件)、Archival Electronic Records(归档的电子文件) 再到 Electronic Archives,(电子档案)。国内外学者对“电子文件”或“电子文档”有着多种不同的定义,但是总的来说有四种主要的模式,即“文件说”“代码序列说”“信息组合或信息集合体说”“文字材、、、料说”。解释如下。
        1 国际档案理事会电子文件委员会对电子文件与文档的定义是:“电子文件是一种通过数字电脑进行操作、传输与处理的文件,并具有文件的一般定义。”
        2 国家档案局《电子文件归档与电子档案管理规范》中的定义是:“能被计算机识别、处理和存储在磁带、磁盘或光盘等介质上,并可以在网络上传递的代码序列。”
        3 从信息的角度对电子文件与文档进行定义,认为电子文件与文档的本质就是信息。电子文件是人们为了某一目的,使用计算机,采用一定的格式和处理方式生成的信息的组合。
        4 从文字材料的角度出发来定义电子文件,即所谓电子文件是指人们在社会活动中形政府电子文档管理4成的,以计算机磁盘和光盘等为载体的文字材料。它主要包括了电子文书、电子信件、电子报表、电子图纸等。

        由上可见,尽管对电子文件与文档这一术语存在不同的表达方式,着眼点与侧重点各有不同,但基本内涵是一致的,即电子文件与文档必须满足以下两个基本条件。

        第一,电子文件与文档首先是文件,必须满足文件的所有条件。国内外在定义电子文件与文档时,都将“文件”作为电子文件与文档的上位概念,认为电子文件与文档是文件的一种类型,应当具有文件的各种属性。这是电子文件与文档与其他形式文件与文档的共同点,也是电子文件与文档与其他类型数字信息的基本区别所在。

        第二,电子文件与文档是由计算机生成与处理,并可经由计算机网络进行传播的,计算机及其网络是电子文件与文档赖以“生存”、“活动”的基础。
        严格地说,“电子”信息也可以用连续的模拟量来表示(也就是说存在模拟计算机),但这些模拟信息往往也不是人们可以直接识读、检测与鉴别的。因此,书中所谓电子文件或电子文档应当是指用电子数字计算机生成和处理的文档,亦即比特文档。

    (2)对电子文档的再认识。随着 IT 技术的发展,特别是数据库技术的广泛应用,组织中的信息呈现出结构化和非结构信息两种形式。结构化的信息通常可以作为数据记录(DataRecord)存储在如层次型数据库、关系型数据库等结构化的数据库中。结构化的信息所描述的是事实对象,即与组织业务过程紧密相关的实体,而实体是通过其属性来描述的,如单位员工的工号、姓名、年龄等数据。非结构化的信息所描述的主要是组织中的概念、思想、方法或知识,如会议决议、施工图纸、施工规范等。

    非结构化的信息记录在文档中,或以组织成员的经验的形式存在于员工的大脑中。数据记录与文档的区别如表 1-1 所示。

    表 1-1 数据记录与文档的区别

    在计算机世界,多媒体信息的产生,并通过对象的链接或嵌入技术以及超文本技术,人们可以将若干分散的电子文件整合成为一个电子文档,从而结合在一起形成一个整体。从这一角度看,文档与文件是有所不同的。但在一般情况下,可以不对文档与文件做严格的区分。

    结构化的信息通常占组织信息的 10%~20%,而非结构化的信息则占 80%~90%,并且大部分的非结构化信息都存在于文档中,因此文档在组织中具有重要的作用。文档的主要作用如下:

    1 作为组织的最终产品或生产产品的支持资料;
    2 作为组织中人、群体或组织之间进行信息交流的一种介质;
    3 作为组织活动过程记录的载体;
    4 组织记忆的一个重要组织部分。第 1 章 文档概述5

    2.电子文件与文档的特点

    从以上的定义,可以归纳出电子文件与文档的基本特点如下。

    (1)信息的非人工可读性。传统文件与文档的内容以字符、数字或图表等形式记录于纸张等载体之上,属于“原子”层次,因而往往人工可直接识读。电子文件与文档的信息以比特(bit)的形式存在,而比特没有体积、没有重量、没有颜色、没有形状,能以光速传播。它是信息的最小单位,是一种存在的状态,是数字化计算中的基本粒子(或称为信息的 DNA)。因此,电子文件与文档是一种人类无法凭借自身的器官直接来识读的信息。

    (2)电子文件与文档的系统依赖性。电子文件与文档信息存储的时候需要进行编码,信息检索和阅读的时候需要进行解码,编码和解码过程中则要借助于一定的硬件设备、系统和应用软件,否则电子文件与文档既无法产生也无法阅读,电子文件与文档真正的形成者和读者是程序。电子文件与文档对系统的依赖性特别是它对特定系统(电子文件与文档产生时运行的软、硬件系统)的依赖性是电子文件与文档的一大特点,也是管理中的难题。

    (3)电子文件文本内容是离散性的合成。传统书写方式形成的文本内容与形式一旦写入纸张等载体,就结合为稳定的复合体,其内容与形式是一个彼此不可分离的统一体,并一起构成了文件与文档原件。而在电子语境中,文件文本内容各个组成部分处于离散状态。内容只是计算机能识读的数字串记录,其全文形式不再以固定的形态整体存在。信息内容不会永远和特定的载体相连,信息内容易于复制、迁移,载体仅仅成为每一个特定时刻文件与文档内容的承载物,它已无法固定某些用以鉴辨文件与文档原始性的外部信息。

    (5)电子文件文本失去了书写的物质痕迹。传统的书写方式语境中,在纸张上书写,用打印机打印出的文件文本等都具有一种不变的痕迹,如油墨、字迹、图形、记号等,个性特征清楚,改变或磨擦掉这些痕迹总会留下印记,如墨水的氧化会留下时间的印记等。然而在电子语境中,当文件内容录入计算机,显示在屏幕上,这种代码是以计算机能够识读的、人无法直接看见和触摸的“非物质”方式存在,并可以根据需要任意改动,而表面上不留痕迹。这样,传统文件写作中的“草稿”“定稿”“正本”等概念的界限在电子环境中变得模糊起、、来,在同一内容的多版电子本文稿中,从外在的形式上无法直接发现具有“原始性”或“本真性”的文本。目前,通过一些专门技术如数字签名、加密等方式可以确认信息内容的真实性,但是,其验证的过程仍是由一系列的计算机程序实现。

    (6)信息的可操作性和不稳定性。电子文件信息是灵活的,可以根据不同的需要进行各种操作,如修改、粘贴、删除,而且还可以不留痕迹。信息可以直接进入计算机辅助制造系统(CAM)进行生产。信息的质量,如清晰度,也可以进行调整。电子文件的可操作性有时会带来难以分辨或预测的易变性,这主要是由信息增、删、改的便捷性,信息系统自动更新和补充功能,计算机病毒的侵袭等原因造成的。电子文件载体性能的不稳定性有可能造成文件中信息的改变,在文件“定型”保存之后,这种易变性容易导致人们对它的真实性和可靠性产生怀疑。此外、电子信息技术的发展,新的信息编码方案、存储格式、存储载体和系统软件的更新换代更加对电子信息稳定性造成巨大的冲击,这要求将原文件迁移到新的技术环境之中,而迁移过程中信息的损失、变异也是难以避免的。

    (7)电子文件形成系统的异构性。异构性是指系统的数据处理与描述方式多样化和不一致性。形成电子文件的系统性质、功能及功能实现方式是多种多样的,建设单位的业务特色、工作环境差异、开发者的技术差异等,都会导致具体系统的概念定义、功能定义、结构方式、政府电子文档管理6技术路线、信息构建等多方面的异构性。解决异构性问题的主要方法是数据处理方式和描述方式的标准化。

    3.电子文档的优势

    用电子文档进行信息管理与沟通相对于使用纸质文档有很多优越性,主要表现在以下几个方面。

    (1)节约成本。利用计算机生成的电子文档进行信息沟通,最直接的好处是节约纸张和打印的费用,对于费用昂贵的工程图纸更是如此。并且电子文档的复制不需要使用复印机,尤其是当文档需要进行很多份的分发时,纸张和复印费用的节约是非常可观的。

    (2)节省存储空间和信息存储的可压缩性。电子文档以数字化的形式存储在存储设备中,如 U 盘、磁盘、光盘等,可以节约存储纸质文档所需的空间。将纸质文档扫描而成的图形化电子文档与原纸质文档所需存储空间的分析对比如下:

    1 张标准的 A4 纸张=50KB(黑白两色扫描,使用 CCITTG4 压缩技术)文字存储量
    1 个文件柜(四个抽屉)=10 000 张纸=500MB=1CD(CD-ROM 或 CD-WROM)
    10 个文件柜=5GB=10CD=1DVD 光盘(DVD-WROM)
    1 张 A0 图纸=16 张 A4 纸=800KB 文字存储量

    计算机直接产生的电子文档比图形化的电子文档占用的空间更小。通常与一张写满中文的 A4 大小的纸张文件(1 000 字左右,字号是小四)相当的 Word 电子文件需占用 25KB 的磁盘空间,包括这一页所包含的文字、文字格式和应用程序对文件整体格式的描述语言所占的空间。一个 100 页的纯文字的电子文件包含大约 10 万小四号大小的中文字,占用的空间为300KB。到目前为止,不但一般计算机的硬盘已经达到了几十 GB 以上,而且光全息信息存储容量将更加惊人。

    电子文件与文档的数字化特征使其通过特殊的编码可以进行压缩处理,使巨量信息可以压缩存储,这不但可以节省大量存储空间,而且可以减轻网络传输的带宽占用(这在政府机关召开视频会议时是很有意义的)。

    (3)传输方便迅速。因特网的出现使信息能以更加低廉的成本迅速地进行传输。文件管理系统通常通过上传的方式把电子文档存储到连接在因特网上的服务器中,服务器收到文件后在很短的时间内用可以在 WWW 上发布或通过 E-mail 通知所有与此文档相关的人员,在收到 E-mail 通告信后,收信人按照 E-mail 中所提供的路径访问文件服务器以获取此文档。从文档作者上传文档到接收人看到文档的整个过程可以在很短的时间内完成,从而提高了信息流动的速度。当然,Internet 的数据传输速度受网络连接设备及采用的传输技术的不同而有很大的差别。

    (4)电子文件与文档的可重用性。存储起来的电子文件可以被反复调取、使用;可以对所存信息整体进行拆分,或提取信息的属性、建立关联、组配使用;通过数据挖掘等方式可以对结构化文件中蕴含的信息进行分析和调用,以发现数据资源库数据中的信息规律。

    (5)电子文本撰写的协同性。在传统文件写作活动中,撰写者始终是写作活动的主体,文本结构的安排、语言的运用、材料的选择等工作都由其独立完成,而且同一文本在同一时间只能由单一撰写者进行写作。而电子文件的写作动摇了文件撰写者的稳定地位。首先,计算机不仅是文件撰写者的工具,而且成为其写作活动的合作者。使作为客体的计算机与主体的书写行为合而为一,计算机按照自身的信息方式参与到文件的书写活动中,它与撰文者共同成为电子文件的创作者。其次,为同一文本的多作者写作活动提供了方便,除了可以通过网络集体讨论和写作,甚至还可以利用协同软件“同时性写作”,即多个撰文者可以在不同的地点同时对同一份文本进行写作和编辑,共同完成文件的写作。

    4.电子文件与文档功能特征

    在办公自动化环境下,电子文件从功能上看,有其显著的特征。

    (1)便于简化文档的处理和管理,实现无纸操作。所有的电子文档都被归档到一个集中的库中,授权的用户可以在任何时候、任何地方进行访问,这将减少有纸办公和储存的空间。另外,可以将不同版本的数据存储在不同的地方,避免了数据之间的矛盾。

    (2)可以进行版本控制,保持文档的完整性。可以提供版本号控制(经验证明,实际 5个版本就已经足够优良),因此,用户能够设置并保存在数据库中的首选版本号,有利于保持文档的完整性。

    (3)可实现集中的文档访问。通过同步多站点服务器,支持多站点访问,这样,在任何一间办公室工作,都可以实现对文档的访问。

    (4)文件查找将更加便捷。检索系统的文件查找功能能帮助工作人员以最快的、最友好的方式准确地访问及找出文件。系统可以设计很多搜索策略实现快速、准确的查询,例如:按标题、按日期、按主题词等。

    (5)多媒体电子文件信息的集成。以往的文件都是平面,文字和图形在平面的纸张或其他载体上呈现。而电子文件是多媒体的、立体的,文字、声音、图像和视频可以融合到一份文件中,或通过超级链接组织在一起。运用多媒体技术,可以使电子文件声形并茂,真实地再现当时的活动情况,从而强化了文件对社会生活记忆和再现功能。可以说,电子文件是一种全方位的记忆和再现,实现了文件功能的革命性变革。

    (6)电子文件管理流程变化。电子文件归档、鉴定、著录环节提前;著录将贯穿于电子文件的编制、处理、归档、迁移、利用等整个生命周期。电子文件管理内容变化:如实体整理仅为对脱机保存的磁盘、磁带、光盘的归类;电子文件管理中,可实现归档与检索的集成、逻辑归档工作与检索工作的集成、归档与鉴定的集成、保护与其他流程之间的集成。电子文件管理顺序发生变化,由串行向并行、线性向非线性变化。


`2 信息生命周期

    信息生命周期这个概念是美国著名的信息资源管理学家霍顿提出的。他认为信息是有生命的,信息生命周期(Information Life Cycle)是指信息运动的自然规律,它一般由信息需求的确定以及信息资源的生产、采集、传播、处理、存储和利用等阶段所组成。信息生命周期主要是从信息管理的角度来着手分析的。但是信息生命周期不仅仅是信息管理周期,更有信息运动周期。而且可以说信息的运动周期是更基础的东西。也就是说,信息生命周期包括信息运动阶段中的生命周期和信息管理阶段中的生命周期,而信息运动周期是信息管理周期的依据。信息生命周期通常分为如下六个阶段。

    `e
        `a数据创建阶段。

        随着社会的发展和技术的进步,数据及其文件也随之迅速增长。所产生的数据和文件需要存储以利于及时的处理、保护和利用。一般数据的价值通常会随着时间逐渐降低,因此所有数据在创建时都应当获得一个由数据的类型、数据的价值和相关法规的要求决定的删除日期。系统将定期清除到期的数据。信息生命周期管理就是要根据应用的要求、数据提供的时间及数据和信息服务的等级,提供相适应的数据产生、存储、管理等条件,以保障数据的及时供应。

        `a数据保护阶段。

        信息的价值与信息的连续可用性、完整性和安全性息息相关。随着越来越多的信息以数字化的形式出现,人们面临着如何以相同或者更少的代价管理迅速增长的信息和存储的挑战。从电子数据处理产生以来,对于数据保护的需求与日俱增:需要防止数据受到无意或者有意的破坏,系统程序和文件被破坏了可以花钱去购买重新安装,而用户的文件(数据)丢失和破坏是花多少钱也买不回来的。越来越多的组织都意识到从数据和文件中心所遭受的重大损失中恢复所需要付出的努力和时间,并从中认识到制定相应计划的重要性。数据保护解决方案是一系列技术和流程的组合,包括备份、远程复制和其他数据保护技术。系统的可用性在一定程度上取决于数据的可用性——即使在技术上服务器和网络都是可用的,但是如果应用系统不能访问到正确的数据,用户将认为系统是不可用的。信息生命周期管理将按照数据和应用系统的等级,采用不同的数据保护措施和技术,以保证各类数据和信息得到及时、有效的保护。

        `a数据访问阶段。

        信息生命周期管理的主要目标是确保信息可以支持决策和提供长期的价值。因此,信息必须便于访问,便于共享。这个阶段将成为信息生命周期管理与业务流程管理的交叉点。

        `a数据迁移阶段。

        信息技术发展是如此快速,以至于信息技术的设备在比较短的时期内就要实现一定程度的更新换代。在当前信息应用的环境中,保持应用系统的全天候运作已第2章文档管理概述19是必需条件。为了对系统进行升级或对系统配置改变而进行的数据迁移就是本阶段的重要工作,并且在将数据从一个存储设备转移到另外一个存储设备时,不能影响系统的正常运行。信息生命周期管理应考虑到这类需求,采用必要的技术加以配合,使数据的迁移简单、自动化而且不影响系统的运作。

        `a数据归档阶段。

        维持一个数据备份和归档可以从多个方面支持系统的运作。它可以防止这些数据记录被无意破坏。它能确保那些仍然具有价值的数据可以得到妥善的保存。人们已经意识到备份数据和文件的重要性。数据备份可以在原始数据和文件因为某种原因被损毁或破坏时进行恢复,是数据存储战略的重要组成部分。

        `a数据销毁阶段。

        许多数据总会在一段时期后,没有再继续保存的价值。这时必须制定相关的政策,对没有必要保存的数据进行销毁,被销毁的数据和文件将从系统或数据库中清除。对一些数据进行销毁必须符合政府的条例和法规。因此,应当建立科学的数据销毁政策。
    `ee

`2 文件生命周期

    `3 文件生命周期理论的基本内容

    文件从现行—半现行—非现行是一个相对独立而完整的运动过程,通常称作文件生命周期,它是客观世界物质运动阶段性在文件运动中的一种表现形式。文件生命周期理论形成于20 世纪 50 年代,最初主要立足于对当时已经出现的新型文件保管机构——文件中心的理论阐释,后经档案学者的逐步深化和扩展,成为在国际档案界具有广泛影响的重要理论之一。1950 年 8 月,英国罗吉尔· 利斯教授在第一届国际档案大会上正式提出了文件生命周期划分的“三阶段理论”,即文件的现行、休眠或暂时保存、销毁或永久保存,它标志着文件生命周期理论的诞生。此后,经过欧美国家的一些档案学者(特别是阿根廷档案学者曼努埃尔· 巴斯克斯教授)的进一步探索,在 20 世纪 80 年代中后期得到定型,并为国际档案界所接受。

    文件生命周期理论的基本内容可以概括为三方面:文件从其形成到销毁或永久保存是一个完整的运动过程;这一过程依文件价值形态的变化可以划分为若干阶段;文件在每一阶段因其特定的价值形态而与服务对象、保存场所和管理形式之间存在一种对应关系。尽管不同学者对于文件阶段的划分和描述不尽相同,但对于文件生命周期理论的这三个要点是基本认同的。

    从系统论、信息论、控制论的角度来认识,文件运动作为社会大系统中的一个信息系统,它是根据利用者需要,通过人工干预使“文件”与“档案”两种不同的社会价值循环转化的一个信息流。其特点有时间的无限性、空间的转换性、内容的一致性、价值的互变性和动态的可控性等;其实质是客观世界物质运动质量互变规律和否定之否定规律在文件运动中的具体表现,也真实地反映了社会主体对文件和档案的需求愿望、利用目的及工作机制的发展变化。

    我国的学者在理论与实践的基础上,进行了相关的研究,并指出:“文件作为人们进行信息交流和储存的一种工具,有其相对稳定的物质载体,也就是说,文件是包含信息、意识、知识内容的一种物质形态,它与其他物质一样是运动的,我们把文件从产生到消亡的整个过程称为文件的运动周期。”为此,提出了文件运动阶段四分法观点,具体介绍如下。

    第一阶段,是文件的制作和产生阶段。是加工处理有关信息材料,生产并记录信息,制成文件正本,使之具有法定效用的过程。

    第二阶段,是文件的现实使用阶段。这时文件处于传递、运转、处理的过程之中,是为实现管理目标以便在现实工作中发挥实际效力的阶段。

    第三阶段,是文件的暂时保存阶段。对办理完毕的文件,除其中失去保存价值者予以销毁外,其余文件(休眠文件)暂时保存在机关档案室,过一段时间视实际需要决定最终归宿。

    第四阶段,是文件的永久保存阶段。即文件存入档案馆阶段,文件经鉴定、筛选后,其中具有永久参考价值者转化为档案,进入档案馆保存。

    不论是三阶段说,还是四阶段说,都是以文件生命周期理论为基础的。从文件生命周期管理理论产生的背景来看,它主要适用于纸质文件,尤其是档案的管理。事实上,文件生命周期理论并不完全适用于非纸质文件,至少是不适用于电子文件的,因为在电子文件的管理过程中,基本上不涉及物理空间的问题。所以,文件生命周期理论现在看来是十分有限的。而信息生命周期理论的适用范围就很广,可以涉及所有的信息资源。实际上文件生命周期就是信息生命周期在文件上的体现。所以说,信息生命周期实际上已经包含了文件生命周期。虽然信息生命周期是基于 IT 技术产生的,但是它是适用于各种信息资源的。

    `3文件运动的基本规律、定律与判断准则

        `s
            `a 基本规律:在一定条件下,文件能直接转化为档案,档案也可直接或间接转化为文件。
            `a 定律:当文件的现实目的已经实现或基本实现且具有备以查考价值和历史文化价值时,现行文件就直接转化为档案。当档案能在现实社会活动中发挥现行文件作用时,档案可以并能够直接或间接转化为现行文件。
            `a 判断准则:凡现实目的已经实现或基本实现了的现行文件都可以直接转化为档案,凡档案都是现实目的已经实现或基本实现了的前现行文件;凡具有查考价值与历史文化价值的现行文件都必然直接转化为档案,凡档案都必是具有查考价值与历史文化价值的前现行文件;凡能在现行社会活动中发挥现行文件作用的档案都可以直接或间接转化为有效现行文件,凡能转化为有效现行文件的档案都必须具有现行文件价值。
        `ss

    `3文件运动规律的再认识

        `z
            `a 文件运动具有多样性和不平衡性的特点。文件运动在一定主观与客观、内部与外部等因素的影响和制约下,通过文件与档案、主体与客体的相互作用,总是在阶段性的质变规律基础上按照螺旋式发展规律不断向前运动的。这一现象正是客观世界物质运动多样性和不平衡性在文件运动中的综合反映。只有认识到这一点,建立在“文件—档案”运动基础上的现行理论中才会有“档案—文件”运动的合法地位。

            `a 文件运动是文件的周期运动和周期连续运动有机结合的过程。从普遍性来说,“文件—档案”的转化是文件运动中普遍存在的显性运动方式,也是文件运动的主流。从特殊性来说,“档案—文件”的转化是文件运动中个别存在的隐性运动方式,也是文件运动的次要方式。由于它们的客观存在,并在其价值规律的作用下统一于一个共同的运动体内,因此才构成了一个完整的文件运动的循环过程。这一事实告诉我们,文件运动规律不仅要揭示“文件—档案”的微观运动规律,而且也要揭示“文件—档案—文件—档案”的宏观运动规律。

            `a 文件与档案相互转化是一定历史条件下文件、档案价值规律和社会需求规律相互作用的结果。

            `a 文件运动规律为正确认识文件与档案的性质及其关系提供了理论根据。文件运动规律说明:文件与档案在物质形式上是一种继承关系,在实质上是社会职能完全不同的两种事物;不同时期的文件、档案是在不同时期的社会活动中根据利用主体的需要而形成的,它们在适用对象和具体作用上都有明确的分工,在时间和空间上是不可替代的。
        `zz

    综上所述,深入探索文件运动规律,对于信息时代深入认识文件和档案的本质属性、科学构建文档管理模式及其理论体系,都是重要的和必要的。随着计算机与网络技术的应用,电子文件开始陆续出现。电子文件具有与纸质文件许多不同的特点,它的出现使许多传统的理论受到了严峻的挑战。于是,人们开始寻找新的理论来支持电子文件管理。在这种情况下,两种新的理论被提出,即电子文件生命周期理论和文件连续体理论。

    `3 电子文件生命周期理论

    国际档案理事会电子文件委员会在 1997 年的一份有关电子文件管理的报告中提出了与档案界传统文件生命周期理论内涵完全不同的电子文件生命周期(the life cycle of electronicrecords)的概念。它指出在文件产生之前,电子文件的生命周期就已开始,并借鉴软件工程学中软件生命周期的划分方法,将电子文件的整个生命周期划分为如下三个阶段。

    第一,概念阶段。概念阶段是指电子信息系统的设计、开发和安装阶段。这是相对于传统文件生命周期较为特殊的阶段,表明数字化条件下的文件档案一体化管理开始向前延伸到了文件管理系统的设计阶段。

    第二,形成阶段。形成阶段是指具体的电子文件在可靠的电子环境中产生出来的阶段。

    第三,维护阶段。维护阶段是指对形成后的电子文件进行技术维护和保管的阶段,亦即文件产生之后直至它被销毁或永久保存的整个过程。

    这一表述借鉴了计算机技术中“软件生命周期”之说,即将软件从开始计划到废弃不用的整个过程划分为计划阶段、设计阶段和维护阶段。它不是以文件内在价值的变化为依据,而是主要着眼于电子文件所处的状态,从电子文件管理系统的设计为开端,追踪电子文件从孕育、生成到保存的过程。其出发点主要在于强调电子文件的管理应实行“超前控制”,一些管理要求应该在设计阶段就加以考虑,并尽可能设计到系统中去。

    这一表述与档案界已有的文件生命周期理论的内涵是完全不同的。针对这一现象,国内外学者进行了深入的研究。不少人认为,电子文件作为文件的一种类型,应该有与其他文件基本相同的生命周期。三段论从技术角度描述了行为人对电子文件实施的行为,而不是每一份文件自身的生命和运动过程,参照软件生命周期理论提出的三段式模型没有清晰地揭示电子文件自身运动的生命周期,但更适合于作为电子文件管理系统生命周期的描述。

    电子文件生命周期理论应该是对以文件价值变化为基础的文件生命周期理论的继承和发展。文件生命周期理论在本质上适用于电子文件,同时,也需要按照电子文件的特点对这一理论进行部分的修正。根据目前的认识,电子文件生命周期理论的基本思想主要有以下几个方面:

    `e
        `a电子文件从其形成到销毁或永久保存是一个完整的、不可割裂的运动过程;
        `a这一过程可以根据电子文件的功能和价值形态的变化划分为若干阶段(如现行期、半现行期、非现行期),对不同阶段的管理需求应该统筹规划;
        `a电子文件在每一阶段因其特定的功能和价值形态而具有不同的服务对象和服务方式,但电子文件运动的阶段性与其物理位置、保存场所没有必然的对应关系;
        `a对电子文件生命周期全程的管理和监控措施由电子文件管理系统实现,因此,电子文件管理活动应该向前延伸到电子文件管理系统的设计阶段。
    `e

    电子文件生命周期理论是在文件价值变化基础上的文件生命周期理论的继承和发展,是电子文件全程管理和前端控制原则的依据。

    `3 文件连续体理论

    文件连续体理论是从文件形成(包括形成前,文件管理系统的设计)到文件作为档案保存和利用全过程中连贯一致的管理方式。文件连续体理论的形成经历了从 20 世纪中叶澳大利亚学者伊恩·麦克莱恩(Lan Maclean)的观点中萌芽,到 80 年代中期加拿大学者杰伊·阿瑟顿(Jay Atherton)明确提出“连续体”概念,再到 90 年代,澳大利亚的弗兰克·厄普沃德、苏·迈克米希(Sue Mckemmish)和英国的费林(Flynn)等学者的研究使文件连续体理论逐渐定型这样一个过程。其核心是一个文件连续体的模型,该模型由文件保管、业务活动、凭证、实体四条坐标轴和文件的生成、具有凭证作用的附加信息、组织记忆、记忆的多元化四维构成。文件连续体模式在加拿大和澳大利亚的研究与倡导下,得到了国际档案界的认可,而且一些电子文件管理专家和学者列出了文件连续体理论的种种优势,声称它应该取代文件生命周期理论。

    文件连续体理论的特点和重点在于从管理角度研究问题。它以文件的“形成(create)、”“捕获(capture)、”“组织(organize)、”“合成(pluralize)”四个管理步骤为主线,在一个多元时空的范围内,运用立体的、多维的研究方法,全方位地考察文件从最小保管单位直到组成最大保管单位的运动和管理过程,研究文件保管形式与业务活动和业务环境的互动。因而更确切地说,它描述的主要是文件管理规律和管理模式,同时也涉及了文件自身的运动规律问题。

    文件连续体理论认为:电子文件由于可以同时在不同的地点、场合发挥不同的作用,故而其运动并非呈线性状态,而是多点、多维、反复、不断进行的,已经无法为之划分明确的运动阶段。事实上,不论电子文件还是普通载体文件,都同样可以因为在任何时候同时被各种社会主体拿去在各种社会活动中利用,故能同时在不同的地点、场合发挥不同的作用,可以在多个背景和用途的范围内同时存在或积累。

    文件生命周期理论与文件连续体理论相比有如下的不同:

    `e
        `a文件生命周期理论侧重于对文件实体的研究,考察的是文件实体从形成直至销毁或永久保存的运动过程及规律;而文件连续体理论的研究视角是文件保管形式与业务活动和业务环境的互动,考察的是文件从最小保管单位直到组成最大保管单位的运动过程和规律。
        `a文件生命周期理论主要运用平面的、单维的研究方法,以文件的运动轨迹为脉络,研究的焦点集中在文件的生命历程,研究思路是线性的;文件连续体理论将文件置身于一个多元时空范围,转而运用立体的、多维的研究方法,展现出文件的整个运动过程。
        `a文件生命周期理论承认文件运动整体性的同时,将文件生命过程划分为若干个相对独立的阶段,实行阶段式管理,相应使得文件管理与档案管理的界限十分清晰;文件连续体理论则更多地突出文件运动的连续性和整体性,将文件视为一个无须明确分割的连续统一体。
        `a文件生命周期理论强调各阶段文件因其特定的价值形态而与保管场所、管理方式之间存在一定内在的联系;而文件连续体理论不再要求相关因素的机械对应。
        `a文件生命周期理论的理论基础是主客体关系论,强调主体与客体的对立和统一;而文件连续体理论以结构化理论为基础,强调结构是社会活动的中介,又是社会活动的结果。作为循环反复地卷入社会系统的生产和再生产要素的结构,不是存在于社会活动之外,而是与它融为一体。
        `a用连续体理论为导向的系统方法管理文件从根本上改变了文件保管者的角色。文件保管者不再在文件形成后才管理文件,而是主动超前地同其他保管者一起共同确定机构活动需要保管哪些文件,然后纳入事务活动体系进行管理。
    `ee

`2 文档一体化与文件中心

    `3 文档一体化

    所谓文档一体化就是将目前档案工作中各机关相对分散独立的文件管理和档案管理统一成一个有机整体进行管理。“文件、档案一体化管理”的基本思路是:文件管理是档案管理的前提;档案管理是文件管理的延伸和发展;文件管理和档案管理是一个统一的系统工程;档案部门和人员的参与,是文件管理质量的重要保证;文件管理人员与档案人员之间也存在相互促进的关系。

    依据文件生命周期理论,文件从其产生形成到最终销毁或作为档案永久保存是一个完整的生命过程。这一过程可分为若干阶段,由于各阶段有其不同的价值、作用和特点,各阶段文件也就存在着与之相对应的保管地点和存放方式。文件生命周期理论强调文件从始至终都是一个完整的、动态的、连续的生命运动过程。而文档管理工作中常因为标准和方法的不统一而出现文件与档案分离的不正常现象。文档分离带来的弊端造成大量失控的具有重要价值的账外文件,档案部门馆藏质量因此降低,进而导致不能很好地发挥参考服务作用,并导致档案部门信息服务功能降低,也导致了档案部门地位的降低。要解决这一实际工作的难题必须实现文档一体化。

    随着信息技术的发展,文件和档案的计算机管理提上了议事日程。文件档案的计算机管理避免了以往工作环节、程序的重复雷同,将文件和档案作为统一的系统进行管理。加强文件管理的超前控制保证了档案的质量从而充分发挥了档案的作用。档案工作者提前介入文件的生命周期从信息源头做起的思想在世界范围内受到一致欢迎。

    `3 文件中心

    在文件运动的不同阶段需要不同的管理方式、保管场所,其中暂时保存阶段文件尤为特殊。这一阶段的文件要经受时间的考验,有的可能已经没有价值可以销毁;有的一段时间内有价值过后再销毁;有的具有永久保存价值。因此需要一个过渡性的中间机构来专门对其进行整理、保管和利用,于是文件中心这一机构便应运而生了。

    文件中心的最初出现,是二战期间美国海军部为解决急剧增加的文件的存放问题而设置的临时库房。而这种临时库房之所以能演变成文件中心,主要有两方面原因。一是战后美国联邦政府各机构新产生的文件数量的迅速增长,其中半现行文件大量积累。二是长期以来,美国联邦机构内部没有设立类似欧洲国家“登记室”这种专职文件管理机构。各机构在探索集中保存这些半现行文件的方法时,发现海军部设置的临时库房经济、有效。因此,各机构纷纷仿效,成立了过渡性保存机构——文件中心。

    最早产生于美国并在国外许多国家广泛流行的文件中心是否适合我国国情、能否在我国建立和推广至今在学术界仍未取得完全一致的认识。我国档案工作在集中统一管理原则指导下在全国范围建立了各级各类档案室、档案馆和档案事业管理机构,构成一个严密完整的组织体系。文件也经历着从现行文件阶段到机关档案室阶段再到档案馆档案阶段的生命运动过程,也存在着类似文件中心的管理中间性和过渡性半现行文件的机关档案室。只要注意文件中心的适用范围,它可以成为我国档案管理机构的有益补充,文件中心能改变中小机关档案工作薄弱的现状。

    `3 文档一体化与文件中心的比较

    文档一体化与文件中心是有区别的,同时二者又是有联系的,但这种联系并不是将文件中心简单地作为文档一体化实现的一种方法模式。

    `z
    `a文档一体化与文件中心的区别。

    从文件生命周期理论角度来分析文档一体化是将文件生命运动阶段中相对分离的第二、三阶段统一起来,将以前分离的两个工作环节作为一个整体系统有序地结合起来使文件在生命运动的各个阶段都能得到连续的、统一的、标准化的管理。文档一体化要求在一个统一的行政管理体制下、在一个统一的组织机构内有统一的制度、统一的互相衔接的工作程序、方法以及统一的控制中心。把现行文件和半现行文件纳入统一的管理体系,将原本相对独立的、具体环节互相雷同的两种管理体系糅合在一起加强文件的超前控制,走信息开发一体化的道路不仅适合于纸质等载体形式的实体档案,也适应电子时代信息化的需要。就现在而言,机关档案室的职能从收集、整理、鉴定、保管、统计、编目、编研和利用等 8 个环节来说,都有其实际工作内容。实行文档一体化以后,档案工作独立的职能将渗入到文件的第一、二阶段,形成一个不可分割的整体,加强了对文件整体宏观的控制,它的功能将更侧重于信息的开发利用而非保管。

    文件中心的建立是坚持将文件生命运动的第三个阶段作为独立环节,对多个单位处于暂时保存阶段的文件进行集中管理。从文件生命运动的整体性来说它使文书处理和档案管理更加分散独立。但是,由于文档的分散,文件中心又对各机关具体工作办理不了解,因而会出现机关不交文件而中心又不知的情况,会形成更多的账外文件,造成有重要价值档案的流失,造成国家档案财富的损失。

    文档一体化强调的是文件生命运动的第二、三阶段文件管理和档案管理实现统一有序,强调文件在其生命运动各阶段应保持连续统一,将相对独立的、环节雷同的两种管理体系交融起来,关注档案工作的提前介入。而文件中心只是文件生命运动第三阶段即暂时保存阶段文件的存放与管理方式,二者是有区别的。只有当文档一体化的研究涉及文件保管与存放的时候,文件中心才能作为除机关档案室外的一种有益补充形式出现。文件中心的研究始终是研究文件生命运动的一个阶段。文件中心只能是按照文档一体化思路进行改革时选择半现行文件管理地点和管理方式时的一种选择,它不能取代文档一体化。简单建立文件中心并不能解决文档一体化问题,文件中心只是文档一体化模式中的一个环节。

    `zz
