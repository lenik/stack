
\section {组织结构设计}

    组织结构(Organizational Structure)是指,对于工作任务如何进行分工、分组和协调合作。 组织结构(organizational structure)是表明组织各部分排列顺序、空间位置、聚散状态、联系方式以及各要素之间相互关系的一种模式,是整个管理系统的“框架”。 组织结构是组织的全体成员为实现组织目标,在管理工作中进行分工协作,在职务范围、责任、权利方面所形成的结构体系。组织结构是组织在职、责、权方面的动态结构体系,其本质是为实现组织战略目标而采取的一种分工协作体系,组织结构必须随着组织的重大战略调整而调整。

\subsection{设计要素}

    管理者在进行组织结构设计时,必须正确考虑6个关键因素:工作专业化、部门化、命令链、控制跨度、集权与分权、正规化。

    \subsubsection {工作专业化}

    20世纪初,亨利·福特(Henry Ford)通过建立汽车生产线而富甲天下,享誉全球。他的做法是,给公司每一位员工分配特定的、重复性的工作,例如,有的员工只负责装配汽车的右前轮,有的则只负责安装右前门。通过把工作分化成较小的、标准化的任务,使工人能够反覆地进行同一种操作,福特利用技能相对有限的员工,每10秒钟就能生产出一辆汽车。

    福特的经验表明,让员工从事专门化的工作,他们的生产效率会提高。今天,我们用工作专门化(work specialization)这个术语或劳动分工这类词汇来描述组织中把工作任务划分成若干步骤来完成的细化程度。

    工作专门化的实质是:一个人不是完成一项工作的全部,解成若干步骤,每一步骤由一个人独立去做。就其实质来讲,工作活动的一部分,而不是全部活动。

    20世纪40年代后期,工业化国家大多数生产领域的工作都是通过工作专门化来完成的。管理人员认为,这是一种最有效地利用员工技能的方式。在大多数组织中,有些工作需要技能很高的员工来完成,有些则不经过训练就可以做好。如果所有的员工都参与组织制造过程的每一个步骤,那么,就要求所有的人不仅具备完成最复杂的任务所需要的技能,而且具备完成最简单的任务所需要的技能。结果,除了从事需要较高的技能或较复杂的任务以外,员工有部分时间花费在完成低技能的工作上。由于高技能员工的报酬比低技能的员工高,而工资一般是反映一个人最高的技能水平的,因此,付给高技能员工高薪,却让他们做简单的工作,这无疑是对组织资源的浪费。

    通过实行工作专门化,管理层还寻求提高组织在其他方面的运行效率。通过重覆性的工作,员工的技能会有所提高,在改变工作任务或在工作过程中安装、拆卸工具及设备所用的时间会减少。同样重要的是,从组织角度来看,实行工作专门化,有利于提高组织的培训效率。挑选并训练从事具体的、重覆性工作的员工比较容易,成本也较低。对于高度精细和复杂的操作工作尤其是这样。例如,如果让一个员工去生产一整架飞机,波音公司一年能造出一架大型波音客机吗?最后,通过鼓励专门领域中进行发明创造,改进机器,工作专门化有助于提高效率和生产率。

    20世纪50年代以前,管理人员把工作专门化看作是提高生产率的不竭之源,或许他们是正确的,因为那时工作专门化的应用尚不够广泛,只要引入它,几乎总是能提高生产率。但到了60年代以后,越来越多的证据表明,好事做过了头就成了坏事。在某些工作领域,达到了这样一个顶点:由于工作专门化,人的非经济性因素的影响(表现为厌烦情绪、疲劳感、压力感、低生产率、低质量、缺勤率上升、流动率上升等)超过了其经济性影响的优势.

    现在,大多数管理人员并不认为工作专门化已经过时,也不认为它还是提高生产率的不竭之源。他们认识到了在某些类型的工作中工作专门化所起到的作用,以及使用过头可能带来的问题。例如,在麦当劳快餐店,管理人员们运用工作专门化来提高生产和售卖汉堡包、炸鸡的效率。大多数卫生保健组织中的医学专家也使用工作专门化。但是,像奥帝康公司和土星公司则通过丰富员工的工作内容,降低工作专门化程度而获得了成功。

    \subsubsection {部门化}

    一旦通过工作专门化完成任务细分之后,就需要按照类别对它们进行分组以便使共同的工作可以进行协调。工作分类的基础是部门化(departmenta—lization)。

    对工作活动进行分类主要是根据活动的职能。制造业的经理通过把工程、会计、制造、人事、采购等方面的专家划分成共同的部门来组织其工厂。当然,根据职能进行部门的划分适用于所有的组织。只有职能的变化可以反映组织的目标和活动。一个医院的主要职能部门可能有研究部、护理部、财会部等;而一个职业足球队则可能设球员人事部、售票部门、旅行及后勤部门等。这种职能分组法的主要优点在于,把同类专家集中在一起,能够提高工作效率。职能性部门化通过把专业技术、研究方向接近的人分配到同一个部门中,来实现规模经济。

    工作任务也可以根据组织生产的产品类型进行部门化,例如,在太阳石油产品公司(Sun Petroleum Products)中,其三大主要领域(原油、润滑油和蜡制品、化工产品)各置于一位副总裁统辖之下,这位副总裁是本领域的专家,对与他的生产线有关的一切问题负责,每一位副总裁都有自己的生产和营销部门。这种分组方法的主要优点在于:提高产品绩效的稳定性,因为公司中与某一特定产品有关的所有活动都由同一主管指挥。如果一个组织的活动是与服务而不是产品有关,每一种服务活动就可以自然地进行分工。比如,一个财会服务公司多半会设有税务部门、管理咨询部门、审计部等等,每个部门都会在一个产品或服务经理的指导下,提供一系列服务项目。

    还有一种部门化方法,即根据地域来进行部门划分。例如,就营销工作来说,根据地域,可分为东、西、南、北4个区域,分片负责。实际上,每个地域是围绕这个地区而形成的一个部门。如果一个公司的顾客分布地域较宽,这种部门化方法就有其独特的价值。

    位于纽约州北部的雷诺兹金属公司(Reynolds Metals)铝试管厂,生产过程由5个部门组成:铸造部、锻压部、制管部、成品部、检验包装运输部。这是一个根据生产过程来进行部门化的例子。公司这样做的主要原因在于,在铝 试管生产过程中, 由每个部门负责一个特定生产环节的工作。金属首先被铸造成巨大的胚料;然后送到锻压部,被挤压成铝管;再把铝管转送到试管部,由试管部负责把它们做成体积各异、形状不同的试管;然后把这些试管送给成品部,由它负责切割、清洗工作;最后,产品进入检验、包装、运输部。由于不同的环节需要不同的技术,因此这种部门化方法对于在生产过程中进行同类活动的归并提供了基础。

    过程部门化方法适用于产品的生产,也适用于顾客的服务。例如,如果你到一家州属机动车辆管理办公室去办驾驶执照,你必须跑好几个部门。在某个州,办理驾照必须经过3个步骤,每个步骤由一个独立部门负责:⑴负责核查工作的机动车辆分部;⑵负责办理驾照具体工作的驾照部;⑶负责收费的财务部。

    最后一种部门化方法是根据顾客的类型来进行部门化。例如,一家销售办公设备的公司可下设3个部门:零售服务部、批发服务部、政府部门服务部比较大的法律事务所可根据其服务对象是公司还是个人来分设部门。

    根据顾客;类型来划分部门的理论假设是,每个部门的顾客存在共同的问题和要求,因此;通过为他们分别配置有关专家,能够满足他们的需要。

    大型组织进行部门化时,可能综合利用上述各种方法,以取得较好的效果。例如,一家大型的日本电子公司在进行部门化时,根据职能类型来组织其各分部;根据生产过程来组织其制造部门;把销售部门分为7个地区的工作单位;又在每个地区根据其顾客类型分为4个顾客小组。但是,90年代有两个倾向较为普遍:第一,以顾客为基础进行部门化越来越受到青睐。为了更好地掌握顾客的需要,并有效地对顾客需要的变化作出反应,许多组织更多地强调以顾客为基础划分部门的方法。例如,施乐公司已取消了公司市场部的设置,把市场研究的专家排除在这个领域之外。这样使得公司能更好地了解谁是它的顾客,并更快地满足他们的需要。第二个倾向是,坚固的职能性部门被跨越传统部门界限的工作团队所替代。

    \subsubsection {命令链}

    20年前,命令链的概念是组织设计的基石,但今天它的重要性大大降低不过在决定如何更好地设计组织结构时,管理者仍需考虑命令链的意义。

    命令链(chain of command)是一种不间断的权力路线,从组织最高层扩展到最基层,澄清谁向谁报告工作。它能够回答员工提出的这种问题:“我有问题时,去找谁?”“我对谁负责?”

    在讨论命令链之前,应先讨论两个辅助性概念:权威和命令统一性。权威(authority)是指管理职位所固有的发布命令并期望命令被执行的权力。为了促进协作,每个管理职位在命令链中都有自己的位置,每位管理者为完成自己的职责任务,都要被授予一定的权威。命令统一性(unity of command)原则有助于保持权威链条的连续性。它意味着,一个人应该对一个主管,且只对一个主管直接负责。如果命令链的统一性遭到破坏,一个下属可能就不得不穷于应付多个主管不同命令之间的冲突或优先次序的选择。

    时代在变化,组织设计的基本原则也在变化。随着电脑技术的发展和给下属充分授权的潮流的冲击,现在,命令链、权威、命令统一性等概念的重要性大大降低了。《商业周刊》〈Business Week〉最近的一篇文章中有两段话为这种变化提供了很好的例证:

    3月中旬一个星期三的上午, 查尔斯·凯瑟困惑地扫视了一眼从公司配送中心送来的存货报告。根据电脑列印出来的报告, 玫瑰牌上光油只能保证3天的供货了, 远远低于公司要求的3周半的库存要求。但凯瑟知道, 公司设在密苏里州杰弗逊城的工厂两天前刚运来346箱(每箱12瓶)上光油, 玫瑰牌上光油一定是被抢购一空了。他便打开自己与生产线相联的电脑, 把批示输进去:在周四上午再生产400箱上光油。

    这是一位计划经理工作日程中的一段小插曲, 对不对?但事实上凯瑟不是管理人员,他只是生产线上的一名工人, 官方的头衔是“生产线协调员”, 是公司上百名工作于电脑网路上的工人中的一员。他们有权检查核对货物运送情况, 安排自己的工作负荷, 并经常从事以前属于管理人员领域的工作。

    现在一个基层雇员能在几秒钟内得到20年前只有高层管理人员才能得到的信息。同样,随着电脑技术的发展, 日益使组织中任何位置的员工都能同任何人进行交流,而不需通过正式渠道。而且,权威的概念和命令链的维持越来越无关紧要,因为过去只能由管理层作出的决策现在已授权给操作员工自己作决策。除此之外,随着自我管理团队、多功能团队和包含多个上司的新型组织设计思想的盛行,命令统一性的概念越来越无关紧要了。当然,有很许多组织仍然认为通过强化命令链可以使组织的生产率最高,但今天这种组织越来越少了。

    \subsubsection {控制跨度}

    一个主管可以有效地指导多少个下属?这种有关控制跨度(span of control)的问题非常重要,因为在很大程度上,它决定着组织要设置多少层次,配备多少管理人员。在其他条件相同时,控制跨度越宽,组织效率越高,这一点可以举例证明。

    假设有两个组织,基层操作员工都是4096名,如果一个控制跨度为4,另一个为8,那么控制跨度宽的组织比控制跨度窄的组织在管理层次上少两层,可以少配备800人左右的管理人员。如果每名管理人员年均薪水为40 000美元,则控制跨度宽的组织每年在管理人员薪水上就可节省3 200万美元。显然,在成本方面,控制跨度宽的组织效率更高。但是,在某些方面宽跨度可能会降低组织的有效性,也就是说,如果控制跨度过宽,由于主管人员没有足够的时间为下属提供必要的领导和支持,员工的绩效会受到不良影响。

    控制跨度窄也有其好处,把控制跨度保持在5人一6人, 管理者就可以对员工实行严密的控制。但控制跨度窄主要有3个缺点:第一,正如前面S指出的,管理层次会因此而增多,管理成本会大大增加。第二,使组织的垂直沟通更加复杂。管理层次增多也会减慢决策速度,并使高层管理人员趋于孤立。第三,控制跨度过窄易造成对下属监督过严,妨碍下属的自主性。

    近几年的趋势是加宽控制跨度。例如,在通用电气公司和雷诺金属公司这样的大公司中,控制跨度已达10人一12人,是15年前的2倍。汤姆·斯密斯是卡伯利恩公司(Carboline Co.)的一名地区经理,直接管辖27人,如果是在20年前,处于他这种职位的人,通常只有12名下属。

    加宽控制跨度,与各个公司努力降低成本、削减企业一般管理费用、加速决策过程、增加灵活性、缩短与顾客的距离、授权给下属等的趋势是一致的。但是,为了避免因控制跨度加宽而使员工绩效降低,各公司都大大加强了员工培训的力度和投入。管理人员已认识到, 自己的下属充分了解了工作之后,或者有问题能够从同事那儿得到帮助时,他们就可以驾驭宽跨度的控制问题。

    \subsubsection {集权与分权}

    在有些组织中,高层管理者制定所有的决策,低层管理人员只管执行高层管理者的指示。另一种极端情况是,组织把决策权下放到最基层管理人员手中。前者是高度集权式的组织,而后者则是高度分权式的。

    集权化(centralization)是指组织中的决策权集中于一点的程度。这个概念只包括正式权威,也就是说,某个位置固有的权力。一般来讲,如果组织的高层管理者不考虑或很少考虑基层人员的意见就决定组织的主要事宜,则这个组织的集权化程度较高。相反,基层人员参与程度越高,或他们能够自主地作出决策,组织的分权化(decentralization)程度就越高。

    集权式与分权式组织在本质上是不同的。在分权式组织中,采取行动、解决问题的速度较快,更多的人为决策提供建议,所以,员工与那些能够影响他们的工作生活的决策者隔膜较少,或几乎没有。

    近年来,分权式决策的趋势比较突出,这与使组织更加灵活和主动地作出反应的管理思想是一致的。在大公司中,基层管理人员更贴近生产实际,对有关问题的了解比高层管理者更详实。因此,像西尔斯和盘尼(J.C.Penny)这样的大型零售公司,在库存货物的选择上,就对他们的商店管理人员授予了较大的决策权。这使得他们的商店可以更有效地与当地商店展开竞争。与之相似,蒙特利尔银行把它在加拿大的1 164家分行组合成236个社区,即在一个有限地域内的一组分行,每个社区设一名经理,他在自己所辖各行之间可以自由巡视,各个分行之间最长距离不过20分钟的路程。他对自己辖区内的问题反应远远快于公司总部的高级主管,处理方式也会更得当。IBM的欧洲总监瑞纳托·瑞沃索采取类似的办法把欧洲大陆的公司分成200个独立自主的商业单位,每个单位都有自己的利润目标、员工激励方式、重点顾客。“以前我们习惯于自上而下的管理,像在军队中一样。”瑞沃索说,“现在,我们尽力使员工学会自我管理。

    \subsubsection {正规化}

    正规化(formalization)是指组织中的工作实行标准化的程度。如果一种工作的正规化程度较高,就意味着做这项工作的人对工作内容、工作时间、工作手段没有多大自主权。人们总是期望员工以同样的方式投入工作,能够保证稳定一致的产出结果。在高度正规化的组织中,有明确的工作说明书,有繁杂的组织规章制度,对于工作过程有详尽的规定。而正规化程度较低的工作,相对来说,工作执行者和日程安排就不是那么僵硬,员工对自己工作的处理许可权就比较宽。由于个人许可权与组织对员工行为的规定成反比,因此工作标准化程度越高,员工决定自己工作方式的权力就越小。工作标准化不仅减少了员工选择工作行为的可能性,而且使员工无需考虑其他行为选择。

    组织之间或组织内部不同工作之间正规化程度差别很大。一种极端情况是,众所周知,某些工作正规化程度很低,如大学书商(向大学教授推销公司新书的出版商代理人)工作自由许可权就比较大,他们的推销用语不要求标准划一。在行为约束上,不过就是每周交一次推销报告,并对新书出版提出建议。另一种极端情况是那些处于同一出版公司的职员与编辑位置的人。他们上午8点要准时上班,否则会被扣掉半小时工资,而且,他们必须遵守管理人员制定的一系列详尽的规章制度。

\subsection {四大结构}

    组织结构一般分为职能结构、层次结构、部门结构、职权结构四个方面。

    1、职能结构:是指实现组织目标所需的各项业务工作以及比例和关系。其考量维度包括职能交叉(重叠)、职能冗余、职能缺失、职能割裂(或衔接不足)、职能分散、职能分工过细、职能错位、职能弱化等方面。

    2、层次结构:是指管理层次的构成及管理者所管理的人数(纵向结构)。其考量维度包括管理人员分管职能的相似性、管理幅度、授权范围、决策复杂性、指导与控制的工作量、下属专业分工的相近性等。

    3、部门结构:是指各管理部门的构成(横向结构)。其考量维度主要是一些关键部门是否缺失或优化。

    从组织总体型态,各部门一、二级结构进行分析。

    4、职权结构:是指各层次、各部门在权力和责任方面的分工及相互关系。主要考量部门、岗位之间权责关系是否对等。

\subsection {三个内容}

    企业组织架构包含三个方面的内容:

    \begin{
    单位、部门和岗位的设置。

    企业组织单位、部门和岗位的设置,不是把一个企业组织分成几个部分,而是企业作为一个服务于特定目标的组织,必须由几个相应的部分构成,就像人要走路就需要脚一样。它不是由整体到部分进行分割,而是整体为了达到特定目标,必须有不同的部分。这种关系不能倒置。

    各个单位、部门和岗位的职责、权力的界定。

    这是对各个部分的目标功能作用的界定。如果一定的构成部分,没有不可或缺的目标功能作用,就像人的尾巴一样会萎缩消失。这种界定就是一种分工,但却是一种有机体内部的分工。嘴巴可以吃饭,也可以用于呼吸。

    单位、部门和岗位角色相互之间关系的界定。

    这就是界定各个部分在发挥作用时,彼此如何协调、配合、补充、替代的关系。

    这三个问题是紧密联系在一起的,在解决第一个问题的同时,实际上就已经解决了后面两个问题。但作为一大项工作,三者存在一种彼此承接的关系。我们要对组织架构进行规范分析,其重点是第一个问题,后面两个问题是对第一个问题的进一步展开。

    企业组织架构设计规范的要求。

    对于这个问题,如果没有一个组织架构设计规范分析工具,就会陷入众说纷纭、莫衷一是的境地。我们讲企业组织架构设计规范化,也就是要达到企业内部系统功能完备、子系统功能担负分配合理、系统功能部门及岗位权责匹配、管理跨度合理四个标准。

\subsection {制度形式}

    \subsubsection {1、直线制}

    直线制是一种最早也是最简单的组织形式。它的特点是企业各级行政单位从上到下实行垂直领导,下属部门只接受一个上级的指令,各级主管负责人对所属单位的一切问题负责。厂部不另设职能机构(可设职能人员协助主管人工作),一切管理职能基本上都由行政主管自己执行。直线制组织结构的优点是:结构比较简单,责任分明,命令统一。缺点是:它要求行政负责人通晓多种知识和技能,亲自处理各种业务。这在业务比较复杂、企业规模比较大的情况下,把所有管理职能都集中到最高主管一人身上,显然是难以胜任的。因此,直线制只适用于规模较小,生产技术比较简单的企业,对生产技术和经营管理比较复杂的企业并不适宜。

    \subsubsection {2、职能制}

    职能制组织结构,是各级行政单位除主管负责人外,还相应地设立一些职能机构。如在厂长下面设立职能机构和人员,协助厂长从事职能管理工作。这种结构要求行政主管把相应的管理职责和权力交给相关的职能机构,各职能机构就有权在自己业务范围内向下级行政单位发号施令。因此,下级行政负责人除了接受上级行政主管人指挥外,还必须接受上级各职能机构的领导。

    职能制的优点是能适应现代化工业企业生产技术比较复杂,管理工作比较精细的特点;能充分发挥职能机构的专业管理作用,减轻直线领导人员的工作负担.但缺点也很明显:它妨碍了必要的集中领导和统一指挥,形成了多头领导;不利于建立和健全各级行政负责人和职能科室的责任制,在中间管理层往往会出现有功大家抢,有过大家推的现象;另外,在上级行政领导和职能机构的指导和命令发生矛盾时,下级就无所适从,影响工作的正常进行,容易造成纪律松弛,生产管理秩序混乱。由于这种组织结构形式的明显的缺陷,现代企业一般都不采用职能制。

    \subsubsection {3、直线-职能制}

    直线-职能制,也叫生产区域制,或直线参谋制。它是在直线制和职能制的基础上,取长补短,吸取这两种形式的优点而建立起来的。目前,我们绝大多数企业都采用这种组织结构形式。这种组织结构形式是把企业管理机构和人员分为两类,一类是直线领导机构和人员,按命令统一原则对各级组织行使指挥权;另一类是职能机构和人员,按专业化原则,从事组织的各项职能管理工作。直线领导机构和人员在自己的职责范围内有一定的决定权和对所属下级的指挥权,并对自己部门的工作负全部责任。而职能机构和人员,则是直线指挥人员的参谋,不能对直接部门发号施令,只能进行业务指导。

    直线-职能制的优点是:既保证了企业管理体系的集中统一,又可以在各级行政负责人的领导下,充分发挥各专业管理机构的作用。其缺点是:职能部门之间的协作和配合性较差,职能部门的许多工作要直接向上层领导报告请示才能处理,这一方面加重了上层领导的工作负担;另一方面也造成办事效率低。为了克服这些缺点,可以设立各种综合委员会,或建立各种会议制度,以协调各方面的工作,起到沟通作用,帮助高层领导出谋划策。

    \subsubsection {4、事业部制}

    事业部制最早是由美国通用汽车公司总裁斯隆于1924年提出的,故有“斯隆模型”之称,也叫“联邦分权化”,是一种高度(层)集权下的分权管理体制。它适用于规模庞大,品种繁多,技术复杂的大型企业,是国外较大的联合公司所采用的一种组织形式,近几年中国一些大型企业集团或公司也引进了这种组织结构形式。事业部制是分级管理 、分级核算、自负盈亏的一种形式,即一个公司按地区或按产品类别分成若干个事业部,从产品的设计,原料采购,成本核算,产品制造,一直到产品销售,均由事业部及所属工厂负责,实行单独核算,独立经营,公司总部只保留人事决策,预算控制和监督大权,并通过利润等指标对事业部进行控制。也有的事业部只负责指挥和组织生产,不负责采购和销售,实行生产和供销分立,但这种事业部正在被产品事业部所取代。还有的事业部则按区域来划分。

    \subsubsection {5、模拟分权制}

    这是一种介于直线职能制和事业部制之间的结构形式。

    许多大型企业,如连续生产的钢铁、化工企业由于产品品种或生产工艺过程所限,难以分解成几个独立的事业部。又由于企业的规模庞大,以致高层管理者感到采用其他组织形态都不容易管理,这时就出现了模拟分权组织结构形式。所谓模拟,就是要模拟事业部制的独立经营,单独核算,而不是真正的事业部,实际上是一个个“生产单位”。这些生产单位有自己的职能机构,享有尽可能大的自主权,负有“模拟性”的盈亏责任,目的是要调动他们的生产经营积极性,达到改善企业生产经营管理的目的。需要指出的是,各生产单位由于生产上的连续性,很难将它们截然分开,就以连续生产的石油化工为例,甲单位生产出来的“产品”直接就成为乙生产单位的原料,这当中无需停顿和中转。因此,它们之间的经济核算,只能依据企业内部的价格,而不是市场价格,也就是说这些生产单位没有自己独立的外部市场,这也是与事业部的差别所在。

    模拟分权制的优点除了调动各生产单位的积极性外,就是解决企业规模过大不易管理的问题。高层管理人员将部分权力分给生产单位,减少了自己的行政事务,从而把精力集中到战略问题上来。其缺点是,不易为模拟的生产单位明确任务,造成考核上的困难;各生产单位领导人不易了解企业的全貌,在信息沟通和决策权力方面也存在着明显的缺陷。

    \subsubsection {6、矩阵制}

    在组织结构上,把既有按职能划分的垂直领导系统,又有按产品(项目)划分的横向领导关系的结构,称为矩阵组织结构。

    矩阵制组织是为了改进直线职能制横向联系差,缺乏弹性的缺点而形成的一种组织形式。它的特点表现在围绕某项专门任务成立跨职能部门的专门机构上,例如组成一个专门的产品(项目)小组去从事新产品开发工作,在研究、设计、试验、制造各个不同阶段,由有关部门派人参加,力图做到条块结合,以协调有关部门的活动,保证任务的完成。这种组织结构形式是固定的,人员却是变动的,需要谁,谁就来,任务完成后就可以离开。项目小组和负责人也是临时组织和委任的。任务完成后就解散,有关人员回原单位工作。因此,这种组织结构非常适用于横向协作和攻关项目。

    矩阵结构的优点是:机动、灵活,可随项目的开发与结束进行组织或解散;由于这种结构是根据项目组织的,任务清楚,目的明确,各方面有专长的人都是有备而来。因此在新的工作小组里,能沟通、融合,能把自己的工作同整体工作联系在一起,为攻克难关,解决问题而献计献策,由于从各方面抽调来的人员有信任感、荣誉感,使他们增加了责任感,激发了工作热情,促进了项目的实现;它还加强了不同部门之间的配合和信息交流,克服了直线职能结构中各部门互相脱节的现象。

    矩阵结构的缺点是:项目负责人的责任大于权力,因为参加项目的人员都来自不同部门,隶属关系仍在原单位,只是为“会战”而来,所以项目负责人对他们管理困难,没有足够的激励手段与惩治手段,这种人员上的双重管理是矩阵结构的先天缺陷;由于项目组成人员来自各个职能部门,当任务完成以后,仍要回原单位,因而容易产生临时观念,对工作有一定影响。

    矩阵结构适用于一些重大攻关项目。企业可用来完成涉及面广的、临时性的、复杂的重大工程项目或管理改革任务。特别适用于以开发与实验为主的单位,例如科学研究,尤其是应用性研究单位等。

    \subsubsection {7、委员会 }

    委员会是组织结构中的一种特殊类型,它是执行某方面管理职能并以集体活动为主要特征的组织形式。实际中的委员会常与上述组织结构相结合,可以起决策、咨询、合作和协调作用。

    优点:①可以集思广益;②利于集体审议与判断;③防止权力过分集中;④利于沟通与协调;⑤能够代表集体利益,容易获得群众信任;⑥促进管理人员成长等。

    缺点:①责任分散;②议而不决;③决策成本高;④少数人专制等。

    \subsubsection {8、多维立体组织结构}

    这种组织结构是事业部制与矩阵制组织结构的有机组合。多用于多种产品,跨地区经营的组织。

    优点:对于众多产品生产机构,按专业、按产品、按地区划分;管理结构清晰,便于组织和管理。

    缺点:机构庞大,管理成本增加、信息沟通困难

    通用咨询(天津)有限公司为多种产品制造企业设计、实施多维立体组织结构的方案与咨询。

\subsection {基本配置}

    明茨伯格提出的五种组织结构配置是创业结构、机器官僚结构、专业官僚结构、事业部结构和特别结构。以下描述了每一种结构配置。表\ref{tab:org1}概括了在战略实施时与适当的配置相关的特定的组织特点。

    \begin{table}[!htb]
        \small \centering
        \begin{tabular}{|*6{c|}}  \hline
            项目 & 创业结构 & 机器官僚结构 & 专业官僚结构 & 事业部结构 & 特别结构 \\ \hline
            战略和目标 & 发展、生存 & 防御、效率 & 分析、有效性 & 事业部、利润 & 探索、创新 \\ \hline
            年龄和规模 & 年轻、小 & 年老、大 & 多样 & 年老、非常大 & 年轻 \\ \hline
            技术 & 简单 & 机器但不自动化 & 服务 & \shortstack{可分开的、像\\机器官僚结构}
                & \shortstack{很复杂、经\\常是自动化} \\ \hline
            环境 & \shortstack{简单且动态的,\\有时是敌意的} & 简单而稳定 & 复杂而稳定
                & \shortstack{相当简单且\\稳定多种市场} & 复杂且动态 \\ \hline
            规范化 & 几乎没有 & 很多 & 几乎没有 & 在事业部内 & 几乎没有 \\ \hline
            结构 & 职能 & 职能 & 职能或产品 & 产品、综合 & 职能和产品 \\ \hline
            合作 & 直接监督 & 纵向联盟 & 横向联盟 & 总部员工 & 双向调整 \\ \hline
            控制 & 小组 & 官僚的 & 小组和官僚的 & 市场和官僚的 & 小组 \\ \hline
            文化 & 发展的 & 薄弱的 & 鲜明的 & 子文化 & 鲜明的 \\ \hline
            技术人员 & 没有 & 许多 & 几乎没有 & \shortstack{总部有许多\\操作控制人员} & 几乎没有 \\ \hline
            行政人员 & 几乎没有 & 许多 & 许多专业人员 & 总部和事业部分开
                & \shortstack{有许多但在\\项目小组内} \\ \hline
            组织的关键部分 & 高层管理 & 技术人员 & 生产核心 & 中心管理 & 支持者和技术核心 \\ \hline
        \end{tabular}
        \caption{战略组织结构} \label{tab:org1}
    \end{table}

    五种组织类型的特点:

    \begin{enumerate}

        \item 创业结构

        创业结构的组织常常是正处于组织生命周期第一阶段的、新的小型公司。组织以机器为核心,由总管理者和工人组成。这种结构只需要少数的辅助人员。不需要专门化和规范化,协调和控制来自于上层。公司的建立者拥有权力,并创造企业文化。没有规范化的工作程序,员工几乎没有决定权。这种组织适合于动态的环境。它可以迅速地调整,并与更大的、不善适应的组织进行成功的竞争。它必须具备适应性以建立市场。但是这种组织没有力量,容易受到突然变化的冲击。除非它很有适应能力,否则将会失败。

        \item 机器官僚结构

        机器官僚结构讲的是官僚制组织,这种组织很大,技术已经规范化,经常是为了大型生产。专门化和规范化程度很强,关键的决策来自于上层。因为这种组织不善调整,其环境简单而稳定。机器官僚结构拥有大量的技术和行政人员。技术人员包括工程师、市场调研人员、财务分析人员和系统分析人员,利用他们来对组织的其他部分进行检查,并对之程式化和规范化。技术人员是组织内的支配团体。机器官僚结构经常因为缺乏较低层员工的管理、缺乏创新、弱势的文化以及工作力量分散而遭到批评,但是它们适合于大型的、稳定的环境和效率目标。

        \item 专业官僚结构

        专业官僚结构的明显特征是生产的核心是由专业人员组成的,如医院、大学和咨询公司。

        尽管组织是官僚制的,但是生产核心的人员拥有自主权。长期的培训和经验促使这种结构的组织形成集体的管理和鲜明的文化,由此减少对官僚管理结构的需要。这些组织经常提供服务而不是有形的产品,它们存在于复杂的环境中。组织的绝大部分力量在于生产核心中的专业人员。技术群体很小或者不存在,但需要大量的行政管理人员来处理组织的日常事务。

        \item 事业部结构

        事业部结构的组织很大,往往根据产品或市场分成若干事业部。在事业部之间几乎没有联络措施进行协调,事业部通过损益报告强调对市场的控制。事业部的划分形式可以是相当规范化的,因为技术经常是有规可循的。

        尽管整个组织要服务于各种市场,但是任何事业部的环境都是简单而稳定的。每一个事业部在一定程度上都是自主的,并拥有自己的文化。在事业部的内部存在分权,总部人员会保留一些职能,如计划和调研。

        \item 特别结构

        特别结构是为了在复杂的动态环境中求得生存而提出的。该技术很复杂,如宇航和电子工业。特别结构组织的年龄就像是年轻人或中年人,其规模相当大,但需要适应。在团队基础上建立的结构有很多横向的联合和被授权的员工。技术人员和生产核心人员对于关键的生产要素均有权力。’组织有详细的劳动分工,但不拘泥于形式。员工的专业化程度很高,文化价值观鲜明,强调群体的控制。,通过分权,在任何层次的人员都可以参与决策。就结构、权力关系和环境而言,特别结构与机器官僚结构几乎正好相反。

    \end{enumerate}

    五个结构配置的要点就是最高管理层能够设计出得以协调和使关键要素互相匹配的组织。例如,机器官僚结构适合于在稳定的环境中争取高效率的战略;但在敌对的和动态的环境中,采用机器官僚结构就是个错误。管理者可以通过设计适合所处环境的正确的结构配置来实施战略。

\subsection {设计规范}

    企业组织架构设计规范方法——目标功能树系统分析模型

    1、如何解决组织架构设计规范化的四个方面的问题?

    专门探索解决组织架构设计的科学方法,到目前为止,还很少有人进行探索,也没有见诸书刊的文献。在这种情况下,对组织架构设计规范化的讨论,就仅仅变成了一种讨论者个人在组织架构上的主观偏好的陈述。

    我们说规范与否,在这里并不是一种简单的价值判断,不是由自己的主观偏好来认定什么是好、什么是坏,而是要把握所规范的对象本身的性质,用所要规范的对象自身的内在逻辑联系和发展规律,来定义规范。也就是说,不是从外部向要规范的对象强加一个什么,而是让要规范的对象本身按照它应有的形式和规律运行和发展。只有这样,才能使规范化管理这一工作达到应该有的提升企业价值的目的。企业规范化管理,必须在企业价值的提升——管理效率和管理效益的改善上有充分体现。我们在探索对企业规范化管理问题时,一再强调要寻求一个科学的方法,其最根本的一点就是如何解决管理的效率和效益问题。

    2、规范企业组织架构设计的科学方法是什么?

    在回答这一问题之前,首先必须明白企业组织本身是什么这一问题。

    前面已作过分析,企业组织是一个有机系统,是存在于一个更大系统之中的有机系统,其内部又可以细分为很多子系统。而企业组织作为一个由人构成的社会组织,本身是具有目的性的,也就是说它的存在是服务于人的特定目的的,它是人为达到特定目的而创建的。企业组织与自然存在物不一样,它是由人创造的,它必须服务于创造它的人的意志和目的。从这个意义上讲,可以说它是人们为达到一种特定目的而由人自己创造的一个工具。

    企业作为由人创建的一个工具性的社会组织,其目的和目标是显然的,其内部架构必须服务于这特定的目的和目标,也是显而易见的。它的这种目的性和功能性特征,为我们寻找到对它进行规范的科学方法提供了线索。这方法就是与它的目标功能特性直接对应的目标功能树系统分析模型。

    所谓目标功能树系统分析模型,也就是通过对分析对象本身所存在的目标功能结构进行系统分析,以分析确定分析对象的内在结构和发展运行的规律。

    由人所创造的存在物有一个共同的特点,这就是它们具有目标和功能这样一种多层次的结构。山水草木本身的存在没有任何目的性,在它的内部也就无法对它区别出目标和功能这样的层次结构来。草木本身没有意志,当它被人选做达成某一特定目的的特殊工具时,也就是人赋予了它特定的目的性。能够实现这一目的的作用也就成了它的功能。当这种自然存在物被选为人的特定工具的时候,它也就不再是完整意义上的自然存在物,而是被注入了人的意志目标的手段和工具。在这里的目标,实际上是人的目标,其功能是它相对于这种目标的作用和性质。

    而目标和功能并不是截然对立的,而是相互依存的。相对于功能作用,目标才成其为目标。功能作用只有相对于一定目标,它才成其为功能。呼吸是肺的功能作用,但它只有相对于需要呼吸的动物才有这种功能作用。需要就是一种特定的目的或目标。并且目标和功能本身的定义也是相对的,在一个复杂的系统结构中,目标和功能是在多重层次上存在的。为实现一定的目标,必须有相应的功能;为保证一定功能的正常发挥,又必须有一系列细小的功能。上一层次的功能相对于下一层次的功能,也就成了目标。

    通过这种目标功能树分析,可很方便有效地理清系统内部的层次结构。就企业组织这一特定系统而言,通过运用目标功能树系统分析模型对它进行分析,就可准确地为企业组织架构的设计提供一个框架性工具。这种分析,不仅有助于我们确定企业不同时段上要达成的目标,而且有助于我们一层一层地选择确定为达成企业目标而必须采取的具体措施办法。

    就企业系统进行分析,企业的目标就是要赚钱。企业通过什么途径赚钱?如何才能赚钱?稍加分析就会发现企业系统,是由信息(信息流)、组织(人流)、营销(物流)、财务(资金流)四大系统构成的。这是就第一个层次的目标功能作用进行的分析。如果要进一步地细分,进入第三和第四个层次的目标功能作用关系的分析,就可得到61个小的子系统。下面就企业的“四流”形成的系统分别进行分析,以明确其内部结构,及其各个层次上的子系统的目标功能作用。

\subsection {规范实施}

    前面已经分析了组织架构的标准以及要规范组织架构必须运用的方法——目标功能树系统分析模型。但究竟如何实施组织架构的规范化,在此略做分析。

    第一步,选择确定组织架构的基础模式。这一步工作要求根据自己企业的实际,选择确定一个典型的组织模式,作为企业的组织架构的基础模式。在当代企业的实践中,选择直线职能式和矩阵式结构的较普遍,并有越来越多的企业选择增加弹性模式的相应特征予以补充其基本模式的局限。

    第二步,分析确定担负各子系统目标功能作用的工作量。这一步工作要求根据目标功能树系统分析模型,分析确定自己企业内部各个子系统目标功能作用的担负工作量。要考虑的变数有二:一是企业的规模;二是企业的行业性质。

    第三步,确定职能部门。这一步工作要求根据自己企业内部各个子系统的工作量大小和不同子系统之间的关系,来确定企业职能管理部门。即把关联关系和独立关系,并且工作量不大的子系统的目标功能作用合并起来,由一个职能管理部门作为主承担单位,负责所合并子系统的目标功能作用工作的协调和汇总。把制衡关系的子系统的目标功能作用分别交由不同单位、部门或岗位角色承担。

    第四步,平衡工作量。这一步工作要求对所拟定的各个单位、部门的工作量进行大体的平衡。因为工作量过大的单位、部门往往会造成管理跨度过大,工作量过小的单位、部门,往往会造成管理跨度过小。所以,需要通过单位、部门之间的工作量平衡来使管理跨度实现合理化。在这里,要注意的一点是:存在制衡关系的子系统,要避免将其目标功能作用划归为同一单位承担,即要优先保证制衡关系子系统的目标功能作用的分开承担。

    第五步,确立下级对口单位、部门或岗位的设置。如果企业下属的子公司、独立公司、分公司规模仍然比较大,上级职能管理部门无法完全承担其相应子系统目标功能作用的工作协调和汇总,就有必要在这个层次上设置对口的职能部门或者专员岗位。

    第六步,绘制组织架构图。这一步工作要求直观地构画出整个企业的单位、部门和岗位之间的关系,及所承担的子系统目标功能作用的相应工作。

    第七步,拟定企业系统分析文件。这一步工作也就是为企业组织架构确立规范。企业系统分析文件是具体描绘企业内部各个子系统的目标功能作用,该由哪些单位、部门或者岗位来具体承担,以及所承担的内容,并对职责和权力进行界定。

    第八步,根据企业系统分析文件撰写组织说明书。这一步工作就是在组织构图的基础上,分析界定各个单位、部门组织和岗位的具体工作职责、所享有的权力、信息传递路线、资源流转路线等。

    第九步,拟定单位、部门和岗位工作标准。明确界定各个单位、部门和岗位的工作职责、工作目标、工作要求。

    第十步,根据企业系统分析文件、组织说明书及单位、部门和岗位工作标准进行工作分析,并撰写工作说明书。除了界定前述内容外,还要明确界定任职的条件和资格。

    第十一步,就上述文件进行汇总讨论,通过后正式颁布,组织架构调整改造工作完成。

\subsection {诊断维度}

    在对企业的组织结构进行系统的审视时我们往往从以下四个维度展开:业务结构、职能结构、层次结构、职权结构。

    \subsubsection {一、业务结构}

    在组织存在多项业务时,我们审视组织各项业务的分工结构及组织资源的配比情况。具体到单项业务,我们从业务流程切入,审视组织部门的设置是否足以覆盖该业务流程且不重叠。按照罗宾斯对组织的研究,业务部门的划分有以下几种方式,各种方式有不同的优缺点,在实际操作中我们可以灵活掌握。

        \begin{enumerate}
            \item 按产品划分部门

            优点:有利于产品改进、有利于部门内协调缺点:部门化倾向(本位主义),管理费用高(机构重叠)

            适用:规模大、产品多、产品之间差异大。

            \item 按地区划分:把某一地区的业务集中于某一部门。

            原因:地理分散带来的交通不便和信息沟通困难优点:针对性强,能对该地区环境变化迅速作出反应

            缺点:与总部之间协调困难(不易控制)

            \item 按顾客划分:其前提每个部门所服务的特定顾客有共同需求、且数量足够多。如:如一家办公用品公司的销售:零售部、批发部、政府部。

            \item 综合标准:实践中往往几种划分方法结合在一起

        \end{enumerate}

    \subsubsection {二、职能结构}

    在该维度中,我们审视两个问题,一是是否存在职能重叠或缺失的现象,尤其是组织所需的关键职能是否具备。二是职能部门是否定位清晰,是否有明确的使命。

    \subsubsection {三、层次结构}

    包括组织的管理层级和管理幅度。

    管理层级是随着组织规模的扩大和关系的复杂化而产生的,与规模、管理幅度密切相关。管理幅度是指一个主管人员能直接有效地管辖的下属人数。管理幅度与层次成反比关系。一般我们认为管理3-20名直接下属比较合适。其中,高层管理者管理3-10名下属;中层管理者管理6-15名下属;基层管理者管理15-20名下属比较合理,但并不绝对,以下因素也会影响到管理幅度:如管理者的素质和能力、下属的素质和能力、工作相似性、工资环境的稳定性、计划的完善程度、授权、人员空间分布、配备助手等等。

    \subsubsection {四、职权结构}

    是指各部门、各层次在权利和责任方面的分工和相互关系。按照罗宾斯的理解,职权分三种:

        \begin{enumerate}
            \item 直线职权:上下级之间的指挥、命令关系。也就是我们通常说的“指挥链”。

            \item 参谋职权:组织成员向管理者提供咨询、建议的权力。该职权源于直线人员对专业知识的需要,如财务、质量、人事、公关等。

            \item 职能职权:参谋部门或参谋人员拥有的原属直线人员的一部分权力。该职权是直线人员由于专业知识不足而将部分指挥授予参谋人员,使他们在某一职能范围内行使指挥权。职能职权只有在其职能范围内才有效。是一种有限指挥权。
        \end{enumerate}

    在对职权结构的审视中,我们需要把握两个要点:一是授权是否合理?二是信息沟通是否顺畅?

    通过以上四个维度,我们通常能够对企业的组织结构进行一个系统的剖析。当然,每个企业都有其自身的特点和背景,所面临的问题各不相同,需要我们在实际工作中灵活掌握。

\subsection {改革时机}

    \begin{enumerate}
        \item 企业战略发生巨大变化,组织结构已经难以适应
        \item 企业所处发展阶段发生变化,组织结构成为发展的制约
        \item 组织人事或管理模式发生变化,急需对组织结构做出调整
        \item 外部市场发生变化,竞争对手的网络发生变化,亟需调整组织结构
        \item 组织结构臃肿、协调困难、沟通不畅、决策缓慢,亟需优化组织结构
        \item 组织人浮于事,官僚作风
        \item 信息不畅,决策执行走样
    \end{enumerate}

\subsection {组织架构}

    首先要衡量公司的组织架构现在是什么状况,你可以看一下你们公司各部门在哪一个位置?我们缺乏什么职能,这是我们考虑宏观组织架构的出发点,这个组织架构来源企业的战略,宏观流程和组织架构。很多企业在变革时,人力资源需要参与。

    任何一个组织架构要反映汇报关系,要反映每个框是一个职位,而不是部门。一是要明确反映出岗位和岗位间的关系,二是要反映出岗位和岗位是如何组合的,是如何组成团队的;三是要这个图上看出大概的层级的关系,分为几个层面。有非常多的人轻视这样的组织架构。

    \subsubsection {功能型组织架构}

        最普遍的是职能型或者说是功能型组织架构,在组织架构或者团队建设最容易接收。这样的架构里,成功者是需要专长的人,对自己的工作了解深入,让做什么就做得非常好。其实这就是岗位设计的意义一一通过不同的组合方式,让员工有更好的工作效率。​

        但这个组织架构更强调个人干个人的事情,缺点是导致层级太多。

    \subsubsection {事业部式}

        以产品或服务为核心的组织架构(事业部式):

        事业部式的架构很时髦,通常是按照产品或者客户、市场来分。所谓产品是指,比如说一个企业既生产汽车,又生产飞机,不同的产品就有不同的组织架构。

    \subsubsection {以窑户或地区为核心}

        按照客户来分,如B-B,B-C的B或C就是按照客户来分。大客户或公众客户,客户是企业客户,他的销售市场是一批人员来做;如果是个人用户,则是另外一批人员做。如果企业客户细分成学校、事业单位、政府机构等,客户群划分不同,做法又有不同。按照市场的做法是比较流行的做法,因为现在非常注意客户服务、客户满意度。这个组织架构可以最好地满足客户地需求。每个客户有一个客户经理,可以按照客户的需要进行工作。

    \subsubsection {业务流程团队组织架构}

        用的比较少,从解决方案,到客户服务、生产、物流,按照每个客户的流程,一步一步地完成。

    \subsubsection {矩阵式组织架构}

        组织机构的基本问题:管理结构

    \subsubsection {管理的层级}

        管理的层级数量多还是少。从CEO到UNE到底需要多少个层级?四个还是八个?要不要设副经理?扁平化组织问题。

    \subsubsection {控制跨度}

        直接下属的数量;窄与宽的控制跨度。

    \subsubsection {决策流程的集中性}

        企业主要决策,控制权力与职权的集中程度。

\subsection {体系}

    在管理学意义上,组织结构实质上是一种职权-职责关系结构。一个现代化的、健全的组织机构一般包括如下关系子系统:

    \begin{enumerate}
        \item 决策子系统

        组织的领导体系和各级决策机构及其决策者组成决策子系统。各级决策机构和决策者是组织决策的核心。

        \item 指挥子系统

        指挥子系统是组织活动的指令中心,在各职能单位或部门,其负责人或行政首脑与其成员组成垂直形态的系统。行政首脑的主要任务是实施决策机构的决定,负责指挥组织的各项活动,保证各项活动顺利而有效地进行。指挥子系统的设计应从组织的实际出发,合理确定管理层次,并根据授权原则,把指挥权逐级下授,建立多层次、有权威的指挥系统,来行使对组织各项活动的统一指挥。

        \item 参谋-职能子系统

        参谋-职能子系统是参谋或职能部门组成的水平形态的系统。各参谋或职能部门,是行政首脑的参谋和助手,分别负责某一方面的业务活动。设计参谋-职能子系统,要根据实际需要,按照专业分工原则,设置必要的参谋或职能机构,并规定其职责范围和工作要求,以保证有效地开展各方面的管理工作。

        \item 执行子系统、监督子系统和反馈子系统

        决策中心决定组织的大政方针,指挥中心是实施计划的起点,而执行子系统、监督子系统和反馈子系统是使计划得以正确无误地推行的机构。

        指挥中心发出指令,这个指令一方面通向执行机构,同时又发向监督机构,让其监督执行的情况。反馈机构通过对信息系统的处理,比较效果与指令的差距后,返回指挥中心。这样,指挥中心便可以根据情况发出新的指令。

        执行机构必须确切无误地贯彻执行指挥中心的指令。为了保证这一点,就应有监督机构监督执行情况,而反馈子系统是反映执行的效果。执行子系统、监督子系统和反馈子系统必须互相独立,不能合而为一。

    \end{enumerate}

\subsection {不合理的表现}

    组织结构设置时也通常会犯的一个普遍错误就是:将一种“理想的”或“普遍适用的”机械组织模式强加给另外一个活生生的企业。随着企业发展到一定的阶段,组织结构不合理的一些矛盾就会日益尖锐,其直接的表现为:

    1)组织中管理层次过多,以至于一个能干的人能够在25岁这样年轻的时候就担任了管理职务,却不能经过正常的升迁阶梯在自己仍相当年轻和富有效率的时候达到高层管理岗位;

    2)组织注意力集中在不恰当的问题上,加剧不必要的争论,小题大做;

    3)使弱点和缺陷加大,而不是使长处和优势加强;

    4)经理人花在会议上的时间,超过了他们工作的1/4或者更多,组织结构不合理的危险性不言而喻。这样的问题,造成了组织虚有其表。基于此,调整组织结构,以使其符合企业实际情况,就摆上了企业的议事日程。

\subsection {演变规律和发展趋势}

    (一)企业组织结构的演变规律

    从企业组织发展的历史来看,企业组织结构的演变过程本身就是一个不断创新、不断发展的过程,先后出现了直线制、矩阵制、事业部制等组织结构形式。当前,金字塔式的层级结构已不能适应现代社会特别是知识经济时代的要求。目前企业发展已经呈现出竞争全球化、顾客主导化和员工知识化等特点。故而,企业组织形式必须是弹性的和分权化的。因此,现代企业十分推崇流程再造、组织重构,以客户的需求和满意度为目标,对企业现有的业务流程进行根本性的再思考和彻底重建,利用先进的制造技术、信息技术以及现代化的管理手段,最大限度地实现技术上的功能集成和管理上的职能集成,以打破传统的职能型组织结构,建立全新的过程型组织结构,从而实现企业经营成本、质量、服务和效率的巨大改善,以更好地适应以顾客、竞争、变化为特征的现代企业经营环境。

    (二)企业组织结构的发展趋势和新型组织结构形态

    从在美国考察的实际情况来看,企业组织结构发展呈现出新的趋势,其特点是:1、重心两极化;2、外形扁平化;3、运作柔性化;4、结构动态化。团队组织、动态联盟、虚拟企业等新型的组织结构形式相继涌现,具体来说,具有这些特点的新型组织结构形态有:

    第一,横向型组织。横向型的组织结构,弱化了纵向的层级,打破刻板的部门边界,注重横向的合作与协调。其特点是:⑴组织结构是围绕工作流程而不是围绕部门职能建立起来的,传统的部门界限被打破;⑵减少了纵向的组织层级,使组织结构扁平化;⑶管理者更多的是授权给较低层次的员工,重视运用自我管理的团队形式;⑷体现顾客和市场导向,围绕顾客和市场的需求,组织工作流程,建立相应的横向联系。

    第二,无边界组织。这种组织结构寻求的是削减命令链,成员的等级秩序降到r最低点,拥有无限的控制跨度,取消各种职能部门,取而代之的是授权的工作团队。无边界的概念,是指打破企业内部和外部边界:打破企业内部边界,主要是在企业内部形成多功能团队,代替传统上割裂开来的职能部门;打破企业外部边界,则是与外部的供应商、客户包括竞争对手进行战略合作,建立合作联盟。

    第三,组织的网络化和虚拟化。无边界组织和虚拟组织是组织网络化和虚拟化的具体形式,组织的虚拟化,既可以是虚拟经营,也可以是虚拟的办公空间。

\section{人员管理方法}

    现代企业要获得较快的发展,内部必须具有相当的凝聚力,有效的人员管理是保证公司凝聚力的重要手段。企业管理的根本就是人员的管理,尤其当你升入公司的高层后,或者当企业发展壮大后,很多人会发现,管理的80\%的时间实在考虑人员管理问题。

    事业有成的老板都对他的各级员工进行系统的培训工作,这不但是由于关心员工们,而且是由于这样做对企业有利。只靠实践本身是不够的。如果某人错误地做某件事,而且还努力苦干,最终会完全错干下去,因为得不到纠正。

    许多公司发现他们有这样或那样使业务得不到改善的强烈欲望,但却并不真正知道该如何帮助公司达成目的。再加上许多有关培训方面的书籍充斥你的办公室(其中提供了各种这样那样的培训方案),这反而使他们感到迷惑。结果,他们只是对此耸耸肩,搁置一旁,继继原来的工作方式。

    考虑怎样使培训工作能提高每个人的工作能力,怎样用提升内部员工或新进员工补充未来的工作岗位,而不是靠挖角竞争对手来为你提供。以下几点将帮你沿着有效路线迈出第一步,使你看到培训员工对你做每件事的重大意义,帮助理解该怎么去做。

    \begin{enumerate}
        \item 帮助员工学习他们本职以外的工作

        以扩大他们的经验,在各种紧急状况时有处理能力,避免令人厌烦的重复性工作并提高效率。

        \item 向员工提供许多小册子,请义,而不要指望他们把一切东西都用笔记下来。

        除非他们会速记,否则就跟不上。假如你请你的员工记下这些最重要的问题,这会对他们有帮助。你可能记得以前对员工举行过考试并试图想起关于他们的信息,如果你过去得到过这种信息是通过眼睛而不是通过耳朵的,那么,现在你就很可能再得到原始记录。你回忆到的应该是视觉形象而不是声音。

        \item 要把培训看成是加速的经验

        它能使你的员工们从别人的成功和错误中学习经验教训,避免以困难方式学习时所付出的代价与辛苦。

        初学者要经过下列四个阶段:
            \begin{enumerate}
                \item 不自觉的不熟练——不能够做某事,甚至不知道自己的无知。
                \item 自觉的不熟练——开始知道了。
                \item 自觉的熟练——学会做某事,但必须特别专心致志于工作进程的每一阶段。
                \item 不自觉的熟练——能够不费力地完成工作。
            \end{enumerate}

        \item 当你的事业向前发展时,新产品、新体制、新政策和新市场都有进行培训的需要

        培训工作是无穷无尽的,没有培训就不能向前发展,而且随着公司发展速度的加快,对培训工作的需求也会增加。从上到下检查每个人的工作成绩 ,通过培训加强实力,克服弱点和发挥潜力。让员工认识到,培训是一种令人兴奋的机遇,不是使人不愉快的或是一种改正性的惩罚措施。

        \item 新来的员工会感到孤象征、紧张和不安,因为开始时他们不可能做出很大贡献。

        做出适当的培训计划,向他们介绍公司,你的部门和他们具体的工作情况。在新员工即将到来的第一天。在他们出发前来时做好准备迎接工作。要亲自欢迎他们,并且要介绍旧有的同事及相关人员。

        \item 记住要培训在公司范围内调转工作的人员。

        不需要对他们再作公司介绍,但必须介绍他们要去的部门和具体工作。从另一家公司转到你公司做类似原来的工作的人也需要按照你的工作方式进行培训,不学习可能会遇到困难。许多公司都有一项政策,宁愿培养自己的人员,不从竞争对手那去招聘。

        注意听新来员工说:“我们”一词,指的是你们公司,而不是他们不久前才离开的先前那家公司。这是一个重要的里程碑,这意味着他们已经把自己看成是你手下的员工了。

        \item 指派一名你的有经验验员工做联络工作指导员。

        新来员工可以向他求得帮助。最理想的人选是与新来者年龄和兴趣味相似,同他们合得来,使员工觉得他不是  他的培训教师,他更像是他们的好朋友。如果你让有经验的人去教新员工,要让这些人先接受培训,使他们不要散布坏习惯。菲雅特人才培训公司许多年来一直培训有经验的美容师,再去叫他们培训别人,然后,他们被提升到店长的职位上,这一制度培训,对美容界享有高质量声誉做出了重要贡献。

        \item 技能与知识能够教会,但人的态度只能靠榜样的力量才能让人学会。

        人的态度是有感染力的。你的员工们只有在你树立榜样的情况下才会是热心的、忠诚的、有用的、守时的和认真负责任的。

        \item 让员工无拘无束的接受培训

        因为,如果他们精神紧张,他们就不能很好学习。如果他们不懂,鼓励他们大胆提问题,不要让他们感觉到提问题就心有愧或显得愚笨。

        \item 评价培训工作的有效性

        它是否达到了你的目的;如果没有达到,原因是什么。强调从你的培训花费中所取得的价值与其他费用相比较,在你进行培训之前,不论是内部的还是外部的,都要认真地向人们作简要通报。要大家一致同意所订的目标并制定培训期间的行动计划。培训后要听取他们的汇报,并对照他们的计划检查他们的学习进展情况。

       \item 有计划地培养你的下属每一员工使之积极向上。

        如果不这样做,那些最具有发展前途的员工就会离职而去。如果你有足够时间去挑选你认为适合的具有更高培训练水平的人,比如说你要的是有三年训练水平的人,而现在有一个适合的但经验并不足三年的人选,您是准备放弃他另找,还是自己培养呢?

        \item 鼓励有雄心壮志的人

        有人采取主动,为了提高专业水平去学习,去读书。如果你认为一项课程是重要的,至少应该给予资助。  如果你们公司没有一个培训开发中心,应该成立一个,不久它将变成那些关心事业的人宝贵财富。(当然小公司可以不考虑,但要有准备思想,要有准备动作。)

       \item 实行工作轮换以加速那些有才能的人的经验传播

        被指定去担任最高职务的人需要有企业中尽可能多的部门的经验。

        日本的公司培养全面的经理人员,而不是专家。应该学习他们的榜样。在日本提升担任最重要经理岗位人员的提升年龄为33岁,而在西方则为27岁。人们必须在各种岗位上长时间任职以证明他们的能力,并看到他们的决策的效果。这样至少要花费12至18个月的时间。在单层的组织机构中,(例如:美容院)大多数工作变更都是水平式的,要考虑增加人们的知识和广度与责任的方法。

       \item 邀请顾客来评论你们的服务水平,并请他们提出可能改善服务的方法

        培训与顾客接触的全部员工,不论是面对面或通过电话或函件与顾客联系,都要他们很好的与顾客进行业务及服务交往。即使有的知识对他们并不重要,但是也能使他们感到自己在关心企业的发展。

    \end{enumerate}
