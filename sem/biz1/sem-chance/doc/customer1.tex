\subsection {客户跟进技巧与实例分析}

    现在无论是做网络营销还是电话营销,客户跟进都很重要,因为客户有时候是需要我们保持长期联系的,一旦你不联系客户,客户有可能就改变主意了或者是觉得你的服务有问题。因此,我们就来看看客户跟进的技巧与话术吧。

    我们举个例子来说明,当我们在电话中与一些客户初步交流过后,客户可能会讲:“好,你给我些资料看看。”而当电话营销人员在发过电子邮件后,再打电话跟进的时候,可能会有如下场景:

    销售人员:“今天给您电话,就是想同您确定下资料是否收到。”

    客户:“收到,谢谢!”

    销售人员:“那有什么疑问的地方没有?”

    客户:“没有,谢谢!”

    销售人员:“如果是这样,那让我们保持联系,如果以后有什么需要的话,请随时与我联系!”客户:“好的,好的,一定,一定!”

    这个跟进电话是否很成功,相信经验丰富的电话销售人员会说:“不。”因为经验告诉我们:这样讲的客户80\%以上不会再主动与你联系。那如何打跟进电话才会既可以推动销售,又可以保持长期关系,又可以加强客户对我们的良好印象呢?

    首先,要在第一次电话中确定这个客户是否值得你再次打电话给他,否则,就是在浪费时间。

    电话目标很重要,像刚才例子中,除了知道客户是否收到资料外,还应尽可能多的提些问题,获取更多的信息。例如:

    “那这个问题您怎么看?”

    “它对有帮助吗?”

    “帮助在什么地方?”

    “您建议我们下一步如何走?”

    “为什么呢?”等等

    跟进电话在开始白中把这次电话与上次电话的要点和结果联系起来,让客户想起上次谈话的要点,如双方都做过的承诺等,同时,陈述这次电话目的。而不是仅仅告诉客户:“我觉得应打个电话给您…”。

    典型的跟进电话:

    “陈经理,我是**公司的***,上周三电话结束时,我们约好今天打电话给您。当时,我们谈到…,今天给您电话是我们对这个问题又进行了深入研究,想同您探讨下这个结果,可能会花15分钟左右,现在打电话方便吗?”

    打跟进电话给客户时,最好能有些新的、有价值的东西给客户,让客户觉得每次与你通完电话后都有收获。关于这一点,最好能与你的同事一起进行头脑风暴,看看可以找出多少有价值的理由与客户保持联系。例如,你公司最新的产品、同客户约好回电、客户在这期间业务上发生了变化、同客户确定价格等等。

    “我们公司最近根据客户的要求,开发了一种新的成本更低的产品…”

    “最近看到您公司业务在调整,所以,想着您可能会需要我们的帮助…”

    “最近在看报纸,其中的一条新闻觉得您可能会感兴趣…”

    “我一看到我们的新产品,我第一个想到的就是您,我觉得您可能从中获得利益…”

    “我昨天看电视,听到一个主持人的声音特别像你,所以,就打电话给你…”

    打跟进电话时以下话语尽可能少讲:

    “打电话给您主要是想看看您最近好不好...”

    “是看看是不是有什么变化…”

    “很久没有联系了,觉得应当给您个电话…”

    “只想看看您是否准备好…”

    “看是不是有些什么东西是您需要的…”

    但这些话可做为已成交客户的回访话术

    跟进电话的一般流程:

    表明身份”我是XX公司的XX…”

    从某点上过渡到这个电话目的“上个星期您提到…”

    打电话目的“今天就是具体同您一起探讨那个降低成本的计划的”

    确认客户时间是否允许“可能要花10分钟时间,现在方便吗?”

    提问问题把客户引入会谈“您对我提交给您的新方案有什么建议?”

    做好计划,识别有价值客户进行跟进,根据不同类型的客户确定电话跟进的频率,最好一个客户联系软件来管理你的客户,以提高效率。
