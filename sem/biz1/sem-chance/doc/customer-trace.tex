\subsection {目标客户跟进}

    我们都知道,生意往往不是一次就能谈成的,需要反复多次的商讨和沟通才能够达到双方都比较满意的效果,而最终签单。这中间的过程,就需要我们的销售人员不断地跟进,使得谈判不断地深入,从而达到我们预计的目的。

    客户的跟进是需要技巧的。首先我们要明白客户跟进的目的是什么:

     \begin{enumerate}
        \item 动态的了解客户方的需求变化,并依此调整自己的销售策略

        \item 不断重申自身媒体价值和能够给客户方带来的收益

        \item 增进与客户方负责人的感情,减少交易障碍
     \end{enumerate}

    在我们理解了客户跟进主要目的后,接下来的问题是怎样进行跟进:

     \begin{enumerate}
        \item 拜访跟进:

        最推荐的客户跟进方式。当面交流可以更直观的了解客户公司的动态和需求变化,同时可以增进彼此的感情和信任度。一般处于售前阶段的跟进以每周一次为宜,但如遇特殊时期(例如:直接竞争对手正在于我们抢份额;客户方制定年度计划期间),可以考虑2-3天进行一次拜访跟进。

        \item 电话跟进:

        比较节省时间的跟进方式。在电话跟进时切忌流于“程式化”,一定要在同电话前做好准备:确定通话目的;要通过电话向对方了解什么;准备通过电话告诉对方哪些有益的或对方感兴趣的信息;设计好通话的开场白和“问题漏斗”等等。

        \item 资料跟进:

        通过向客户方发送相关资料,达到增加与客户合理接触机会的目的;达到增加销售机会的目的;达到与对方增进感情、消除隔阂的目的;并最终达到刺激客户购买本媒体广告的目的。这些资料包括:新的招商计划方案;价格变动信息;已发布过的同业网络广告检测记录;已执行过的目标客户竞争对手的专项活动总结报告;最近计划进行的公关或促销活动或论坛活动方案等等。

        \item 服务跟进:

        邀请客户相关负责人观摩我们组织的各项活动,有限度的参加我们组织的公关或促销活动,让客户感受我们的平台价值和服务价值,从而刺激其购买欲,最终促成成交。

        \item 负责人公关:

        了解客户方“关键人物”的脾气秉性、兴趣爱好、生活习惯、性格弱点等等,投其所好,消除隔阂和对立情绪,与之建立良好的、可以互相信任的个人关系。
     \end{enumerate}

    客户跟进的最终目的是:通过不断的、高频率的信息和服务“轰炸”,与目标客户达成“互信”,并使目标客户越来越清楚的认识到搜房平台与专业服务的价值,刺激目标客户与本媒体合作的欲望,从而最终达到“销售签约、顺利收款”的目的。因此,以上所列举的各种跟进方式可以根据目标客户的不同特点,组合运用。
