\subsection {销售机会发现:寻求已存在的销售机会}

    企业要从被人们忽略和丢弃的未被满足的市场需求中寻找销售机会,其途径有:

    \subsubsection {从供需缺口中寻找销售机会}

    某类产品在市场上供不应求时,就表明了可供产品在数量、品种方面的短缺,反映了消费者的需求尚未得到满足,这种供需缺口对于企业来说就是一种市场机会。寻求供需缺口可采用以下方法:

    \begin{enumerate}
        \item 需求差额法。即从市场需求总量与供应总量的差额来识别市场机会。可用公式表示:

        需求差额=市场需求量-产品供应量

        产品供应量=国内产量+进口产量+库存量

        \item 结构差异法。即从市场供应的产品结构与市场需求结构的差异找寻市场机会。产品的结构包括品种、规格、款式、花色等。有时供需总量平衡,但供需结构不平衡,仍然会留下需求空缺,分析供需结构差异,企业便可从中发现市场销售机会。

        \item 层次填补法。即从需求层次方面来寻求市场机会。市场需求层次可分为高、中、低三档。马斯诺又把需求层次分为五个等级,可以通过分析各层次需求满足的情况,找出未被满足的“空档”,并生产相应产品予以填补。
    \end{enumerate}

    \subsubsection {从市场细分中寻求销售机会}

    从细分市场中寻找销售机会的主要方法有:
    \begin{enumerate}
        \item 深度细分。即把某项细分标准的细分程度加深拉长。如按服装型号可分为小、中、大号。但如果采用延伸法把细分度拉长,则可分为:特小号、小号、中号、大号、特大号和特型号等,也可将细分度加深,如特型号中分宽长型、宽短型、窄短型等。深度细分,照顾了消费者复杂的需求差异,通过这次细分,可以发现未被满足的市场。

        \item 交叉细分。即采用两上标准细分。例如家俱可以用收入、年龄两个标准细分市场,可以把市场分成若干个次市场。

        \item 立体细分法。即采用三个细分法。如按收入、年龄、人口三个标准细分,可以把市场分成(3×3×3)27个次市场,这27个市场各不相干。

        \item 多维细分法。即采用多种标准对市场进行细分,如按文化程度、购买动机、生活方式、年龄、职业、收入等细分市场,可以细分出更多的市场,从中发现被他人忽略的市场机会。
    \end{enumerate}

    \subsubsection {从产品缺陷中寻求销售机会}

    产品缺陷往往影响消费者的购买兴趣及两次购买的可能,不断弥补产品的缺陷则可能给企业带来新的生机。例如:某生产照相机企业,最初投放市场照相机机型复杂、笨重、不易操作,针对上述缺陷,研制推出一种快速自动照相机、弥补了原照相机的不足,产品一上市大受欢迎。以后,他们又在消费者调查中了解到该相机闪光灯需用电池带动,经常更换电池给消费者带来麻烦,而且电池留在机内时间长了会变质漏水,损坏照机。为克服这些缺点,研制人员研制出一种更趋完美的能够弥补上述缺陷的超小型、匣式、不用电池的新型照相机,新式相机很快行销市场。

    \subsubsection {从竞争对手的弱点中寻求销售机会}

    研究竞争对手,从中找出竞争对手产品的弱点及销售的薄弱环节,也是寻找机会的有效方法之一。如:某家大电器公司,以“取竞争者之长,补竞争者之短“的方式,参与市场竞争。在竞争对手成功地开发出自动洗碗机之后,就把这种洗碗机带回实验室,从产品功率、性能、零件数量和种类及成本构成等逐一进行评估,该公司将机器拆散,对每个零件加以研究,以发现弱点加以改进。这样,这家公司很快地开发出一种性能更好,价格更低的全自动洗碗机,从而取代了竞争对手。

\subsection {销售机会的捕捉诀窍}

    捕捉销售机会,对销售人员各方面的能力有较高的要求,它要求销售人员能够及时收集并分析研究影响推销环境变化的因素的信息和资料,从中发现销售机会出现的可能性和具体内容;要求销售人员能看准时机,以恰到好处地捕捉销售机会;要求销售人员能发挥主观能动性和创造性,善于打破常规,创造出独具特色的销售机会;要求销售人员注意把握因地制宜的原则,研究销售机会的空间适应性。

    捕捉销售机会虽然并不容易,但也并非无窍门可寻。以下介绍几种捕捉销售机会的诀窍:

    \begin{enumerate}

    \item 谨思慎行

    推销工作的每一个步骤对销售人员来说都极为重要。销售人员处理得当,有可能促成推销机会的出现和形成;但若销售人员急于求成,鲁莽行事,则有可能亲手毁灭销售机会出现的可能性。

    因此,销售人员在推销的每一环节都应保持冷静,随时掌握局势的变化,利用自己的常识和经验,充分分析思考,然后再谨慎行动。切勿信口开河,鲁莽行事,使顾客产生不信任感,或由于压力过高而丧失购买信心,从而失去有利的销售机会。

    \item 察言观色

    在推销过程中,销售机会往往都是潜生的,具有相当的隐藏性,而不会明显地显现出来,但也并非是完全无迹可寻的。顾客的购买倾向和成交意愿往往会从顾客的表情、语言、行为等方面显现出来,因而销售人员应善于观察和分辨,依据自己的推销经验及时捕捉推销机会。

    \item 多听少讲

    销售人员一接近顾客,马上口若悬河,恨不得将推销品全部优点一股脑儿告诉顾客,这是推销工作的一大忌。销售人员应虚心听取顾客的意见和要求,而不要只顾自己讲话(应在必要时予以回答)。这样不但让顾客感到受到尊重,从而有利于创造良好的推销氛围,并且可以从顾客的言谈中获得推销的线索和答案,从而控制推销机会。

    \item 循序渐进

    推销交易有简有繁,复杂的交易往往需要多个回合才可能完成。因此,销售人员应有足够的耐心和恒心。循序渐进,按部就班,配合推销活动的每个阶段适时地把握机会,调整推销工作的方式和内容。不要急于求成而破坏了有利的销售机会。

    \item 耐心等待

    耐心是销售人员必须具备的重要品质。急功近利,行事冲动极易导致推销失败。这是因为,顾客在做出买不买,买多少,何时买等购买决策时,都不是一时冲动可以决定的。他需要权衡各种客观因素,如产品特征,购买能力等,同时还要受到主观因素的影响,如心情好坏等。因此,购买决策过程是一个极其复杂的过程,并不是一蹴而就的。销售人员应设身处地地为顾客着想,体会顾客的难处,耐心地等待时机。

    另外,销售人员和顾客双方有各自不同的习惯和想法,考虑问题和行事的方法和程序也都各不相同。在推销过程中,销售人员不能将自己办事的程序强加于顾客,而应注意顾客的思路,调整自己以与之相配合。因此,有足够的耐心,是选择竞争时机的关键。但销售人员也不应一味地消极等待,在关键时刻要发挥推波逐澜的作用,以免贻误时机。

    \item 坐山观虎斗

    当别人出现失误时,可能你的好机会就来了。在推销活动当中,推销的参与者之间往往存在着种种矛盾,利用这些矛盾经常能为自己创造出难得的机会。

    首先,要利用顾客和竞争对方之间的矛盾。在顾客向你抱怨竞争对手时,你应该乘机而入,向顾客推销自己的商品,有可能会获得成功。这虽然是落井下石的作法,但切不可给顾客以落井下石的感觉,以免顾客产生反感。因为你作为竞争对手的身份是比较敏感的,一定要把握分寸,不要不择手段,把对方贬低得一无是处,并注意一定要以事实为根据来说话,才会更有说服力,从而赢得顾客的信任和好感。

    其次,要利用竞争对手之间的矛盾。竞争对手之间,出于利益的争夺,往往存在着种种矛盾。有时,双方会你争我夺,在顾客面前互相贬损,甚至有时会不惜亏本,一决雌雄,最终很可能弄得两败俱伤。在这种情况下,你应冷静观察,在双方争夺激烈时按兵不动,当双方筋疲力尽之际,再伺机出击,展开推销攻势,争取顾客。

    \item 伺机而动

    一些特殊日子和事件往往是推销商品的大好时机,如我国传统节日春节、元旦、国庆节等重大节日,各种体育盛会,纪念活动等都有可能成为推销商品的大好机会。有经验的销售人员往往能事先就做好充足准备,拟定销售计划,做好万全之策,把握并利用这些机会,极力宣传商品,刺激顾客的购买欲望,促进商品销售。

    \item 环境烘托

    销售人员还可为顾客创造良好的购物环境,并根据商品的特点,设计柜台摆放,商店装潢、灯光设计、商品包装、背景音乐、环境卫生等环境条件,来衬托并突出商品,增强商品竞争能力,激发顾客的购买欲望,从而促进商品销售。

    \item 节奏缓急

    销售人员把握推销节奏的能力极为重要。在该给予顾客思考权衡时就应放缓节奏,给顾客喘息的机会。而在销售人员发现有迹象显示出顾客的购买意图时,则应抓住时机,一鼓作气,劝说顾客,达成交易。

    \end{enumerate}

\subsection {创造新的销售机会}
    创造销售机会在于能对销售环境变化做出敏捷的反应,善于在许多寻常事物中迸发灵感,巧于利用技术优势开发出新产品。

    \subsubsection {从市场发展趋势中创造销售机会}

    市场发展趋势包含两方面内容:一是指某类产品市场(包括销售、消费、需求)增长比率;二是指市场客观环境的变化动向。

    1. 增长比率法。市场增长比率的正变化,表明了未来市场需求的增长,企业应以超前的眼光,创造销售机会。例如据科威特有关机构预计今后5年科威特电讯设备需求量平均增长率为22~25\%,电讯辅助设备需求量将平均增长11~15\%,空调冷藏设备需求量平均增长18~23\%,而上述三种产品的60\%需要进口。这是企业电子产品进入科威特市场的良机。

    2. 环境变化法。即从市场宏观环境变化中创造机会。环境往往使机会与挑战并存,经营者既要以敏锐的眼光从变化动向中预测未来,把握销售机会,还要以非凡的创造力,善于把挑战转化为机会。例如:在我国大中城市中,人口出现老龄化趋热,这意味着老年人市场逐步扩大,企业可把握此动向,深入细分老年人市场,开发出能最大程度地满足他们要求的各种产品。

    \subsubsection {从社会时代潮流中创造销售机会}

    社会发展的各个时代都会形成流行的涡心,例如当今时代的潮流是回归大自然。在这种社会大潮的冲击下,许多企业顺应潮流,把握机遇,推出了“自然产品”,如用植物原料制造出的药品、化妆品、饮料;开发出“绿色产品”(即减少环境污染、保护生态环境、节约使用自然资源的产品)。由于这些产品迎合了当代人们的心态,从而激发了人们新的需求。

    \subsubsection {用科学技术创造销售机会}

    现代科学技术的发展趋势,表现出三大特征:

    1. 新材料的应用。近年来世界新材料的开发主要集中在高功能聚合物、精密陶瓷、复合材料和高级合金上。目前这类新材料市场规模达2000亿美元,预测到2000年,上述4种材料成交额可超过100亿美元,研制并抢先运用这种新材料,推出新产品,能创造新的市场机会。

    2. 新能源的利用。即用新的能源取代旧的能源。如国际鉴于石油资源的短缺,正在研制新的能源汽车等,如电动汽车、甲醇汽车、天然气汽车、太阳能汽车和氢气汽车等。

    3. 新技术的应用。21世纪将是电子化的时代,将电子技术广泛地应用到生产领域可以创造新的市场机会。总之,随时关注世界科学技术发展动态,及时地将这些技术引入生产领域,将给企业带来无限生机。

    \subsubsection {用销售手段创造销售机会}

    通过采用创新的销售手段,创造新的销售机会。例如,日本阿托搬家中心,改变过去搬家方式,别出心裁,决意要将“烦恼的搬家”变为“愉快的旅行”由此设计出一种命名为“21世纪的梦”的搬家用车,这种车分为上下两层,上层前半部是豪华客厅和休息场所,娱乐设备齐全,下层是驾驶室,车的后半部为行李车厢。阿托中心还同时提供3000多项与搬家有关的服务,此车一推出,预约搬家者蜂拥而至。这是通过创新的服务来创造新的销售机会。此时还可通过预报商品流行,来引发消费者的需求,如预报服装家俱流行款式、流行色;还可以利用广告宣传、新闻报道等创造销售机会。

    企业一旦识别和寻求到恰当的销售机会,必将为企业的经营与发展带来勃勃生机。

\subsection {销售机会的评估}

\begin{enumerate}
    \item 分析销售机会的性质
        \begin{enumerate}
            \item 明确成功利用某一机会的必要条件;
            \item 以利用这一机会的基本要求为参照,分析本企业从事相关业务的能力,确认利用这一机会的优势与不足,弥补差距的可能性;
            \item 同样分析每一个最有可能的竞争者;
            \item 比较本企业与竞争者,各自在成功利用这一机会条件方面的优劣,确认本企业对这一机会是否拥有相对竞争,差距大小。
        \end{enumerate}

    \item 分析销售机会的质量

        \begin{enumerate}
            \item 分析这一机会所显示的,仅仅是人们的需要和欲望,还是已经形成市场。暂时条件不允许,或人们不打算经由市场满足的需要,至多只是一个潜在的市场。
            \item 分析这一市场是否拥有起码的规模,即是否有了足够数量的顾客;
            \item 确认本企业进入这一已有足够数量顾客的市场,是否具备必需的销售能力。
        \end{enumerate}
\end{enumerate}
