
    企业在市场销售工作中,往往有时碰到老销售员不听话,比较头痛,大家普遍认为老销售员对公司的各项市场运作非常熟悉,对客户的关系也很好,一时不听话,也拿他没有办法,碰到新上任的销售经理,很多时候一上火,全给开了,其实这样是不太妥的,因为我们从事的市场销售工作,是来不得半点情绪化的,那么,遇到以前的老销售员不听话,我们做为销售经理该如何进行老销售员管理呢?

  一、了解老销售员的现阶段的状况

  做为老销售员,可能不象刚来那样,冲劲十分大,他可能每月做完销售指标,不多也不少,刚刚好,十分自在的种自己那块一亩三分地,但是人非草木,每一个从事销售工作的人,内心世界都渴望对自己有所发展,如果作为一个销售经理去找老销售员谈话,不会有任何一个老销售员说,我来公司从事销售是来混日子的。对老销售员管理时,要知道老销售员当前最大的问题,找出合理的办法加以引导,如薪金,工作环境,和老销售员在公司的地位等等,要帮他们解决问题,不要对他们不管不问,自动催化他们成"老油条"。

  二、要给老销售员一顶合适的高帽子

  每一个人的内心世界,都有对官本位的追求,做为老销售员是十分希望对他工作的肯定,作为管理者在进行老销售员管理时要明确这点。有一个非常油条的老销售员,他在公司做了三年的销售,城市里的商超大小买手,经理,非常熟悉,称兄到弟。好多时,新的上司搞不定,他一去就马上行,他和他的新上级关系很紧张,新上级多次要开他,说他缺纪律,经常迟到,不能服从统一管理,还会影响其它人员,经理知道后,对这位老销售进行了了解,原来此君来了公司三年,职位一直没变,他懊恼的说,我和那些商场经理在一起,连名片都不敢掏,来公司这样久,啥头衔都没有,真没面子,经理发现他工作不错,就是有点散慢,经理对这位老销售增加了工作量,让他带了两个新手,并给他的名片上印着主管,没想到从第二天开始,天天第一个来公司,每天详细的工作报告写完,最后一个离开,过了一段时间,经理表扬他,问他进步很快的原因,他说,我有两个兵,要做表帅啊,两个小兄弟看着我呢,一定要做好。

  三、不要让老销售员独自为王

  企业有时和老销售员发生不愉快时,常常听到老销售员这样说,公司的客户是我开发的,销售网络是我编织的,我不干了,看你们怎么办?等等来威胁企业,其实这是一个销售人的误区,很多销售员认为这些客户是自己的,是非常错误的,因为你现有的网络和客户是有公司为你在背后支称的平台,一但你离开,网络和客户会自行消失,不信你试一试。但做为企业在进行销售员管理时,要注意的是销售员在外销售公司的产品,是代表公司,而不是个人行为,这样才能让客户知道合作的是与公司发生交易,而不是看着某某人。

  总之,我们对老销售员管理的态度要正确,对待老的销售员要正确的引导,管理,控制,但决不要动不动就炒掉不听话的老销售,存在既合理,每一个人都是有用的,看你如何用好他和她。
