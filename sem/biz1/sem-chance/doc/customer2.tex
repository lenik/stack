\subsection {销售中的沟通方法}

    人们往往认为,商品市场中的销售者只是在销售商品。其实,从沟通学的角度,从更高的层次来分析,销售者销售的其实是“人”。这才是买卖成功的秘诀,也是商品销售的最高境界。因为买卖双方如果认可了对方的“为人”,才更会在欲望的基础上形成动机,采取行动,完成买卖。古今中外,莫不如此。

    销售是一项沟通的艺术,把话说到客户心里,也就有了成交的希望。良好的沟通将会贯穿于销售工作的整个过程,而沟通能力的强弱,也将在每一个环节上,对销售工作的成败产生决定性的影响。销售不懂沟通学,就犹如在茫茫的黑夜里行走,永远只能误打误撞。事实上,销售高手往往都是沟通专家。

    你看“悟”字,竖心旁,五个口,那你就经常跟人沟通嘛,用心跟五个人交流,这五个人也用心跟你交流。如果你能找到五个跟你用心沟通的朋友,那你这一辈子真的就能悟到道了。了解对方想听和不想听的、喜欢和不喜欢的,以及对方的担心、顾虑等,如此便打开了人与人之间沟通的大门。高品质的沟通,应把注意力放在结果上,而不是情绪上,沟通从心开始。沟通能力是评价一个人素质高低的重要指标,沟通对于企业的重要性更是不言而喻。

    专家研究表明,优秀经理70\%以上的时间都用在了沟通上。经理如此,销售代表也不例外。因为销售代表不但要管理客户,还要管理客户的下家,也就是要协助客户做好“助销”工作,工作的内容和难度增加了。具体来说销售代表要管理好客户的物流、资金流、信息流,还有客户的下家。这“三流”的管理是建立在和客户的良好沟通上。只有良好的沟通才能落实公司的策略、定货、促销等。对客户如此,对内部也同样,在日常工作中,销售代表要和上司沟通,争取销售政策、促销活动经费等资源;和同事沟通,来争取到物流的配合、财务的配合、培训支持的配合、行政的配合等。

    “企”字由“人”和“止”组成,企业无“人”则“止”。可见:“人”是企业的第一要素!而毛主席说:有人群的地方则一定有左中右。陶行知和学生的沟通明显属于团队沟通——在企业团队,复杂的人际关系使很多人往往不善于沟通胡一夫老师认为,人与人之间的交流和沟通是一门重要的管理艺术。

    松下幸之助有句名言:“企业管理过去是沟通,现在是沟通,未来还是沟通。”因此,管理离不开沟通,沟通已渗透于管理的各个方面。正如人体内的血液循环一样,如果没有沟通的话,企业就会趋于衰亡和倒闭。胡一夫老师认为,我们的营销团队十分需要沟通,在一个团队里,如果听不到一点异响,听不到一点反对意见,那是不正常的。水,在污泥塘里,不动不响,那是死的;在清江河里,汹涌奔腾,那是活的。有一点逆耳的话在耳边响着,警钟常鸣,不见得就是坏事。甚至可以说,是好事。

    如果说“传统”逢人只说三分话,而“现代”不过是“不管三七二十一,反正有话就要直说”。那么,传统与现代的区别,几乎局限于“成熟”与“浅薄”,根本和进步与否无关,我们怎么能够盲目地反传统、崇现代呢?反过来说,“现代”的“有话直说”若是“也要适当地配合情境来掌握分寸”,请问与“传统”有什么不同?难道“由不懂得传统道理的人,将自己认为西方有的、我们没有的翻译过来,就成为现代”吗?偏偏现代社会,充满了“知东不知西”或“知西不知东”的人,又何以沟通东西两方的文化呢?

    全世界的人,都希望有话直说。却由于各地的风土人情有所差异,因而产生不同的沟通方式,这是民族性的区别使然。中国人喜欢自由自在、不受约束,当然也乐于有话直说。但是太多“先说先死”的案例,使得我们深切体会“祸从口出”的道理;因而主张“慎言”,做到“应该有话直说的时候,当然应该有话直说;不应该有话直说的时候,当然不应该有话直说”的“中道”境界,形成中国人的沟通功夫。
