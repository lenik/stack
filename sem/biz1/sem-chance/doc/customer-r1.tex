\subsection { 销售如何做好客户关系管理}

    现在不少的销售人员都在为寻找自己的客户发愁,有时候不是因为找不到客户,而是有了太多的客户不知道怎么办。如何做好客户管理工作呢?   其实对于客户的管理无非也就是以下几个阶段:
      \begin{enumerate}
      \item 客户信息收集
      \item 客户划分
      \item 客户跟踪处理。
        \end{enumerate}

    这其中最关键的就应该是客户的划分和跟踪处理了。那对于客户的如何划分也就决定了怎么样跟踪处理客户信息了。

    我们首先来看客户的划分,对于手上现有一个客户信息,我们可以从以下四个角度产生四种不同的划分方式。

    第一,我们可以从客户是否已经和我们成交的状态把客户分为:
        \begin{itemize}
            \item 已成交客户
            \item 正在谈判客户
            \item 潜在客户
        \end{itemize}

    第二,我们可以从客户的重要性(一般用可成交额度或者业务潜在量来衡量)来划分为重要客户和非重要客户。

    第三,从需要处理客户信息的时间段上可以把客户分为:

    \begin{enumerate}
        \item 紧急客户(一般需要在一周内做出处理)
        \item 缓急客户(一般指一周到1个月内需要对该客户作出处理)
        \item 不紧急客户(1个月以上3个月以内必须处理的客户)
        \item 可慢反应客户(3个月以后才可能发生关系的客户)
    \end{enumerate}

    第四,我们还可以从客户的需求状况上把客户分为:目标客户(现在就有需求)、潜在客户(未来有需求)和死亡客户(不会有任何需求)

    以上就是通常的四种划分方式,不同的划分有不同的管理方式。像上面的分法,我们如何管理客户呢?可以根据上面的信息经过整理后形成如下的等级划分:

    \begin{itemize}
        \item  A级客户:有明显的业务需求,并且预计能够在一个月内成交;
        \item  B级客户:有明显的业务需求,并且预计能够在三个月内成交;
        \item  C级客户:有明显的业务需求,并且预计能够在半年内成交;
        \item  D级客户:有潜在的业务需求的客户或者有明显需求但需要在至少半年后才可能成交;
        \item  E级客户:没有需求或者没有任何成交机会,也叫死亡客户。
    \end{itemize}

    基于明晰的客户分类,我们可以进一步建立客户追踪志,用追踪日志来跟进客户的方法称为客户追踪志管理法。那到底建立什么样的客户追踪志呢?对于每个级别的客户又如何区分对待呢?

    我们现在先来介绍都有那些客户追踪志,客户的追踪志一般有以下几种:

    \begin{itemize}
        \item 客户追踪日志:也就是需要每天将客户的信息重新跟踪处理,并刷新记录;
        \item 客户追踪周志:就是每周内至少对客户的信息处理一次,并刷新记录;
        \item 客户追踪半月志:也就是每15天对客户的信息处理一次,并刷新信息记录;
        \item 客户追踪月志:也就是每30天需要至少对客户的信息处理一次,并刷新信息记录。
        \item 客户追踪年志:也就每一年需要至少对客户的信息处理一次,并刷新信息记录。
    \end{itemize}

    有了客户追踪志以后,我们只需要对相应等级的客户用相应追踪志做管理,那我们的客户管理就游刃有余了。一般来说,对于A级客户我们 需要用客户追踪日志,对B级客户我们使用客户追踪周志,对C级客户我们使用客户追踪半月志,对于D级的客户我们使用客户追踪月志,而对于D级的客户我们则 使用客户追踪年志。而且每次客户追踪以后就对客户信息重新定格划分等级,并且用新的等级所对应的管理方法来处理。
