\section{日志}
日志是都i销售员日程的记录
    \ul {
    \item \opset {新建行动日志} {
        \ops {
        \item  单击\button{新建}按钮,进入创建行动日志界面
        \screenshot{1.png}
        \item  选择行程的\button{开始时间}和\button{结束时间}
        \item  选择\button{洽谈方式}
        \item  选择\button{拜访的客户}
        \item  如果有工作伙伴一起作业的,选择\button{工作伙伴}
        \item  输入此次行程的\textbox{具体内容}和\textbox{花费详细}
        \item  如果此次行程是对某个销售机会的推进
            \ul {
            \item  选择相应的\button{销售机会}
            \item  设置此次行程结束后的\button{机会阶段}
            }
        \screenshot{2.png}
        \item  点击\button{保存}按钮,保存新建的行动日志
        }
    }
    \item \opset {修改行动日志} {
        \screenshot{3.png}
        \ops {
        \item  选择需要修改的行动日志
        \item  点击工具栏上的\button{修改}按钮,进入到修改行动日志界面
        \item  修改相关内容
        \item  点击\button{保存}按钮,保存更改后的行动日志
        }
    }
    \item \opset {删除行动日志} {
        \ops {
        \item  选择需要删除的行动日志
        \item  点击工具栏上的\button{删除}按钮,在弹出的确认对话框中点击\button{确定},删除行动日志
        }
    }
    }

\section{销售机会}
    \ul {
    \item \opset {新建销售机会} {
        \ops {
        \item  点击\button{新建}按钮,进入到新建销售机会界面
        \screenshot{19.png}
        \item  在机会基本信息标签下
        \screenshot{20.png}
            \ul {
            \item  输入\textbox{机会标题}
            \item  选择\button{机会分类}以及\button{机会来源}
            \item  输入\textbox{机会内容},\button{机会地址}
            }
        \item  在客户标签下
        \screenshot{21.png}
            \ul {
            \item  点击\button{添加}按钮,弹出添加客户对话框
            \item  选择相关\button{客户}
            \item  输入客户在这个机会中担任的\textbox{角色}
            \item  点击\button{确定}按钮添加客户
            }
        \item  如果该机会由机会行动产生,在行动跟踪标签下
        \screenshot{22.png}
            \ul {
            \item  点击\button{连接到}按钮,关联相关行动日志
            \item  选择行动日志,点击\button{断开}按钮,断开行动日志跟机会的连接
            }
        \item  点击\button{保存}按钮,保存新建的销售机会
        }
    }
    \item \opset {修改销售机会} {
    \screenshot{23.png}
        \ops {
        \item  选择需要修改的销售机会
        \item  点击工具栏上的\button{修改}按钮,进入到修改销售机会界面
        \item  修改相关信息
        \item  点击\button{保存}按钮,保存更改后的销售机会
        }
    }
    \item \opset {删除销售机会} {
        \ops {
        \item  选择需要删除的销售机会
        \item  点击工具栏的\button{删除}按钮,在弹出的确认对话框中点击\button{确定}按钮,删除相应销售机会
        \screenshot{24.png}
        }
    }
    }


\section{机会字典}

\subsection{机会分类}
系统预设有三个销售分类:
重要,一般,其他
    \ul {
    \item \opset {企业可以根据自己的需要,自由设置机会分类} {
        \ops {
        \item  点击\button{添加新的}按钮,进入到添加新的机会分类界面
        \screenshot{4.png}
            \ul {
            \item  输入\textbox{代码}
            \item  输入\textbox{显示名称}
            \item  输入\textbox{描述}
            \screenshot{5.png}
            \item  点击\button{保存}按钮,保存新建的机会分类
            }
        \item  点击操作列表中的\button{编辑}图标,进入修改对应会分类界面
        \screenshot{6.png}
            \ul {
            \item  修改相关信息
            \item  点击\button{保存}按钮,保存更改后的机会分类
            }
        \item  点击操作列表中的\button{删除}图标,在弹出的确认对话框中点击\button{确定},删除对应机会分类
        \screenshot{7.png}
        }
    }
    }

\subsection{机会来源}
系统预设有十一个销售机会来源:
电话来访,客户介绍,独立开发,媒体宣传,促销活动,老客户,代理商,合作伙伴,公开招标,互联网,其他
    \ul {
    \item \opset {企业可以根据自己的需要,自由设置销售机会的来源} {
        \ops {
        \item  点击添加新的按钮,进入到新建机会来源界面
        \screenshot{11.png}
            \ul {
            \item  输入\textbox{代码}
            \item  输入\textbox{显示名称}
            \item  输入\textbox{描述}
            \screenshot{12.png}
            \item  点击\button{保存}按钮,保存新建的机会来源
            }
        \item  点击操作列表中的\button{查看}图标,查看机会来源的详细内容
        \item  点击操作列表中的\button{编辑}图标,进入编辑机会来源界面
        \screenshot{13.png}
            \ul {
            \item  修改相关信息
            \item  点击\button{保存}按钮,保存更改后的机会来源
            }
        \item  点击操作列表中的\button{删除}图标,在弹出的确认对话框中点击\button{确定},删除对应机会来源
        \screenshot{14.png}
        }
    }
    }

\subsection{洽谈类型}
系统预设有六种洽谈类型:
电话洽谈,面谈,走访,通过互联网,信件洽谈,其他方式洽谈
    \ul {
    \item \opset {企业可以根据自己的需要,自由设置洽谈的类型} {
        \ops {
            \item  点击添加新的按钮,进入到新建洽谈类型界面
            \screenshot{15.png}
                \ul {
                \item  输入\textbox{代码}
                \item  输入\textbox{显示名称}
                \item  输入\textbox{描述}
                \screenshot{16.png}
                \item  点击\button{保存}按钮,保存新建的洽谈类型
                }
            \item  点击操作列表中的\button{查看}图标,查看洽谈类型的详细内容
            \item  点击操作列表中的\button{编辑}图标,进入编辑洽谈类型界面
            \screenshot{17.png}
                \ul {
                \item  修改相关内容
                \item  点击\button{保存}按钮,保存更改后的洽谈类型
                }
            \item  点击操作列表中的\button{删除}图标,在弹出的确认对话框中点击\button{确定},删除对应的洽谈类型
            \screenshot{18.png}
        }
    }
    }

\subsection{机会阶段}
系统预设有七个机会阶段:
初始化,初步沟通,已经报价,合同付款洽谈,合同签订,终止
    \ul {
    \item \opset {企业可以根据自己的需要,自由设置机会阶段} {
        \ops {
        \item  点击\button{添加新的}按钮,进入到添加新的机会阶段类型界面
        \screenshot{8.png}
            \ul {
            \item  输入\textbox{代码}
            \item  输入\textbox{显示名称}
            \item  输入\textbox{描述}
            \screenshot{5.png}
            \item  点击\button{保存}按钮,保存新建的机会阶段
            }
        \item  点击操作列表中的\button{查看}图标,查看机会阶段的详细内容
        \item  点击操作列表中的\button{编辑}图标,进入编辑机会阶段界面
        \screenshot{9.png}
            \ul {
            \item  修改相关信息
            \item  点击\button{保存}按钮,保存更改后的机会阶段
            }
        \item  点击操作列表中的\button{删除}图标,在弹出的确认对话框中点击\button{确定},删除对应的机会阶段
        \screenshot{10.png}
        }
    }
    }
