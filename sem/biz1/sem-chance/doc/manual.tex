\section{日志}
日志是都销售员日程的记录
    \opset{行动日志}{
    \item \ops{新建行动日志}{
        \item  单击\button{新建}按钮,进入创建行动日志界面
            \screenshot{1.png}
        \item  选择行程的\button{开始时间}和\button{结束时间}
        \item  选择\button{洽谈方式}
        \item  选择\button{拜访的客户}
        \item  如果有工作伙伴一起作业的,选择\button{工作伙伴}
        \item  输入此次行程的\textbox{具体内容}和\textbox{花费详细}
        \item  如果此次行程是对某个销售机会的推进
            \ul {
            \item  选择相应的\button{销售机会}
            \item  设置此次行程结束后的\button{机会阶段}
            }
            \screenshot{2.png}
        \item  点击\button{保存}按钮,保存新建的行动日志
    }
    \item \ops{修改行动日志}{
        \item  选择需要修改的行动日志
        \item  点击工具栏上的\button{修改}按钮,进入到修改行动日志界面
            \screenshot{3.png}
        \item  修改相关内容
        \item  点击\button{保存}按钮,保存更改后的行动日志
    }
    \item \ops{删除行动日志}{
        \item  选择需要删除的行动日志
        \item  点击工具栏上的\button{删除}按钮,在弹出的确认对话框中点击\button{确定},删除行动日志
    }
    }

\section{销售机会}
    \opset{销售机会}{
    \item \ops{新建销售机会}{
        \item  点击\button{新建}按钮,进入到新建销售机会界面
            \screenshot{19.png}
        \item  在机会基本信息标签下
            \screenshot{20.png}
            \ul {
            \item  输入\textbox{机会标题}
            \item  选择\button{机会分类}以及\button{机会来源}
            \item  输入\textbox{机会内容},\button{机会地址}
            }
        \item  在客户标签下
            \screenshot{21.png}
            \ul {
            \item  点击\button{添加}按钮,弹出添加客户对话框
            \item  选择相关\button{客户}
            \item  输入客户在这个机会中担任的\textbox{角色}
            \item  点击\button{确定}按钮添加客户
            }
        \item  如果该机会由机会行动产生,在行动跟踪标签下
            \screenshot{22.png}
            \ul {
            \item  点击\button{连接到}按钮,关联相关行动日志
            \item  选择行动日志,点击\button{断开}按钮,断开行动日志跟机会的连接
            }
        \item  点击\button{保存}按钮,保存新建的销售机会
    }
    \item \ops{修改销售机会}{
        \item  选择需要修改的销售机会
        \item  点击工具栏上的\button{修改}按钮,进入到修改销售机会界面
            \screenshot{23.png}
        \item  修改相关信息
        \item  点击\button{保存}按钮,保存更改后的销售机会
    }
    \item \ops{删除销售机会}{
        \item  选择需要删除的销售机会
        \item  点击工具栏的\button{删除}按钮,在弹出的确认对话框中点击\button{确定}按钮,删除相应销售机会
            \screenshot{24.png}
    }
    }


\section{机会字典}

\subsection{机会分类设置}
系统预设有三个销售分类:
重要,一般,其他,企业可以根据自己的需要,自由设置机会分类
    \opset{机会分类设置}{
        \item \ops{新建机会分类}{
            \item  点击\button{添加新的}按钮,进入到添加新的机会分类界面
            \screenshot{4.png}
                \ul {
                \item  输入\textbox{代码}
                \item  输入\textbox{显示名称}
                \item  输入\textbox{描述}
                \screenshot{5.png}
                \item  点击\button{保存}按钮,保存新建的机会分类
                }
        }
        \item \ops{编辑机会分类}{
            \item  选择需要编辑的机会分类
            \item  点击操作列表中的\button{编辑}图标,进入修改对应会分类界面
                \screenshot{6.png}
            \ul {
                \item  修改相关信息
                \item  点击\button{保存}按钮,保存更改后的机会分类
            }
        }
        \item \ops{删除机会分类}{
            \item  选择要删除的机会分类
                \screenshot{7.png}
            \item  点击操作列表中的\button{删除}图标
            \item  在弹出的确认对话框中点击\button{确定},删除对应机会分类
        }
    }

\subsection{机会来源设置}
系统预设有十一个销售机会来源:
电话来访,客户介绍,独立开发,媒体宣传,促销活动,老客户,代理商,合作伙伴,公开招标,互联网,其他
%企业可以根据自己的需要,自由设置销售机会的来源
\opset{机会来源设置}{
    \item \ops{新建销售机会来源}{
        \item  点击\button{添加新的}按钮,进入到新建机会来源界面
            \screenshot{11.png}
        \ul {
            \item  输入\textbox{代码}
            \item  输入\textbox{显示名称}
            \item  输入\textbox{描述}
                \screenshot{12.png}
            \item  点击\button{保存}按钮,保存新建的机会来源
        }
    }
    \item \ops{查看机会来源}{
        \item  点击操作列表中的\button{查看}图标,查看机会来源的详细内容
    }
    \item \ops{编辑机会来源}{
        \item  点击操作列表中的\button{编辑}图标,进入编辑机会来源界面
            \screenshot{13.png}
            \ul {
            \item  修改相关信息
            \item  点击\button{保存}按钮,保存更改后的机会来源
            }
    }
    \item \ops{删除机会来源}{
        \item  选择需要删除的销售机会
        \item  点击操作列表中的\button{删除}图标
        \item  在弹出的确认对话框中点击\button{确定},删除对应机会来源
            \screenshot{14.png}
    }
}

\subsection{洽谈类型设置}
系统预设有六种洽谈类型:
电话洽谈,面谈,走访,通过互联网,信件洽谈,其他方式洽谈
%企业可以根据自己的需要,自由设置洽谈的类型
\opset{洽谈类型设置}{
    \item \ops{新建洽谈类型}{
        \item  点击\button{添加新的}按钮,进入到新建洽谈类型界面
            \screenshot{15.png}
        \ul {
        \item  输入\textbox{代码}
        \item  输入\textbox{显示名称}
        \item  输入\textbox{描述}
            \screenshot{16.png}
        \item  点击\button{保存}按钮,保存新建的洽谈类型
        }
    }
    \item \ops{查看洽谈类型}{
        \item  选择需要查看的洽谈类型
        \item  点击操作列表中的\button{查看}图标,查看洽谈类型的详细内容
    }
    \item \ops{编辑洽谈类型}{
        \item  点击操作列表中的\button{编辑}图标,进入编辑洽谈类型界面
            \screenshot{17.png}
        \ul {
            \item  修改相关内容
            \item  点击\button{保存}按钮,保存更改后的洽谈类型
        }
    }
    \item \ops{删除洽谈类型}{
        \item  选择需要删除的洽谈类型
        \item  点击操作列表中的\button{删除}图标
        \item  在弹出的确认对话框中点击\button{确定},删除对应的洽谈类型
            \screenshot{18.png}
    }
}

\subsection{机会阶段设置}
系统预设有七个机会阶段:
初始化,初步沟通,已经报价,合同付款洽谈,合同签订,终止
%企业可以根据自己的需要,自由设置机会阶段
\opset{机会阶段设置}{
    \item \ops{新建机会阶段}{
        \item  点击\button{添加新的}按钮,进入到添加新的机会阶段类型界面
            \screenshot{8.png}
        \ul {
            \item  输入\textbox{代码}
            \item  输入\textbox{显示名称}
            \item  输入\textbox{描述}
            \screenshot{5.png}
            \item  点击\button{保存}按钮,保存新建的机会阶段
        }
    }
    \item \ops{查看机会阶段}{
        \item  选择需要删除的机会阶段
        \item  点击操作列表中的\button{查看}图标,查看机会阶段的详细内容
    }
    \item \ops{编辑机会阶段}{
        \item  选择需要编辑的机会阶段
        \item  点击操作列表中的\button{编辑}图标,进入编辑机会阶段界面
            \screenshot{9.png}
        \ul {
            \item  修改相关信息
            \item  点击\button{保存}按钮,保存更改后的机会阶段
        }
    }
    \item \ops{删除机会阶段}{
        \item  选择需要删除的机会阶段
        \item  点击操作列表中的\button{删除}图标
        \item  在弹出的确认对话框中点击\button{确定},删除对应的机会阶段
        \screenshot{10.png}
    }
}
