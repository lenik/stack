\section {什么是销售机会}

\subsection {销售机会}

    在推销过程中,销售人员必须充分把握随时出现的各种机会。所谓机会,是指由于环境的变化,而为人们提供的实现某种目的的可能性。而销售机会则是指在推销过程中,由于环境经常发生变化,给销售人员提供的实现其推销目的的一种可能性的统称。

\subsection {销售机会的种类}
    销售机会是多种多样、纷繁复杂的。根据不同的标准,可将推销机会划分为不同种类。

    \begin{enumerate}
        \item 从对销售机会的认识程度上来看,可将其划分为偶然性销售机会和非偶然性销售机会。

            1)偶然性销售机会是指不可预测的,出于预料之外的一种销售机会。这种机会是非常难能可贵的,对销售人员的观察能力和应变能力的要求比较高。

            2)非偶然性销售机会,则是指销售人员通过对各方面因素的分析和研究,可在一定的时间和范围内预测到的一种销售机会。这种销售机会对销售人员的创新能力要求较高。如每个推销员都知道春节前是购物热潮,但关键在于销售人员能发挥自己的创造性,使自己的推销品在众多的商品中独树一帜,吸引顾客的注意,从而实现推销目的。

        \item 从销售机会作用和影响的范围及程度来看,可将销售机会划分为战略性销售机会和战术性销售机会。

            1)所谓战略性销售机会是指从长远、整体和全局上影响推销品销售的一种销售机会。若能捕捉到这种机会,将对企业长远的、全局的发展产生深远的影响。

            2)战术性销售机会,则是指从眼前、局部来影响推销品销售的一种销售机会,具有灵活机动的特点。

        \item 从销售机会的表现方式来看,可将其划分为潜在的销售机会和显露的销售机会。

            1)所谓潜在的销售机会是指销售机会不突出,需要销售人员深入分析、挖掘才会发现的一种销售机会。这种销售机会具有很强的隐蔽性,对销售人员各方面的素质和能力的要求比较高。因此,这种机会一旦为销售人员发现并把握,就会形成相对于其他销售人员的强有力的竞争优势。

            2)所谓显露的销售机会是指推销机会表现得比较明显,易于发现的一种销售机会。这种销售机会一般易于察觉,它对销售人员为自己创造更有利的机会条件的能力要求比较高。

        \item 从销售环境的变化内容来看,可将其划分为政治性销售机会和非政治性销售机会;经济性销售机会和非经济性销售机会;时间性销售机会和非时间性销售机会;季节性销售机会和非季节性销售机会等。

        \end{enumerate}

\subsection {销售机会的特征}
    销售机会一般来说具有以下特征:

    \begin{enumerate.zh}

        \item 客观性。销售机会的出现与否,是不以销售人员的主观意志为转移的。它是由于客观环境的变化而发生的,其大小由客观环境变化的内容、程度、范围和性质等因素决定。因此,销售人员必须注意观察,并及时采取有效的措施来认识机会、把握机会并利用机会。

        \item 平等性。从事同一领域推销活动的销售人员,所面临的市场竞争环境是基本相同的。由于客观环境的变化,为每个推销员带来的机会也是基本一致的。因此,可以说机会面前人人平等。在这种情况下,谁能及时并充分地把握机会,创造销售佳绩,则完全依赖于销售人员自身的观察能力、分析能力、应变能力和创造能力。

        \item 可创造性。销售人员不应一味地消极适应环境变化,而是要充分发挥自己的主观能动性,积极 采取各种措施来诱导和创造有利于自己的销售机会。

        \item 时间性和空间性。销售环境的变化一般会带来销售机会,但这种销售机会不会无限期地持续下去,而是有一定的时间界限。错过了时间,也就错过了机会。而且,销售机会从地域上也不是可以无限延伸的,它有一定的空间范围限制,离开了特定的空间范围,销售机会就不存在了。

        \item 两面性。销售机会具有两面性。一方面,销售人员若及时采取恰当的措施,充分把握销售机会,就有可能获得推销成功。但若贻误时机,或决策失误,则有可能变主动为被动,陷于推销危机当中。另一方面,销售人员彼此之间存在着竞争的关系。同样的销售机会,若由于销售人员自己的原因而未能及时利用,则有可能成为其他销售人员实现交易的良机。

    \end{enumerate.zh}
