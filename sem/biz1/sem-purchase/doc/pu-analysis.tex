\section {采购需求分析概述}

    \subsection {采购需求分析}
    要进行采购,首先要分析弄清采购管理机构所代理的需求者究竟需要什么、需要多少、什么时候需要的间题,从而明确应当采购什么、采购多少、什么时候采购以及怎样采购的问题,得到一份确实可靠科学合理的采购清单。这个环节的工作,就叫做采购需求分析。需求分析是采购工作的第一步,是制定采购计划的基础和前提。

    在极简单的情况下,需求分析是很简单的。例如,在偶尔的单独采购中,单一品种并且与其他需求没有关联的情况下,需要什么、需要多少、什么时候需要的问题常明确,不需要进行复杂的需求分析也就清楚了。

    但是,某些生产制造企业或相关舫性较强的商品销售企业,在这种较复杂的采购背景下,进行必要合理的需求分析,就构成采购部门或需求部门的重要工作环节了。例如,一个电脑生产制造企业,有几千甚至上万个零部件、由各种不同类别的材料构成,有很多的加工车间、经历很多的生产工序,每个车间、每个工序生产这些零部件,都需要不同品种不同数量的原材料、工具、设备、辅助材料等,这么多的零部件,什么时候需要什么材料、需要多少、哪些品种要单独采购,哪些品种要联台采购、哪些品种先采购、哪些品种后采购、采购多少,这些问题不迸行认真的分析研究,就不可能进行科学的采购工作。

    因此,科学的采购必须要首先做出正确的采购决策。而采购决策的确定,主要就是采购需求的确定。也就是要确定正确的商品采购时间、正确的采购数和正确的采购品种。为了达到这个目的,需要有合理的和科学的需求分析方法才行。

    \subsection {采购需求分析的作用及主要方法}

    采购需求分析是采购部门或需求部门执行采购的第一步工作,通过对采购需求的分析,确定采购品类、数量及采购时间,主要是为了实现以下目的。

    \begin{enumerate}
        \item 从需求方角度来讲,精细化的需求描述,可以使需求方得到更为符台自己要求的产品,以达到提高需求方的使用目的。

        \item 从需求时间的分析来看,通过把握准确的供货时间,既保证需求部门的产品供应,避免缺货,同时也能够在库存控制中,降低库存成本,提高企业利润。

        \item 从供应商角度来看,准确及时地分析商品需求,有助于供应商提前备货,保证备货过程中商品的需求数量和质量要求,减少逆向物流运作,提高供应效率和经济效益。

        \item 准确的需求描述也是提高供应商和采购商服务水平,提升企业信誉的有效途径。进行采购需求分析与确定的方法有很多种,常见的有:采购需求表、统计分析预测法、ABC分析法、物资消耗定额管理分析方法、经济订货批量法、固定期间法、固定数量法、批对批法、物料需求计划法等。
    \end{enumerate}

