\subsection {采购的工作岗位及主要职责}

    根据企业性质不同,采购工作的主要指责也不尽相同,如生产制企业原材料采购、销售企业商品采购、政府设备采购等。

    但是,不管是何种商品采购,采购人员的工作能容基本类似。一般采购部门人员的业务主要包含:报表审批、文书、联络工作、谈判、跟催、价值分析、商品验收、财务结算等工作。表\ref{tab:workingTimeAlloc}和表\ref{tab:strategyTimeAlloc}分别是从作业层面和策略层面就不同的工作内容所占用的时间比例进行了对比介绍。

    \begin{table}[bcht]
        \centering
        \begin{tabular}{c|c|c}
            \hline
            序号 & 采购部门的工作内容 & 占用时间比例(\%) \\ \hline
            1 & 文书 & 50 \\ \hline
            2 & 跟催 & 20 \\ \hline
            3 & 推销员访谈 & 10 \\ \hline
            4 & 商品品质问题 & 10 \\ \hline
            5 & 开会 & 5 \\ \hline
            6 & 电话联络 & 5 \\ \hline
        \end{tabular}
        \caption{作业性工作的时间分配对比} \label{tab:workingTimeAlloc}
    \end{table}

    \begin{table}[bcht]
        \centering
        \begin{tabular}{c|c|c}
            \hline
            序号 & 采购部门的工作内容 & 占用时间比例 (\%) \\ \hline
            1 & 调研选择供应商 & 20 \\ \hline
            2 & 谈判 & 20 \\ \hline
            3 & 价值分析 & 15 \\ \hline
            4 & 品质改善 & 10 \\ \hline
            5 & 寻找替代来源 & 15 \\ \hline
            6 & 推销员访谈 & 10 \\ \hline
            7 & 文书作业 & 5 \\ \hline
            8 & 电话联络 & 5 \\ \hline
        \end{tabular}
        \caption{策略性时间分配} \label{tab:strategyTimeAlloc}
    \end{table}

    从采购部门的管理阶层考虑,一般可划分为几个层次:采购经理、采购科长、采购员、采购助理。采购部门人员的岗位指责描述如下。

    \begin{enumerate.zh}
        \item 采购经理:负责主要材料和关键项目采购,部门的统筹管理,主要工作内容包括拟定采购部门工作方针与目标,负责主要原料或无聊的采购,编制年度采购计划于预算,签订订单与合约,采购制度的建立与完善。

        \item 采购科长:负责安排、协调、协助采购员的工作,对并非关键无聊的采购员负责,部门关键问题的处理等,维持部门采购业务顺利开展。其主要工作内容包括分派采购人员及助理的日常工作,负责次要原料或无聊的采购,协助采购人员与供应商谈判价格、付款方式、交货日期等,采购进度的追踪;保险、公正、索赔的督导,审核一般物料的采购案,市场调查,供应商的考核等。

        \item 采购员:负责准备和发出制定商品供应或服务的采购订单。追求最低采购总成本,在价值最大化的同时,识别和发展合格的供应商。如果发生供应终端要恰当地加以协调。主要工作内容包括一般物料的采购,查访厂商,与供应商谈判价格、付款方式、交货日期等,要求供应商执行价值工程的工作,去认定交货日期,一般索赔案件的处理,处理退货,手机价格情报及替代品资料等工作。

        \item 采购助理:协助采购员做好采购订单定制工作,尤其是采购单操作过程中单据的登记和事项的记录,安排与接待访客等协调性工作。其主要工作内容包括请购单、验收单登记,订购单与合约的登记,交货记录及催促交货,访客的安排与节点,采购费用的申请与报支,电脑作业与裆案管理,办保险、公证事宜等。
    \end{enumerate.zh}
