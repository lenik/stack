\subsection {采购谈判的技巧}

    谈判技巧是采购人员的利器,谈判高手通常都愿意花时间去研究这些技巧,以求事半功倍,下列谈判技巧值得采购人员研究。

    \begin{enumerate}

        \item 谈判前要有充分的准备:知己知彼,百战百胜。成功的谈判最重要的步骤就是要先有充分的准备。采购人员的商品知识,对市场及价格的了解,对供需状况的了解,对本公司的了,对供货商的了解,对本公司能作出的价格底线、目标、上限,以及其他谈判的目标都必须先有所准备并列出优先级,将重点简短列在纸上,在谈判时随时参考,以提自己。

        \item 谈判是要避免谈判破裂:有经验的采购人员,不会让谈判完全破裂,否则根本不必谈判,他总会让对方留一点后路,一代下次谈判达成协议。没有达成协议总比勉强达成协议好。

        \item 只与有权决定的人谈判: 本公司的采购人员的接触对象可能有:业务代表、业务各级主管、经理、协理、副总经理,总经理或懂事长,支具体谈判对象要看供货商的规模大小而定。这些人的权限都不一样,采购人员应避免与没权决定事务的人谈判,以免浪费自己的时间,向时可避免先将本公司的立场透露绐对方。谈判之前,最好问清楚对方的权限。

        \item 尽量在本公司办公室内谈判:在自已的公司内谈判除了有心理上的优势外,还可随时得到其他同事、部门或主管的必要支持,同时还可节省时间与旅行的开支。

        \item 放长线钓大鱼:有经验的采购人员知道对手的需要,故尽量在小处满足对方,然后渐渐引导对方满足采购人员自己需要。避免先让对手知道自已的需要,否则对手会利用此弱点要求采购人员先做出让步。

        \item 采取主动,但避免让对方了解本公司的立场:攻击是最佳的防御,采购人员应尽量将自己预先准备的问题,你开放式的问话方式让对方尽量暴露出对方的立场,然后再采取主动,乘胜追击,给对方足够的压力,对方若难以招架,自然会做出让步。

        \item 必要时转移话题:若买卖双方对某一细节争论不休,无法淡拢,有经验的采购人员会转移话题,或喝个茶暂停,以缓和紧张气氛。

        \item 尽量以肯定的语气与对方谈话:否定的语气容易激怒对方,让对方没有面子,谈判因而难以进行。故采购人员应尽量肯定对方,称赞对方,给对方面子,因而给对方面子,也会给自己面子。

        \item 尽量成为一个好的倾听者:一般而言,业务人员总是认为自己是能言善道,比较喜欢讲话。采购人员应尽量让他们讲,从他们的言谈及肢体语言之中,采购人员可听出他们的优势劣势,也可了解他们的谈判立场。

        \item 尽量为对手着想:全世界只有极少数的人认为谈判时,丝毫不能让步。事实证明,大部分成功的采购谈判都是要在彼此和谐的气氛下进行才可能达成的。人都是爱面子的,任何人都不愿意在威胁的气氛下谈判,何况本公司与良好的供应商应有细水长流的合作关系,而不是对抗的关系。

        \item 以退为进:有些事情可能超出采购人员的权限或知识范围,采购人员不应操之过急,装出自己有权了解某事,做出不应做出的决定。此时不妨以退为进,与主管或同时研究或弄清楚事实情况后,再答复或决定也不迟,毕竟没有人是万事通的。草率仓促的决定大部分都是不好的决定,智者总是先深思熟虑,再做决定。

        \item 不要误认为50/50最好:有些采购人员认为谈判结果50/50最好,批次不伤和气。这是错误的想法。事实上,有经验的采购人员总会设法为自己公司争取最好的条件,然后让对方也得到一点好处,能对他们的公司有个交代。

        \item 谈判的十二戒:采购人员若能避免下列十二戒,谈判成功的机会大增。
            \begin{itemize}
                \item 准备不周;\item 缺乏警觉;\item 脾气暴躁;\item 自鸣得意;\item 过分谦虚;\item 不留情面;\item 轻诺寡信;\item 过分沉默;\item 无精打采;\item 仓促草率;\item 过分紧张;\item 贪得无厌。
            \end{itemize}

    \end{enumerate}


