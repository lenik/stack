\section {询价采购}

    询价采购,是指询价小组(由采购人的代表和有关专家共3人以上的单数组成,其中专家的人数不得少于成员总数的三分之二)根据采购需求,从符合相应资格条件的供应商名单中确定不少于三家的供应商向其发出询价单让其报价,由供应商一次报出不得更改的报价,然后询价小组在报价的基础上进行比较,并确定最优供应商的一种采购方式,也就是我们通常所说的货比三家,它是一种相对简单而又快速的采购方式。政府采购法规定实行询价采购方式的,应符合采购的货物规格、标准统一、现货货源充足且价格变化幅度小的政府采购项目。

    询价采购有如下几种类型:
        \begin{itemize}
        \item  询价采购还可分为报价采购、议价采购和订购。
        \item  报价采购是指采购方向供应商发出询价或征购函,请其正式报价的一种采购方法。
        \item  议价采购是指与供应商进行个别谈判,商定价格的一种采购方法。
        \item  订购是指利用订购单或订购函,列出采购所需物资及标准寄给供应商的一种采购方法。
        \end{itemize}

\subsection {询价采购过程}

    询价采购的具体过程如下:
    \begin{enumerate}
        \item  成立询价小组。

        这是执行询价采购方式的重要环节,询价小组由采购人的代表和有关专家共3人以上的单数组成,其中专家的人数不得少于成员总数的 2/3。要选择专业水平较高、素质全面的人士参加,专家组成的询价小组应对采购项目的价格构成和评定、成交标准等事项做出规定。询价小组根据所要采购的内容选定从符合相应价格条件下的供应商名单中选定3家以上的供应商,并且为询价采购做好充分的事前准备,如确定采购的需求、预测采购的风险等等。

        \item  确定被询价的供应商名单。

        询价小组根据所采购商品的特点及对供应商、承包商或服务提供者的要求,特别是根据要采购的内容,从符合相应资格条件的供应商名单中,选定3者以上的供应商。选择时必须依据所要采购的内容,同时考察各供应商的供应能力和资格条件,做出慎重选择。

        \item  询价。

        对所选定的供应商分别发出询价单,询价单的内容除了价格以外,还应包括商品品质、数量、规格、交货时间、交货方式、售后服务等内容,供应商应就询价单的内容如实填报。询价小组要求被询价的供应商一次报出不得更改的价格。这适用于采购现成的而并非按采购人要求的特定规格特别制造或提供的标准化货物,货源丰富且价格弹性变化不大的采购项目;在询价采购实践中,存在着与采购人等有串通现象,报出的价格不符合市场经济规律性;甚至出现有特殊的供应商报价后又更改的不良倾向。所以,这就要求询价小组一定要规范供应商的报价,坚持依法规定一次性报价;没有更改的机会,切防相互之的勾结腐败行为。

        \item  确定成交供应商。

        采购人根据符合采购需求、质量和服务相等且报价最低的原则确定成交供应商,并将结果通知所有被询价的未成交的供应商。采购人要在依法、科学、规范化地情况下进行;

          \begin{enumerate}
              \item  询价小组完成询价工作后,要形成询价报告,提交给采购人;
              \item  采购人要依法、科学、规范化地进行供应商的成交确定;
              \item  采购人在确定成交供应商时,必须严格执行事先确定的成交供应商评定标准;
              \item  采购人确定成交供应商后,要将结果通知所有被询价的未成交供应商。
            \end{enumerate}
    \end{enumerate}

\subsection {询价采购的风险形式}

    \begin{enumerate}
    \item  价格风险。

        询价采购遵循的是货比三家,通过多家供应商之间的竞争获取质优价廉的物品,但由于信息不对称的存在,加之询价采购报价方式的缺陷,导致了采购过程中的价格风险,具体表现为以下几个方面:一是价格垄断。市场经济的飞速发展,使得一批产品质量过硬、品牌信誉高的企业脱颖而出,这类企业往往是一些大型企业,具备完善的销售渠道和营销体系,在某个城市和地区都设有总代理商,掌握了一定区域市场价格的话语权,这使得在采购此类产品时,采购机构所询价的供应商都向总代理商申请特价,无形中形成了一个价格垄断体系,最终询价的对象实质上只有一家,即某品牌产品的地区总代理商,表面上公开.公平的询价采购只能流于形式,无法有效实现通过价格竞争获取最优价格的功能。

        合谋报价。合谋报价是指被询价的供应商互通信息,联手操纵价格给集中采购带来的风险。从笔者参与的多次询价采购来看,以办公自动化产品为例,由于被询价的供应商所在的地域相对集中,相互之间常有调剂产品的现象发生,某些供应商之间会达成一定的策略同盟,在回应采购机构的询价时会私下达成协议,促成某个供应商赢得采购合同,不难想见,此时所获取的价格就不是最优价格,这给集中采购的效益蒙上了一层阴影。

        低价陷阱。低价陷阱是指某些供应商为赢得采购合同和提供售后价策略,但在执行合同和提供售后服务上大打折扣,给采购的后续环节带来许多不确定因素,影响了采购整体效益的提高。从采购实践来看,此类风险在询价采购中并不鲜见,某些供应商在询价单上报出低价后,在执行合同时却以各种理由推脱,如无法按期交货、产品的配件或附加设备不全,有的甚至以次充优,偷梁换柱,给询价采购带来了隐患。

    \item  采购机构的道德风险

        由于信息不对称的普遍存在,询价采购中采购机构的道德风险也是不容回避的问题,采购机构的道德风险是指采购机构利用供应商所不知道的私有信息采取不利于供应商的行为,在询价采购中主要表现为:一是采购信息的不公开。在询价采购中,采购机构出于机构利益或个人利益的考虑,对本应公开的信息隐而不发,使某些具备实力的供应商失去参与机会,损害了潜在供应商的知情权。二是保护主义引发的对供应商的歧视政策。在采购活动中,采购机构往往出于保护老客户和本地关系户的目的,在制定询价采购方案和沟通采购信息方面对熟悉的供应商给予倾斜照顾,无形中造成了对外地供应商和潜在供应商的歧视,影响了供应商之间的公平竞争。三是确定询价供应商过程中的暗箱操作。《政府采购法》规定,确定供应商的程序、标准、依据等应该及时向供应商公布,然而在实际工作中,采购机构可能采取拖延信息发布,模糊选择标准等办法给供应商知晓信息造成壁垒,从而为确定询价供应商中的暗箱操作创造机会。

    \item  询价流程的失范风险

        现行《政府采购法》对询价采购的具体操作流程没有规范,各单位在进行询价采购时的程序各不相同,价流程缺乏统一性和规范性,这使得询价采购可能存在失范风险,影响了采购效益的提高。一是询价对象选择过程的失范。确定询价对象是询价采购的首要环节,也是影响询价采购效益的重要因素,在具体采购活动中,由于没有相关法律条文规范询价对象的选取方式,各单位在选取询价对象时的方法不尽相同,有的从熟悉的供应商中挑选,有的从以往打过交道的供应商中选取,没能较好体现询价对象选择的公开透明,对潜在合格供应商造成了一定的歧视。二是报价过程的失范。供应商在回应采购机构的询价时往往采取的是传真方式,由于各供应商报价时间不一,有些供应商便企图利用报价时间差获取其他供应商的报价信息,以调整自己的报价,影响了最优采购价格的实现。三是供应商评定标准的失范。在确定预成交供应商时,通常采取的是最低价法,即报价最低的供应商赢得采购合同,但在实际与机会,损害了潜在供应商的知情光采购画上了一个问号,这需要从询操作中,某些供应商虽然报价很低,但全寿命采购理念来看,此类供应商是不可取的,因此亟需从法规的角度规范供应商的评定标准,确保“物有所目标的实现。

    \item  供应商履约风险

        询价采购具有采购周期短的优点,但换个角度考虑,正是由于其组织设计上的缺陷,无法在事前和事中对供应商进行全面细致的考察,为后续履约环节埋下了隐患,表现在实际采购活动中,则是供应商在以低价获得采购合同后,却不能按期或保质保量地履行合同,主要体现在:一是有价无货,在笔者参与的多次询价采购中,有时供应商在以低价成为预成交商后,却以暂时无货为理由拒签合同,给采购任务的完成制造障碍;二是低价劣货,由于信息不对称,供应商在供货时往往采取以劣充优,以次充好来蒙蔽往往采取以劣充优,以次充好来蒙蔽用户,这类货物往往是返修率和故障率双高,造成了用户对采购机构的不满三是售后服务大打折扣,某些供应商在履行售后服务义务时往往推三阻四,以各种理由降低服务质量,使得所购产品的服务质量无法得到保障。

    \end{enumerate}

\subsection {询价采购风险的防范对策}

        询价采购的风险一定程度上给阳价采购制度和运行机制等方面加以完善和防范,以促进政府采购健康有序地发展。

    \begin{enumerate}
    \item  加强供应商资质审查,合理选择询价对象

        供应商资质审查是询价采购的一个重要环节,只有所询价的供应商均是资质优良、诚信良好的企业,才能有效地化解后续采购风险的发生,保证采购质量。但由于询价采购周期短,而供应商的资质却是一个长期的系统的考察过程,不可能在短期内对某个供应商的资质做出准确的评判,这就需要采购机构立足平时,做实做细这项工作。一是要建立完备的供应商数据库,供应商资料的收集非一朝一夕之功,需要采购机构在每次采购完成后,及时整理参与采购的供应商资料,按照产品分类建立相关供应商数据库,定期对产品市场进行考察,掌握产品的主要品牌、主要生产企业和区域总代理商。二是可以采取供应商资质年审制度。供应商资质年审制度符合供应商动态管理的要求,采购机构每年初可在指定媒体上发布年审公告,注明供应商所需呈交的资质材料,而后组织相关领域的专家对供应商进行评定,对供应商实施分级分类管理。三是加强与中介机构的信息沟通。某些行业协会或诚信评估机构对供应商的了解是非常全面的,采购机构应加强与此类机构的横向协作和信息沟通,既可以及时掌握丰富翔实的供应商资料,还能大大减少信息收集成本,可以有效提高供应商资质审查的效益。

    \item  建立健全采购机构内部监督制约机制

        采购机构内部监督制约机制的完善是有效防范采购机构道德风险的一剂良方,只有从制度上堵住采购人员徇私舞弊的入口,才能确保询价采购的公开、公正和公平。建立健全采购机构内部监督制约机制,一是要科学分工,明晰职责,在采购机构内部建立起分工协作、相互制约的工作机制。每次询价采购前要成立询价小组,小组应由包括采购人员和有关专家在内的三人以上单数构成,小组在询价前要明确各人分工、询价供应商的资质标准及选取方式,评价成交供应商的标准以及合同管理办法等。询价小组的运作要保证相互牵制、协调高效,以杜绝采购中可能出现的"一言堂"和暗箱操作。二是要充分发挥纪检审价部门的作用,采购机构内部的纪检审价人员应参与询价采购的整个流程,监督询价采购的各个环节是否按规程运作,在供应商选择和供应商评定等敏感环节是否有舞弊行为发生,最后确定的成交价是否合理等,以保证对询价采购的实时有效监督。

    \item  规范询价采购流程

        规范的询价采购流程是询价采购健康发展的关键所在,只有询价采购流程实现了规范化、制度化,才能有效提高采购效益。一是要明确询价对象的选取方式,按照"三公"原则的要求,采购机构应对所有资质合格的供应商一视同仁,确保机会均等,基于采购成本的考虑,不可能对所有供应商进行询价,因此在资质合格的供应商库中,只有通过随机抽取的方式,才能保证询价对象选取的公正。二是规范供应商报价过程。要有效杜绝供应商利用时间差获取其他供应商报价信息,必须规定所有被询价供应商在规定时间以密封形式报价,然后由询价小组统一拆封,进行比价。三是科学合理地制定供应商评定标准,要从全寿命角度考虑产品的成本,确保所采购产品在寿命期间内性价比最高。

    \item  发展与供应商的合作伙伴关系

        发展与供应商的合作伙伴关系是采购机构供应商管理内容上一次质的飞跃,通过建立采购机构与供应商之间共享信息、共担风险、共同获利的合作伙伴关系,可以有效降低双方之间的信息不对称程度,抑制询价采购中供应商违约等不利行为的发生。在合作伙伴关系建立的过程中,需要重点解决好以下几个问题:

        \begin{itemize}
            \item  充分利用同实施监督与激励。采购合同是供应商和采购机构为进行产品交易而签订的明确双方权利义务的书面协议,具有法律效力。在设计合同条款时,采购机构应充分考虑合作中可能出现的意外情况,如产品成本控制不力、交货不反时、产品质量有缺陆等。在合同中明确出现此类情况时的处理方式反双方责任、风险的分担。

            \item  参与供应商的长期规划,增加其有害选择的机会成本。采购机构通过参与供应商的长期规划,不仅可以加深相互的了解,而且使供应商相信长期合作的真实性,在其选择不利行为时,会充分考虑行为造成的对其长期利益的损害,增加这种有害选择的机会成本,保证双赢结果的实现。

            \item  及时收集用户反馈信息,加强对供应商监督。采购机构应当定期收集用户对供应商供货情况、产品质量情况和售后服务情况的反馈意见,井要求供应商&时处理。对一些可观测的供应商行为,如产品质量控制、产品交付能力等,采购机构可采取定期检查或抽查等形式,如供应商不符合标准,则采取相应的惩罚措施。对不可观泪IJ的供应商行为,如技术创新、产品成本控制等,可根据实际结果设计奖励合同,充分调动其改革积极性。

        \end{itemize}
    \end{enumerate.zh}

\subsection {询价采购的注意事项}

    第一,最大程度地公开询价信息。参照公开招标做法,金额较大或技术复杂的询价项目,其采购信息也应在省级、中央级媒体上发布,最起码应当在地级市的党报、采购网、电视台发布,扩大询价信息的知晓率,信息发布要保证时效性,让供应商有足够的响应时间,询价结果也应及时公布。通过公开信息从源头上减少“消息迟滞型”“不速之客”现象的出现。

    第二,更多地邀请符合条件的供应商参加询价。被询价对象确定要由询价小组集体确定。询价小组应根据采购需求,从符合相应资格条件的供应商名单确定不少于三家的供应商,被询价对象的数量不能仅满足三家的要求,力求让更多的符合条件的供应商参加到询价活动中来,以增加询价竞争的激烈程度。推行网上询价、传真报价、电话询价等多种询价方式,让路途较远不便亲来现场的供应商也能参加询价。

    第三,实质响应的供应商并非要拘泥于“三家以上”。政府采购法规定,只要发出询价邀请的供应商达三家以上即可,前来参加并对询价文件作实质响应的供应商并非要人为硬性地达到三家,但是起码要达到两家以上,询价采购由于项目一般较小往往让大牌供应商提不起兴趣,如果非得要达三家,询价极可能陷入“僵局”,重要的是要形成竞争,而非在供应商数量上斤斤计较。

    第四,不得定牌采购。指定品牌询价是询价采购中的最大弊病,并由此带来操控市场价格和货源等一系列连锁反应,在询价采购中定项目定配置定质量定服务而不定品牌,真正引入品牌竞争,将沉重打击陪询串标行为,让“木偶型”“不速之客”绝迹于询价采购活动,让采购人真正享用到政府采购带来的质优价廉的好东西。

    第五,不单纯以价格取舍供应商。法律规定“采购人根据符合采购需求、质量和服务相等且报价最低的原则确定成交供应商”,这是询价采购成交供应商确定的基本原则,但是不少人片面地认为既然是询价嘛,那么谁价格低谁“中标”,供应商在恶性的“价格战”中获利无几,忽视产品的质量和售后服务。过低的价格是以牺牲可靠的产品质量和良好的售后服务为条件的,无论是采购人还是供应商都应理性地对待价格问题。不可否认,价格是询价中的关键因素,但绝非惟一因素,在成交供应商确定上要综合评审比较价格、技术性指标和售后服务等,在此基础上依法确定。

\subsubsection {以“价格”论英雄 不是竞价采购与询价采购的初衷}

    在以买方市场为主的现实经济环境下,不管是采购企业还是供应商都在想方设法降低企业的生产与管理成本,以提高自身企业的市场竞争力。开始有一些企业应用网上竞价采购与询价采购的电子商务模式,然而有部分企业在使用竞价采购平台和询价采购时在最终评审过程中,存在仅以价格定标的现象,产生了产品质量与后继服务跟上等一些问题。而出现的这些现象并不是我们设计询价支持系统所想看到的,设计初衷是为了企业通过竞价采购平台从更多的优质供应商中找到最优价格的供应商,而非最低价的供应商。

      在实际的竞价平台的具体运用中,不少企业在实施“竞价采购平台”与“询价采购平台”时,往往只重视“价格”指标,看谁的“报价”低,就让谁中标。这不仅引起了不少供应商的异议,一些采购人也因中标商的后续服务不力等而不满这种采购行为。即使是竞价与询价采购,也不能仅仅以“价格”作为唯一的评标标准。

      一、以“价格”论英雄的危害性

    \begin{enumerate}
        \item  容易造成被询价对象低价抢标,进而影响到采购项目的质量。

        从客观上讲,对政府采购这块蛋糕,没有一个供应商不是垂涎欲滴的,谁都想中标,也都想赚钱,这是他们办企业搞经营的目的和宗旨。而在实际工作中,不少的采购企业及其采购代理机构却偏偏就是以供应商的“报价”来作为评定他们中标与否的唯一依据,这就“迫使”被询价供应商要想中标,就只能降低报价,并且也只能从“价格”指标上去“努力”,才能提升他们的“竞争力”。对此,不少的供应商面对“难得”的被询价机遇,他们就只能采用“低价”的手段去抢标,而不想过多去考虑这宗供应业务是否能“赚”多少钱,甚至于暂时也不考虑是否能赚钱,取得中标资格是首位的,而一旦他们中标后,就开始算起“细账”,项目供应就全部围绕其“赚钱”的目的,并大肆削减项目的供应成本,不是变相减低项目的配置标准,就是减少附件功能,或是采用低质甚至于劣质材料等措施,结果,采购项目的质量肯定受到严重的影响,产品的使用寿命也必将因此而下降,采购人的权益受到了很大的侵害。这就是以“低价”论英雄带来的弊端。

        \item  容易造成中标商后续服务不力,导致采购人权益难以得到保障。

        正是由于采购人或其采购代理机构对采购项目中标条件的确认,过分地集中在供应商的“报价”上,而不少的供应商在竞标时,为了能够中标,在“价格方面也已“被迫”作出了较大的“让步”,因此,他们在中标后,就不会再关心供应项目的后续服务了,因为,一方面是采购人自己没有将“服务”等指标提高到应有的重视程度,供应商就更不会主动去服务;另一方面,中标商也要赚钱,在较低的中标价位下,再持续提供那些“免费”服务,必然加大额外的成本开支。因此,如果单单就以“价格”论英雄,就必然会导致采购人得不到周到的后续服务。

        \item  容易造成企业采购只是追求资金节约的假象,进而影响企业采购形象。

        众所周知,企业采购的目的和宗旨不仅仅是为了节约采购资金,提高财政资金的使用效率,而是还必须要考虑到其他诸多的因素,如维护国家和社会利益、保护环境、保障当事人的正当权益等。而询价采购又是一种使用频率较高的企业采购方式。据不完全了解,在不少的县、市,有85\%以上的采购项目都是通过“询价”方式采购的,因此,如果对待仅仅“价格”来判断或认定供应商中标与否,势必就会给供应商造成一种企业采购的宗旨就是为了追求资金节约率,而不顾供应商应分享的利润,更不顾其他当事人正当权益的错觉,这就弯曲了企业采购的宗旨,直接影响了企业采购的外在形象。
    \end{enumerate}

    二、防范仅以“价格”指标作为中标条件的根本措施

    \begin{enumerate}
        \item  科学设置评标指标“体系”,杜绝以“价格”为单一评标标准。采购代理机构为采购人代理任何一个采购项目时,都应同时为采购人考虑到以下几个关键性的指标参数:

        \begin{itemize}
            \item  价格指标,即投标人(被询价人)对采购项目的报价情况,这是衡量采购资金使用效率(即节约额)的重要指标;
            \item  服务指标,即被询价人为采购人提供的售后服务工作,这是衡量采购人权益保障程度的重要指标;
            \item  质量要求,即对采购项目配置、性能等提出的具体要求;
            \item  采购人的需求,对任何一个采购项目,其质量、标准、配置等有高有低,档次有高、中、下,产品功能也有多有少等等,但是否适合于特定采购人的需求,还必须由采购人根据具体情况而定。
        \end{itemize}

        这些都是与采购人权益息息相关的缺一不可的重要指标,因此,作为用来评标的标准就必须要围绕这些基本因素来展开,针对不同采购人的具体情况,分清各个指标的轻重程度,采取综合评分法,评定出各供应商的综合竞争力,而不能仅仅以“价格”作为唯一的评标因素。

        \item  全面公示评标结果,遏制片面追求低价的采购行为。众所周知,询价采购是企业采购活动中常用的一种采购手段,但其不属于也不同于公开招标的采购方式,相对来说,询价采购的不少必要环节和特定的操作步骤都是在比较小的范围内实施的,这样,只有对询价采购过程中的相关信息充分对外公开,才能有效提高其采购操作的透明度,才能遏制住各种暗箱操作行为。因此,我们可以将询价采购中的投标人对各项指标的响应情况都一一公示出来,让所有参与投标的供应商都能相互对比和比较,以充分提高评标工作的公正性,这样就能有效抑制那些只重视价格上的效益,而忽视其他因素的采购行为。

        \item  严厉打击各种低价采购行为,维护采购法律法规的严肃性。对实际工作中无视法律规定只看中眼前供应商“让利”的短期采购行为,采购监督管理部门要责成有关单位和人员及时纠正,对因追求低价采购而影响项目质量,甚至于给采购人带来重大经济损失的,则要严肃追究有关人员的责任,触犯法律的,要追究法律责任,甚至于刑事责任等;对违反规定的采购代理机构,则要给予罚款,暂停、甚至于取消其采购代理资格的处罚等。

    \end{enumerate}

        如果说竞价采购与询价采购是降低采购成本的有效手段,那么也需要企业根据自身采购商品的实际属性与市场规律来全面审核供应商,而非仅以"价格"论英雄,以免给企业带来不必要的麻烦。

\subsubsection {询价采购的常见问题}

    \begin{enumerate}
        \item  询价信息公开面较狭窄,局限在有限少数供应商,一般很少在采购信息发布指定媒体上发布询价公告,满足于三家的最低要求,排外现象较严重。从财政部指定的采购信息发布媒体上很难发现询价信息,很多询价项目信息不公开,不但外地供应商无从知晓相关的采购信息,而且当地的供应商也会遭遇“信息失灵”,不少询价项目的金额还挺大,但是信息却处于“保密”状态,为代理机构和采购人实施“暗箱操作”提供了极大便利,一些实力雄厚的供应商只能靠边站,“望询兴叹”,采购人意见纷纷却很无奈。
        \item  询价采购出现超范围适用,法律规定适用通用、价格变化小、市场货源充足的采购项目,实际工作中则是以采购项目的概算大小来决定是否采用询价方式。询价并不是通用的“灵丹妙药”,有着确切的适用条件,实际工作中一些代理机构和采购人将询价作为主要采购方式,错误地认为只要招标搞不了的,就采用询价方式,普通存在滥用、错用、乱用询价方式问题,代理机构隔三差五搞询价,忙得“不亦乐乎”,被琐碎的事务缠身,采购效率和规模效应低下,还有些人借询价规避招标。
        \item  询价过于倾向报价,忽视对供应商资格性审查和服务质量的考察。法律规定“采购人根据符合采购需求、质量和服务相等且报价最低的原则确定成交供应商”,这是询价采购成交供应商确定的基本原则,但是不少人片面地认为既然是询价嘛,那么谁价格低谁“中标”,供应商在恶性的“价格战”中获利无几,忽视产品的质量和售后服务。指定品牌询价现象比较突出。
        \item  确定被询价的供应商主观性和随意性大。被询价对象应由询价小组确定,但是往往被采购人或代理机构“代劳”,在确定询价对象时会凭个人好恶取舍,主观性较大。法律还规定从符合相应资格条件的供应商名单中确定不少于三家的供应商,一些采购人和代理机构怕麻烦不愿意邀请过多的供应商,只执行法律规定的“下限”,某代理机构的询价资料中被询价的供应商一律为三家,还有些询价项目,参与的供应商只有二家,甚至仅有一家。询价一般不设询价保证金。
        \item  询价采购的文件过于单薄,往往就是一张报价表,基本的合同条款也会被省略。法律规定询价采购应制作询价通知书,在一些询价采购活动中,询价方一般不会制作询价通知书,多采取电话通知方式,即使制作询价通知书,内容也不够完整,且规范性较差,价格构成、评标成交标准、保证金、合同条款等关键性的内容表述不全,影响了询价的公正性,不少询价采购结束后采购双方不签合同,权利义务不明确,引发了不必要的纠纷。
        \item  询价小组组成存在问题,采购代理机构人员介入小组,专家数量和比例不足法定要求。法律规定“询价小组由采购人的代表和有关专家共三人以上的单数组成,其中专家的人数不得少于成员总数的三分之二”。询价的主体应是询价小组,但有些代理机构却直接操作,既不通知采购人代表参加,也不商请有关专家,还有些代理机构虽然依法组成了询价小组,但是小组的专业化水准很低,更多的是专家人数根本无法达到三分之二,试想让“外行”来从事询价,确实让人不放心。
        \item  采购活动的后续工作比较薄弱。不搞询价采购活动记录,不现场公布询价结果,询价方式随意性大。一些地方尝试采用电话询价、传真报价、网上竞价等方式搞询价采购,尽管这些有便利之处,但不宜过多地使用,法律规定在询价过程中供应商一次报出不得更改的价格,采用非现场方式搞询价存在舞弊漏洞,采购方有机会随意更改任何一家供应商的报价,或者给有关供应商“通风报信”。
    \end{enumerate}
