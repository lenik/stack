\section{定单管理}
    \ul {
    \item \opset {新建生产订单} {
        \ops {
        \item  点击工具栏上的\button{新建}按钮,进入新建生产订单界面
        \screenshot{11.png}
        \item  输入\button{摘要}
        \item  输入\textbox{附加要求}
        \item 选择\button{客户}
        \item 选择对应\button{销售机会}
        \item  点击\button{添加}按钮弹出添加订单项目对话框
        \screenshot{12.png}
            \ul {
            \item  选择订单项目对应的\button{物料}
            \item  输入\textbox{单价}
            \item  输入\textbox{数量}
            \item  输入\textbox{对外名称}
            \item  输入\textbox{对外规格型号}
            \item  输入\textbox{备注}
            \item  选择\button{币种},\button{铭牌}和\button{交货时间}
            \item  点击\button{确定}按钮,增加一个生产订单项目
            }
        \screenshot{13.png}
        \item  点击\button{保存}按钮,保存新建的生产订单
        }
    }
    \item \opset {修改生产订单} {
    \screenshot{14.png}
        \ops {
        \item  选择需要修改的生产订单
        \item  点击工具栏上的\button{修改}按钮,进入修改生产订单页面
        \item  修改相应需要修改的内容
        \item  点击\button{保存}按钮,保存更改后的生产订单
        }
    }
    \item \opset {删除生产订单} {
    \screenshot{15.png}
        \ops {
        \item  选择需要删除的生产订单
        \item  点击工具栏上的\button{删除}按钮,在弹出的确认对话框中点击\button{确定},删除对应的生产订单
        }
    }
    \item \opset {将生产任务单转换为PDF格式} {}
    }
选择需要打印的生产任务单,点击工具栏上的PDF按钮。系统会生成此生产任务单对应的PDF文件。方便打印。


\section{生产任务管理}
    \ul {
    \item \opset {新建生产任务单} {
        \ops {
        \item  点击工具栏上的\button{新建}按钮,进入到新建生产任务单界面
        \screenshot{16.png}
        \item  选择相应的\button{生产订单},系统自动导入生产项目
        \screenshot{17.png}
        \item  编辑\button{摘要信息}
        \item  如果需要修改生产项目
            \ul {
            \item  选中需要修改的生产项目
            \item  点击\button{修改}按钮,进入到修改生产项目界面
            \item  修改相应需要修改的项目
            \item  点击\button{确定}按钮,保存修改后的生产项目
            \screenshot{18.png}
            }
        \item  点击\button{保存}按钮,保存新建的生产任务单
        }
    }
    \item \opset {修改生产任务单} {
        \ops {
        \item  选择需要修改的生产任务单
        \item  点击工具栏上的\button{修改}按钮,进入到生产任务单修改页面
        \item  修改相应需要修改的内容
        \item  点击\button{保存}按钮,保存更改后的胜场任务单
        }
    }
    \item \opset {删除生产任务单} {
        \ops {
        \item  选择需要删除的生产任务单
        \item  点击工具栏上的\button{删除}按钮,在弹出的确认对话框中点击\button{确定}按钮,删除相应的生产任务单
        \screenshot{19.png}
        }
    }
    }


\section{物料计划管理}
    \ul {
    \item \opset {新建物料计划单} {
        \ops {
        \item  点击工具栏上的\button{新建}按钮,进入新建物料计划单界面
        \screenshot{20.png}
        \item  选择对应的\button{生产任务单}或者\button{订单},系统会自动计算出物料计划项
        \screenshot{21.png}
        \item  编辑\button{物料计划项}明细
        \item  点击库存锁定标签,点击\button{库存导入}按钮,导入仓库中相关原料
        \screenshot{22.png}
        \item  编辑\button{库存锁定项}明细
        \item  点击\button{保存}按钮,保存新建的物料计划单
        }
    }
    \item \opset {修改物料计划单} {
        \ops {
        \item  选择需要修改的物料计划单
        \item  点击工具栏上的\button{修改}按钮,进入修改物料计划单页面
        \item  修改相应的内容
        \item  点击\button{保存}按钮,保存修改后的物料计划单
        }
    }
    \item \opset {删除物料计划单} {
        \ops {
        \item  选择需要删除的物料计划单
        \item  点击工具栏上的\button{删除}按钮,在弹出的确认对话框中,点击\button{确认}按钮,删除相应的物料计划单
        \screenshot{23.png}
        }
    }
    }
