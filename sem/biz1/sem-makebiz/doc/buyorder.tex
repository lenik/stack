\section {采购订单管理}

    采购订单是存货在采购业务中流动的起点,是详细记录企业物流的循环流动轨迹、累积企业管理决策所需要的经营运作信息的关键。通过它可以直接向供应商订货并可查询采购订单的收货情况和订单执行状况,通过采购订单的关联跟踪,采购业务的处理过程可以一目了然。

    采购订单标明了
        \begin{itemize}
            \item  供应商
            \item  要订购的物料或服务
            \item  数量
            \item  价格
            \item  供货日期和供货条款
            \item  支付条款
        \end{itemize}
    此外,采购订单确定订购的物料是存入库存还是在收货时就直接被消耗。采购订单须经核准。

\subsection {采购订单的分单}

        采购业务的目标,一是满足对物料的需求,二是降低成本,第三还包括质量,质量是采购业务能够发生的前提条件,肯定要满足。在采购管理中还有一个重要管理内容,那就是如何做内控,因为采购往往是企业花钱最多的部门,内控做得不好,最终影响产品成本和质量。

        订单分配是采购管理的一个核心问题,也是一个难题,因为分单需要综合平衡内外各种关系以实现采购业务目标,同时还要加强内控。假设采购一种物料N吨,有三个供应商ABC,如何给三个供应商分采购量?

    \subsubsection {策略方式一}

        80/20方法。绝大部分给A企业(例如80\%),另外小部分给另外B企业,或者B、C企业;这样做的好处是绝大部分订单量给A,可以获得批量经济折扣,获得更低成本。B、C做为培养关系的供应商,在A出问题时候可以快速补充,相当于备份的大供应商,可以避免供应中断。

        这种操作方式存在一种内控上的难题,最大的采购量到底给哪一家?面向各个供应商采购不同的量,采购价格如何确定?供应商可以通过向采购决策人公关以获取最大的采购量,而供应量小的供应商也可以公关获得好的采购价格,因为量不一样,价格和采购量大的不能完全比较。实际上大小供应商都有发挥空间。

        解决的方法:通过供应商评估确定优先供应商,备份供应商,明确主供应商和备份供应商的量的分配比例原则,减少了第一层次的内控(不过,这需要定期评估,以评估小组操作);第二层次的内控,那只有参考市场价格了,很难内控;或许可以通过对采购人员整体部门KPI的相关考核来加强内控,比如考核年采购成本降低金额等。

    \subsubsection {策略方式二}

        平分方法。一些企业的分单策略是平分:采购订单在三个供应商之间平分,这是一个非常简单的处理原则,很容易操作,很容易检查。

        操作过程:采购量在三个供应商之间平分。在了解市场价格基础上,和三个供应商谈判价格,取三个供应商价格最低者,要求另外两家也达到最低价格(因为别的供应商能够达到,要求你达到这个价格是合理要求);在谈判中的策略是:“A供应商已经做到10元了,B你也应该可以做到10元。”在和B谈判妥当的时候,三家供应商一起签订合同,和三家供应商价格大家都能够看到,有一定的透明性,证明不是压价策略。这种谈判最多做一轮,不会重复要求供应商压价。出价高的供应商晚获得订单,如果不降价,失去订单也是可能的。

        策略分析:这样做的原因很重要一点是加强内控。采购内控主要有两个点:一是决定哪个供应商可以入围成为合格供应商,另外一个是分单多少。入围合格供应商是产品开发部门确定的,一次性的;分单是经常性的,是内控重点。采取平分原则,则是的分单透明化,供应商不用公关,可以减少公关成本,消除一个在订单数量分配上的内控问题。其次三个供应商均要求实现最低价,消除在价格上的内控问题。如果供应商在数量和价格上均是凭实力,不需要公关,那么成为企业的合格供应商也需要凭实力,即使不能完全消除第一个方面内控问题(成为合格供应商)也关系不大,毕竟只是一次性的事情。

        看来,这种貌似不好的分单策略(不能获得规模效益),还有深刻的内控优势,也是一个不错的策略。当然,应用这个策略的企业还未规范开展供应商管理,假如有规范的供应商管理,这是否是一种好的策略,值得探讨。

\subsection {采购订单的更新控制}

    虽然系统会自动根据采购计划生成采购订单,但是在实际工作中,采购员往往还需要对采购订单中的部分内容进行更新。为了加强控制,无疑需要对这个更新作业进行更细致、更深入的管理。那么在采购订单中更新部分内容会触发那些业务逻辑呢?笔者在这里根据自己的了解作一番分析。希望这个分析报告可以帮助大家更加深入的了解单据的更新操作。

    采购订单中有很多内容都是从产品基本信息表中取得。如产品的规格描述、产品的计量单位、产品的包装要求等等。有时候企业根据实际需要,可能需要在采购订单上更新这些信息。如以前供应商送原材料过来的时候是4个一包的。现在企业为了某种原因,如生产线的要求,需要把这个包装数量改为3个一包或者5个一包。此时,在采购订单的包装说明中可以进行更改。更改保存后,就完事了吗?其实,这背后还暗藏着一个业务逻辑。即这个包装信息、产品规格等等属于产品的基本信息范畴。那么在采购订单上进行更改后,是否要求采购订单在保存更新的同时去更新产品基本信息表中的内容吗?还是只是采购订单上更新即可?

    到底采用什么方式,主要根据企业业务性质来判断的。如果这个更新只是一个临时的调整,那么就不需要更新产品基本信息中的内容。但是,如果这个更新是永久的,以后都要采用这个包装方式的话,则就需要在更新采购订单的同时更新产品基本信息表中的内容。不过产品基本信息表中的内容毕竟比较敏感,如果想通过订单关联更新产品基本信息表中的内容,则要符合一些业务逻辑的控制规则。具体来说,主要有三个方面。

    \begin{enumerate}
        \item  产品信息表中必须指明某个字段可以被其他单据所更改。

        这主要是为了保证数据的一致性。因为在企业中,原材料等基本信息往往不是采购人员建立,而是研发部门等建立。为此,研发部门有这个权利那些内容可以被其他人员更改。如此的话,当其他部门通过其他单据更改了某部分内容之后,作为产品信息的主人,就比较容易追踪。在实际工作中,笔者建议企业用户,把一些关系不是很大的内容,如包装方式等等可以让其他员工进行更改,以减少信息建立人员的工作量。但是,对于一些关键的参数,如原材料检验标准、原材料规格等字段的话,最好还是谁建立谁更改。

        \item  用户需要有这张表对应的权限

        如在ERP系统权限设计中有一个排它权限。如果某个用户做了这个限制之后,则他建立的信息可能就只有他自己能够进行更改。其他用户无权进行修改。如果有这个限制的话,则其他用户就无法通过采购订单等相关单据更新这个产品基本信息表。

        \item  是采购订单中的控制

        在实际工作中,可能需要更新与不需要更新两种情况同时存在。是否需要更新产品基本信息表的内容需要采购员根据实际情况来进行判断。为此,在采购订单更新用户按保存后,系统就会进行判断。更改的内容是否涉及到产品基本信息表中的内容。如果涉及到而且产品基本信息表中又指定这个字段可以被更改的话,则系统就会提示用户是否需要把这个更新同步到产品基本信息表中。如果用户选择是的话,则这个更新会被同步到产品基本信息表中。如果选择否的话,则只是在采购订单上进行更新,而不会涉及到产品基本信息表。
    \end{enumerate}

    也就是说,要同时满足以上三个条件,产品基本信息表中的内容才能够被采购订单所更新。笔者在项目推广中,对于用户的建议是这个功能要慎用。对于一些共享程度比较高的信息可以通过级联更新来节省数据维护的工作量。但是对于一些技术性比较强的数据,则最好还是采取专人维护专人负责制比较好。
与产品价格信息表关系

    除了会对采购订单中的产品基本信息如包装信息进行更改,采购员改的最多的还要算是采购订单价格。这个采购订单价格在采购订单管理中又是一个比较敏感的字段。为了保障企业资金的安全性,采购订单在这个字段的更新上采取了比较多的控制措施。
    \begin{enumerate}

        \item  采购订单价格更新

        首先,采购订单会控制采购员是否有这个权限更改这个采购价格以及更改的幅度有多大。有些企业对于这个采购价格控制的比较严格,普通采购员无法更改这个采购价格。因为采购订单中的价格自动会从供应商产品价格表中带出来。也就是说,默认的价格就是跟供应商协商好的价格。若要进行价格变动的话,则必须由采购经理来完成。有些企业则相对宽松一点,采购员可以更改采购订单的价格,但是其有一个幅度的限制。如某个原材料标准价格为10元,而采购员可以在1\%的范围内修改这个采购价格。也就是说,其向供应商采购时,其价格的最大修改权利只有11元。如果超过这个价格的话,就需要采购经理或者其他人员的授权才行。可见通过对采购价格这个字段本身的权限控制,能够大大的提高采购订单价格的准确性。明显消除采购员与供应商串通的徇私舞弊行为。

        \item  采购订单价格更新对其他数据表的影响

        采购订单的价格更新是否会对其他相关表产生影响呢?这里的相关表主要包括两张数据表,分别为产品价格信息表与产品供应商价格信息表。产品价格信息表中定义了产品的计划价格;产品供应商价格信息表则定义了某个供应商的具体价格信息。他们之间有彼此的联系。如产品供应商价格信息表默认情况下,会继承产品计划价格表中的价格。而在建立采购订单的时候,默认情况下其价格来自于产品供应商价格信息表。如果这张表中没有相应数据的话,则其会采用产品计划价格表中内容。而现在反过来,如果采购订单中物料价格的变化,是否会更新以上这些表中的内容呢?
    \end{enumerate}

    笔者在项目实施中,给用户第一个建议是最好不要通过采购订单来更新这两张基本价格表中的信息。若产品计划价格表中或者供应商价格信息表中的内容有改变的话,最好通过独立的变更单据来进行变更。这主要是出于数据的一致性考虑。而且也易于后续的查询追踪。如果在采购订单中进行级联更新的话,以后很难查询到这个更改记录。虽然这会增加一些数据维护的工作量。但是却可以保持数据的一致性,笔者认为这个交易还是值得的。

    \textbf {建议}

    笔者给用户的第二个建议是如果供应商价格变化真的很频繁。就像前一段时间,原材料的价格几乎是一月一变。在这种情况下,如果每次价格变化时都要先更改供应商价格信息表中的内容,那显然工作量会变得很大。而每次去更改采购订单的价格也不利于采购订单的管理。对于这种情况,如果原材料市场价格变换比较频繁,那么笔者建议企业用户,可以让采购订单更改供应商价格信息表,但是级联更新只限于此,而不能够更改产品基本价格表。因为产品基本价格表不仅跟采购订单或者供应商价格表相关;而且还是计划成本计算的来源。而产品计划成本在一段时间内必须是稳定的。如故其也是随着市场价格波动而波动的话,就不利于进行计划成本与实际成本的对比分析。故在任何情况下,产品基本价格表的信息都不能够通过采购订单或者采购订单变更单来更新。而只能够通过独立的单据来进行更新。

    而且,产品标准价格表中的价格信息也不能够经常变更。最好其更改的频率不要超过每月一次。否则的话,会导致计划成本的不稳定。不利于后续的统计分析。同时,也会增加维护的工作量。如笔者有一家客户,即使在价格变动频繁的市场环境中,其原材料计划价格也是一个季度调整一次。

    由于采购订单中价格更新涉及到三个层次的价格表信息,所以更新控制就会更加的严格。笔者维护的多个ERP项目中,其用的最多的控制方式如下:

    \begin{enumerate}
        \item  在采购订单控制上,采购员可以更改价格。但是其更改后的价格不能够超过供应商基本价格表中的价格。也就是说,采购员降低采购价格是允许的,但是若要调高采购价格则是不允许的。并且系统会有一张采购订单实际价格与供应商价格信息表的分析报表,用来对采购员进行考核。

        \item  采购订单价格更新一般情况下不会更新其他基础价格表中的内容。最多只是更改供应商价格表中的价格信息。
    \end{enumerate}
