\subsubsection{ \nequiv (无标题) } {
    \addtolength{\leftskip}{5mm}
    (无描述)


    \maketxtable{(无标题)}{palette}{TX.Palette.tab}
}
\subsubsection{ \nequiv 调色板条目 } {
    \addtolength{\leftskip}{5mm}
    (无描述)


    \maketxtable{调色板条目}{palette_entry}{TX.PaletteEntry.tab}
}
\subsubsection{ \nequiv 银行 } {
    \addtolength{\leftskip}{5mm}
    (无描述)


    \maketxtable{银行}{bank}{TX.Bank.tab}
}
\subsubsection{ \nequiv 外汇的相关术语。
<ul>
<li>卖出价是银行将外币卖给客户的牌价,也就是客户到银行购汇时的牌价;
<li>而买入价则是银行向客户买入外汇或外币时的牌价,它分为现钞买入价和现汇买入价两种。
<li>现汇买入价是银行买入现汇时的牌价, 而现钞买入价则是银行买入外币现钞时的牌价。
<li>现汇和现钞是不同的概念,是外币存入银行后的两种不同形态。钞是可以存入取出的,汇不行,只能兑换成钞才可以取; 但汇是可以像汇款一样汇往国外的,钞不行,一定要兑换成现汇。
至于为什么会不同,那是因为银行在外汇交易的过程中会承担风险,所以要控制价差赚取提供服务的费用。 现汇卖出价和现钞卖出价是相同的,即卖出价。
<li>中间价= 现汇买入价+现汇卖出价)/2,中间价是市场所形成的。
<li>基准价是人民银行公布的一种中间价,其他商业银行可在基准价基础上,按照人行规定的浮动范围制定自己的买入、卖出价。 基准价是人民银行公布的。
</ul> } {
    \addtolength{\leftskip}{5mm}
    (无描述)


    \maketxtable{外汇的相关术语。
<ul>
<li>卖出价是银行将外币卖给客户的牌价,也就是客户到银行购汇时的牌价;
<li>而买入价则是银行向客户买入外汇或外币时的牌价,它分为现钞买入价和现汇买入价两种。
<li>现汇买入价是银行买入现汇时的牌价, 而现钞买入价则是银行买入外币现钞时的牌价。
<li>现汇和现钞是不同的概念,是外币存入银行后的两种不同形态。钞是可以存入取出的,汇不行,只能兑换成钞才可以取; 但汇是可以像汇款一样汇往国外的,钞不行,一定要兑换成现汇。
至于为什么会不同,那是因为银行在外汇交易的过程中会承担风险,所以要控制价差赚取提供服务的费用。 现汇卖出价和现钞卖出价是相同的,即卖出价。
<li>中间价= 现汇买入价+现汇卖出价)/2,中间价是市场所形成的。
<li>基准价是人民银行公布的一种中间价,其他商业银行可在基准价基础上,按照人行规定的浮动范围制定自己的买入、卖出价。 基准价是人民银行公布的。
</ul>}{fxr_record}{TX.FxrRecord.tab}
}
\subsubsection{ \nequiv (无标题) } {
    \addtolength{\leftskip}{5mm}
    (无描述)


    \maketxtable{(无标题)}{abstract_item}{TX.AbstractItem.tab}
}
\subsubsection{ \nequiv (无标题) } {
    \addtolength{\leftskip}{5mm}
    (无描述)


    \maketxtable{(无标题)}{abstract_item_list}{TX.AbstractItemList.tab}
}
\subsubsection{ \nequiv (无标题) } {
    \addtolength{\leftskip}{5mm}
    (无描述)


    \maketxtable{(无标题)}{thing}{TX.Thing.tab}
}
\subsubsection{ \nequiv [字典] 单位 } {
    \addtolength{\leftskip}{5mm}
    (无描述)


    \maketxtable{[字典] 单位}{unit}{TX.Unit.tab}
}
\subsubsection{ \nequiv 单位换算表 } {
    \addtolength{\leftskip}{5mm}
    (无描述)


    \maketxtable{单位换算表}{unit_conv}{TX.UnitConv.tab}
}
