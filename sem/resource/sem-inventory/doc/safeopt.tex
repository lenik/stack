\subsection { 优化安全库存}

\subsubsection { 术语和计算}

    \begin{itemize}
        \item  正态分布(Normal distribution):概率论中最重要的一种分布,也是自然界最常见的一种分布。该分布由两个参数——平均值和方差决定。概率密度函数曲线以均值为对称中线,方差越小,分布越集中在均值附近。

        \item  标准差(Standard deviation):也称均方差(Mean square error),是各数据偏离平均数的距离的平均数,它是离均差平方和平均后的方根,用σ表示。标准差是方差的算术平方根。计算步骤如下:

        \begin{enumerate}
            \item  算出一组数据的平均值。
            \item  计算该组每个数据与第一步的到的平均值之间的差值。
            \item  将第二步得到的差值进行平方。
            \item  计算由第三步得到的所有平方值的平均值。
            \item  将第四步计算得到的平均值再开平方,即得到标准差。
        \end{enumerate}

        在Excel中,可以使用STDEVPA函数计算一组数据的标准差。在计算安全库存时,可以利用未来一段时间内的需求预测数据来确定标准差。

        \item  交付周期(Lead time):自确认物料需求直至物料交付到库并可以使用的全过程所需的时间。包括了供应商或采购方自制生产周期,制作并下达采购订单或生产工单的周期(含订单处理时间和所需批准时间),及收货和检验时间。

        \item  交付周期期间需求量(Lead-time demand):上述交付周期期间的预测需求量。假设交付周期为10天,每天的预测用量为100件,则交付周期期间需求量为1000件。

        \item  预测(Forecast):很多组织都有专门的预测部门和有关未来一段时期内的需求预测数据。如果没有预测数据,可以使用过去一段时期内的平均用量来代替。

        \item  预测期间(Forecast period):预测所基于的一段时期,如预测未来4周,或3个月的需求。

        \item  需求历史(Demand history):一般而言,某项物料的需求历史越久远,提供的需求信息也越丰富,并能从中看出销售模型,如季节性、周期性、趋势性等。

        \item  订购周期(Order cycle),又称补给周期(Replenishment cycle):指两次订购之间的间隔时间。最简单的计算方法是:

        \begin{enumerate}
            \item  用全年的总用量除以每次订购的数量,得到订购次数。
            \item  用365天除以订购次数,计算得到每两次订购之间的平均间隔时间(天数)。
        \end{enumerate}

        \item  再订购点(Reorder point):触发库存补充的最低库存水平。再订购点=交付周期期间需求量+安全库存

        \item  服务水平(Service level):用百分数表示的所期望的对客户(内部和外部)的服务水平。

        \item  服务因子(Service factor):用来乘以标准差从而计算可以满足所期望服务水平的特点存货数量。在Excel中,可以使用NORMSINV函数将服务水平百分数转换成服务因子。

    \end{itemize}

\subsubsection { 安全库存的计算公式及影响因素}

    假设需求服从正态分布,未来一周内的日均预测需求为100件,如果期初库存数保持在500件,那么从概率上来说,可以满足实际需求的可能性为50\%,因为实际需求大于一周500件或小于一周500件的概率各位50\%。此时的标准差为零。假如我们增加一个标准差,也就是在平均需求的基础上乘以一个标准差,我们能够满足实际需求的可能性则提高到84\%;如果乘以两个标准差,那么满足实际需求的可能性则提高到98\%;如果乘以三个标准差,可能性则可以提高到99.85\%。给平均需求加上越高的标准差乘数,则满足实际需求的可能性越高,同时也意味着越高的库存数量水平。

    在计算安全库存时,我们将上面说到的标准差乘数作为服务因子,并使用预测期间需求数据或历史需求数据计算出标准差。因此,最简单的安全库存计算方法为:安全库存量 = 标准差 X 服务因子。这个公式成立的条件是,交付周期(Lead Time)、订购周期(Order Cycle)以及预测期间(Forecast Period)三者都相同。

    而实际上,这三个周期都相同的情况是难以达成的。因此,我们需要将这三个周期都加以量化,得到一个更加普遍适用的安全库存计算公式,即:安全库存量=标准差X服务因子X交付周期因子X订购周期因子X预测平均需求因子。

    下面是两个新的因子的一般计算方法:

    \begin{enumerate}
        \item  交付周期因子(Lead Time factor):用交付周期除以预测期间,再开平方。
        \item  订购周期因子(Order-cycle factor):用预测期间除以订购周期,再开平方。
    \end{enumerate}

\subsubsection { 安全库存设定中的其他问题}

    预测平均需求差异因素:当安全库存中用到的标准差的计算是基于预测平均需求时,实际平均需求与预测平均需求之间的差异会给整个安全库存的计算和设定也带来偏差。理论上可以用实际需求平均值减去预测需求平均值再除以日最高需求量和日最低需求量这二者的平均值计算,然而在设定安全库存的时候,无法获得将来实际需求的数据,因此变通的方法可以用历史期间实际需求平均值减去对那段历史期间原来的预测需求平均值,再除历史期间日最高用量和最低用量的平均值的到的因子加以替代。

    交付周期变动因素:实务中,实际交付周期与标准交付周期也存在着差异。与上述预测平均需求因子相类似,可以用过去若干个历史期间的差异进行估计。至于取值需要在满足更高的服务水平和相应的更高库存导致的库存持有成本中进行权衡后决定。
