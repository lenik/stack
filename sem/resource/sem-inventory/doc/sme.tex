\subsection {中小企业的库存管理问题}

    物资储备以保证企业正常的生产和销售。库存是任何企业必不可少的基本要数,但是又占用着经营者的资金。企业要想提高效益,必须有效地控制库存。于是,企业的经营者们想尽一切办法来降低库存,在占有尽量少的资金前提下,最大限度地满足企业的正常生产和经营。甚至有人提出了零库存的概念,但零库存也还是有库存,只是通过类似VMI的方式将库存占用的资金降低趋向为零。对于企业管理层特别是采购供应链管理者而言,两头平衡库存与资金占用量是不可缺少的功课。

    中小企业是我国经济建没的重要力量,目前中国的中小企业占全部企业数的99\%以上、国内生产总值的50\%。但是我国许多中小企业不同程度地存在影响其发展的问题,主要是企业信息化水平不高。为此,在首届中国西部地区对外经济合作暨中小企业信息化论坛上,国家提出要把推进中小企业信息化进程放在企业发展的优先位置,充分利用信息技术,不断提高企业应用水平,提高企业的开发创新能力、经营管理能力和竞争力。然而,中小企业在其信息化的进程中却举步维艰。信息化程度很低,没有通过使用信息系统改变企业的管理模式和决策程序。致使企业对家底摸不清,对市场变化反应不灵。这种特点是由中小企业库存管理的不足引起的。

    这里分析一下目前中小企业中库存管理存在哪些问题,通过分析企业在库存管理方面存在的问题,提出一些解决的思路。并通过调研青岛胶南几家经营企业以及青岛三江源建筑添加剂有限公司获得的数据进行分析说明。

    \subsubsection { 物资储备过高,库存结构不合理。}

    现在很多中小企业还是完全凭经验管理,对所有存货采用统一的库存控制方式。没有对重点存货进行重点管理,是一种较为粗放的库存管理。没有科学规范的方法,只能设置较高的安全库存以应付企业经常面临的预期外需求,而不是根据不同存货的服务水平以及各自提前期来确定。因此,在物资采购过程中就会缺乏科学的预测,也没有先进的库存数据分析系统,采购人员只能靠经验、订单和季节因素来确定目前企业所需的物资,缺乏具体的市场调研和需求调查,造成库存积压。以青岛三江源建筑添加剂有限公司为例:从2009年底财务报表的数据来看,该企业月产量600公吨,年营业额380多万的情况下,现有资金储备160多万元,其中由于长期积压物资占用的资金约为100万,严重影响企业资金流通。

    \subsubsection { 信息化程度不高,信息化进程缓慢。}

    据《中国中小企业信息化发展报告(2007)》显示,目前中小企业信息化存在的主要问题是,仍有相当数量的中小企业对信息化促进企业发展的作用、效果以及政府支持信息化建设的政策措施了解不够。虽然有高达调研数量80\%的中小企业具有接入互联网的能力,但用于业务应用的只占44.2\%,只有9\%的中小企业实施了电子商务,4.8\%的企业应用了管理软件。统计数据表明,目前,在我国1000万家中小企业中,实施信息化的比例还不到10\%,甚至有些企业仍然采用纯手工操作。另外还有一些原因也制约着中小企业信息化的进程。如:决策层缺乏库存控制意识,忽略库存成本;缺乏成熟的专业库存管理软件;研究开发一套信息化管理系统需要大量的资金。实行一套管理系统涉及到整个企业的各部门利益;现有的信息系统的优势可以被廉价的劳动力取代,从业人员素质低下,难以掌握先进管理技术和优化技术,需要占用大培的时间培训等。

    如瑞鑫物流作为胶南一个小有名气的物流公司,其所谓的库存就是几间仓库,货物的堆放混乱,导致配送人员在货物堆中寻找所需的物件。如中联水泥位于胶南的分公司,由于信息化程度不高,对于同一种材料的多次入库和多次领用,很难计算实时库存,经常由于原材料库存数日不清造成采购不及时或过量采购,从而引发缺货损失或库存积压问题。笔者认为:要想构筑先进的物流系统,提高物流管理水平,单靠物流设备和一个简单的软件来管理库存是远远不足的。

    \subsubsection { 信息不能共享,企业自身库存控制失衡。}

    由于信息不能共享,各部门之间的沟通又不及时,生产部门根本不能及时了解库存状况,因此对于生产的组织和计划就显得被动和盲目;采购部门不能及时了解原材料的消耗和库存,无法把握和控制原材料采购的进度和时机;财务部门就无法真正进行成本核算和成本控制。在这种情况下,企业领导无法了解库存积压情况、缺货情况、现有生产情况等方面的准确信息,企业就无法下达有效的生产计划。可见,企业自身在制定生产计划、需求计划、采购计划等物流决策过程中的行为违背对企业的库存进行控制的思想,这类企业往往平均库存水平高,紧急补货频繁,以目标定生产,以生产滚动需求,以滚动需求和供应市场价格定物料采购,成品则看销售的业绩。这样造成仓库被动发生出入库操作,本质上没有库存控制。

\subsubsection { 解决的思路}

    综上所述,目前我国中小企业在库存管理方面存在着缺乏正确的库存控制管理理念和技术手段,以及信息化程度不高等问题。结合我国中小企业的特点,针对当前库存管理存在的问题,提出几点思路:

    \begin{enumerate}
        \item  企业高层领导需要转变观念,适应时代发展的需求,不断学习先进的管理理念和提高决策能力。
        \item  从产品比例、决策参与度、贡献大小等多个角度探讨库存所有权、库存管理权、库存成本以及库存风险的分担模式。
        \item  探讨多样化的库存控制协作功能机构设立和人员组织方法。研究严谨规范的信息共享方案,探寻使用的信息技术,不仅可能提供技术支持,还能被大多数中小企业所接受和承受。
        \item  进一步开拓先进技术。如仿真技术在供应链库存控制领域中的应用。
    \end{enumerate}

    在我国中小企业信息化比例不到lO\%。库存管理在企业管理中占据重要地位的前提下。为了适应全球经济形势。促使我国经济又好又快发展,提高当期中小企业库存管理水平,加快供应链库存管理信息化程度势不可待。
