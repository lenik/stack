\subsection { 负库存的成因和解决方法}

  在应用ERP(Enterprise Resource Planning,企业资源计划/规划)电子信息系统进行管理时,“负库存”这一问题显得尤为突出。

    生产经营部门在向原物料部门领用所需材料时,出现了这样一种情况:所需的材料明明摆在库房里,数量也满足要求。而此时在库管员的帐上这种材料的数量却没有这么多,帐上的库存数量比领料单上的数量还少,库管员按领料单上的数量将材料发放给所需部门后,帐上就出现了负数。库管员为了把帐做平,甚至于还要向领用者打欠条。这种现象被业内人士称作“负库存”。

    库存帐基本上都是些简单的加减法,简单的加减法是绝大多数小学生都不会搞错的。然而,目前在不少的国有企业、民营企业、私营企业、中外合资企业、外商独资企业等等等等各色各样的企业中,这种“负库存”现象在周而复始、反反复复、不同程度地出现。

    让我们走进一家发生“负库存”的经营企业,看看他们是怎样接收供方单位送货的。

    当供方(卖方)向经营企业(买方)送上原物料时,经营企业收货人员收货(一般是库管员在供方收货单上签收认可,有的还要由经营企业质检人员检测后放行),经营企业并未在收货时向供方支付与所收货物等值的资金,同时也不将与所收货物等值的含税发 票在财务入帐。这批原物料以实物形式进入经营企业库房,在企业生产经营过程中使用,却未及时出现在经营企业的财务帐上(一般是延期出现)。

\subsubsection {    为什么会出现“负库存”?}

    一.供方企业的激烈竞争,导致他们不惜代价拼抢市场。有的供方企业向经营企业“铺底”销售,就是相当于向经营企业提供周转这部分物品的流动资金。

    二.经营企业借机占用供方资金,相当于使用无息贷款。所以库房越大越好,库存越多越好,库房相当于银行,负库存数量越多,相当于银行存款越多。银行存款越多,用起来越方便。

    这是双方利益驱动,就象“周瑜打黄盖,一个愿打,一个愿挨。”

    “负库存”会带来什么?

    一.“负库存”可能带来大量物资的堆积。日本人说过一句很经典的话:“库存是万恶之源”。大量的物资库存会导致如下结果:

    \begin{enumerate}
        \item  因市场发展、技术进步带来库存物资价值损失。库存零部件是按过去要求型号做的,装在按瞬息万变的市场要求开发出的新品上就不合适了。这些零部件除用作售后服务及可改作其他用途以外,大量的都只好报废。
        \item  运输、搬运过程中的破损,存放过程中的变质、锈蚀等,导致物资报废、质量下降、价值降低等损失。
        \item  失火、盗窃等情况的发生,导致价值损失。
    \end{enumerate}

    二.“负库存”的精华被日资企业、美资企业甚至港台企业拿去得心应手地在中国本土使用,对付国内企业。比如一家供货实行JIT(Just In Time 准时制)的外资企业,在进出口时,它会使用信用证。而在要我们国内企业给它供货时,无一例外地要每家供货企业铺底。

    三.“负库存”在企业中导致管理混乱。一家给重庆摩托车成车生产名企供货的摩托车配件厂商说:“成车厂库管员厉害,他要不高兴,就把你的件压住,总向制造部发放别的厂送的件,高兴了再给你的发几个出去,你还真拿他没办法。你看看去,那多少个库:一库,二库,三库,四库,五库,六库, 七库,八库,九库,十库,十一库,十二库,十三库,十四库,十五库,海了,里面压了多少厂家的东西。”站在供应商的角度,他拿“负库存”没办法,只好望成车厂的配套部多给自己一点份额,只好望成车厂的库管员多向制造部门发一点自己的货。因为经营企业多用自已的件,结款时就会多收入一些。

    四.个别不规范的经营企业,玩“负库存”就象玩“空手道”,找茬对供方企业罚款,搞得供方不盈反亏,甚至于血本无归。这是有违“建立互利的供方关系”这一原则的。

    五.个别有犯罪心理的经营企业老板,将供方的物资弄到手后,人间蒸发,同样搞得供方血本无归。好歹被警察抓了回来,物资早已以跳楼价脱手,余款也被挥霍一光。

    六.由于经营企业帐面上的数字与实际发生额不一致,导致国家和地方税收流失。

    自古以来,做买卖都是一手交钱一手交货。因为在交换之时,买方向卖方支付货币,获得了物品的所有权。所有权的内容有占有、使用、处分三项权能。“负库存”中,经营企业(买方)还并未支付货款就得到了供方(卖方)的物资,那么,物资的所有权在谁手里呢?什么时候才转移呢? 我们从三个时期进行分析:

    \begin{enumerate}
        \item  供方将原物料送达经营企业由库管员在送货单上签收直至使用前。这一时期内,所有权仍在供方,经营企业属代保管性质。可以认为其占有,但未使用、处分。
        \item  经营企业使用部门领用至经营企业产品形成。这一时期内,经营企业占有、使用、处分了供方提供的原物料。所以,我市一知名汽车生产企业认为,供方企业提供的原物料,在汽车从生产线上下线经检验合格之时,完成了所有权的转移。
        \item  经营企业向供方支付货款或供方收到货款时。这时是完全地实现了所有权的转移。
    \end{enumerate}

    正因为“负库存”,在供方送货之后至经营企业付款之前这一时期内,供方向经营企业所提供物资的所有权转移显得模糊不清,才使一些不轨行为大行其道,“负库存现象”的出现,还有其深刻的社会经济方面的原因。

    \begin{enumerate}
        \item  不少企业物资管理水平乃至经营管理水平低下,一是将自己企业当作银行,把吸纳供应商供货当成吸纳储户存款;二是向银行贷款后没法交差,故意存一大堆废物糊弄银行,银行又去糊弄上面。

        \item  与发达国家比较,我国的相关产业链、供应链并不成熟,正在逐步形成并完善,管理水平和技术水平参差不齐。供方企业所提供的零部件质量往往达不到经营企业所提出的技术要求,或者由于管理水平低下导致零部件质量不稳定,或是送货不及时。这叫经营企业怎么能及时付款来承担这种风险呢?在经营企业收货现场时常会看见供方企业将不合格原物料拉走又将合格原物料拉回来。

        \item  立法不健全,执法有偏差,行政不得力。一个省的高等法院,去执行一个已判决生效的收款案子,不按正规程序叫执行庭的人去,却叫其他庭的法官去。结果这笔款子被变着法儿收进了院长儿子的腰包里。那家花力气下功夫打官司要钱的企业,官司的确打赢了,但看不见钱在哪里,几百号职工连工资也发不出。立法行政司法这些上层建筑,都应该是为企业的永续经营这个经济基础服务的。如果裁判是黑哨,游戏规则又有漏洞,运动员自然就很受伤。社会的诚信也就无从谈起。所以一些企业想方设法拖欠款项,为了自己日子好过,似乎也不足为奇了。
    \end{enumerate}

    “负库存”这样一个看起来是绝大多数小学生都不会搞错的简单加減法,却折射出了一幅复杂的社会经济图象。这是客观存在的一个事实,简单地用好或不好来肯定或者否定都是不对的。 我们用心分析“负库存”这一现象的目的,是为了让我们的企业运行得更好,获得更大的企业效益和社会效益。
