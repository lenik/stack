\section{物料分类管理}
物料分类的模型图:
\screenshot{category.png}
某个物料分类可以0到多个子分类,子分类可以还有自己的子分类。

\subsection{物料分类设置}
在其中新增分类时,有一个“分类性质”,这个属性是在BOM模块中起作用的。只有在分类性质为成品或半成品下面的物料才能设置BOM
    \ul {
    \item  新建物料分类
        \ul {
        \item  点击工具单上的\button{新建}按钮,进入新建物料分类界面
        \screenshot{9.png}
        \item  选择上级分类
        \item  输入\textbox{分类名称}
        \item  选择分类性质
        \item  输入\textbox{分类描述}
        \item  点击\button{保存}按钮,保存新建的物料分类
        }
    \item  查看物料分类
    \screenshot{11.png}
        \ul {
        \item  选择需要查看的物料分类
        \item  点击工具栏上的\button{查看}按钮,查看分类的详细信息
        }
    \item  修改物料分类
        \ul {
        \item  选择需要修改的物料分类
        \item  点击工具栏上的\button{修改}按钮,进入编辑物料分类界面
        \item  修改相应的信息
        \item  点击\button{保存}按钮,保存更改后的物料分类
        }
    \item  删除物料分类
        \ul {
        \item  选择需要删除的物料分类
        \item  点击工具栏上的\button{删除}按钮,在弹出的对话框中点击确定,删除物料分类
        \screenshot{10.png}
        }
    }


\section{物料管理}
物料的模型图:
\screenshot{material.png}
    \ol {
    \item  物料属于某个物料分类
    \item  基准价格用于销售于参考报价用
    \item  单位换算表用于一个物料有多种计量单位的情况
    \item  自定义属性可以给物料添加更多的属性值
    \item  推荐库位可以指定物料的常用的存放在仓库中的库位
    \item  物料附件可以上传任何类型的文件,为物料提供技术图纸,详细文档说明等
    }
    \ul {
    \item  新建物料
        \ul {
        \item  点击工具条上的\button{新建}按钮,进入新建物料界面
        \screenshot{12.png}
        \item  选择\button{物料分类}
        \item  输入\textbox{物料名称}
        \item  选择\button{物料单位},输入\textbox{单位提示}
        \item  输入\textbox{条形码}
        \item  输入\textbox{规格型号}
        \item  输入\textbox{备注}
        \item  如果物料有其他属性需要输入,点击\button{自定义属性}标签
        \screenshot{13.png}
            \ul {
            \item  点击\button{添加}按钮,弹出自定义属性界面
            \item  输入\textbox{属性名称}
            \item 输入\textbox{属性值}
            \item  点击\button{确定}按钮,添加自定义属性
            }
        \item  点击\button{保存}按钮,保存新建的物料
        }
    \item  查看物料信息
    \screenshot{20.png}
        \ul {
        \item  选择需要查看的物料
        \item  点击工具栏上的\button{查看}按钮,或者点击工具里面的\button{查看图标}
        }
    \item  修改物料信息
        \ul {
        \item  选择需要修改的物料
        \item  点击工具栏上的\button{修改}按钮,进入修改物料信息界面
        \item  修改物料相关信息
        \item  点击\button{保存}按钮,保存更改后的物料信息
        }
    \item  设置物料价格
        \ul {
        \item  点击\button{价格设置}图标,弹出物料价格设置对话框,增加物料最新报价
        \screenshot{14.png}
        }
    \item  单位换算设置
        \ul {
        \item  点击\button{单位换算表设置}图标,弹出单位换算表设置对话框
        \screenshot{15.png}
        \item  点击\button{增加换算关系}按钮,弹出新增单位换算对话框
        \item  选择\button{目标单位}
        \item  输入\textbox{换算率}
        \item  点击\button{保存}完成新的换算关系添加
        }
    \item  推荐库位设置
        \ul {
        \item  点击\button{推荐库位设置}图标,弹出推荐库位设置对话框
        \screenshot{16.png}
        \item  点击\button{增加推荐库位}按钮,弹出库位选择对话框,选择库位
        \item  点击\button{保存},完成新的推荐库位添加
        }
    \item  设置仓库选项
        \ul {
        \item  点击\button{仓库选项设置}图标,弹出仓库设置对话框
        \screenshot{17.png}
        \item  选择\button{仓库},设置\button{安全库存}及\button{循环盘点周期}
        \item  点击\button{保存},完成仓库设置
        }
    \item  设置物料附件
        \ul {
        \item  点击\button{物料附件设置}图标,弹出附件设置对话框
        \screenshot{18.png}
        \item  点击\button{选择文件}按钮,选择需要上传的物料附件
        \item  点击\button{上传}按钮,上传附件
        }
    \item  删除物料
        \ul {
        \item  选择需要删除的物料
        \item  点击工具栏上的\button{删除}按钮,在弹出的对话框中点击\button{确定}按钮,删除相应的物料
        \screenshot{19.png}
        }
    \item  检索物料
        \ul {
        \item  在左侧物料分类栏中直接选择物料分类,会显示该分类下的物料
        \item  点击右侧搜索边条,根据实际需要检索
        }
    }


\section{设置}

\subsection{仓库设置}
    \ul {
    \item  新建仓库
        \ul {
        \item  点击工具栏上\button{新建}按钮,弹出新建仓库对话框
        \screenshot{1.png}
        \item  输入\textbox{名称}
        \item  输入\textbox{地址}
        \item  输入\textbox{电话}
        \item  选择仓库\button{负责人}
        \item  点击\button{保存}按钮,保存新建仓库
        }
    \item  查看仓库
        \ul {
        \item  选择需要查看的仓库
        \item  点击工具栏上的\button{查看按钮}或者列表中的\button{查看工具},查看选择的仓库信息
        }
    \item  修改仓库
        \ul {
        \item  选择需要修改的仓库
        \item  点击工具栏上的\button{修改}按钮,进入到修改仓库信息对话框
        \item  修改相应的信息
        \item  点击\button{保存}按钮,保存更改后的仓库信息
        }
    \item  删除仓库
        \ul {
        \item  选择需要删除的仓库
        \item  点击工具栏上的\button{删除}按钮,在弹出的对话框中点击\button{确定},删除相应的仓库信息
        \screenshot{2.png}
        }
    }

\subsection{库位设置}
    \ul {
    \item  新建库位
    \screenshot{7.png}
        \ul {
        \item  选择仓库
        \item  点击工具栏上的新建按钮,弹出新建库位对话框
            \ul {
            \item  在对话框中选择上级库位
            \item  输入\textbox{库位名称}
            \item  选择\button{库位容量单位}
            \item  输入\textbox{库位容量}
            \item  输入\textbox{容量单位提示}
            \item  输入\textbox{等级}
            }
        \item  点击保存按钮,保存新建的库位
        }
    \item  查看库位
    \screenshot{6.png}
        \ul {
        \item  选择需要查看的库位
        \item  点击工具栏上的\button{查看}按钮,查看库位的详细信息
        }
    \item  修改库位
        \ul {
        \item  选择需要修改的库位
        \item  点击工具栏上的\button{修改}按钮,进入库位修改对话框
        \item  修改相应内容
        \item  点击\button{保存}按钮,保存修改后的库位信息
        }
    \item  删除库位
        \ul {
        \item  选择需要删除的库位
        \item  点击工具栏上的\button{删除}按钮,在弹出的确认对话框中点击\button{确定},删除相应的库位
        \screenshot{8.png}
        }
    }


\section{日常业务}

\subsection{库存初始化}
    \ul {
    \item  新建库存初始化条目
        \ul {
        \item  选择要初始化库存的仓库
        \item  点击工具栏上的\button{新建}按钮,进入库存初始化界面
        \screenshot{5.png}
        \item  输入\textbox{摘要}
        \item  输入\textbox{描述}
        \item  选择\button{对应公司}及\button{对应部门}
        \item  点击\button{添加}按钮,弹出库存初始化条目对话框
        \screenshot{3.png}
            \ul {
            \item  选择\button{对应物料}
            \item  输入\textbox{批号}
            \item  输入\textbox{编组}
            \item  输入\textbox{备注}
            \item  输入\textbox{数量}
            \item  输入\textbox{单价}
            \item  选择\button{有效期限}
            \item  选择\button{库位}
            \item  选择\button{状态}
            \item  点击\button{确定}按钮,添加一条库存初始化条目
            }
        \item  点击\button{保存}按钮,保存新建的库存初始化单据
        }
    \item  修改库存初始化条目
        \ul {
        \item  选择需要修改的库存初始化条目
        \item  点击工具栏上的\button{修改}按钮,进入库存初始化单据修改界面
        \item  修改相应的内容
        \item  点击\button{保存}按钮,保存更改后的库存初始化单据
        }
    \item  删除库存初始化条目
        \ul {
        \item  选择需要删除的库存初始化单据
        \item  点击工具栏上的\button{删除}按钮,在弹出的确认对话框中点击\button{确定},删除对应的库存初始化单据
        \screenshot{4.png}
        }
    }

\subsection{物料出入库}
本系统对不同类型的入库,直接从菜单上进行区分。例如:采购入库,生产入库,委外入库,销售出库,生产出库,计划出库,委外出库等
点击相应的菜单,进入到不同类型的入库单管理界面
    \ul {
    \item  新建出入库单
    \screenshot{27.png}
    \ol {
        \item 选择需要出入库的仓库,选择仓库后,会显示最新的出入库单据。
        \screenshot{28.png}
        \item 点击工具栏上的\button{新建}按钮,进入新建出入库单界面
        \item 输入入库单\textbox{简要}
        \item 输入\textbox{描述信息}
        \item 点击按钮,弹出选择对应公司(供应商/客户)对话框
        \screenshot{29.png}
        \item 选择相应的公司,点击\button{确定}按钮完成选择。
        \item 点击按钮,弹出选择对应公司下对应部门的对话框。
        \screenshot{30.png}
        \item 如果已经在对应公司下设置了部门,选择部门,点击\button{确定}按钮完成选择。
        \item 击\button{添加}按钮,弹出添加出入库项目对话框。
        \screenshot{31.png}
        \item 点击按钮,弹出选择物料对话框
        \screenshot{32.png}
        \item 选择一个物料,点击\button{确定}按钮,完成物料选择
        \item 输入\textbox{批号}
        \item 输入\textbox{编组}
        \item 选择\button{有效期}
        \item 选择\button{库位}(如果仓库下已经建立了库位),点击按钮,弹出选择库位对话框
        \screenshot{33.png}
        \item 如果仓库下已经建立了库位,会列出所有库位供选择。点击\button{确定}按钮完成选择
        \item 输入\textbox{数量}
        \item 输入\textbox{单价}
        \item 选择\button{币种}
        \item 选择物料\button{状态}
        \item 填写\button{备注}信息
        \item 点击\button{确定}按钮,完成入库项目添加
        \item 点击\button{修改}按钮,弹出修改入库项目对话框
        \item 点击\button{移除}按钮,移除入库项目
        \item 点击\button{保存}按钮,完成入库单添加。
    }
    \item  修改出入库单
        \ul {
        \item  选择需要修改的出入库单,点击工具栏上的\button{修改}按钮,进入修改出入库单界面。
        过程类似于新建出入库单
        }
    \item  查看出入库单
    \item  删除出入库单
    \screenshot{34.png}
        \ol {
            \item 选择要删除的出入库单,点击工具栏上的\button{删除}按钮,弹出确认删除对话框
            \item 点击确定按钮,完成删除。
        }
    }

\subsection{库存调拨}
    \ul {
    \item  新建库存调拨单
    \screenshot{27.png}
    \ol {
        \item 选择需要调拨的仓库,选择仓库后,会显示最新的调拨单据。
        \screenshot{35.png}
        \item 点击工具栏上的\button{新建}按钮,进入新建调拨单界面.
        \item 输入\textbox{作业简要}
        \item 输入\textbox{作业描述}
        \item 选择\button{目标仓库}
        \item 点击按钮,选择调拨\button{操作人员}
        \screenshot{36.png}
        \item 选择调拨人员,点击\button{确定}按钮,完成选择
        \item 点击\button{添加}按钮,弹出添加调拨项目对话框
        \screenshot{37.png}
        \item 点击按钮,弹出选择物料对话框
        \screenshot{38.png}
        \item 选择需要调拨的物料,点击\button{确定}按钮,完成选择
        \item 输入\textbox{批号}
        \item 输入\textbox{编组}
        \item 选择\button{有效期}
        \item 点击按钮,弹出库位选择对话框
        \screenshot{39.png}
        \item 点击\button{确定}按钮,完成库位选择
        \item 输入\textbox{数量}
        \item 输入\textbox{单价}
        \item 选择\button{币种}
        \item 选择\button{状态}
        \item 输入\textbox{备注}
        \item 点击\button{确定},完成调拨项目添加
        \item 点击\button{修改},修改调拨项目
        \item 点击\button{移除},移除调拨项目
        \item 点击\button{保存}按钮,完成碉堡单输入
    }
    \item  修改库存调拨单
    \ol {
        \item 选择需要修改的库存调拨单
        \item 点击工具栏上的\button{修改}按钮,进入调拨单修改界面
        \item 修改过程类似新建库存调拨单
    }
    \item  查看库存调拨单
    \item  删除库存调拨单
    \ol {
        \item 选择需要删除的\textbf{库存调拨单}
        \screenshot{43.png}
        \item 点击工具栏上的\button{删除}按钮,弹出确认删除对话框
        \item 点击\button{确定}按钮,完成删除
    }
    }

\subsection{库存盘点}
    \ul {
    \item  新建库存盘点表
    \ol {
        \item 选择需要盘点的仓库
        \screenshot{40.png}
        \item 点击\button{新建}按钮,新建库存盘点表
        \screenshot{41.png}
        \item 点击\button{查询}按钮,弹出物料查询对话框
        \screenshot{42.png}
        \item 点击\button{添加}按钮,弹出物料选择对话框
        \screenshot{23.png}
        \item 选择物料,点击\button{确定}按钮,完成选择
        \item 点击\button{移除},移除不需要盘点的物料
        \item 如果要盘点所有物料,勾选\textbf{所有物料}选项
        \item 点击\{确定}按钮,把选择的物料添加到盘点表中
        \item 点击\button{查询}按钮,查询选择物料在仓库中的状态
        \item 盘点实际库存,和盘点表中的数据对比
        \item 选择物料,点击修改按钮,修改盘点项,录入盘点数据
        \item 选择物料,点击移除按钮,移除盘点项目
        \item 点击保存按钮,保存物料盘点表
    }
    \item  修改库存盘点表
    \ol {
        \item 选择相应的库存盘点表,点击工具栏上的\button{修改}按钮,进入修改盘点表界面
        \item 修改相应的内容,过程类似于新建库存盘点表
    }
    \item  查看库存盘点表
    \item  删除库存盘点表
    \screenshot{43.png}
    \ol {
        \item 点击工具栏上的删除按钮,弹出确认删除对话框
        \item 点击确定按钮,完成删除
    }
    }


\section{查询}

\subsection{库存查询(含明细)}
库存查询,可以查询仓库的全部物料,也可以选择目标物料进行独立的查询。
\screenshot{21.png}
\ol {
    \item 选择需要查询的仓库
    \item 选择时间
    \item 点击查询按钮,弹出查询对话框,如下图
    \screenshot{22.png}
    \item 点击\button{添加}按钮,弹出选择物料对话框
    \item 点击\button{确定}按钮,选择需要查询的物料
    \item 点击\button{移除}按钮,移除不需要查询的物料
    \item 勾选\button{所有物料},本次查询即为查询仓库下所有物料
    \item 点击\button{查询}按钮,在查询结果表单中列出物料列表
    \screenshot{23.png}
    \screenshot{24.png}
    \item 点击\button{确定}按钮,完成物料查询,列出查询的物料
    \screenshot{25.png}
    \item 在操作列中点击\button{明细},弹出该物料的出入库详细信息
    \screenshot{26.png}
    \item 在出入库明细列表中,点击\button{所在},显示出入库所在单据以及所在仓库等详细信息
    }
