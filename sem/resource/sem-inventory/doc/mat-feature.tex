\section {物料的定义和特性}

    组成产品结构的最小单元是物料。在英文里“物料”有多种叫法,当我们说“物料需求计划” 时,“物料” 在英文里对应的是 materia1。 当我们说 “物料编码” 时,英文里的物料是 item 或 part。在一些国产 ERP 软件里,也有用 “物项”、 “物品”、 “物件”的,很不统一。我们顺着“物料需求计划”的词义,采用“物料”的叫法。

    这里所说的物料,指的是:凡是要列入计划、控制库存、控制成本的物件的统称。包括所有的原材料、配套件、毛坯、半成品、联产品、 副产品、回收品、需要处理的废品、包装材料、标签、说明书、技术文件、合格证、产成品、工艺装备、甚至可以是不能存储的某些能源等。换句话说,物料是计划的对象,库存的对象和成本的对象。

    从管理学角度看,物料的管理特性(家乡材料有物理性能和化学性能一样)主要是3个方面。

    \begin{enumerate}
        \item 相关性

        从供需链的概念出发,任何一种物料都是由于某种需求而存在的,没有需求的物料,就没有产生或保存的必要。一种物料的消耗量受另一种物料的需求量制约,如购买原材料是为了加工零件,而生产零件又是为了装配产品; MRP 理念的基本出发点就是平衡供需关系。从大范围来讲,一个企业的原料是另一个企业的产品= 一一个企业的产品,又是另一个企业的原料。

        只有当市场有需求时,企业生产的产品才有价值。这种相关需求不但有品种规格、性能、质量和数量的要求,而且有时间和空间 (需用时间和地点) 的要求。

        \item 流动性

        既然任何物料都是由于某种需要而存在,它就必然处于经常流动的状态,而不应当在某个地点长期滞留。物料的相关性必然形成物料的流动性,不流动的物料只能是一种没有需求的积压浪费。通过物料的流动性来检查物料在相关性上存在的问题,是物料管理或物流管理的一项重要内容。

        \item 价值

        物料是有价值的。库存或存货是流动资产,要占用资金; 而资金又是有时间价值的,使用了资金就应体现资金成本,要产生利润。因此,不仅要把库存物料看成是一种资产,还要看到它也是一种 "负债”(尤其是超储物料); 它占用了企业本来可以用来在其他方面获取利润的资金 应当计算机会成本。产品研发人员需要知道每个零部件的价值,从设计源头把住成本关,这也是价值工程学要做的分析工作。

    \end{enumerate}

    只有理解物料的这些管理特性 才能应用好 ERP 系统,做好计划管理、物料管理。
