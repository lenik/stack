\subsection { 高库存、低周转率的本源}

    -- \textit{ Inventory is the result of lack of information coupled with incompetence(大意:库存是信息匮乏的结果,无能使之更甚)。}

    拿信息换库存,着眼点就在信息的重要性。举个例子:英特尔实行高标准、高要求,供应商如不按时交货,就有罚款;但是,他们又不愿共享预测信息。供应商没法,就只有提高安全库存。为支持英特尔,有个供应商竟然备了10个星期的安全库存,而对别的客户来说,备个三到四个星期就绰绰有余。有趣的是,英特尔屡屡告诉这个供应商说,英特尔的产能飙升,用料将大幅增加,你准备好了吗?供应商问,那你的预测用量究竟要增加百分之几十?英特尔就没了下文,不知是因为是机密呢还是真的不知道。这个供应商就只有蒙头猜测,准备产能和库存。运气好、准备地正好的机会就如中了彩票;运气不好的时候自然居多。结果无非两种:要么库存积压,一年半载用不完;要么短缺,"明明告诉你要多备料,你还准备不好,实在该死,不罚你罚谁"。

    斯坦福大学的李效良教授(Hau Lee)有一系列的文章,讲的是供应链的牛鞭效应,表现就是分布在供应链各节点的库存,根子就是信息不对称。可以说,库存管理、供应链管理的根本就是信息,不管是不愿共享信息还是愿意共享但太低效。相关文献、案例一个图书馆估计都放不下。

    对于无能的提法则相对就少多了,但并不意味着就不存在,或者不普遍。先说执行上的无能。供应商能力低下,采购方管理不力、选择不善,交货、质量都不能保证。那怎么办?库存是供应链的缓冲剂,多放库存就是了。举个例子。通用电气和罗尔斯·罗伊斯都生产飞机引擎。2000年前后,后者从接单到交货,平均需要260天时间,最长达1年,库存周转率只有3.4次,库存一度接近30亿英镑。通用电气的生产周期短的多,库存周转8次左右,光库存成本一项就比罗尔斯·罗伊斯一年少开支2.5亿英镑。罗尔斯·罗伊斯生产周期长,跟它的产品设计、生产管理和供应商交货上的低效不无关系。生产周期长,自然就增加库存数量;交货、质量不稳定,更加增加安全库存数量。执行上的无能就这样成了公司竞争力的软肋。痛定思痛,罗尔斯·罗伊斯推行"四十天引擎计划"(即从接单到交货只要40天),系统降低生产周期和改进按时交货率,这是后话。

    再说规划上的无能。与执行上的无能相比,规划上的无能危害更大,而且更隐蔽。"一将无能,累死千军"其实说得就是规划上的无能。产品设计上,工程师们"语不惊人死不休",整出一堆一堆的非标准设计,后面跟着一长串独特的零件号,还有供应商生产、库存的一系列问题,大家都是有目共睹,深受其苦,就不赘述。预测、生产规划上的无能则并不一定能被深刻理解。有趣的是,在有些公司,与供应链执行部门相比,规划部门往往被忽视。为什么呢?执行部门例如采购部门是公司和供应商的窗口,需要有高超的组织、协调和业务能力,所以也尽量配备一流的人才。但规划部门呢,都是跟自家人打交道,"难度"小,能过得去就行了,配备的大多是些老弱病残。结果就可想而知:规划是供应链的引擎,规划不足,供应链执行就是救火,越救越急,供应商则永无宁日,交货、质量就更没保证;相应地,规划就采取更多的缓冲做法,供应商就更难得到真实需求信号,信息不对称,互相博弈,博准的自然少于博不准的,进一步增加执行上的不可靠。整个供应链就陷入恶性循环。

    采购、供应商有搞砸的时候,而且多的是,但相对规划部门来说,都是小巫见大巫了。举个简单的例子。同一个供应商、同样类型的产品、生产周期相同、预测也大致相同,一个规划员设两周的安全库存,另一个设四周。这也罢了。大跌眼镜的是,同一个规划员,一个产品设四周的安全库存,另一个设两周,而产品的属性都一模一样。也难怪,这也是为什么那些大型设备、飞机制造商的库存周转率往往连一都不到。这些公司动辄有几十亿美金的库存,注销起来过期库存动辄都是以千万美金计。规划上的罪恶可想而知了。
