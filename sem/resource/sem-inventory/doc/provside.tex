\subsection { 供应商库存改进方法}

    供应商库存管理是以采购物资分析为基础,对于价格波动大、市场预测分析难度大、仓储占用空间大、必须形成库存的采购物资,通过与有实力的供应商合作、利用供应商库存进行的物资管理。目前,通过搭建信息系统等手段实现“零库存”是企业管理追求的一项目标。但是,对无法实现“零库存”的物资应尽量充分利用供应商的库存,并对其进行物资管理,将供应商库存管理纳入到企业自身的物流管理系统中来。

\subsubsection { 一、企业实施供应商库存管理的优势}

    对于企业来讲,将部分库存管理工作转移给供应商,首先可以借助供应商的市场分析和预测的能力和结果,减小信息不对称的劣势,从而减小采购支出、降低采购风险;其次,可以减少自有库存的占用面积和扩建成本;再次,可以降低仓库工作人员、仓储环境管理、仓储物资管理等综合管理成本。因此,企业应该将供应商库存管理纳入企业仓储管理中,提高对供应商库存的管理能力。

\subsubsection { 二、供应商库存管理的关键因素}

    对供应商库存进行有效管理取决于四个关键因素,首先是建立与供应商的合作模式,并界定合同要约。通过协商确定双赢的合作模式是企业与供应商合作的基础。供应商的库存管理职责、服务费用在双方的年度合作协议中确定和表明。

    其次是搭建信息平台,对供应商库存管理必须以实时共享的信息平台为保障,保证企业与供应商信息沟通及时、顺畅。

    再次,供应商必须具备市场预测分析能力或安全库存预测能力,这些能力是企业采用供应商库存管理的核心因素。供应商依据原材料市场变化及价格走势做出物资价格预测,并将预测结果每周通报给对口管理部门。供应商还需要按照市场预测对库存进行调整,将库存调整结果和物资库存的变化情况每周通报给物资采购管理部门。

    第四,明确界定供应商库存管理职责。在供应商库存管理过程中对供应商进行考核评估,对其进行监督,促进双方合作。

\subsubsection { 三、供应商库存管理的职责界定}

    供应商库存管理的内容主要包括五个方面:市场分析预测、数据更新、仓储管理、对异常情况管理和业绩评报。在这五个方面,企业和供应商应明确界定和履行好各自职责。

    企业在供应商库存管理方面的职责:在市场分析预测方面,企业应提供每个月的物资需求,并确保需求统计的准确性和时效性,根据供应商提供的价格趋势信息,与供应商协商,做好物资储备管理工作。在数据更新方面,对供应商库存信息进行综合统计,准确掌握各类物资的库存数量。在仓储管理方面,为供应商提供仓储技术参数并进行定期检查;明确对特殊保护物资的规定。

    对异常情况管理方面,对物资需求突发性增加或减少,以及物料因质量问题需要返工情况及时与供应商沟通,以便供应商迅速反应。在业绩评报方面,企业应对供应商库存管理表现进行定期评报,对表现优秀供应商给予表扬和奖励,对供应商存在的问题提出改进意见,对表现差的供应商给予撤销。

    供应商在库存管理方面的职责:供应商在市场分析与预测方面,为企业提供未来物资价格波动趋势,为物资储备调整提供参考依据;确保预测数据在供应商库存信息系统中的准确性和实效性。在数据更新方面,确保信息平台库存数据的及时更新,按照库存数据和预测信息调整库存量。

    在仓储方面,按照企业要求改善仓储环境,保证库存物资的质量。在异常情况管理方面,确保对突发情况的应急处理,尽量满足企业对于采购物资需求突发性增加或减少,并对企业要求返工的物资保质保量按时完成。业绩评报方面,供应商应根据企业反馈的评报结果进行及时改进,不断提升管理能力。

\subsubsection { 四、合同终止前供应商库存管理}

    如果企业需要与供应商终止库存管理的合同,应该做好供应商库存的消化工作。在合同终止前的最后2个月,企业停止向供应商提供下个月的需求预测,冻结需求预测的第一天通知供应商即时存量。同时,供应商负责统计剩余库存,并将信息提报给企业。

    企业可以在供应商库存管理的基本要求上,结合企业自身特点,不断完善供应商库存管理工作,发挥供应商管理在企业价值创造中的作用。
