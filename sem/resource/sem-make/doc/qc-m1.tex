\section {什么是质量}

    品管应该完成的所有工作,可以说就是质量管理的中心业务,而其业务又以防止不良的发生为重点,从这个意义上看,关于质量的机能,它有如图\ref{fig:qcMech}所示的业务。

    \begin{figure}[h]
        \centering
        \begin{tikzpicture}[
                fun/.style={
                    draw,
                    inner sep=1em,
                    fill=white,
                    drop shadow,
                    },
                nc/.style={
                    edge from parent/.style={draw=none},
                    level distance=2.2em,
                    font=\itshape,
                    },
                ]
            \node[fun]{品质技能}
            [edge from parent fork down, level distance=6em, sibling distance=8em]
                child{node[fun]{品质检查}
                    child[nc]{ node{(验收机能)} }
                    }
                child{node[fun]{品质管理}
                    child[nc]{ node{(预防机能)} }
                    }
                child{node[fun]{品质保证}
                    child[nc]{ node{(保证机能)} }
                    }
                ;
        \end{tikzpicture}
        \caption{质量管理的品质机能}    \label{fig:qcMech}
    \end{figure}

    质量不是光靠检查就能确保的,要确保质量必须能正确地执行合理的设计、正当的工程管理等品管(预防机能)业务。也就是说,质量保证工作就是调查品管业务是否实行得当,调查设计、制造、销售等各部门是否确保了目标的质量,并将此结果向经营者报告。

\section {质量保证活动范围}

  质量保证活动不只是各部门在各阶段的业务执行(部门内的活动)与管理动,它还包含部门与部门之间的活动及其管理(机能管理)。更重要的是,不只是明确活动的管理方法,并要求确实执行,同时还必须对质量的管理,即质计划、质量传达、质量确认等采取相同措施。以下以项目形式列举了质量保证活动的范围:

    \begin{enumerate}
        \item 质量的设计,新品种、新制品质量的设定,规格的设定、修正与废除。

        \item 材料的购人与保管: 材料管理、库存管理。

        \item 标准化。

        \item 工程的解析与管理。

        \item 检查及不合格产品的处理。

        \item 客诉处理、质量稽查。

        \item 例设备管理:设备的建设、预防保养、计测管理。

        \item 人事劳务管埋: 适当正确的职务分配、教育训练。

        \item  外包、转包管理。

        \item 技术开发: 新制品的开发、研究管理、技术管理。

        \item 诊断与稽查: 质量管理实施状况的诊断、品管关系业务稽查。
    \end{enumerate}

\section {质量保证体系设立}

    在设立质量保证体系时,应注意以下几点

    \begin{enumerate}
        \item 回馈的方法必须明确。

        \item 体系图的纵轴表示开发的阶段,横轴表示职别,此职别的负责人必须明确有关内容。

        \item 对体系运作的手段,用具(表单类)及运作规则必须予以确定。

        \item 决定是否可以向下一阶段运作的评价项目与评价方法必须明确。

        \item 必须由体系运作所带来的经验来修正体系。
    \end{enumerate}

\section {质量组织计划}

    在质量管理的推进计划中,最重要的是“组织计划”。

    \begin{enumerate}
        \item 组织计划应害虑的事顶:

            在组织计划中要考虑以下事项

            \begin{enumerate}
                \item 质量管理的组织,并不是指成立质量管理部或品管科,而必须是:
                    \begin{itemize}
                        \item 让所有相关人员都知道有关产品质量情报的信息系统。

                        \item 在质量管理活动中,动员企业内所有的部门,所有阶层人员的方法。
                    \end{itemize}

                \item 是否有因人而设立组织,因组织而设立工作的倾向?

                \item 是否充分进行授权? 其授权工作不能只是口头说说,必须以文件加以明确。

                \item 组织之间的结合是否完全,品管活动中动员企业所有部门,所有阶层的方法是否确立? 从品管业务角度来看 横向的联系是否充分?

                \item 变更组织时的步骤是否明确?

             \end{enumerate}
        \item 组织计划的原则

        如前所述,组织化并不仅仅是将业务细分,成立各种职能部门,为了达到质量目标,还必须使其分担质量管理的责任与权限。 因此,组织必须满足以下原则:

            \begin{enumerate}
                \item 组织的上级与下级之间必须有明确的权限关系。这里所说的权限是指要求其他人活动的正常权利。

                \item 组织中的每一个人应固定地向生产线的某一主管报告,并且要明白自己必须向谁或谁向自己报告。

                \item 经营者、管理者的责任与权限必须以文件形式明确限定。

                \item 权责要相符。

                \item 必须尽量缩减管理层次。

                \item 管理者必须致力于标准化,防止例外或异常情况的发生。

                \item 每一位管理者能够使其协力帮忙的职位数(也即管理幅度)有一定限制。一般情况下,高层的管理幅度为3\~6人,中层为5\~9人,低层为7\~15人。

                \item 组织应具有弹性,必须能顺应形势的变化而有所改变。

                \item 制度必须简明,即阶层的数目不要太多。
            \end{enumerate}

        在这种情形下,下层职位的人只要向自己生产线的主管报告,接受其指示就行了。职位间指挥命令的混乱,只要能明确职位所应管理的项目,即能化解。
    \end{enumerate}
