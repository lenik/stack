
    资金管理(Funds Management)的原则主要是:划清固定资金、流动资金、专项资金的使用界限,一般不能相互流用;实行计划管理,对各项资金的使用,既要适应国家计划任务的要求,又要按照企业的经营决策有效地利用资金;统一集中与分口、分级管理相结合,建立使用资金的责任制,促使企业内部各单位合理、节约地使用资金;专业管理与群众管理相结合,财务会计部门与使用资金的有关部门分工协作,共同管好用好资金。

\subsection {资金管理中的常见问题}

    \subsubsection {资金管理意识淡薄,管理体系不健全}

    一些企业领导资金管理意识淡薄和没有资金时间价值观念。企业有钱时,不知如何规划使用,没有钱时就发愁,没有一种长期预算资金管理意识。在资金的整个循环过程中缺乏科学性和统一协调性,财务管理工作中没有充分考虑时间价值问题。现金流量是企业筹资、用资的关键。资金管理主要包括:余额管理和资金流量管理。一些企业对资金管理认识不到位,片面性追求产量和产值。尤其是在当前金融危机背景下,企业应对产品开发和未来风险进行合理评估和日常监控,避免产品占用大量资金和投资于风险大的项目。

    \subsubsection {资金管理模式不适应企业实际}

    有些企业没有建立完善的资金管理模式,有的建立了,但是不适应企业的实际。一般来说,资金管理模式分为两种:资金集中管理模式和分散模式。而企业一般分为单一企业和集团式企业。企业应根据实际情况建立适合企业资金管理的资金管理模式。一般来说集中式的资金管理对资金的集中控制和统一调配有利,但不利于发挥成员企业或分公司的积极性。成员企业或分公司在资金上过分依赖企业集团,若配套措施不到位,可能影响资金的周转速度,影响其对市场的应变能力。分散式的资金管理,有利于调动成员企业或分公司的积极性,但又难以避免资金分散、资金使用率低、沉淀资金比例大、资金使用成本高等固有缺点。由于集中式和分散式固有的不足,企业应有着重,结合两者模式的优点,根据企业的实际动态的混合的选用。我国企业的资金管理模式不科学,如有的集团公司内部还没有成立资金中心,没有运行资金集中管理机制;子公司多数没有大局意识和全盘观念,反映在资金日常管理上是五花八门、各自为政、存贷双高,出现了一些真空地带。

    \subsubsection {资金管理手段落后,使用效益低下}

    随着各个企业的迅速发展,跨区域经营的企业越来越多,这就出现如何对企业资金进行控制管理的问题。有的企业由于资金管理手段落后,资金控制能力不足,使得资金使用效益低下,在企业经营过程中付出了沉重的代价。尤其是大型集团公司由于成员企业众多,地域分布广泛,资金管理失控,监控缺乏手段,资金使用率低。其突出表现是:由于企业集团缺少统一资金管理系统,各个子公司、各种业务对资金流动的影响没有形成相关联的完整信息,难于有效监督,风险较大;由于缺少资金统一管理和规划使用,有的企业资金出现缺口,只能向银行贷款,而有的企业资金富裕,却无法给缺少资金的企业使用,只能向银行贷款,增加企业的财务费用,这导致了整个企业集团的资金成本上升,资源浪费和资金使用效益低下。这些问题严重制约了企业的健康发展,在当前这种形势下,甚至有可能威胁到企业的生存。

    \subsubsection {资金风险管理不足,引发严重财务风险}

    当前形势下,企业系统和非系统风险较大。而有些企业对资金风险管理没有引起足够的重视,主要表现在财务制度不健全,账户管理混乱,资金调拨不按流程和权限办理,造成企业资金的损失;违规对外担保或连环担保,给企业带来不必要的经济责任和法律责任,从而导致企业深陷担保诉讼泥潭。

\subsection {金融危机的影响}

    \subsubsection {市场需求萎缩,企业库存增加}

    金融危机引发的全球经济增速放缓,投资者信心不足,一方面造成企业投资不足,生产资料需求降低,企业效益下降,另一方面是消费者储蓄动机加大,减少了个人消费。这造成了整个市场的市场需求大幅萎缩,企业销售不畅,企业库存积压,同时由于市场价格加速下行,导致企业经营风险陡增,资金周转速度缓慢,资金链不能有效运行,也从不同程度上影响了企业的经营业绩。

    \subsubsection {企业间接融资难}

    根据《证券市场周刊》测算,在本次金融危机中我国有些金融机构海外投资出现了亏损,其中中行亏损额最大,约为38.5亿元。建行、工行、交行、招行及中信银行依次亏损5.76亿元、1.20亿元、2.52亿元、1.03亿元、0.19亿元。这些巨额的损失使各大商业银行在今后的发展中提高了自身经营的谨慎性,一方面严格规范和执行借贷标准,加强内部控制制度建设,提高自身管理能力;另一方面,虽然2009年我国实施积极宽松的货币政策,增加信贷规模和货币投放量,同时根据近日中国人民银行公布数据显示,新增贷款逐步回升,但是受金融危机持续升温、企业市场化经营形势恶化的影响,银行惜贷现象依旧,短期内难以见效,企业融资前景不容乐观,金融危机环境下企业资金紧张的矛盾还将继续。

    \subsubsection {企业通过直接融资获得资金更加困难}

    受金融危机的影响,我国股票市场受到冲击,导致了上证指数从2007年10月份的6124点暴跌至1638点,虽然2009年上证指数在回升,但是一直在3000点左右徘徊。金融危机打击了投资者的信心并在金融市场产生了示范效应。这使我国企业发行股票或者债券进行融资更加困难,企业通过直接融资缓解企业营运资金的渠道受阻。

    \subsubsection {金融危机引发了企业资金链断裂危机}

    所谓资金链,是指维系企业正常生产经营运转所需要的基本循环资金链条。资金链犹如一个企业的血脉,其循环过程可以简短的表述为:现金——资产——现金。企业要生存发展,就必须保持这个循环良性运转。当企业在资产难以变现或根本没有资产可以变现的情况下,由于现金流通不畅或没有可支配现金不能偿付到期债务,由此引发资金链断裂的危机。由于当前企业出现了直接融资、间接融资难,产品需求下降以及企业内部经营管理方面的原因造成企业资金链断裂。

\subsection {参考解决措施}

    \subsubsection {增强资金管理意识}

    首先,企业应树立资金进行统一管理的观念,不管是单一企业还是集团企业,在当前形势下,要对资金进行统一管理,规划使用。其次,树立现金流量的观念,在财务管理的具体工作中,为管理人员提供现金流量的信息,除年终提供的现金流量表之外,在日常工作中可根据不同情况,编制现金流且计划,以及短期现金流量预测报告和长期现金流量报告。要加强现金流量的控制和分析,严格控制现金流入和流出,保证企业始终具备支付能力和偿债能力。最后,提高企业资金管理的风险意识,要充分估计各个项目风险,谨慎投入资金。

    \subsubsection {全面提升企业资金管理水平}

    在当前形势下,不管是单一企业还是集团企业,资金集中管理是发展的必然趋势。实施企业资金集中管理,对于企业的生存与发展具有重要作用。它有助于企业完善整体资金链,实现整体利益的最大化,有利于集中进行战略方向的调整,有效地降低企业控制成本,提高企业整体信用等级,降低财务成本,优化资金管理体系,提高资金的使用效率。目前,一旦企业进行资金集中管理,企业可以按照自身需求对各个子账户的资金进行相应地归集,加强内部资金的整合和统筹管理,实现内部资金的相互平衡,提高资金的使用效益。

    \subsubsection {积极利用计算机技术,保证资金管理}

    随着计算机信息技术的发展,企业在实施资金的集中管理和监控中可以大力应用计算机技术。在企业中大力利用新的资金管理手段和模式,而计算机网络技术和统一的财务管理软件是先进的管理思想、管理模式和管理方法的有效载体,也是实施资金集中管理和有效监督控制的必然选择。其次,借助ERP系统的优势,提升企业资金管理效率。ERP 系统简化了采购、销售与财务之间的流程过程,充分利用财务与其他业务之间数据信息的互通,提高管理效率,使资金管理贯穿于整个企业业务流程的每个环节,对企业各个环节进行实时监控,有效发挥财务监管机制。

    \subsubsection {强化财务监督}

    首先,加强内部管理,对合同事前审批、事中执行、事后评价,进行严格地监控与分析。增强风险防范意识,建立风险预警机制。对经营业务的采购、销售、库存等各风险节点资金使用情况进行认真梳理,实时监控可能发生的风险,有效应对,将风险控制在最低。其次,积极开展内部审计,前移监督关口。企业的内部审计是严格监督、考核企业财务资金管理的重要环节,是强化监督约束机制、使预算取得实效的保障。健全内部审计监督考核制度,保证企业财务信息的真实可靠。变过去的“事后监督”为事前、事中监督和适时监督,围绕企业的发展目标和年度预算,对公司资金流向、财务状况变动等情况实施全过程的跟踪和监控,定期检查,及时反馈,防微杜渐,确保预算的严肃性和企业发展目标的如期实现。通过以上两方面的努力保证企业资金安全和完整。

    \subsubsection {拓宽融资渠道}

    有条件的企业积极取得政府和金融机构的支持,获得资金支持。同时企业应根据自身的实际情况,积极拓展融资渠道。在内源融资方面可以实施员工持股计划、预收账款融资、质押应收账款融资、专利权质押融资以及产权交易等内源性融资方式。在外源融资方面,除了依赖传统外部资金支持,如银行存款、风险基金、发行债券或企业上市等,努力寻求企业间金融互助合作,有条件的企业可以考虑风险投资、融资租赁和股权融资等创新型的融资方式 。
