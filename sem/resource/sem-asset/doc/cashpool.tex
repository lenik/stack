    “现金为王”的理念在当前企业资金管理中倍受关注。随着企业集团全球化业务的不断拓展,企业对于内部资金的集中化管理和流动性管理的需求变得日益迫切。企业资金管理正逐渐从成本中心向利润中心转换。管理好企业的现金流已经不单单拘泥于管理好库存现金的层面,更多的指的是营运资金。如何强化企业的现金流管理,提升企业价值成为了越来越多资金管理人的工作重点。近年来,随着我国监管体系的不断放宽,一种基于委托贷款模式的创新型资金管理技术—“现金池”正逐渐兴起,成为了集团企业实现资产流动性、集中化管理的首选工具。

    什么是企业的现金池?

    现金池是建立在委托贷款模式下的集团企业内部的资金融通。在现金池结构中有一个母帐户和若干个子帐户。以公司总部的名义设立集团现金池母帐户,通过子帐户向母帐户以委托贷款的方式,每日定时将子帐户的资金余额上划到母帐户中。资金上划后,子公司账户上保持零余额或目标余额。这个限额的设定通常是由企业根据自身资金管理的需求和现金存量的额度与银行协商确定。在现金池结构中,子账户资金上划一般是在固定的时点,一天一次。但是企业也可以根据实际需求向银行要求一天多次的子账户余额上划。

    日间,若子公司对外付款时账户余额不足,银行可以根据以其上存母账户的资金头寸额度为限进行透支支付。一般而言,为企业提供现金池业务的银行都会为其子账户提供日间透支服务,以确保子账户日间对外的正常支付。当然,银行为成员单位提供日间透支服务,是基于企业集团自身良好的信誉和相应的资产担保,必要时企业还需事先和银行签订相关的法律文书。银行规定,集团企业为其下属子账户提供担保的必须要大于或者等于其所有子账户日间透支额度的总和。子账户的日间透支会由现金池的母账户在日终自动向子账户划拨相应的款项来结清欠款。企业集团还可以根据每个子账户的资金需求情况合理分配他们的日间透支额度。只要子帐户的日间透支额度不超过企业集团担保的额度,银行是不会就这一透支来征收利息的。但是在实际操作中,往往会有突发情况的发生,子账户对外支付金额可能要超过其日间可透支额度。这就需要企业集团向银行发出人工拨付的请求,由银行将款项拨付到需要付款的子账户。这样企业集团既能及时满足子账户应急的需要,也不会使子账户支付需求超出额度,而承担额外的利息费用。这种临时性的拨付一般是由企业人员通过网上操作完成的,计算机网络的发达,系统化,足以使得这类款项实时到帐。

    企业现金池业务的基本流程图

    以银行和财务公司为委托贷款中介的现金池

    企业现金池可以以银行,也可以以企业集团财务公司为委托贷款中介进行开设。现金池业务中涉及到的事项主要有子帐户余额的上划、子帐户的日间透支、主动拨付、母子帐户之间的委托贷款,以及子帐户向集团总部的上存、下借分别计息等。如下图所示:
      

    以银行为委托贷款中介

    在此模式下的现金池,集团各成员企业均需在现金池服务提供的银行开立结算账户、委托存款账户和委托贷款账户,其中结算账户为法人透支账户,具体透支额度可由银行对各家成员企业逐个单独授信,也可由集团核心企业统一申请和担保的集团授信并由核心企业将授信额度分配给各成员企业,授信提款可采用法人账户隔夜透支的方式,也可采用发放一笔流动资金贷款的方式。

    集团企业和法人关联公司的账户超过目标余额的部分,首先归还最早借入的委托贷款,其次划拨到委托存款账户。如果出现导致现金池中的汇总委托存款余额小于汇总委托贷款余额的借入委托贷款请求,委托存款余额小于委托贷款余额的部分形成资金需求企业的账户透支。

    在该种现金池中,集团财务公司可作为一个成员企业的角色出现,并代表集团各成员企业与银行沟通。当委托存款虚拟汇总账户余额大于委托贷款汇总余额的部分或存差,该差额部分可用于隔夜投资。当经常性出现委托存款虚拟汇总账户余额大于委托贷款余额,其差额中部分稳定的余额,可由财务公司作为牵头企业代表集团成员投资于定期存款、通知存款、协议存款,或投资于国债市场、货币市场、基金市场,或向其他企业发放委托贷款。

    以财务公司为委托贷款中介

    其模式与以银行为委托贷款中介的现金池一致。由于财务公司支付中介的局限,财务公司一般需要选择一家银行作为结算代理,其成员企业也需要在银行开户与集团之外的其他企业进行结算。

    银行为财务公司提供现金池代理结算服务,此时银行并不是以委托贷款中介出现,而是以支付结算中介和或者现金池软件供应商的身份出现。由于委托贷款由财务公司处理,集团内的法人公司将富余资金划到在财务公司开立的委托存款专户账户,需要资金的企业在委托存款大于委托贷款的金额内借入委托贷款,银行按照财务公司和成员企业的指令划拨款项。如企业借入的委托贷款不足支付请求的部分,可由财务公司提供透支服务,也可通过银行或财务公司提供流动资金贷款解决。

    现金池业务给企业资金管理带来的积极作用

    企业通过建立现金池主要有三方面的好处:

    1. 利息成本的降低

    现金池实现了企业内部资金的融通,将外部融资变成了内部融资。在现金池中,不同帐户上的正负余额可以进行有效的相互抵消,资金充裕的帐户资金就能自动转移到那些资金缺乏的帐户上去,这样一来,既简化了融资手续,又降低了整个企业集团对于贷款的需求。而且内部融资的成本要远远低于银行的同期的贷款利率,这样也大大降低了融资成本和财务费用。但是为了避嫌涉及转移定价,内部融资的利率也应当遵守公平交易原则,采用的利率不宜过高也不宜过低。

    我们来举个例子,假设A,B,C同属一个集团公司,A日均盈余RMB400,000,B日均盈余RMB200,000,而C日均透支RMB300,000。我们假设现金池中的存款利率为2%,贷款利率为5%。很明显,集团通过设立现金池较之后就能够获得较多的利息收益。(我们暂且不计算相关税费,仅从利息收支方面出发)

    2. 获取最大的资金回报

    现金池将企业闲散的资金进行了回笼,充分提高了资金的使用效率。企业可以将集中后的资金进行更有效的投资活动,如投资于定期存款、通知存款、协议存款,或投资于国债市场、货币市场、基金市场,或向其他企业发放委托贷款等,从而为企业增加额外的投资收益。即使企业不进行投资活动,大额的存款也可以使企业从银行这里获得较高的协定存款利率。

    3.改善整个集团的资金管理水平

    在每日日终,各子帐户将余额上存母帐户,提高了整个集团资金的透明度。通过现金池,集团总部能够及时了解到各个子帐户的资金流量情况,便于企业司库们从整个集团层面管理和控制好流动性资金的需求,加强了内部控制的效力。
