\section {资金的现金流量表}

    现金流量表作为企业财务报告的重要组成部分,是反映企业“血液循环”——现金及其等价物的增减及变化后的状况的动态报表,具有鲜明的综合性、可比性。但是原始的财务报表数据资料只有科学的运用专门方法,加以疏理、分析,才能达到揭示企业财务状况变动的原因、后果及趋势,为报表使用者预、决策提供正确的依据的目的。就现金流量表而言,可运用结构分析法、比率分析法和趋势分析法三种分析方法,现分别试述如下:

\subsection {制作企业现金流量表应包含的主要内容}

    \begin{enumerate.zh}
        \item  概况 分析现金流量采用的最基本的方法是编制百分数现金流量表。即分别计算出各项现金流入、流出及净值占现金总流入、流出及净值的比重。通过结构百分比分析,可反映出企业现金流入(流出)的渠道流出有几条,要增加现金流入主要靠什么;流出的渠道分别占有多大比例,通过分析了解企业财务状况的形成、变动及原因等。一般来讲,经营活动现金流占总现金流比重大的企业,经营状况较好,财务风险较低,现金流入结构也较为合理。特别当现金净流量为正时,判断企业现金流入是否强劲,要注意现金净流量是由经营活动产生的还是融资活动产生的,从而深入探究经营活动产生的现金流的源泉是否稳定、可靠。

        \item  经营活动产生的数据流分析 进行此项分析时,可对经营活动产生的数据流细分项进行结构分析,同利润表中的主营业务收入和其他经营利润相结合,若两者相差不大,说明企业会计帐面上的收入数值非常有效、及时地转成了实际现金流入,应收账款的管理也比较有效(可参考应收账款周转率分析),未来为保证其经营活动产生的现金源源流入提供了有力保障;反之,则应加强经营收入的实现与变现,确保资金的有效运转,否则,即使账面资产较大,也存在潜亏的可能。

        \item  投资活动产生的现金流分析 进行此项分析时,应充分考虑企业预算、投资计划,可对应资产负债表中的长期投资及历年投资收益等相关情况进行分析,判断是否存在潜在风险。例如,企业投资活动现金流出是用于购建资产且自有资金足以支付的,可分析应是按计划扩大生产规模或改进生产工艺,以降低经营成本,预测会为企业发展奠定良好的基础。

        \item  融资活动产生的现金流分析 此项分析可反映出企业融资来源与用途及结构比率,一般与现金流量表概况分析相结合,考虑现金流入、流出的总体趋势,对举借债务的结合比率分析法进一步分析偿债能力。
    \end{enumerate.zh}

\subsection {现金流量表的分析方法}

    现金流量表比率分析是以经营活动现金净流量与资产负债表等财务报表中的相关指标进行对比分析,全面揭示企业的经营水平,测定企业的偿债能力,反映企业的支付能力。大致可分为盈利质量分析、偿债能力分析和支付能力分析三部分。具体分析如下:

    \begin{enumerate.zh}
    \item  盈利质量分析。

    盈利质量分析是指企业根据经营活动现金净流量与净利润、资本支出等之间的关系,揭示企业保持现有经营水平,创造未来盈利能力的一种分析方法。包含以下指标:

        \begin{itemize}
            \item  盈利现金比率:=经营现金净流量/净利润。一般来讲,该比率越大,企业盈利质量越强。若该比率<1,说明企业本期净利润中存在未实现的现金收入,即使盈利,也可能发生现金短缺,严重时亦可导致亏损。分析时可参考应收账款,若本期增幅较大,需及时改进相关政策,确保应收款按时回收,消除潜在风险。

            \item  再投资比率:=经营现金净流量/资本性支出,反映企业当期经营净现金流量是否足以支付资本性支出(固定资产的投资)所需现金。一般来讲,该比率越高,说明企业扩大生产规模、创造未来现金流量或利润的能力就越强,若<1,说明企业资本性支出包含融资以弥补支出的不足。
        \end{itemize}

    \item  偿债能力分析。

    企业经常通过举债来弥补自有资金的不足,但最终用于偿债的最直接的资产是现金,因此,用现金流量和债务比较可以更好的反映企业的偿债能力。

        \begin{itemize}
            \item  现金流动负债比:=现金净流量/流动负债总额,若此指标偏低,反映企业依靠现金偿还债务的压力较大;偏高,则说明企业能轻松的依靠现金偿债。

            \item  现金债务总额比:=现金净流量/债务总额,该指标反映当年现金净流量负荷总债务的能力,可衡量当年现金净流量对全部债务偿还的满足程度。分析时应与债务平均偿还期相结合,若平均偿还期越短,则比率越高越好,反之则此比率要求越低。企业债权人可凭此指标衡量债务偿还的安全程度。

            \item  现金到期债务比:=现金净流量/本期到期债务。这里,本期到期债务是指即将到期的长期债务和应付票据。对这个指标进行考察,可根据其大小直接判断公司的即期偿债能力。
        \end{itemize}

    \item  支付能力分析。

    支付能力的分析,主要是通过企业当期取得的现金收入特别是其中的经营活动现金收入,同其各种开支来进行分析和比较的。

    将企业本期经营活动现金收入同本期偿还的债务、发生支出进行对比后,其余额即为可用于投资及分配的现金。在不考虑筹资活动的情况下,它们的关系是:可用于投资、分配股利(利润)的现金=本期经营活动的现金收入+投资活动取得的现金收入-偿还债务的现金支出-经营活动的各项开支。

    一般来讲,若企业本期可用于投资、分配股利(利润)的现金>0,说明企业当期经营活动现金收入加上投资活动现金收入足以支付本期债务及日常活动支出,且尚有结余用于再投资或利润分配;反之指标<0,说明尚需通过筹资来弥补支出的不足。

    \end{enumerate.zh}

\subsection {从现金流量看企业的发展趋势}

    现金流量表趋势分析主要是通过观察连续报告期(至少为2年,比较期越长,越能客观反映情况及趋势)的现金流量表,对报表中的全部或部分重要项目进行对比,比较分析各期指标的增减变化,并在此基础上判断其发展趋势,对未来做出预测的一种方法。趋势分析注重可比性,具体问题具体分析。例如,正常经营的同一企业在不同时期如果采用不同的财务政策,其现金流量表的变化并不能完全说明其财务状况的变化趋势;筹建期或投产期的企业,经营活动现金流量较少,筹资活动的现金流量较多,成熟期一般则相反。因此,正确计算运用趋势百分比,可使报表使用者了解有关项目变动的基本趋势及其变动原因,在此基础上预测企业未来的财务状况,为其决策提供可靠的依据。

\subsection {小结}

    综上论述,我们可以看出,现金流量表在分析企业财务状况时,确实是一个不可多得的工具。在实际操作中,注意现金流量表分析与资产负债表和利润表等财务报表分析相结合,可以更清晰、全面地了解企业的财务状况及发展趋势,了解其与同行的差距,及时发现问题,正确评价企业当前、未来的偿债能力、支付能力,以及企业当前和前期所取得的利润的质量,科学的预测企业未来财务状况,为报表使用者做出决策的提供正确的依据。
