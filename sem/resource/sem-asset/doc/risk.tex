\section {资金风险管理}

    资金风险的管理技巧,是投资成败的关键之一。很多非常聪明的投资者,都是在资金管理上犯了情绪化的错误。我们建议您:
    \begin{enumerate}
        \item  在资金管理上最重要是保障本金的安全,其次才是获利最大化的追求。
        \item  止损/止盈纪律的执行要坚决实行程式化。
        \item  失误后要先执行纪律,再进行场外反思。
    \end{enumerate}

\subsubsection {根据自身情况调节风险}

    如果您是刚开始交易的新手,希望您掌握好账户资金以及开仓所占保证金的比例。

    一般风险规则告诉我们:每笔交易的风险不能超过5\%-15\%。比如说,你用所有保证金的5\%开仓位。如果损失,你就用剩下的95\%的5\%来开仓位。

\subsubsection {风险回报率}

    目标盈利和所能承受损失比率一般设为2:1,但3:1更好。

\subsubsection {止损}

    不要忘记止损。通常,当市场跟您预期的背道而驰时,投资者总是不自觉地想调整止损位。但是止损一旦设定就请不要更改,否则止损就失去了保护资产的作用。

    我们鼓励每个投资者在场外反思如下问题:

    \begin{enumerate}
        \item  我能承受多大风险?
        \item  我最多要赚多少,最多能亏多少?
        \item  目前市场情况如何,很动荡还是很平静?
        \item  我做这个仓位的逻辑是什么?
        \item  需要多长时间可以下结论我做一个仓位的逻辑是否正确?
    \end{enumerate}
