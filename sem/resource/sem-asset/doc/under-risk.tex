    金融危机下的资金管理

    华尔街的大风暴吹到了远远不止美林,雷曼等几家投行,美国人一直引以为傲的金融体系特别是风险控制体系终于不堪美国人自己的贪婪而垮塌。多年的眼花缭乱的金融创新事实只是带来了一些虚幻的假象而已。在这么纷繁复杂的环境中,企业该如何生存,如何看待风险,而我们职业的风险管理者—资金管理人该做些什么呢?

    笔者联系了一些跨国公司和本土公司的职业经理人,试图作出一些归纳,但我想肯定还有不足之处,也希望大家能够补充:

    1, 从新审视交易对手风险

    这里的交易对手包括银行和公司的业务伙伴。

    金融风暴的影响力之大,都能让银行破产,企业当然要小心,别把自己的血汗放在不安全的篮子里,如果交易对手银行也有类似冰岛的landsbank这样的机构而预先不有所知的话,那将给企业带来的会是灭顶之灾。在中国境内,笔者得知境内很多外资银行因为要拯救母公司汇回利润,同时又因为前些年存贷比拉得太高,现阶段急需境内融资以解决流动性问题,而境内银行因考虑到风险因素谨慎拆借导致外资银行头寸短缺,如果企业的合作银行只是外资银行的话,可能即使有额度也放不出款,就像2007年底发生的银监会窗口指导产生的效果一样。于是以前完全依赖外资银行的机构现金流就产生了相当大的问题,而现在再去修补和中资银行的关系却总有临时抱佛脚的味道,而且这个“佛脚”还不是那么好抱的。

    交易对手风险其实也应该重视起来。有些核心的大企业基本现金流还是正常,银行额度也还够充足,但是他们的上下游的中小厂家却未见得。随着国内大量的成规模的中小型制造型企业的倒闭,核心大企业的零配件,原材料或者某些大超市某些产品的稳定供应商出现生产问题,均会发生一系列连锁性的问题并同时对核心企业本身产生巨大影响,所以一方面需要尽可能的去了解供应链上关键机构的生存状况,并适时给与帮助,同时未雨绸缪,总是为公司想好Plan B,也是资金管理人应该干的事情。

    2, 重新制定资金管理战略

    这个和上个交易对手风险有密切的相关。

    企业需要重新审定合作银行,并对和他们发生的业务金额额度作出相应的调整。同时原先资金管理重要的内容例如尽量压缩财务成本这条可能要重新审视。例如通用电气,在危机发生之初就有敏感的意识并提早发行大量公司债券,帮助公司提早实现财务的灵活度;通用汽车虽然债券已经早就进入了垃圾级别,但是在危机之初他们也已经把在各个银行的额度提取干净,宁可承受相对高的财务成本,也要保证公司现金流的充足和稳定。过冬的心态促使企业对于资金管理产生了与在经济繁荣时期完全不一样的战略。虽然中国并不是本次金融漩涡的中心,但是中国的资金管理人也一样需要根据世界经济形势的变化为自己公司的战略安全保驾护航。比如充分考虑再融资风险,减少以前经常性的短贷长用的方法,同时要广开财路,宁可资金管理人麻烦一点也要根据业务的需要多找一两家合作银行以做备用。有资深的资金管理人在去年宏观调控之前就用承诺费的方式固定了银行的借贷利率水平和信贷额度,虽然花费了一定的承诺费用,但在之后的调控紧急期大多数机构都有额度贷不到钱的状况下,他们公司就可以安安心心的按照承诺合同规定的额度和利率进行融资安排。

    3, 谨慎制定风险战略

    近期报出国航在石油期货上的亏损和香港中信泰富在衍生品交易上的失误造成的巨额亏损。其中均有时机把握和工具选择上的失误之处。失误点我们会另外行文仔细分析。但可以想见的是在选取金融工具的时候一定要再三考虑各种极端状况。今年早先欧洲银行主导中资银行大卖的CMS产品的大面积亏损,就在于欧元区长短期利率的倒挂现象。这样的极端状况几十年不见而一旦显现就会让实际金融工具的使用者遭受巨大损失。

    各种理财工具在背后都有各自的风险结构构成。而董事会必须清楚这背后的风险意味着什么而且承受的起。银行毕竟有自己赚钱的立场,而企业资金管理人的职责应该是分析清楚这些工具的结构同时告知董事会并提出自己的专业意见。必须高度警惕风险,尽可能的用常理去理解问题,远离那些错综复杂的交易程式,类似香港最近出事的的雷曼迷你债券,任凭销售人员吹得天花乱坠,最后还是投资者自己为损失埋单。从来都不睬纳复杂理财工具的Nike这次就可以笑得很大声,因为他们的钱从来都是非常老实的呆在银行里吃利息,最多做点结构性存款。虽然在经济繁荣时期看起来司库并没有为企业赚取足够的利润,但是在经济风暴的今天他们完全可以笑看那些卷在各种债券风波中的企业资金管理人。

    4, 密切关注全球金融动向并为决策者提供战略决策参考。

    大动荡的时期也是意味着大机会的时期,很多准备充分的机构已经出手或准备出手在这次动荡中抄底突击。怎么抄底?花多大代价?整合并购怎样的资源?现金流的支持状况?上市可行吗?发债的成本比银行融资怎样?什么样的利息?长的短的?司库们其实有大堆的事情可以做,千万不要被边缘化,也不要整日埋头在琐碎的银行公文和财务数据的收集整理报告中。看看我们的资金管理网,参与一下资金管理者沙龙的活动,和同业交流交流心得体会和对未来趋势的判断,不仅是对自身见闻的扩充,更是可以一直搭着这个行业最前沿的脉搏。

    几点建议而已,相信大多数的朋友已经在做这样的工作。我们一直希望能够成为资金管理人专业群体的全方面的伙伴,帮助大家的成长。

    结伴而行,即使在最寒冷的冬天也不会无助。
