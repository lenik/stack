\section {财务预算}

\subsection {新阶段,实行财务预算管理的重要性}

    就我国目前企业经营管理的实际成效调查分析来看,部分企业中的财务管理仍处于低层次管理阶段,甚至在一些企业中仍未使用实际性、可持续操作的预算管理办法。据调查,在大型企业经营管理中财务预算管理越来越发挥着重要的作用。新阶段,随着我国企业种类、企业数目的扩大,财务预算管理的普及势在必行。

  \subsubsection { (一)有利于提升企业的整体竞争力}

    财务预算管理是企业管理的核心,而预算就是以货币表示的预算结果,既是决策的具体化,又是控制生产经营的活动的依据。它把企业经营管理的计划与控制结合了起来:一方面为做出这份方案提供可靠而充实的资金的计划;另一方面对该方案进行监督与考核,权衡利弊。

   \subsubsection {(二)有利于强化企业资金管理,其本质就是获得资源利用的最大化}

    财务预算管理是实现对产品流和资本流的管理控制,其目的在于用最低的投入获得最高额的收益。通常,就是使得日常资金预算达到资金运用的最高效益。在此基础上,营业收入预算是整个财务预算管理的中枢环节,营业收入关系到整个预算方案的合理性和可持续性。

    \subsubsection {(三)有利于评价各部门各单位工作绩效}

    财务预算是企业在预测和决策的基础上,围绕着战略计划,对年度预算内各种经济资源和经济行为的合理预计、测算并进行的财务和监督的活动。其侧重点更在于对各部门年度工作成绩总结的一种考核,对其部门能力的一种鉴定。

    \subsubsection {(四)推行财务预算管理,促进企业管理创新是企业发展的必然选择}

    现代企业经营理念更注重的是“产权明晰、权责明确、管理科学”,企业财务管理内部功能、外延及其地位发生了深刻的变化,强化财务预算管理已成为现代管理企业在激烈的市场竞争中得以生存和发展的一大法宝,也是现代企业管理制度得以实施的重要保障。企业应当积极按照国家相关规定,组织好财务预算工作,培养相关人才,明确责权限,完善编制程序,强化监督,从内因解决问题。

\subsection {财务预算管理在企业管理中的现状分析}

   \subsubsection { (一)对财务预算管理观念淡薄,财务管理执行力弱}

    就目前来看,不少电缆加工企业管理者把企业生存发展的重中之重放在产量上面,追逐订单量,也就是交易量,过于看重营业收入而不考虑各个方面的隐性成本,比如说对环境卫生影响、质量不过关召回等所另外担负的成本。此外,从业职工培训不过关,缺乏相关理论知识的指导,使得财务预算部门的执行力大打折扣。

    \subsubsection {(二)收入预算、支出预算达不到统一}

    在实际生产过程中,很多企业很难以做到一方面挖掘足够的收入,清理欠款并合理占用资金;另一方面紧紧抓住资金管理的关键点,全面实施财政收支预算管理。预算已经确立就相当于具有了权威性,换句话说就是“法律效力”,正确的权威性才能引领企业走向成功,而当今不合理的预算依然存在,对企业的生存发展构成一定的威胁。

    \subsubsection {(三)财务预算工作已完成上级部门下达的任务为目标,预算执行往往超标}

    据了解,很多企业在财务预算方面对各部门采取比较传统的承包式的模式,最终预算执行往往超标,几经调整也无济于事,特别在年终考核,实际日数与预算期限大有出入,收入不能确保,成本费用得不到很好地控制,尤其是利润增长缓慢出现明显停滞现象。
