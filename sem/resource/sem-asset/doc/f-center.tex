\subsection {浅析“以财务为中心”}

    不少公司都谈到类似“企业管理以财务为中心,管理信息化应当先上财务系统”的论点。抱有这类论点的个人和公司很多,其中不乏由电算化会计转型到 ERP 系统的公司,甚至有些外国公司的发言也是如此讲。

    企业管理涉及的业务很多,其中有一些属于核心业务,如,制造业的产品研发、销售、生产、采购、财务等; 没有这些业务就谈不上企业运作。 就 ERP 内部集成而言就包括了所有的核心业务。财务与其他业务一样,都属于核心业务。任何类型的企业,核心业务的内容可能会有变化,但却都少不了财务; 也许,这正是人们容易把财务作为“中心”的一个原因。

    但是我们要注意到,每一项核心业务都是一种流程,而不同流程的发生是有先有先后的,就是项目管理中经常强调的“先导”与“后续”的关系。在制造业,财务成本又是销售、生产和采购3项核心业务运作的结果,相对于销售、生产和采购来讲是一种“后续”流程。通过财务分析再反过来,指导或修正经营业务。 这里的信息流程是闭环和“互动”的,可见,各个业务系统之间是一种“相互依存、相互作用”的集成关系,而不是以哪个为中心的关系。

    财务管理大量的工作是账务处理,最后生成国家财政部要求的3种报表——损益表、资产负债表和现金流量表。这些报表反映的是一个时期企业经营状况的结果,是在一定的会计期末向企业外部公开发布的信息。这里的关键词,一个是“结果”,一个是“期末”,既不能反映 “过程”,又不能做到 “实时”。

    在其他业务系统还没有实施之前,所有进入财务系统的原始数据,都是手工录入的,甚至是“人工合成” 的,不能体现信息集成,实时性和准确性都得不到保证。报表上所有的数据都要来自各个业务系统,没有供销产等各种业务信息,财务系统将成为无源之水,无本之木。

    业界有的人士说得对 “财务报表只是一个结果,只知道结果而不了解过程,对制造业改进管理来说没有多大用处的”。只看结果,不看过程,是无法控制和优化流程的。与实现信息化的初衷相悖,这本是很容易理解的浅近道理,但目前社会上的认识却非常混乱。

    因此,以“财务为中心”的提法会影响企业依据自身的特点实事求是地确定实施 ERP 系统的步骤。而且,如果作为软件设计的指导思想,会成为束缚产品的发展的思想桎梏。

    不是说企业在实施 ERP 系统的进程中,不可以先上财务系统,不能绝对化。但是,如果先上财务系统,就必须对上面提到的现实情况要有一个清楚的认识。
