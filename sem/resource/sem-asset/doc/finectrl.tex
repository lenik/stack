\subsection {精细化资金管理}

    \subsubsection {为什么要对资金进行精细化管理}

    很简单,因为资金的节省就是净利润。因为省钱比挣钱要容易得多。对于企业来说你实现100元的销售收入可能只获得10元的利润,但如果你省10元,10元同样也是利润,但付出的努力是完全不一样的。

    另外对资金进行精细化管理的另一个原因是理解资金和业绩的关系。比如销售人员的资金和销售业绩的关系,研发资金和技术成果的关系等等。通过对这种关系的分析从而更好的配置有限的资金预算。

    精细化资金可以进行业绩评估。尤其同等资金的产出指标。

    \subsubsection {目前资金管理的模式}

    目前可以说所有的公司都是进行大呼隆式的的粗放式管理,包括年度的资金预算和实际资金的核算。顶多核算到部门就算了,资金成了流水账。很多的信息都被淹没了。

    我们可以参考SAP的资金核算大体如下:

    年度预算:可以针对部门按照成本中心和成本要素进行预算,另外也可以使用内部订单进行一些项目预算。

    日常核算:所有的资金都是记到成本中心或内部订单,简单的流水记账,如果你想知道某种资金或个人的资金就比较麻烦。可能需要导出数据做很多处理。

    可以说从预算和实际核算都是粗放型的。为什么,没有按照部门,人员,业务活动和业绩进行。最终也是糊涂账。

    \subsubsection {精细化的资金管理模式}

    对于一些资金处理较多的组织和部门,比如销售公司或者每月需要处理很多资金的财务中心,同时需要对不同的资金进行分析和评估,楼主推荐如下模式:

    \begin{enumerate}
        \item  报销人员网上申报个人资金(包括年初的预算和实际的报销),主管人员审批预算和实际报销并评估业务,月度资金报告和业绩报告。资金系统用工作流驱动整个流程,可以结合移动商务时时批准,减少高层人员出差的审批问题。同时实际的资金单据全部使用条形码管理,十分便于财务人员集中处理。

        \item  数据接口方式。资金系统实现与各种现有ERP系统的无缝转接。采用批处理方式,资金系统可以实现输出用户需要的格式进行数据导入,方便的实现数据录入,没有任何冗余动作和风险。

        \item  简单明了的驾驶舱分析报告。高层管理人员可以方便的调阅部门和人员的资金执行报告,随时进行管理和调控。
    \end{enumerate}

    总之,精细化的资金管理就是要从目前粗放型的管理实现由部门和项目的管理更象预算,个人和资金类型的多纬度的精细管理演进。
