\subsection {资金管理的误区}

    “现金为王”这样的经营理念慢慢的被更多的企业家所接受,大家对如何做好资金管理也都有自已的看法.在财务管理的理论中,也有很多计算模型和观念.其中有一点说企业最佳的现金持有状态是_通过计算企业最小的现金持有量,除此之外,账上不保留更多的钱,但是需要的时候就可能拿到钱.

    理论有的时候只是计算一个理想化状态,很多时候它只存在于书本和考试中.可是有很多人迷信书本,在现实的企业管理追求出现这种理想状态,认为只有这样才是企业资金管理的最佳状态,这是非常可怕的.

    我们都知道,企业自身永远是处于一个动态发展的过程中,同时,企业也处于社会动态变化的过程中,在二个动态过程里面,找到一个静态或不断变化的静态理想状态,几乎是不可能的事情.所以在实践中,我们可以看到,那些活得最久的企业,它们的资金管理对比理论来说,都不是最佳的状态.华人首富李嘉诚企业永远保留大量的现金,远大在很多年前就宣称企业有5亿现金是不动的.

    对远大这种做法的质疑有很多,认为企业这种做法失去了很多发展的机会.这种说法当然有它的道理,包括现金是收益性最差的资产,这种说法都是对的.

    但是,在说这些话的时候,很多人忘记了一点:任何事情都有它的二面性.保有大量现金,失去发展机会和收益差的同时,这也是企业长久发展最安全的方式,企业真的很难做到二者的兼顾.

    在这个时候,往往是考验企业家智慧的时候,你把企业做成了短跑英雄,还是做一个长跑的冠军.我们眼前闪过了太多的短跑英雄,顺驰、飞龙、三株等等。

    在一些企业管理的核心理念上具有什么样的态度,决定了企业发展的心态.做为企业管理核心的资金管理上,更是如此.企业其实就是一项马拉松,短跑心态可能有开始的风光,但那一定是流星一样的短暂,笑到最后才是最好的!
