\section {资金的现金流管理}

    企业是以盈利为目的的,是人尽皆知的。当前不乏有一些企业刻意的追求高收益、高利润。因此往往会有这样一种错误的思想,认为企业利润显示的数值高就是经营有成效的表现。从而很大一定程度上忽略了利润中所应该体现出来的流动性。作为企业的资金管理者应当要能够充分、正确地界定现金与利润之间的差异。利润并不代表企业自身有充裕的流动资金。流动资金也就是我们俗称的现金流,对于企业的健康发展有着重要的作用。

    现金流(Cash Flow)是指一段时间内企业现金流入和流出的数量。企业在销售商品、提供劳务、或是出售固定资产、向银行借款的时候都会取得现金,形成现金的流入。而企业为了生存、发展、扩大需要购买原材料、支付工资、构建固定资产、对外投资、偿还债务等,这些活动都会导致企业现金流的流出。如果企业手头上没有足够的现金流来面对这些业务的支出,其结果是可想而知的。从企业整体发展来看,现金流比利润更为重要,它贯穿于企业的每个环节。在现实生活中我们可以看到,有些企业虽然帐面盈利颇丰,却因为现金流量不充沛而倒闭;有的企业虽然长期处于亏损当中,但其却可以依赖着自身拥有的现金流得以长期生存。企业的持续性发展经营,靠的不是高利润而是良好、充足的现金流。

    传统意义上的现金管理主要是涉及企业资金的流入流出。然而广义上的现金管理,其所涉及的范围就要广的多,通常包括企业帐户及交易管理、流动性管理、投资管理、融资管理额风险管理等。企业现金管理主要可以从规划现金流、控制现金流出发。

    规划现金流主要是通过运用现金预算的手段,并结合企业以往的经验,来确定一个合理的现金预算额度和最佳现金持有量。如果企业能够精确的预测现金流,就可以保证充足的流动性。企业的现金流预测可以根据时间的长短分为短期、中期和长期预测。通常期限越长,预测的准确性就越差。到底选择何种现金流的预测方式就要纵观企业的整体的发展战略和实际要求。同时企业的现金流预测还可以现金的流入和流出两方面的出发,来推断一个合理的现金存量。

    控制现金流量是对企业现金流的内部控制。控制企业的现金流是在正确规划的基础上展开的,主要包括企业现金流的集中控制、收付款的控制等。现金的集中管理将更有利于企业资金管理者了解企业资金的整体情况,在更广的范围内迅速而有效地控制好这部分现金流,从而使这些现金的保存和运用达到最佳状态。
