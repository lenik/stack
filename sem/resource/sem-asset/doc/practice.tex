\practices

\subsection{会计科目管理}
管理现有的会计科目
\opset{会计科目管理}{
    \item \ops{新增会计科目}{
        \item  点击\button{新建}按钮,进入编辑会计科目界面
            \screenshot{1.png}
        \item  输入\button{编号}及\button{科目名称}
        \item  选择\button{上级科目}(图)
        \item  点击\button{保存}按钮,保存新的会计科目
    }
    \item \ops{查看会计科目}{
        \item  选择需要查看的会计科目
        \item  点击工具条上的\button{查看}按钮
        \item  查看会计科目的详细内容
    }
    \item \ops{修改会计科目}{
        \item  选择需要修改的会计科目
        \item  点击工具条上的\button{修改}按钮,进入编辑会计科目界面
        \item  修改需要修改的会计科目明细
        \item  点击\button{保存}按钮,保存更改后的会计科目
    }
    \item \ops{删除会计科目}{
        \item  选择需要删除的会计科目
        \item  点击工具条上的\button{删除}按钮
        \item  在弹出的对话框中点击确定按钮,删除对应会计科目
            \screenshot{2.png}
    }
}

\subsection{资产初始化}
初始化资产管理,设置企业出世资产
\opset{资产初始化}{
    \item \ops{新建初始资产单}{
        \item  点击\button{新建}按钮,进入新建初始资产单界面
            \screenshot{9.png}
        \item  点击\button{添加}按钮,弹出添加初始资产单项目对话框
        \item  选择\button{会计项目}
        \item  选择\button{二级会计项目}
        \item  输入\button{金额}
        \item  选择\button{币种}
        \item  点击\button{确定}按钮,添加初始化资产单项目
        \item  点击\button{保存}按钮,保存初始化资产单
    }
    \item  \ops{修改初始资产单}{
        \item  选择需要修改的初始化资产单
        \item  点击工具栏上的\button{修改}按钮
        \item  修改初始化资产单的相应内容
        \item  点击\button{保存}按钮,保存更改后的初始化资产单
    }
    \item \ops{删除初始资产单}{
        \item  选择需要删除的初始化资产单
        \item  点击工具栏上的\button{删除}按钮,在弹出的对话框中点击确定,删除对应的初始化资产单
            \screenshot{10.png}
    }
}

\subsection{资金流业务管理}
\opset{资金业务管理}{
    \item \ops{新建资金流业务单}{
        \item  点击工具栏上的\button{新建}按钮,进入新建资金流业务单界面
            \screenshot{6.png}
        \item  输入\button{金额},\button{摘要},\button{业务详细}
        \item  点击\button{保存}按钮,保存新的资金流业务单
    }
    \item \ops{修改资金流业务单}{
        \item  选择需要修改的资金流业务单
        \item  点击工具栏上的\button{修改}按钮,进入到修改资金流业务单界面
        \item  修改相关内容
        \item  点击\button{保存}按钮,保存更改后的的资金流业务单
    }
    \item \ops{删除资金流业务单}{
        \item  选择需要删除的资金流业务但
        \item  点击工具栏上的\button{删除}按钮
        \item  在弹出的对话框中点击\button{确定}按钮,删除对应资金流业务单
            \screenshot{7.png}
    }
}

\subsection{凭证管理}
凭证管理用来管理资金流业务单,实现收支平衡
\opset{凭证管理}{
    \item \ops{新建凭证}{
        \item  点击\button{新建}按钮进入新建凭证界面
            \screenshot{3.png}
        \item  选择对应的资金流业务单
        \item  点击\button{添加}按钮添加凭证项目,确保借贷平衡
            \screenshot{4.png}
        \item  点击\button{保存}按钮,保存新凭证
    }
    \item \ops{修改凭证}{
        \item  选择需要修改的凭证
        \item  单击\button{修改}按钮
        \item  修改需要修改的项目或对应的资金流业务单
        \item  点击\button{保存}按钮,保存修改后的凭证
    }
    \item \ops{删除凭证}{
        \item  选择需要删除的凭证
        \item  点击\button{删除}按钮
        \item  在弹出的对话框中点击\button{确定}按钮,删除凭证
            \screenshot{5.png}
    }
}

\subsection{资产查询}
查询当前资产
\screenshot{8.png}
\opset{资产查询}{
        \item \ops{查询步骤}{
            \item  勾选全部选择框,点击\button{查询}按钮,查询全部资产
            \item  没勾选全部选择框的情况下
            \ul {
            \item  点击按钮选择会计科目
            \item  点击\button{查询}按钮,查询相应会计科目下的资产情况
            }
    }
}
