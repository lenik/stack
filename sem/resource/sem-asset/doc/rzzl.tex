融资租赁

    随着国家对融资租赁行业的支持力度加大,融资租赁企业如雨后春笋般涌现,发展迅速的同时,面对的国内外竞争对手也越来越多,业务操作和管理的复杂性加大,强化行业的管理能力势在必行。近年来,科学管理受到普遍重视,特别是我国加入WTO后,受到全球经济一体化的挑战,越来越多的企业认识到信息化建设的迫切性。信息化对提升企业管理水平的重要性已经经国际众多大型企业验证,融资租赁在我国作为一个新兴行业,要想进一步发展和提高管理能力,就必须加强行业的信息化建设。

     加快信息流转,突破资源壁垒

    利用信息系统将客户信息、业务信息、设备信息、合同信息、资金信息等项目信息统一管理,形成一个完整的企业信息资源库。当项目开始启动,整个信息资源库就会被调用起来,消除了以前人工管理引发的信息不通畅,资源无法充分利用的被动局面,信息流动始终贯穿企业经营活动的整个过程。

    顺畅业务流程,提高工作效率

    融资租赁业务涉及多个业务对象,如租赁公司、设备供应商、客户、代理商等,每个环节的业务流程都有其运转的规律,相互之间缺乏对接。利用IT手段将专业的融资租赁系统与企业内部系统,如1104系统、CRM系统、财务系统、网银系统等整合起来,可简化和优化企业的业务流程,某个系统中的流程结束后,将自动触发另一个流程,形成闭环顺畅的业务处理程序,提高工作效率,增强企业的业务处理能力。

    规范业务处理,控制项目风险

    融资租赁的业务处理,对内要符合企业的管理理念和业务制度,控制项目风险;对外要遵循国家监管机构(如银监会等)制定的各类法律规范,降低行业风险。是以,企业需要一套深刻融合内外部的各类规章制度的业务管理系统,通过表单、流程、模块等将其规范到日常的工作中,通过与监管部门的系统实现对接,来规范业务处理。系统需要具备随需应变的功能,如自定义流程、灵活的权限管理等来应对与时俱进的法律法规及企业的业务规范,使项目始终业务合规、风险可控。

     数据报表分析,提供决策支持

    由于融资租赁行业的特殊性,在项目运作过程中,需要丰富多样的业务报表来对管理层的决策提供支持,如行业逾期率图表、时序曲线图、逾期租金清单、授信额度使用率报表、租息差额汇总等等,管理层需要通过这些报表上的信息了解项目进展,并制定下一步的决策。缺乏信息化管理的企业,无法实时获取数据报表,且数据报表的结论具有人为的主观性。一套了解融资租赁行业业务规则和风险机制的业务管理系统能为企业管理决策提供实时、客观、科学的数据报表,使企业的决策高效高质。

    多语言多架构,助力国际化发展

    随着融资租赁行业进入快速发展期,外资进驻中国市场,内资投资国外市场,融资租赁行业的国际化趋势越来越明显,企业的员工组成和组织架构的复杂程度也随之增加。企业的信息化系统需要能支持不同的语言环境,如简体中文、繁体中文、英文、日文等,以此满足不同地区员工的工作习惯。系统中的权限管理要能与企业的组织架构相匹配,以此来实现横向纵向的矩阵式管理,既满足管理需求,也满足业务需求,助力企业国际化发展。

    通过建设信息化系统促进融资租赁企业的健康生存和发展,提高企业的管理能力和业务水平,是企业发展的必由之路。信息化能帮助融资租赁企业实现资源整合,流程严谨,业务合规,项目监控,管理灵活,为融资租赁企业提供一个风险防控平台,协助融资租赁企业在业务效益和管理效益上收获双赢!
