\subsection { 营运资金优化}

    四种加强企业营运资金管理方式:

    \subsubsection {一、通过对购销环节中的风险进行事先预测、分析和控制,尽量规避风险}

    许多企业为了实现利润、销售更多产品,经常采用赊销形式。片面追求销售业绩,可能会忽视对应收账款的管理或管理效率低下。例如对赊销对象的现金流动情况及信用状况缺乏,以及未能及时催收货款,容易出现货款被客户拖欠,造成账面利润很高,但实际资金却很少。对此,财务部门应加强对赊销和预购业务的控制,制定相应应收账、预付货款控制制度,加强对应收账款的管理,及时收回应收账款,减少风险,从而提高企业资金管理使用率。

    \subsubsection {二、控制成本费用,提高企业利润,增加企业价值}

    会计利润是当期收入和费用成本配比的结果。在任何收入水平下,企业都要做好对内部成本、费用的控制,做好预算,加强管理力度,减少不必要的支出,才能够提高利润,增加企业价值。

    \subsubsection {三、加强企业财务预算,提高企业营运效率}

    财务管理应站在企业全局的角度,构建科学的预测体系,进行科学预算。预算包括销售预算、采购预算、投资预算、人工预算、费用预算等等,这些预算使企业能预测风险,及时得到资金管理的各种信息,及时采取一定措施,防范风险,提高效益。同时,制定的各种预算,不仅可以协调企业各部门的工作,避免部门间冲突,提高内部协作的效率。而且,销售部门在销售、费用等预算指导下,还可事先对市场有一定了解,把握市场变化,减少存货的市场风险。

    \subsubsection {四、完善财务制度建设,进一步加强营运资金管理}

    \begin{enumerate.zh}
        \item  明确内部管理责任制。

        很多企业认为催收货款是财务部门的事,与销售部门无关,其实这是一种错误的观点。事实上,销售人员对催收应收账款应负主要责任。如果销售人员在提供赊销商品时,还要负收回应收账款的责任,那么,他会谨慎对待每一项应收账款。销售人员应在征得财务部门的认可后才能发货,并且要按购销合同及时收回货款。

        \item  建立客户信用档案。

        企业应在财务部门中设置风险控制员,应通过风险控制员对供应商、客户的信用情况进行深入调查和建档,并进行信用等级设置,对处于不同等级的客户实行不同的信用政策,减少购货和赊销风险。风险管理员对客户可从以下方面进行信用等级评定:考察企业的注册资本;偿还账款的信用情况;有没拖欠税款而被罚款的记录;有没有拖欠供货企业货款的情况;其他企业的综合评价。风险管理员根据考察结果向总经理汇报情况,再由风险管理员、财务部门经理、销售部门经理、总经理讨论后确定给予各供应商及客户的货款信用数量。如果提供超过核定的信用数量时,销售人员必须取得财务经理、风险管理员及总经理的特别批准。如果无法取得批准,销售人员只能降低信用规模或者放弃这桩业务,这样就能控制销售中出现过度多的应收账款现象,减少风险。

        \item  严格控制信用期。

        应规定应收账款的收款时间,并将这些信用条款写进合同,以合同形式约束对方。如果出现未能在规定时间内收回应收账款,企业可依据合同,对拖欠货款企业采取一定措施,以及时收回货款。

        \item  通过信用折扣鼓励欠款企业在规定时间内偿还账款。

        很多企业之所以不能及时归还欠款是因为他们及时归还得不到什么好处,拖欠也不会有什么影响。这种状况会导致企业应收账款回收的效率比较低下。为了改善这种局面,企业可以采取相应的鼓励措施,对积极回款的企业给予一定的信用折扣。

        \item  实施审批制度。

        对不同信用规模、信用对象实施不同的审批级别。一般可设置三级审批制度。由销售经理、财务经理和风险管理员、总经理三级审核。销售部门如采用赊销方式时,应先由财务部门根据赊销带来的经济利益与产生的成本风险进行衡量,可行时再交总经理审核。这样可以提高决策的效率,降低企业经营的风险。

        \item  加强补救措施。

        一旦发生货款拖欠现象,财务部门应要求销售人员加紧催收货款,同时风险管理员要降低该企业的信用等级;拖欠严重的,销售部门应责令销售人员与该企业取消购销业务。

        \item  建立企业内部控制制度。

        主要包括存货、应收账款、现金、固定资产、管理费用等一系列的控制制度。对违反控制制度的,要给予相关责任人一定的惩罚。

        \item  严格控制开支,对各种开支采用计划成本预算。

        对各种容易产生浪费的开支要采取严格的控制措施。例如,很多企业业务招待费在管理费用中占很大比例,在计征所得税时无法全额在税前扣除。对此,企业应该要求销售人员控制招待费支出,由财务部门按其月销售收入核定适当的招待费标准。
    \end{enumerate.zh}

    总之,营运资金管理在企业销售及采购业务中处于重要地位,对企业利润目标的实现会产生重大影响。营运资金管理应是对销售工作的控制而不是限制,促进销售部门减少销售风险,提高利润水平。企业领导人不仅应重视财务管理,还要协同搞好资金营运管理。
