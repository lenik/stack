    许多咨询公司认为,公司经营以人力资本为主,不需太多流动资金,更不需过高的资本投入。真是这样吗?

    现金流管理目前在生产型或商业流通型企业得到经营管理人员越来越多的重视,近日EVA概念在国内的推广者——思腾思特公司所面临着可能退出中国市场的尴尬局面,却不得不使许多公司引以为鉴,在现金流管理方面予以更多的重视。

    咨询公司很少提及现金流的问题,与思腾思特公司类似,典型的专业咨询公司,在现金流管理方面也和其他企业一样,存在问题。如果您的企业出现如下问题,那么,你的现金流已经亮起红灯。

    你的现金流能有效满足业务发展吗?

    ●资本金不足,无后续资金注入

    思腾思特公司,成立时启动资金仅一两百万美元,股东以财务性投资为主要动机,对于公司在运营中发生的现金流耗损没有通过进一步的注资来支持。许多咨询公司也都认为,公司经营以人力资本为主,不需太多流动资金,更不需过高的资本投入。孰不知,咨询公司运营流动资金的充足也是公司业务持续发展的基本条件,没有充足的现金流量,就似人体血脉供养不足,必将给公司持续经营带来问题。

    ●市场拓展并未带来足够的现金回流

    目前国内对咨询公司专业服务虽然十分认同,但对服务价值的认定却仍未充分考虑,许多咨询公司付出的专业劳动并不能获取充分的现金回报。思腾思特公司进入中国不久,就推出了“中国上市公司财富创造和毁灭排行榜”,并且以管理咨询、财务咨询等手段向中国不少国有大中型企业推广EVA概念及应用方法,市场增长潜力的确不小。然而由于处于市场拓展初期项目定价往往打有折扣,且客户来源于国内,咨询服务难度大,现金收益却来得低微,项目做了不少,年咨询业务收入却仅有1000余万元,扣除相关付现成本后,留存公司现金流量已很有限。

    ●缺乏其他有效的融资手段

    资本金较小,经营收益又不理想,且无足额的有效实物资产用于抵押,一般也就很难通过银行流动奖金贷款等其他融资手段解决资金缺口。

    现金支出是否难以控制?

    ●人力成本支出没有有效的规划

    许多咨询公司已经意识到人力资源对公司业务发展的重要性,但在人力成本支出方面却没有合理有效的规划,难以对员工形成有效的激励。员工对薪金收入不满,工作积极性不高,同时还会在公司财务控制不严的情况下设法增加自身各种费用开支,从而导致公司不能有效控制人力成本,现金支出不确定性增加,公司经营风险增大。

    ●项目运作成本没有有效的控制

    项目运作中,也没有有效的成本核算与支出控制措施,项目运作成本往往高出项目收入,导致在项目运作层面上的现金收支不平衡。

    ●公司财务控制体系并不健全

    咨询公司与生产型企业、商业流通性企业相比,规模要小许多。因而,公司往往忽视在财务控制体系方面的建设,现金预算制度、项目成本核算制度、费用报销制度等不甚健全,最终致使公司成本结束软化,现金开支难以预期与控制。现金流入的不足与现金支出的难以控制,会使咨询公司的现金趋于匮乏,最终将使公司正常的人力成本开支、项目成本开支以及办公成本开支难以维继,从而使公司陷于资金链断裂,公司面临破产的局面。

    现金流量管理的目标是否明确?

    要有效加强咨询公司现金流管理,为其经营发展创造良好条件,需从首先明确现金流量管理的目标开始。

    咨询公司主动进行现金流管理的目标包括:

    1保证公司日常运营合理开支;2增强现金流量的可预见性;3为不可预期的现金支出提供合理的内外部融资支持;4为闲置资金进行有效增值;5为公司业务拓展提供决策支持;6信用管理与融资能力管理。

    建立一个有效的财务控制体系

    建立一个有效的财务控制体系也是加强现金流管理的一个必要工具,可以从三方面来进行控制,让你的现金流管理拉响警报。

    ●人力成本控制

    人力成本控制应注重对人力资源的有效激励和成本支出的相对可控。人力成本主要包括工资、奖金、项目分成、福利费用、日常费用开销、特殊费用开销等。其中,对于工资、奖金、福利费用、日常费用开销等,属于可预期的日常人工成本,应在公司现金流安排中优先保证,对于奖金、分成、特殊开销等不属日常人工成本,存在一定程度的预期偏差,但公司应根据有关员工提出的建议及实际相关项目时间安排节点,纳入年度或月度计划,从计划上给予事先保证。

    ●项目成本控制

    项目成本控制是咨询公司实际赢利和现金收支平衡的重要手段。项目成本控制不仅责任在于公司财务部门,更在于培养业务部门在项目运作中的收入成本意识,既要有效节约成本,又要切实保证项目成本合理开支所需现金。

    ●现金预算制度

    咨询公司同样应加强现金流管理的计划性,为此可由财务部门牵头,各业务部门参与,以人和项目为基本要素,建立现金预算制度。由各业务部门编制现金收支计划,由财务部门汇总并监督执行。现金收支计划包括年度计划与月度计划。年度计划主要为公司整体业务进行提供决策参考,月度计划主要为项目实际进程中的有关收支提供具体支持。

    为经营发展创造良好条件

    健康的现金流量管理可以从经营、投资、融资等三方面为业务发展提供支持:

    ●对经营活动的支持

    1公司经营活动提供基本的现金流量保证;2公司进行业务拓展和市场扩张提供资金决策依据。

    ●对投资能力的支持

    1通过灵活运用各种短期投资工具,既保证资金的流动性,又使闲置资金有效增值;2通过对公司积累资金的充分利用和有关项目或操作平台的财务分析,寻求长期投资机会,帮助公司实际业务经营的多元化。

    ●对融资能力的支持

    1通过现金流的良好管理,实现公司财务指标的优化,提升企业财务资信;2通过积极拓展内源融资以外的其他外部融资,提升公司资金调度能力,合理利用财务杠杆,最大化公司价值。
