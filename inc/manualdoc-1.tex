\documentclass[hyperref, oneside]{book}

\usepackage{xunicode,xltxtra}

\usepackage[slantfont,boldfont]{xeCJK}
\usepackage[geometry, misc]{ifsym} % DiamondShadowC, Cube
\usepackage{manfnt}             % dbend
\usepackage{mathabx}            % boxplus
\usepackage{mathrsfs}           % mathscr
\usepackage{keystroke}          % keystroke
% \usepackage{MnSymbol}           % nequiv
\usepackage[cm-default]{fontspec}

\usepackage[top=1in, bottom=1in, left=1in, right=1in]{geometry}
\usepackage[pdfborder={0 0 0}]{hyperref}
\usepackage{pdflscape}          % landscape
% \usepackage{multicol}           % Multi column article

\usepackage{fancyhdr}
\usepackage{fancybox}
\usepackage[vario]{fancyref}
\usepackage{url}
\usepackage{xstring}            % StrSubstritute, IfSubStr
\usepackage{enumitem}           % itemize
% \usepackage{float}
\usepackage{framed}

\usepackage{graphicx}           % includegraphics
\usepackage{tikz}

% \usepackage[table]{xcolor}
\usepackage{pgfplotstable}
\usepackage{booktabs}           % toprule, bottomrule
%\usepackage{tabularx}           % {X}
\usepackage{longtable}          % Alt. to tabular, but not tabularx
\usepackage{multirow}           % Used by: history.tex, TX tables.
\usepackage{arydshln}           % hdashline

% Kill the unavailable in EU1 warnings.
\renewcommand\nobreakspace{}

\setCJKmainfont{WenQuanYi Micro Hei}

\XeTeXlinebreaklocale "zh"
\XeTeXlinebreakskip = 0pt plus 1pt
\linespread{1.2}
\setlength{\columnsep}{.5in}

\newcommand*\ajax{AJAX}

\newcommand{\example}[1]{
    \vskip .2cm
    \doublebox{
        \begin{minipage}{.9\linewidth}
        \textbf{例:} #1
        \end{minipage}
    }
    \vskip .5cm
}

\newcommand{\syntax}[1]{
    \vskip .2cm
    \doublebox{
        \begin{minipage}{.9\linewidth}
        #1
        \end{minipage}
    }
    \vskip .5cm
}

\newcommand\screenshot[1]{
    \vskip .1cm
    \shadowbox{
        \includegraphics[scale=0.35]{#1}
    }
    \vskip .5cm
}

\newcommand\button[1]{ \keystroke{#1} }
\newcommand\textbox[1]{ \ovalbox{#1} }

\newcommand\ol[1]{\begin{enumerate}{#1}\end{enumerate}}
\newcommand\ul[1]{\begin{itemize}{#1}\end{itemize}}

\newcounter{operation}
\renewcommand\theoperation{\Roman{operation}}

\newcommand\opset[2]{
    \large{与 #1 相关的工作流有:}\normalsize
    \setcounter{operation}{1}
    \begin{itemize}[label=\DiamondShadowC]
        \addtolength{\leftskip}{5mm}
        #2
    \end{itemize}
}

\newcommand\ops[2]{
    \large{工作流\theoperation:按照下面所列的步骤进行#1。}\normalsize
    \addtocounter{operation}{1}
    \noexpandarg\IfStrEq{#2}{}{
        (略)
    }{
        \begin{enumerate}
            \addtolength{\leftskip}{8mm}
            #2
        \end{enumerate}
    }
}

\newcommand\Fyes{$\bigcirc$}
\newcommand\Fno{$\times$}

\newcommand\makefeaturetable[1]{
    \centering
    %\begin{table}
    \pgfplotstabletypeset[
        begin table=\begin{longtable},
        end table=\end{longtable},
        every first row/.append style={
            before row={
                \caption{数据特性} \\ \toprule
                & 对象 & 类型 & 设计规模 & 访问控制 & 缓存 & 预置数据 & 审核支持 & 会计对象 & 工资采集 \\ \midrule
              \endfirsthead
                         \multicolumn{10}{l}{(……续前)} \\ \midrule
                & 对象 & 类型 & 设计规模 & 访问控制 & 缓存 & 预置数据 & 审核支持 & 会计对象 & 工资采集 \\ \midrule
              \endhead
                \midrule \multicolumn{10}{r}{(参考后面继续……)} \\ \bottomrule
              \endfoot
                \midrule \multicolumn{10}{r}{(完)} \\ \bottomrule
              \endlastfoot
            }},
        % name, label, type, scale, samples, attrs
        alias/访问控制/.initial=2,
        alias/缓存/.initial=5,
        alias/审核支持/.initial=5,
        alias/会计对象/.initial=5,
        alias/工资采集/.initial=5,
        columns={ 0, 1, 2, 3, 访问控制, 缓存, 4, 审核支持, 会计对象, 工资采集 },
        columns/0/.style={ column name={}, column type=r|, string type },
        columns/1/.style={ column name={对象}, string type },
        columns/2/.style={ column name={类型}, string type,
                string replace={E}{关系},
                string replace={C}{系统},
                string replace={Dict}{目录},
                string replace={UI}{实体},
                string replace={Tree}{结点},
                string replace={Process}{过程},
                string replace={MomentInterval}{记录},
                string replace={Ext}{高级},
                },
        columns/3/.style={ column name={设计规模}, column type=r, std=0:6 },
        columns/4/.style={ column name={预置数据}, string type,
            preproc cell content/.code={
                \pgfkeyssetvalue{/pgfplots/table/@cell content}{
                    ####1
                }}},
        columns/访问控制/.style={ string type,
            preproc cell content/.code={
                \pgfkeyssetvalue{/pgfplots/table/@cell content}{
                    \IfStrEqCase{####1}{
                        {E}{\Fno}
                        {Pool}{\Fno}}[\Fyes]
                }}},
        columns/缓存/.style={ string type,
            preproc cell content/.code={
                \pgfkeyssetvalue{/pgfplots/table/@cell content}{
                    \IfSubStr{####1}{C}{\Fyes}{\Fno}
                }}},
        columns/审核支持/.style={ string type,
            preproc cell content/.code={
                \pgfkeyssetvalue{/pgfplots/table/@cell content}{
                    \IfSubStr{####1}{V}{\Fyes}{\Fno}
                }}},
        columns/会计对象/.style={ string type,
            preproc cell content/.code={
                \pgfkeyssetvalue{/pgfplots/table/@cell content}{
                    \IfSubStr{####1}{A}{\Fyes}{\Fno}
                }}},
        columns/工资采集/.style={ string type,
            preproc cell content/.code={
                \pgfkeyssetvalue{/pgfplots/table/@cell content}{
                    \IfSubStr{####1}{S}{\Fyes}{\Fno}
                }}},
        %every head row/.style={ before row=\toprule, after row=\midrule },
        %every last row/.style={ after row=\bottomrule },
        col sep=comma, header=false
        ]{#1}
    %\end{table}
}

\newcommand\maketxtable[4]{
    $\boxplus$ {\textbf{#1}}
    \vskip 3mm
    { \addtolength{\leftskip}{5mm} #4 }
    \vskip 1mm
    {
    %\centering
    \pgfplotstabletypeset[
        begin table=\begin{longtable},
        end table=\end{longtable},
        %empty header,
        every first row/.append style={
            before row={
                \caption{#1 (#2) 的数据交换规范}  \\ \cline{3-11}
                & & & 名称 & SQL 字段 & 要求 & 类型 & \multicolumn{2}{c}{长度} & 系统 & 说明 \\ \cline{3-11}
              \endfirsthead
                         \multicolumn{11}{l}{(……续前)} \\ \hdashline
                & & & 名称 & SQL 字段 & 要求 & 类型 & \multicolumn{2}{c}{长度} & 系统 & 说明 \\ \cline{3-11}
              \endhead
                \cline{3-11} \multicolumn{11}{r}{(参考后面继续……)} \\ \hdashline
              \endfoot
                \cline{3-11}
              \endlastfoot
            }},
        % 0      1    2     3      4        5      6    7
        % group, num, name, label, sqlname, flags, len, description
        alias/类型/.initial=5,
        alias/系统/.initial=5,
        alias/要求/.initial=5,
        columns={ 0, 1, 2, 3, 4, 要求, 类型, 6, 系统, 7 },
        columns/0/.style={ column name={}, column type=r@{}, string type,
            preproc cell content/.code={
                \pgfkeyssetvalue{/pgfplots/table/@cell content}{
                    \IfStrEqCase{####1}{
                        {}{}
                        {Entity}{$\mathscr{I}$}
                        {EntityAuto}{$\mathscr{I}$+}
                        {EntitySpec}{$\mathscr{I}$+}
                        {CEntity}{$\mathscr{C}$}
                        {CEntityAuto}{$\mathscr{C}$+}
                        {CEntitySpec}{$\mathscr{C}$+}
                        {UIEntity}{$\mathscr{U}$}
                        {UIEntityAuto}{$\mathscr{U}$+}
                        {UIEntitySpec}{$\mathscr{U}$+}
                        {TreeEntity}{$\mathscr{T}$}
                        {TreeEntityAuto}{$\mathscr{T}$+}
                        {TreeEntitySpec}{$\mathscr{T}$+}
                        {DictEntity}{典}
                        {NameDict}{典+}
                        {ProcessEntity}{过程}}[--]
                }}},
        columns/1/.style={ column name={}, column type=c@{ }, string type,
            preproc cell content/.code={
                \pgfkeyssetvalue{/pgfplots/table/@cell content}{
                }}},
        columns/2/.style={ column name={}, column type=r|, verb string type },
        columns/3/.style={ column name={名称}, column type=c, string type },
        columns/4/.style={ column name={SQL 字段}, column type=c, verb string type },
        columns/系统/.style={ column type=c, string type,
            preproc cell content/.code={
                \pgfkeyssetvalue{/pgfplots/table/@cell content}{
                    \IfSubStr{####1}{t}{\Fyes}{\Fno}
                }}},
        columns/类型/.style={ column type=c, string type,
            preproc cell content/.code={
                \pgfkeyssetvalue{/pgfplots/table/@cell content}{%
                    \IfSubStr{####1}{DT}{日期/时间}{%
                        \IfSubStr{####1}{D}{日期}{}%
                        \IfSubStr{####1}{T}{时间}{}%
                        }%
                    \IfSubStr{####1}{N}{数值}{}%
                    \IfSubStr{####1}{S}{字符串}{}%
                    \IfSubStr{####1}{R}{引用}{}%
                    \IfSubStr{####1}{Z}{数据流}{}%
                }}},
        columns/要求/.style={ column type=c, string type,
            preproc cell content/.code={
                \pgfkeyssetvalue{/pgfplots/table/@cell content}{
                    \IfSubStr{####1}{o}{\Fno}{\Fyes}
                }}},
        columns/6/.style={ column name={长度}, dec sep align },
        columns/7/.style={ column name={描述}, column type=m{8cm}, verb string type },
        %every head row/.style={ before row=\cline{3-11}, after row=\cline{3-11} },
        %every last row/.style={ after row=\cline{3-11} },
        col sep=colon, header=false
    ]{#3}
    }
    \vskip 8mm
}

\title {\modtitle \\ \modsubtitle \\ 使用说明书}
\author{SECCA项目组 \\ $\infty$ \small{海宁市智恒软件有限公司}}

\begin{document}

\renewcommand*\sectionmark[1]{\markright{\thesection. #1}}
%\renewcommand*\subsectionmark[1]{\markright{\thesubsection. #1}}
\renewcommand*\thesection{\arabic{section}}

\maketitle

\renewcommand\contentsname{目录/Contents}
\tableofcontents
\clearpage

\section{工艺管理}
    \ul {
    \item  新建工艺
        \ul {
        \item  点击\button{新建}按钮,弹出新建工艺对话框
        \screenshot{1.png}
        \item  输入工艺\textbox{名称}
        \item  输入工艺\textbox{描述}
        \item  点击\button{保存}完成新增工艺
        }
    \item  查看工艺
    \item  修改工艺
        \ul {
        \item  选择需要修改的工艺,点击\button{修改}按钮,弹出修改对话框
        \screenshot{2.png}
        \item  修改相应的工艺\textbox{名称}
        \item  修改相应的工艺\textbox{描述}
        \item  点击\button{保存}按钮完成修改
        }
    \item  删除工艺
        \ul {
        \item  选择需要删除的工艺
        \item  点击\button{删除}按钮,弹出确认删除对话框
        \screenshot{3.png}
        \item  点击确定完成删除
        }
    }


\section{BOM管理}
    \ul {
    \item  新建BOM物料清单
        \ul {
        \item  点击工具栏上的\button{新建}按钮,弹出新建BOM物料清单对话框
        \screenshot{4.png}
        \item  在\button{基本信息}标签下选择\button{物料}
        \item  输入\textbox{生效}
        \item  输入\textbox{失效}日期
        \item  输入\textbox{工资}
        \item  输入\textbox{电费}
        \item  输入\textbox{设备费}
        \item  输入\textbox{其他费用}
        \item  在\button{BOM明细}标签下,点击\button{添加}按钮,弹出添加明细对话框
        \screenshot{5.png}
        \item  \button{选择成品}或者\button{半成品}作为清单项目之一
        \item  勾选\button{是否有效}
        \item  选择BOM清单项目的\button{生效},\button{失效日期}
        \item  输入\textbox{数量}
        \item  输入\textbox{描述}
        \item  点击\button{确定}按钮,完成一个清单项目的添加
        \item  点击\button{保存}按钮,完成新建BOM物料清单
        }
    }
    \ul {
    \item  选择物料清单,点击工具栏上的\button{查看}按钮,查看BOM物料清单详细
    }
    \ul {
    \item  选择物料清单,点击工具栏上的\button{修改}按钮,弹出修改BOM清单对话框
        \ul {
        \item  修改BOM物料清单的\textbox{基本信息}
        \item  修改BOM清单\textbox{明细项目}
        \item  点击\button{保存}按钮,完成BOM物料清单的修改
        }
    }
    \ul {
    \item  工艺设置
        \ul {
        \item  选择BOM物料清单,点击工具栏上的\button{BOM树}按钮,弹出工艺设置对话框
        \screenshot{6.png}
        \item  点击\button{设置}按钮,弹出工艺流程设置对话框
        \screenshot{7.png}
        \item  点击\button{添加}按钮,弹出工艺明细对话框
        \screenshot{8.png}
        \item  在\textbox{基本信息}标签下,输入\textbox{工艺基本信息}
        \screenshot{9.png}
        \item  在\button{消耗材料}标签下,点击\button{添加}按钮,弹出工艺消耗材料对话框
        \item  选择\button{消耗材料},输入\item  输入{数量}
        \item  点击\button{确定},添加一种消耗材料
        \screenshot{10.png}
        \item  在\button{质检标准值}标签下,点击\button{添加}按钮,弹出添加质检标准对话框
        \item  输入\textbox{质检内容}
        \item  输入\textbox{标准值}
        \item  输入\textbox{标准描述}
        \item  点击\button{确定},添加一项质检标准
        \item  点击工艺明细对话框下的\button{确定}按钮,完成工艺项目明细设置
        \item  点击\button{保存}按钮,保存工艺设置
        }
    }


\section{附录}

    \subsection{设计参数}
        \begin{landscape}
            \makefeaturetable{matrix.csv}
        \end{landscape}

    \subsection{数据依赖}
        (略)

    \subsection{数据交换格式}
        \begin{landscape}
            \subsubsection{ \nequiv 销售机会 } {
    \addtolength{\leftskip}{5mm}
    销售员在跑销售时发掘的潜在的销售可能。


    \maketxtable{销售机会}{chance}{TX.Chance.tab}
}
\subsubsection{ \nequiv 行动记录,日志 } {
    \addtolength{\leftskip}{5mm}
    销售员每天记录的日志,计划。


    \maketxtable{行动记录,日志}{chance_action}{TX.ChanceAction.tab}
}
\subsubsection{ \nequiv 洽谈类型 } {
    \addtolength{\leftskip}{5mm}
    行动日志发生时,和客户所产用的交流方式。


    \maketxtable{洽谈类型}{chance_action_style}{TX.ChanceActionStyle.tab}
}
\subsubsection{ \nequiv 机会分类 } {
    \addtolength{\leftskip}{5mm}
    普通,重要,次要,其他。


    \maketxtable{机会分类}{chance_category}{TX.ChanceCategory.tab}
}
\subsubsection{ \nequiv 竞争对手 } {
    \addtolength{\leftskip}{5mm}
    在向一个客户销售产品时,会碰到很多对手。


    \maketxtable{竞争对手}{chance_competitor}{TX.ChanceCompetitor.tab}
}
\subsubsection{ \nequiv 机会客户关联类 } {
    \addtolength{\leftskip}{5mm}
    一个销售机会对应很多客户,以本类建立起关联。


    \maketxtable{机会客户关联类}{chance_party}{TX.ChanceParty.tab}
}
\subsubsection{ \nequiv 机会来源 } {
    \addtolength{\leftskip}{5mm}
    别人介绍,互联网,电话扫。


    \maketxtable{机会来源}{chance_source_type}{TX.ChanceSourceType.tab}
}
\subsubsection{ \nequiv 机会阶段 } {
    \addtolength{\leftskip}{5mm}
    描述机会当前的阶段,如:初始化,交涉中,合同签订。


    \maketxtable{机会阶段}{chance_stage}{TX.ChanceStage.tab}
}
\subsubsection{ \nequiv 供货方式 } {
    \addtolength{\leftskip}{5mm}
    产品供货方式字典类。


    \maketxtable{供货方式}{procurement_method}{TX.ProcurementMethod.tab}
}
\subsubsection{ \nequiv 采购原则 } {
    \addtolength{\leftskip}{5mm}
    产品采购原则字典类。


    \maketxtable{采购原则}{purchase_regulation}{TX.PurchaseRegulation.tab}
}
\subsubsection{ \nequiv 选型明细 } {
    \addtolength{\leftskip}{5mm}
    产品选型,一般包含产品的外部名称,外部规格等,主要给客户说明用。


    \maketxtable{选型明细}{wanted_product}{TX.WantedProduct.tab}
}
\subsubsection{ \nequiv 选型产品的附加属性 } {
    \addtolength{\leftskip}{5mm}
    除了规格型号外,其他的附加属性。


    \maketxtable{选型产品的附加属性}{wanted_product_attribute}{TX.WantedProductAttribute.tab}
}
\subsubsection{ \nequiv 报价 } {
    \addtolength{\leftskip}{5mm}
    选型产品的报价。


    \maketxtable{报价}{wanted_product_quotation}{TX.WantedProductQuotation.tab}
}
\subsubsection{ \nequiv 选型产品基类。 } {
    \addtolength{\leftskip}{5mm}
    (无描述)


    \maketxtable{选型产品基类。}{wanted_productxp}{TX.WantedProductXP.tab}
}

        \end{landscape}

    \clearpage
    \subsection{版本历史}
        \begin{center}
        % \rowcolors{1}{white}{lightgray}
        \caption{版本历史} \centering
        \input{history}
        \end{center}

\section{索引}

    \listoftables
    \listoffigures

\end{document}
