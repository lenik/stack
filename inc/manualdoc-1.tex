\documentclass[hyperref, oneside]{book}

\usepackage[top=1in, bottom=1in, left=1in, right=1in]{geometry}
\usepackage[cm-default]{fontspec}
\usepackage{xunicode,xltxtra}
\usepackage[slantfont,boldfont]{xeCJK}
\usepackage[pdfborder={0 0 0}]{hyperref}
\usepackage{multicol}
% \usepackage{float}
\usepackage{framed}
\usepackage{fancybox}
\usepackage{fancyhdr}
\usepackage[vario]{fancyref}
\usepackage{url}
\usepackage{graphicx}   % includegraphics
\usepackage{keystroke}  % keystroke
\usepackage{manfnt}     % dbend
\usepackage{tikz}
\usepackage{pgfplotstable}
\usepackage{booktabs}   % toprule, bottomrule
\usepackage{xstring}    % StrSubstritute, IfSubStr
\usepackage{enumitem}   % itemize
\usepackage[geometry]{ifsym}

\setCJKmainfont{WenQuanYi Micro Hei}

\XeTeXlinebreaklocale "zh"
\XeTeXlinebreakskip = 0pt plus 1pt
\linespread{1.2}
\setlength{\columnsep}{.5in}

\newcommand*\ajax{AJAX}

\newcommand{\example}[1]{
    \vskip .2cm
    \doublebox{
        \begin{minipage}{.9\linewidth}
        \textbf{例:} #1
        \end{minipage}
    }
    \vskip .5cm
}

\newcommand{\syntax}[1]{
    \vskip .2cm
    \doublebox{
        \begin{minipage}{.9\linewidth}
        #1
        \end{minipage}
    }
    \vskip .5cm
}

\newcommand\screenshot[1]{
    \vskip .1cm
    \shadowbox{
        \includegraphics[scale=0.35]{#1}
    }
    \vskip .5cm
}

\newcommand\button[1]{ \keystroke{#1} }
\newcommand\textbox[1]{ \ovalbox{#1} }

\newcommand\ol[1]{\begin{enumerate}{#1}\end{enumerate}}
\newcommand\ul[1]{\begin{itemize}{#1}\end{itemize}}

\newcounter{operation}
\renewcommand\theoperation{\Roman{operation}}

\newcommand\opset[2]{
    \large{与 #1 相关的工作流有:}\normalsize
    \setcounter{operation}{1}
    \begin{itemize}[label=\DiamondShadowC]
        \addtolength{\leftskip}{5mm}
        #2
    \end{itemize}
}

\newcommand\ops[2]{
    \large{工作流\theoperation:按照下面所列的步骤进行#1。}\normalsize
    \addtocounter{operation}{1}
    \noexpandarg\IfStrEq{#2}{}{
        (略)
    }{
        \begin{enumerate}
            #2
        \end{enumerate}
    }
}

\newcommand\Fyes{$\bigcirc$}
\newcommand\Fno{$\times$}

\newcommand\makefeaturetable{
    \begin{table} \caption{数据特性} \centering
    \pgfplotstabletypeset[
        % name, label, type, scale, samples, attrs
        alias/访问控制/.initial=2,
        alias/缓存/.initial=5,
        alias/审核支持/.initial=5,
        alias/会计对象/.initial=5,
        alias/工资采集/.initial=5,
        columns={ 0, 1, 2, 3, 访问控制, 缓存, 4, 审核支持, 会计对象, 工资采集 },
        columns/0/.style={ column name={}, column type=r|, string type },
        columns/1/.style={ column name={对象}, string type },
        columns/2/.style={ column name={类型}, string type,
                string replace={E}{系统},
                string replace={C}{系统},
                string replace={Dict}{注册表},
                string replace={UI}{实体},
                string replace={Tree}{结点},
                string replace={Process}{过程},
                string replace={MomentInterval}{事件},
                string replace={Ext}{扩展池},
                },
        columns/3/.style={ column name={设计规模}, column type=r },
        columns/4/.style={ column name={预置数据}, string type,
            preproc cell content/.code={
                \pgfkeyssetvalue{/pgfplots/table/@cell content}{
                    ###1
                }}},
        columns/访问控制/.style={ string type,
            preproc cell content/.code={
                \pgfkeyssetvalue{/pgfplots/table/@cell content}{
                    \IfStrEqCase{###1}{
                        {E}{\Fno}
                        {Ext}{\Fno}}[\Fyes]
                }}},
        columns/缓存/.style={ string type,
            preproc cell content/.code={
                \pgfkeyssetvalue{/pgfplots/table/@cell content}{
                    \IfSubStr{###1}{C}{\Fyes}{\Fno}
                }}},
        columns/审核支持/.style={ string type,
            preproc cell content/.code={
                \pgfkeyssetvalue{/pgfplots/table/@cell content}{
                    \IfSubStr{###1}{V}{\Fyes}{\Fno}
                }}},
        columns/会计对象/.style={ string type,
            preproc cell content/.code={
                \pgfkeyssetvalue{/pgfplots/table/@cell content}{
                    \IfSubStr{###1}{A}{\Fyes}{\Fno}
                }}},
        columns/工资采集/.style={ string type,
            preproc cell content/.code={
                \pgfkeyssetvalue{/pgfplots/table/@cell content}{
                    \IfSubStr{###1}{S}{\Fyes}{\Fno}
                }}},
        every head row/.style={ before row=\toprule, after row=\midrule },
        every last row/.style={ after row=\bottomrule },
        col sep=comma, header=false
        ]{matrix.csv}
    \end{table}
}

\title {\modtitle \\ \modsubtitle}
\author{SECCA项目组 \\ $\infty$ \small{海宁市智恒软件有限公司}}

\begin{document}

\renewcommand*\sectionmark[1]{\markright{\thesection. #1}}
%\renewcommand*\subsectionmark[1]{\markright{\thesubsection. #1}}
\renewcommand*\thesection{\arabic{section}}

\maketitle

\renewcommand\contentsname{目录/Contents}
\tableofcontents
\clearpage

\section{工艺管理}
    \ul {
    \item  新建工艺
        \ul {
        \item  点击\button{新建}按钮,弹出新建工艺对话框
        \screenshot{1.png}
        \item  输入工艺\textbox{名称}
        \item  输入工艺\textbox{描述}
        \item  点击\button{保存}完成新增工艺
        }
    \item  查看工艺
    \item  修改工艺
        \ul {
        \item  选择需要修改的工艺,点击\button{修改}按钮,弹出修改对话框
        \screenshot{2.png}
        \item  修改相应的工艺\textbox{名称}
        \item  修改相应的工艺\textbox{描述}
        \item  点击\button{保存}按钮完成修改
        }
    \item  删除工艺
        \ul {
        \item  选择需要删除的工艺
        \item  点击\button{删除}按钮,弹出确认删除对话框
        \screenshot{3.png}
        \item  点击确定完成删除
        }
    }


\section{BOM管理}
    \ul {
    \item  新建BOM物料清单
        \ul {
        \item  点击工具栏上的\button{新建}按钮,弹出新建BOM物料清单对话框
        \screenshot{4.png}
        \item  在\button{基本信息}标签下选择\button{物料}
        \item  输入\textbox{生效}
        \item  输入\textbox{失效}日期
        \item  输入\textbox{工资}
        \item  输入\textbox{电费}
        \item  输入\textbox{设备费}
        \item  输入\textbox{其他费用}
        \item  在\button{BOM明细}标签下,点击\button{添加}按钮,弹出添加明细对话框
        \screenshot{5.png}
        \item  \button{选择成品}或者\button{半成品}作为清单项目之一
        \item  勾选\button{是否有效}
        \item  选择BOM清单项目的\button{生效},\button{失效日期}
        \item  输入\textbox{数量}
        \item  输入\textbox{描述}
        \item  点击\button{确定}按钮,完成一个清单项目的添加
        \item  点击\button{保存}按钮,完成新建BOM物料清单
        }
    }
    \ul {
    \item  选择物料清单,点击工具栏上的\button{查看}按钮,查看BOM物料清单详细
    }
    \ul {
    \item  选择物料清单,点击工具栏上的\button{修改}按钮,弹出修改BOM清单对话框
        \ul {
        \item  修改BOM物料清单的\textbox{基本信息}
        \item  修改BOM清单\textbox{明细项目}
        \item  点击\button{保存}按钮,完成BOM物料清单的修改
        }
    }
    \ul {
    \item  工艺设置
        \ul {
        \item  选择BOM物料清单,点击工具栏上的\button{BOM树}按钮,弹出工艺设置对话框
        \screenshot{6.png}
        \item  点击\button{设置}按钮,弹出工艺流程设置对话框
        \screenshot{7.png}
        \item  点击\button{添加}按钮,弹出工艺明细对话框
        \screenshot{8.png}
        \item  在\textbox{基本信息}标签下,输入\textbox{工艺基本信息}
        \screenshot{9.png}
        \item  在\button{消耗材料}标签下,点击\button{添加}按钮,弹出工艺消耗材料对话框
        \item  选择\button{消耗材料},输入\item  输入{数量}
        \item  点击\button{确定},添加一种消耗材料
        \screenshot{10.png}
        \item  在\button{质检标准值}标签下,点击\button{添加}按钮,弹出添加质检标准对话框
        \item  输入\textbox{质检内容}
        \item  输入\textbox{标准值}
        \item  输入\textbox{标准描述}
        \item  点击\button{确定},添加一项质检标准
        \item  点击工艺明细对话框下的\button{确定}按钮,完成工艺项目明细设置
        \item  点击\button{保存}按钮,保存工艺设置
        }
    }


\section{附录}

    \subsection{设计参数}
    \makefeaturetable

    \subsection{数据依赖}
    (略)

    \subsection{数据交换格式}

    \subsection{版本历史}
    %\input{changes}

\section{索引}

    \listoftables
    \listoffigures

\end{document}
